\chapter{Premières définitions et constructions}

\section{Définitions}

Déroulons les définitions.

\subsection{Monoïdes et groupes}

\begin{definition}
    Un \emph{monoïde}\index{Monoïde} est un couple $(M,\times)$ formé d'un
    ensemble $M$ et d'une application $\times:M\times M\rightarrow M$ qui
    vérifient les axiomes suivants :
    \begin{enumerate}
      \item associativité: $(l\cdot m)\cdot n=l\cdot (m\cdot n)$, $l,m,n\in M$;
      \item élément neutre: il existe $e_M\in M$ tel que $m\cdot e_M=m=e_M\cdot
          m$, $m\in M$.
    \end{enumerate}

    On dit qu'un monoïde $(M,\times)$ est un \emph{groupe}\index{Groupe} si, de
    plus
    \begin{enumerate}
    \setcounter{enumi}{2}
        \item Inverse: pour tout $m\in M$ il existe $n\in M$ tel  que $m\cdot
        n=e_M=n\cdot m$.
    \end{enumerate}
\end{definition}

Un monoïde $(M,\cdot)$ est dit \emph{abélien} ou \emph{commutatif} si $m\cdot
n=n\cdot m$, $m,n\in M$.

\begin{definition}
    Étant donnés deux monoïdes $M,N$, un \emph{morphisme de monoïdes} est une
    application $\phi:M\rightarrow N$ qui vérifie:
    \begin{enumerate}
        \item $\phi(m\cdot n)=\phi(m)\cdot\phi(n)$, $m,n\in M$;
        \item $\phi(e_M )=e_N$.
    \end{enumerate}
\end{definition}

Remarquons que l'application identité $Id:M\rightarrow M$ est un morphisme de
monoïdes et que si $\phi:L\rightarrow M$ et $\psi:M\rightarrow N$ sont des
morphismes de monoïdes alors $\psi\circ \phi:L\rightarrow N$ est un morphisme
de monoïdes.

On notera $Hom_{Mono}(M,N)$ l'ensemble des morphismes de monoïdes
$\phi:M\rightarrow N$ et, si $M=N$, $End_{Mono}(M):=Hom_{Mono}(M,M)$. Étant
donnés deux groupes $M,N$, un morphisme de groupes $\phi:M\rightarrow N$ est un
morphisme entre les monoïdes sous-jacents. Dans ce cas, on notera plutôt
$Hom_{Grp}(M,N)$ et $End_{Grp}(M)$ que $Hom_{Mono}(M,N)$, $End_{Mono}(M,N)$.

\subsection{Anneaux}

\begin{definition}
    Un \emph{anneau}\index{Anneau} est  un triplet $(A,+,\cdot)$ formé d'un
    ensemble $A$ et de deux applications $+,\cdot :A\times A\rightarrow A$ -
    appelées respectivement l'addition et la multiplication -  vérifiant les
    axiomes suivants :

    \begin{enumerate}
        \item $(A,+)$ est un groupe abélien; on note $0_A$ (ou simplement $0$)
            son élément neutre (appelé zéro) et $-a$ l'inverse d'un élément
            $a\in A$.
        \item $(A ,\cdot)$ est un monoïde; on note $1_A$ (ou simplement $1$)
            son élément neutre (appelé unité). 
        \item La multiplication est distributive par rapport à l'addition
            \textit{i.e.} $a\cdot (b+c)=a\cdot b+a\cdot c$ et $(b+c)\cdot
            a=b\cdot a+c\cdot a$, $a,b,c\in A$.
    \end{enumerate}
\end{definition}

Un anneau $A$ est dit \emph{commutatif} si $a\circ b=b\circ a$, $\forall a,b\in
A$.

\begin{remarque}
    Dans la suite, on écrira presque toujours $ab$ au lieu de $a\cdot b$,
    $0:=0_A$, $1:=1_A$. Par ailleurs, on omet presque toujours les données
    $+,\cdot$ des notations.
\end{remarque}

Le monoïde $(A ,\cdot)$ n'est pas un groupe en général; on note
$A^\times\subset A $ le sous-ensemble des éléments inversibles \textit{i.e.}
l'ensemble des $ a\in A$ tel qu'il existe $b\in A$ tel que $ab=1=ba$ ; c'est un
groupe d'élément neutre $1$. On note alors $a^{-1}\in A^\times$ l'inverse d'un
élément de $a\in A^\times$.

On dit qu'un anneau $A$ est \emph{un anneau à division} ou un \emph{corps
gauche} si $1\not=0$ et $A\setminus \lbrace 0\rbrace=A^\times$. Si $A$ est de
plus commutatif, on dit simplement que $A$ est un \emph{corps}.

\begin{exemples}
    \begin{itemize}[leftmargin=* ,parsep=0cm,itemsep=0cm,topsep=0cm]
        \item L'\emph{anneau nul} $A=\lbrace 0\rbrace$ (on n'a pas exclu
            $1\not=0$ dans la définition d'anneaux).
        \item L'anneau  $\Z$ des entiers. Dans ce cas  $\Z^\times=\lbrace \pm
            1\rbrace$.
        \item Les corps commutatifs, par exemple $\Q$, $\R$, $\CC$.  
        \item Si $M$ est un groupe abélien, l'ensemble $End_{Grp}(M)$ des
            endomorphismes du groupe abélien $M$ muni de
            $(\phi+\psi)(m)=\phi(m)+\psi(m)$ et $(\psi\cdot\phi)(m)=\psi\circ
            \phi(m)$ est  un anneau (non commutatif en général) de zéro
            l'application nulle et d'unité l'application identité. Dans ce cas,
            $End_{Grp}(M)^\times=Aut_{Grp}(M)$.  
        \item Si $M$ est un espace vectoriel sur un corps commutatif $k$,
            l'ensemble $End_k(M)$ des endomorphismes du $k$-espace vectoriel
            $M$ muni de $(\phi+\psi)(m)=\phi(m)+\psi(m)$ et
            $(\psi\cdot\phi)(m)=\psi\circ \phi(m)$ est un anneau (non
            commutatif si $M$ est de $k$-dimension $\geq 2$) de zéro
            l'application nulle et d'unité l'application identité. Dans ce cas,
            $End_k(M)^\times=GL_k(M)$.
        \item On rencontre aussi beaucoup d'anneaux en analyse: les anneaux
            $\mathcal{C}(X,\R)$ ou $\mathcal{C}(X,\CC)$ de fonctions continues à
            valeurs réelles  ou complexes sur un espace topologique $X$, les
            anneaux $L^p(X,\mu)$ de fonctions intégrables  sur un espace mesuré
            $(X,\mu)$, les anneaux de séries entières \textit{etc}.
    \end{itemize}
\end{exemples}

\begin{definition}
    Étant donnés deux anneaux $A,B$, un \emph{morphisme d'anneaux} est une
    application $\phi:A\rightarrow B$ qui induit à la fois un morphisme de
    groupes $\phi:(A,+)\rightarrow (B,+)$ et de monoides unitaires
    $\phi:(A,\cdot)\rightarrow (B,\cdot)$ \textit{i.e} qui vérifie:
    \begin{enumerate}
        \item $\phi(a+b)=\phi(a)+\phi(b)$, $a,b\in A$;q
        \item $\phi(a  b)=\phi(a) \phi(b)$, $a,b\in A$ et $\phi(1)=1$;
    \end{enumerate}
\end{definition}

On remarquera que l'application identité $Id:A\rightarrow A$ est un morphisme
d'anneaux et que si $\phi:A\rightarrow B$ et $\psi:B\rightarrow C$ sont des
morphismes d'anneaux alors $\psi\circ \phi:A\rightarrow C$ est un morphisme
d'anneaux. On notera $Hom(A,B)$ l'ensemble des morphismes d'anneaux
$\phi:A\rightarrow B$ et, si $A=B$, $End(A):=Hom(A,A)$.

On dit qu'un morphisme d'anneaux $\phi:A\rightarrow B$ est \emph{injectif},
(resp.  \emph{surjectif}, resp. un \emph{isomorphisme}) si l'application
d'ensembles sous-jacente est injective (resp. surjective, resp. bijective). On
vérifie que si $\phi:A\rightarrow B$ est un isomorphisme d'anneaux,
l'application inverse $\phi^{-1}:B\rightarrow A$ est automatiquement un
morphisme d'anneaux. Comme un morphisme d'anneaux $\phi:A\rightarrow B$ est en
particulier un morphisme de groupes, $\phi:A\rightarrow B$ est injectif si et
seulement si $\ker(\phi):=\phi^{-1}(0_B)=\lbrace 0_A\rbrace$. On notera aussi
$\im(\phi):=\phi(A)$.

Si $\phi:A\rightarrow B$ est un morphisme d'anneaux, on vérifie que
$\phi(A^\times)\subset B^\times$ et que $\phi:A\rightarrow B$ induit par
restriction un morphisme de groupes $\phi:A^\times\rightarrow B^\times$.

\subsubsection{Sous-anneaux}

Si $A$ est un anneau, un sous-anneau\index{Sous-anneau} de $A$ est un
sous-ensemble $A'\subset A$ tel que $1_A\in A'$ et $a'-b'\in A'$, $a'\cdot
b'\in A'$, $a',b'\in A'$.

\begin{exemples}
    \begin{itemize}[leftmargin=* ,parsep=0cm,itemsep=0cm,topsep=0cm]
        \item $\Z$ est un sous anneau de $\Q$, $\Q$ est un sous-anneau de $\R$,
            $\R$ est un sous-anneau de $\CC$. 
        \item Si $M$ est un espace vectoriel sur un corps commutatif $k$,
            $End_k(M)$ est un sous-anneau de $End_{Grp}(M)$. 
        \item $Z(A):=\lbrace a\in A\;|\; a\cdot b=b\cdot a,\; b\in
            A\rbrace\subset A$ est un sous-anneau de $A$, appelé le
            centre\index{Centre (Anneau)} de $A$. Par exemple
            $Z(End_k(M))=kId_M$ et $Z(A)=A$ si et seulement si $A$ est
            commutatif. 
        \item Si $\phi:A\rightarrow B$ est un morphisme d'anneaux, et
            $A'\subset A$ (resp. $B'\subset B$) est un sous-anneau alors
            $\phi(A')\subset B$ (resp. $\phi^{-1}(B')\subset A$) est un
            sous-anneau. En particulier, $\im(\phi)\subset B$ est un
            sous-anneau mais  $\ker(\phi)\subset A$ n'est un sous-anneau que si
            $A$ ou $B$ est l'anneau nul, sinon il ne contient pas $1$ (on verra
            un peu plus loin que $\ker(\phi)$ est ce qu'on appelle un idéal).
    \end{itemize}
\end{exemples}

\subsection{Algèbres sur un anneau commutatif}

Soit $A$ un anneau commutatif.

\begin{definition}
    Une \emph{$A$-algèbre}\index{Algèbre sur un anneau commutatif} est un
    couple $(B,\phi)$ où $B$ est  un anneau et $\phi:A\rightarrow B$ est un
    morphisme d'anneaux tel que $\hbox{\rm im}(\phi)\subset Z(B)$.
\end{definition}

On notera en général $\phi:A\rightarrow B$ ou simplement  (lorsque la donnée de
$\phi:A\rightarrow B$ ne peut prêter à confusion) $B$  la $A$-algèbre
$(B,\phi)$.

\begin{definition}
    Étant données deux $A$-algèbres $\phi_B:A\rightarrow B$,
    $\phi_C:A\rightarrow C$, un \emph{morphisme de $A$-algèbres} est un
    morphisme d'anneaux $\phi:B\rightarrow C$ tel que $\phi\circ \phi_B=\phi_C$
    :

    % TODO: Faire ça autrement
    \[\xymatrix{ & B \ar[dd]^\Phi \\ A \ar[ru]^{\Phi_B} \ar[rd]_{\Phi_C} \\ & C
    }\]
\end{definition}

On remarquera que l'application identité $Id:B\rightarrow B$ est un morphisme
de $A$-algèbres et que si $\phi:B\rightarrow C$ et $\psi:C\rightarrow D$ sont
des morphismes de $A$-algèbres alors $\psi\circ \phi:B\rightarrow D$ est un
morphisme de $A$-algèbres. On notera $Hom_A(B,C)$ l'ensemble des morphismes de
$A$-algèbres $\phi:B\rightarrow C$ et, si $B=C$, $End_A(B):=Hom_A(B,C)$. On dit
encore qu'un morphisme de $A$-algèbres $\phi:B\rightarrow C$ est
\emph{injectif}, (resp. \emph{surjectif}, resp. un \emph{isomorphisme}) si
l'application d'ensembles sous-jacente est  injective (resp. surjective, resp.
bijective). On vérifie que si $\phi:B\rightarrow C$ est un isomorphisme de
$A$-algèbres l'application inverse $\phi^{-1}:C\rightarrow B$ est
automatiquement un morphisme de $A$-algèbres.

\begin{remarque}
    On verra dans la partie II du cours, qu'une $A$-algèbre $\phi:A\rightarrow
    B$ est aussi la même chose qu'un anneau $B$ muni d'une structure de
    $A$-module  et qu'avec cette terminologie un morphisme de $A$-algèbres est
    un morphisme d'anneaux qui est aussi un morphisme de $A$-modules.
\end{remarque}

\begin{exemples}
\begin{itemize}[leftmargin=* ,parsep=0cm,itemsep=0cm,topsep=0cm]
    \item Le \emph{morphisme caractéristique}\index{Caractéristique
        (Morphisme)} $c_A:\Z\rightarrow A$, $1\rightarrow 1_A$ munit tout
        anneau $A$ d'une structure de $\Z$-algèbre canonique et tout morphisme
        d'anneaux $\phi:A\rightarrow B$ est automatiquement un morphisme de
        $\Z$-algèbres pour ces structures (\textit{i.e.} $\phi\circ c_A=c_B$).   
    \item L'inclusion $\iota_A: Z(A)\hookrightarrow  A $  munit tout anneau $A$
        d'une structure de $Z(A)$-algèbre canonique. 
    \item Si $A,B$ sont des anneaux commutatifs, tout morphisme d'anneaux
        $\phi:A\rightarrow B$ munit $B$ d'une structure de $A$-algèbre.
\end{itemize}
\end{exemples}

\begin{exercice}[Quaternions]
    On considère le $\R$-espace vectoriel $\HH$ de base $1,i,j,k$ muni du produit
    $\HH\times \HH\rightarrow \HH$ définie par $i^2=j^2=k^2=-1$, $ij=-ji=k$,
    $jk=-kj=i$, $ki=-ik=j$. 
    \begin{enumerate}
        \item Montrer que $(\HH,+,\cdot)$ est un anneau à division, non
            commutatif. Déterminer son centre   et en déduire que c'est  une
            $\R$-algèbre. 
        \item On note $i$ une racine carré de $-1$ dans $\CC$ et on considère les matrices

            % TODO: Utiliser matrix (pmatrix ?), pas tabular
            \[I:=\left(\begin{tabular}[c]{ll}
            $i$&$0$\\
            $0$&$-i$
            \end{tabular}\right),\; J:=\left(\begin{tabular}[c]{ll}
            $0$&$1$\\
            $-1$&$0$
            \end{tabular}\right),K:=\left(\begin{tabular}[c]{ll}
            $0$&$i$\\
            $i$&$0$
            \end{tabular}\right).\]

            Montrer que le sous-$\R$-espace vectoriel de $M_2(\CC)$ engendré par
            $Id$, $I$, $J$, $K$ est un sous-$\R$-algèbre de $(M_2(\CC),+,\cdot)$
            isomorphe à $\HH$.
     \end{enumerate}
\end{exercice}
 
\section{Produits}\label{Produit (Anneaux)}\label{Produit}\index{Produit}

Si $A_i$, $i\in I$ est une famille d'anneaux, on peut munir le produit
ensembliste $\prod_{i\in I}A_i$ d'une structure d'anneau en posant, pour
$\underline{a}=(a_i)_{i\in I}$, $\underline{b}=(b_i)_{i\in I}\in \prod_{i\in
I}A_i$ $$\underline{a}+\underline{b}=(a_i+b_i)_{i\in I},\;\; \underline{a}\cdot
\underline{b}=(a_i\cdot b_i)_{i\in I}$$ On a alors $0 =(0_{A_i})_{i\in I}$, $1
=(1_{A_i})_{i\in I}$. De plus, les \emph{projections} $p_i:\prod_{i\in
I}A_i\rightarrow A_i$, $\underline{a}\rightarrow a_i$, $i\in I$ sont
automatiquement des morphismes d'anneaux.

\begin{proposition}[Propriété universelle du produit]
    Pour toute famille d'anneaux $A_i$, $i\in I$ il existe un anneau $\Pi$ et
    une famille de morphisme d'anneaux $p_i:\Pi\rightarrow A_i$, $i\in I$ tels
    que pour tout anneau $A$ et famille de morphisme d'anneaux
    $\phi_i:A\rightarrow A_i$, $i\in I$, il existe un unique morphisme
    d'anneaux $\phi:A\rightarrow \Pi$ tel que $p_i\circ \phi=\phi_i$, $i\in I$.

    \[\xymatrix{ A \ar[rr]^{\Phi_i} \ar[rd]_\Phi && A_i \\ & \Pi
    \ar[ru]_{p_i}}\]
\end{proposition}

\begin{proof}
    Vérifions que $\Pi:= \prod_{i\in I}A_i$ et les $p_i:\prod_{i\in
    I}A_i\rightarrow A_i$, $\underline{a}\rightarrow a_i$, $i\in I$
    conviennent. Si $\phi:A\rightarrow \prod_{i\in I}A_i$ existe, la condition
    $p_i\circ \phi =\phi_i$, $i\in I$ force  $\phi(a)=(\phi_i(a))_{i\in I}$,
    $a\in A$. Cela montre l'unicité de $\phi$  sous réserve de son existence.
    Pour conclure, il faut vérifier que $\phi$ défini par
    $\phi(a)=(\phi_i(a))_{i\in I}$, $a\in A$ est bien un morphisme d'anneaux,
    ce qui résulte immédiatement des définitions.
\end{proof}

On peut aussi réécrire \ref{Produit} en disant que, pour tout anneau $A$
l'application canonique

\[\hbox{\rm Hom}(A,\prod_{i\in I}A_{i})\rightarrow \prod_{i\in I}\hbox{\rm Hom}
(A,A_{i}), \; \phi\rightarrow (p_i\circ \phi)_{i\in I}\]

est bijective ou encore, plus visuellement :

% TODO: Ne pas utiliser xymatrix
\[\xymatrix{&&A_i \\
A\ar@{.>}[r]^{\exists ! \phi}\ar@/^1pc/[urr]^{\phi_i}\ar@/_1pc/[drr]_{\phi_j}&\prod_{i\in I}A_i\ar[ur]^{p_i}\ar[dr]_{p_j}\\
&&A_j& }\]

Supposons que l'on ait un autre anneau  $\Pi'$ et une famille de morphisme
d'anneaux $p_i':\Pi'\rightarrow A_i$, $i\in I$ vérifiant aussi la propriété du
Lemme \ref{Produit}.1. On a alors, formellement:

\begin{enumerate}[leftmargin=* ,parsep=0cm,itemsep=0cm,topsep=0cm]
    \item un unique morphisme d'anneaux $\phi:\Pi\rightarrow \Pi'$ tel que
        $p_i'\circ \phi=p_i$, $i\in I$;
    \item un unique morphisme d'anneaux $\phi':\Pi'\rightarrow \Pi$ tel que
        $p_i\circ \phi'=p_i'$, $i\in I$;
    \item un unique morphisme d'anneaux $\psi:\Pi\rightarrow \Pi$ tel que
        $p_i\circ \psi=p_i$, $i\in I$;
    \item un unique morphisme d'anneaux $\psi':\Pi'\rightarrow \Pi'$ tel
        que $p_i'\circ \psi'=p_i'$, $i\in I$.
\end{enumerate}

Mais on voit que dans (3) $\psi=\phi'\circ \phi$ et $\psi=Id_\Pi$ conviennent.
L'unicité de $\psi$ dans (3) impose donc $\phi'\circ \phi=Id_\Pi$. Le même
argument dans (4) montre que $\phi\circ \phi'=Id_{\Pi'}$. Autrement dit, les
morphismes d'anneaux $\phi:\Pi\rightarrow \Pi'$ de (1) et
$\phi':\Pi'\rightarrow \Pi$ de (2) sont inverses l'un de l'autre. On dit de
façon un peu informelle que l'anneau produit  $p_i:\prod_{i\in I}A_i\rightarrow
A_i$, $i\in I$ est \emph{unique à unique isomorphisme près}. On rencontrera
beaucoup d'autres constructions de ce type dans la suite.

\begin{remarque}
    Soit $\phi_i:A_i\rightarrow B_i$, $i\in I$ une famille de morphismes
    d'anneaux.  En appliquant la propriété universelle des $p_j:\prod_{i\in
    I}B_i\rightarrow B_j$, $j\in I$ à la famille de morphismes d'anneaux

    \[\prod_{i\in
    I}A_i\stackrel{p_j}{\rightarrow}A_j\stackrel{\phi_j}{\rightarrow} B_j,\;
    j\in I\]

    on obtient un unique morphisme d'anneaux $\phi:=\prod_{i\in
    I}\phi_i:\prod_{i\in I}A_i\rightarrow \prod_{i\in I} B_i$ tel que $p_i\circ
    \phi=\phi_i\circ p_i$, $i\in I$.

    \[\xymatrix{ A_i \ar[r]^{\Phi_i} & B_i \\ \prod_{i\in I} A_i
    \ar[u]^{\pi^A_i} \ar[r]^\Phi & \prod_{i\in I}B_i \ar[u]_{\pi^B_i} }\]
\end{remarque}

Si $A_i=A$, $i\in I$, on note $\prod_{i\in I}A_i=A^I$. On peut  voir $A^I$
comme l'anneau des fonctions $a:I\rightarrow A$ muni de $(a+b)(i)=a(i)+b(i)$
et $(a\cdot b)(i)=a(i)\cdot b(i)$  de zéro l'application nulle et d'unité
l'application constante de valeur $1_A$. On notera qu'on a un morphisme
d'anneaux injectif canonique $\Delta_A:A\hookrightarrow A^I$, $a\rightarrow
(i\rightarrow a(i)=a)$ appelé morphisme diagonal (et qui, si $A$ est
commutatif,   fait de $A^I$  une $A$-algèbre de fa\c{c}on canonique).\\

Pour tout $\underline{a}=(a_i)_{i\in I}\in A^I$ notons
$supp(\underline{a}):=\lbrace i\in I\; |\; a_i\not= 0\rbrace\subset I$ le
\textit{support} de $\underline{a}$. Notons $$A^{(I)}:=\lbrace \underline{a}\in
A^I\; |\; |supp(\underline{a})|<+\infty\rbrace \subset A^I.$$ On observera que
$A^{(I)}\subset A^I$ est stable par différence et produit mais que, si $I$ est
infini, ce n'est pas un sous-anneau de $A^I$ car il ne contient pas $1_{A^I}$. 

% ENDPROOFREAD

\section{Algèbres de polynômes}\label{Poly}\index{Polynômes}Soit $A$ un anneau commutatif.   
Comme on vient de l'observer, le sous-ensemble $A^{(\N)}$ de $ A^\N$ est stable par différence et produit mais ce n'est pas un sous-anneau de $A^\N$ car il ne contient pas $1_{A^\N}$. En utilisant que $(\N,+)$ est un monoide on peut cependant    faire un anneau de $A^{(\N)}$, en le munissant d'une autre multiplication que celle héritée de $A^{\N}$. Notons  $e_n:=(\delta_{m,n}1_{A})_{m\in \N}$, $n\in \N$ et pour $a\in A$, $ae_n:=(\delta_{m,n}a)_{m\in \N}$, $n\in \N$ ; $A^{(\N)}$ contient les $ae_n$, $n\in \N$, $a\in A$ et, par définition,  tout élément $\underline{a}\in A^{(\N)}$ s'écrit de fa\c{c}on unique sous la forme $\underline{a}=\sum_{n\in \N}a_ne_n$. Munissons donc $A^{(\N)}$ de l'addition héritée de celle de $A^\N$ et du produit `de convolution' $*$ défini sur les éléments $e_n$, $n\in \N$ par 
$e_m*e_n=e_{m+n} $ et en général par
$$(\ref{Poly}.1)\;\; (\sum_{n\in\N}a_ne_n) *(\sum_{n\in \N}b_ne_n)=\sum_{n\in \N}(\sum_{i,j\in \N, i+j=n}a_ib_j)e_n.$$
On vérifie facilement que $(A^{(\N)},+,*)$ est un anneau commutatif ayant pour unité $e_0$. L'application canonique $\iota_A:A\rightarrow A^{(\N)}$, $a\rightarrow ae_0  $  est un morphisme d'anneaux.
 On note traditionnellement cet anneau $(A[X],+,\cdot)$ et on dit que $\iota:A\rightarrow A[X]$ est la $A$-algèbre des polynômes à une inderminée. On pose aussi $X^n:=e_n$, $n\in \N$ et $1:=X^0$ de sorte que (\ref{Poly}.1) se réécrit de fa\c{c}on plus intuitive sous la forme
 $$(\ref{Poly}.2)\;\; (\sum_{n\in\N}a_nX^n)(\sum_{n\in \N}b_nX^n)=\sum_{n\in \N}(\sum_{i,j\in \N, i+j=n}a_ib_j)X^n.$$
 
 \ref{Poly}.3 \textbf{Lemme.} (Propriété universelle de la $A$-algèbre des polynômes à une indéterminée) \textit{Pour tout anneau commutatif $A$, il existe une $A$-algèbre $\iota_A: A\rightarrow P$ munie d'un élément $p\in P$ tels que pour toute $A$-algèbre $\phi: A\rightarrow B$ et  $b\in B$, il existe un unique  morphisme de $A$-algèbres $ev^\phi_b:P\rightarrow B$  tel que $ ev^\phi_b(p)=b$. }

	$$ \xymatrix{A \ar[rd]_{\iota_A} \ar[rr]^\varphi && (B, b) \\ & (P, p) \ar[ru]_{ev_b^\Phi}} $$

\begin{proof} Vérifions que $\iota_A:A\rightarrow A[X]$ munie de $X$ conviennent. Si $ev_b^\phi:A[X]\rightarrow B$ existe,  on a par définition d'un morphisme de $A$-algèbres:
$$ev^\phi_b(\sum_{n\geq 0}a_nX^n)=\sum_{n\geq 0}ev_b^\phi(a_n)ev_b^\phi(X)^n=\sum_{n\geq 0}\phi(a_n)b^n,$$
d'où l'unicité de $ev_b^\phi$ sous réserve d'existence.  Pour conclure, il faut vérifier que $ev_b^\phi$ défini par  $ev^\phi_b(\sum_{n\geq 0}a_nX^n)= \sum_{n\geq 0}\phi(a_n)b^n,$ est bien un morphisme d'anneaux, ce qui là encore résulte immédiatement des définitions.
\end{proof}

  Le même argument  formel que celui utilisé dans \ref{Produit}.2 montre que la $A$-algèbre $\iota_A:A\rightarrow A[X]$ est unique à unique isomorphisme  près.\\
 
  On peut aussi réécrire \ref{Poly}.3 en disant que, pour toute $A$-algèbre $\phi:A\rightarrow B$  l'application canonique
$$\hbox{\rm Hom}_A(A[X],B)\rightarrow B,\; f\rightarrow f(X)$$
 est bijective. On adopte  en général la notation plus intuitive $ev_{b}^\phi(P)=P( b)$ et on dit que $ev_{b}^\phi$ est le morphisme d'évaluation en $b$.\\
\\
 Soit $\phi:A  \rightarrow B $ un morphisme  d'anneaux commutatifs. En appliquant la propriété universelle des $\iota_A:A\rightarrow A[X]$  à la $A$-algèbre
$$A\stackrel{\phi}{\rightarrow} B\stackrel{\iota_B}{\rightarrow}B[X]$$
on obtient un unique morphisme d'anneaux  $\tilde{\phi}:A[X]\rightarrow B[X]$ tel que $\iota_B\circ \phi=\phi\circ \iota_A$; explicitement $\tilde{\phi}(\sum_{n\geq 0}a_xX^n)=\sum_{n\geq 0}\phi(a_n)X^n$.\\ 

	$$ \xymatrix{ A\ar[r]^\Phi \ar[d]^{\iota_A} & B \ar[d]_{\iota_B} \\ A[X]\ar[r]_{\tilde{\Phi}} & B[X] } $$
 
 
   \ref{Poly}.4 \textbf{Remarque.} Ce qui nous a permis de définir le produit $*$ sur $A^{(\N)}$ et le fait que $(\N,+)$ est un monoïde: on a utilisé l'addition pour définir $e_n*e_m=e_{n+m}$, l'associativité de $*$ résulte de celle de $+$ sur $\N$ et le fait que $e_0$ soit l'unité de $A^{(\N)}$ du fait que $0$ est l'unité de $\N$.  Pour un monoïde $(M,\cdot)$ quelconque, l'application
 $$\hbox{\rm Hom}_{Mono}(\N,M)\rightarrow M,\; f\rightarrow f(1)$$
 est bijective d'inverse l'application qui à $m\in M$ associe le morphisme de monoïdes $f_m:(\N,+)\rightarrow (M,\cdot)$, $n\rightarrow m^n(=m\cdots m$ $n$ fois). Dans \ref{Poly}.3, se donner $p\in P$ et $b\in B$ revient donc à se donner des morphismes de monoïdes $\nu_A:(\N,+)\rightarrow (P,\cdot)$, $n\rightarrow p^n$ et $\nu:(\N,+)\rightarrow (B,\cdot)$, $n\rightarrow b^n$ et la condition $ev^\phi_b(p)=b$ signifie que $ev^\phi_b\circ \nu_A=\nu$. Avec ce point de vue, on peut reformuler \ref{Poly}.3 comme suit. \\
 
  \ref{Poly}.3' \textbf{Lemme.} (Propriété universelle de la $A$-algèbre des polynômes à une indéterminée) \textit{Pour tout anneau commutatif $A$, il existe une $A$-algèbre $\iota_A: A\rightarrow P$ et un morphisme de monoïdes $\nu_A:(\N,+)\rightarrow (P,\cdot)$ tels que pour toute $A$-algèbre $\phi: A\rightarrow B$ et tout morphisme de monoïdes $\nu:(\N,+)\rightarrow (B,\cdot)$, il existe un unique  morphisme de $A$-algèbres $ev^\phi_b:P\rightarrow B$  tel que $ ev^\phi_b\circ \nu_A=\nu$. }\\

  $$ \xymatrix{ & B \\ A \ar[ru]^\Phi \ar[rd]_{\iota_A} & & \mathbb{N} \ar[lu]_\nu \ar[ld]^{\nu_A} \\ & P \ar[uu]_{ev} } $$

 Ou encore: pour toute $A$-algèbre $\phi:A\rightarrow B$  l'application canonique
$$\hbox{\rm Hom}_A(A[X],B)\rightarrow \hbox{\rm Hom}_{Mono}(\N,B),\; f\rightarrow f\circ \nu_A.$$
 est bijective. Explicitement, $\nu_A:(\N,+)\rightarrow (A[X],\cdot)$ est le morphisme qui envoie $n$ sur $X^n$ donc si $f:A[X]\rightarrow B$ est un morphisme de $A$-algèbres, $f\circ \nu_A:(\N,+)\rightarrow (B,\cdot)$  est le morphisme qui envoie $n$ sur $f(X)^n$.\\

 
 
  \ref{Poly}.5 Avec le point de vue développé dans la Remarque  \ref{Poly}.4, on peut faire la construction précédente en rempla\c{c}ant $(\N,+)$ par n'importe quel monoïde $(N,\cdot)$ (non nécessairement commutatif, non nécessairement dénombrable) d'unité $1_N$.   Notons toujours $e_{n}:=(\delta_{m,n}1_{A})_{m\in N}$, $n\in N$ et pour $a\in A$, $ae_{n}:=(\delta_{m,n}a)_{m\in  N}$, $n\in N$ ; $A^{(N)}$ contient les $ae_{n}$, $n\in N$, $a\in A$ et, par définition,  tout élément $\underline{a}\in A^{(N)}$ s'écrit de fa\c{c}on unique sous la forme $\underline{a}=\sum_{\underline{n}\in N^r}a_ne_{n}$. En munissant $A^{(N)}$ de l'addition héritée de celle de $A^{N}$ et du produit `de convolution' $*$ défini sur les éléments $e_{n}$, $n\in N$ par 
$e_{m}*e_{n}=e_{m\cdot n} $ et en général par
$$(\ref{Poly}.6)\;\; (\sum_{n\in N}a_{n}e_{n}) *(\sum_{n\in N}b_{n}e_{n})=\sum_{n\in N}(\sum_{i,j\in N,i\cdot j=n}a_{i}b_{j})e_{n}.$$
on obtient un anneau (commutatif si $(N,\cdot)$ est commutatif) $(A^{(N)},+,*)$  ayant pour unité $e_{\underline{1_N}}$. L'application canonique $\iota_A:A\rightarrow A^{(N)}$, $a\rightarrow ae_{1_N}  $  est un morphisme d'anneaux et l'application $\nu_A:N\rightarrow A^{(N)}$, $n\rightarrow e_n$ prend ses valeur dans $A^{(N)}\setminus\lbrace 0\rbrace$ et induit un morphisme de monoïdes $\nu_A:(N,\cdot)\rightarrow (A^{(N)}\setminus\lbrace 0\rbrace,*)$.
 On note traditionnellement cet anneau $(A[N],+,\cdot) $ et on dit que $\iota_A:A\rightarrow A[N]$ est la $A$-algèbre du monoïde $(N,\cdot)$. On pose aussi $n:=e_{n}$, $n\in N$    et $1:=1_N$ de sorte que (\ref{Poly}.5) se réécrit de fa\c{c}on plus intuitive sous la forme
 $$(\ref{Poly}.7)\;\; (\sum_{n\in N}a_{n}n) *(\sum_{n\in N}b_{n}n)=\sum_{n\in N}(\sum_{i,j\in N,i\cdot j=n}a_{i}b_{j})n.$$

 \ref{Poly}.8 \textbf{Lemme.} (Propriété universelle de la $A$-algèbre du monoïde $(N,\cdot)$) \textit{Pour tout anneau commutatif $A$, il existe une $A$-algèbre $\iota_A: A\rightarrow P$ et un  morphisme de monoïdes $\nu_A:(N,\cdot)\rightarrow (P  ,\cdot)$ tels que pour toute $A$-algèbre   $\phi: A\rightarrow B$ et tout morphisme de monoïdes $\nu:(N,\cdot)\rightarrow (B  ,\cdot)$  il existe un unique  morphisme de $A$-algèbres $\tilde{\nu}:P\rightarrow B$  tel que $ \tilde{\nu}\circ \nu_A=\nu$. }

	$$ \xymatrix{ & B \\ A \ar[ru]^\Phi \ar[rd]_\iota & & N \ar[lu]_\nu \ar[ld]^{\nu_A} \\ & P \ar[uu]_{\tilde{\nu} } } $$

\begin{proof} Similaire à celle de \ref{Poly}.3 en vérifiant que $\iota_A:A\rightarrow A[N]$ convient. \end{proof}

  Le même argument  formel que celui utilisé dans \ref{Produit}.2 montre que la $A$-algèbre $\iota_A: A\rightarrow A[N]$ est unique à unique isomorphisme  près.\\
 
  On peut aussi réécrire \ref{Poly}.8 en disant que, pour toute $A$-algèbre $\phi:A\rightarrow B$  l'application canonique
$$\hbox{\rm Hom}_A(A[N],B)\rightarrow\hbox{\rm Hom}_{Mono}(N,B),\; f\rightarrow f\circ\nu_A$$
 est bijective. Son inverse est l'application qui à $\nu:(N,\cdot)\rightarrow (B,\cdot) $ associe l'unique morphisme de $A$-algèbres $ \tilde{\nu}:A[N]\rightarrow B$ tel que $ \tilde{\nu}(n)=\nu(n)$ (donc $\tilde{\nu}(\sum_{n\in N}a_nn)=\sum_{n\in N}\phi(a_n)\nu(n)$).\\
 
  \textbf{Exemples.} Si on prend\\
  
  \begin{enumerate}[leftmargin=* ,parsep=0cm,itemsep=0cm,topsep=0cm]
 \item  $(N,\cdot)=(\N,+)$ on retrouve $A[\N]=A[X]$.\\
 % (observer que se donner un morphisme de monoïdes $\nu:(\N,+)\rightarrow (B ,\cdot)$ revient à se donner l'image $b\in B$ de $1\in \N$).\\
 \item $(N,\cdot)=(\N^r,+)$ où $+$ est l'addition termes à termes (pour $\underline{m}=(m_1,\dots,m_r), \underline{n}:=(n_1,\dots, n_r)\in \N^r$,   $\underline{m}+\underline{n}=(m_1+n_1,\dots,m_r+n_r)\in \N^r$). Dans ce cas, on note $\underline{X}^{\underline{n}}:=X_1^{n_1}\cdots X_r^{n_r}:=e_{\underline{n}}$, $\underline{n}\in \N^r$ avec la convention $X_i^0=1$, $i=1,\dots, r$,  et $1:=\underline{X}^{\underline{0}}$ de sorte que (\ref{Poly}.5) se réécrit de fa\c{c}on plus intuitive sous la forme
 $$ (\sum_{\underline{n}\in\N^r}a_{\underline{n}}\underline{X}^{\underline{n}}) (\sum_{\underline{n}\in \N^r}b_{\underline{n}}\underline{X}^{\underline{n}})=\sum_{\underline{n}\in \N}(\sum_{\underline{i},\underline{j}\in \N^r, \underline{i}+\underline{j}=\underline{n}}a_{\underline{i}}b_{\underline{j}})\underline{X}^{\underline{n}}.$$
On note également $A[X_1,\dots, X_r]:=A[\N^r]$ et on dit que $\iota_A:A\rightarrow A[X_1,\dots, X_r]$ est la $A$-algèbre des polynômes à $r$ inderminées. Comme se donner un morphisme de monoïdes $\nu:(\N^r,+)\rightarrow (B ,\cdot)$ revient à se donner les images $b_i\in B $ de $(\delta_{i,j})_{1\leq j\leq r}\in \N^r$, on peut reformuler \ref{Poly}.7 de la fa\c{c}on suivante.\\

   Pour toute $A$-algèbre $\phi:A\rightarrow B$, en notant  
 $$\frak{B}_r:= \lbrace \underline{b}=(b_1,\dots, b_r)\in B^r\;|\; b_ib_j=b_jb_i,\; 1\leq i,j\leq r\rbrace,$$
$$\hbox{\rm Hom}_A(A[X_1,\dots,X_r],B)\rightarrow \frak{B}_r,\; f\rightarrow (f(X_1),\dots, f(X_r))$$
 est bijective. Son inverse est l'application qui à $\underline{b}=(b_1,\dots, b_r)\in \frak{B}_r$ associe l'unique morphisme de $A$-algèbres $ev_{\underline{b}}^\phi:A[X_1,\dots, X_r]\rightarrow B$ tel que $ev_{\underline{b}}^\phi(X_i)=b_i$, $i=1,\dots r$ (donc $ev_{\underline{b}}^\phi(\sum_{\underline{n}\in \N^r}a_{\underline{n}}\underline{X}^{\underline{n}})=\sum_{\underline{n}\in \N^r}\phi(a_{\underline{n}})\underline{b}^{\underline{n}}$). On adopte  en général la notation plus intuitive $ev_{\underline{b}}^\phi(P)=P(b_1,\dots, b_r)$ et on dit que $ev_{\underline{b}}^\phi$ est le morphisme d'évaluation en $\underline{b}$.\\
 
 
 \item Pour $(N,\cdot)$ un groupe, pour toute $A$-algèbre $\phi:A\rightarrow B$, tout morphisme de monoïdes $\nu:(N,\cdot)\rightarrow (B,\cdot)$ est automatiquement à valeur dans le groupe $(B^\times,\cdot)$. On dit dans ce cas que $ A[N]$ est la $A$-algèbre du groupe $(N,\cdot)$.\\
 
  Par exemple, pour $(N,\cdot)=(\Z,+)$, on obtient la $A$-algèbre (notations:  $A[X,X^{-1}]:=A[\Z]$, $X^n:=e_n$, $n\in \Z$ donc en particulier $X^nX^{-n}=e_ne_{-n}=e_{n-n}=e_0=1$) des polynômes de Laurent à une indéterminée. Comme se donner    un morphisme de monoïdes $\nu:(\Z,+)\rightarrow (B ,\cdot)$ revient à se donner l'image  $b \in B^\times $ de $1\in \Z$, on peut reformuler \ref{Poly}.7 de la fa\c{c}on suivante.\\

   Pour toute $A$-algèbre $\phi:A\rightarrow B$,  l'application canonique
$$\hbox{\rm Hom}_A(A[X,X^{-1}],B)\rightarrow B^\times,\; f\rightarrow f(X)$$
 est bijective. Son inverse est l'application qui à $b \in B^\times$ associe l'unique morphisme de $A$-algèbres $ev_{\underline{b}}^\phi:A[X,X^{-1}]\rightarrow B$ tel que $ev_{\underline{b}}^\phi(X)=b$ (donc $ev_{\underline{b}}^\phi(\sum_{n\in \Z}a_n\underline{X}^{\underline{n}})=\sum_{n\in \Z}\phi(a_n)b^n$).\\ 
 
  De même, pour $(N,\cdot)=(\Z^r,+)$, on obtient la $A$-algèbre (notations:  $A[X_1,X_1^{-1},\cdots, X_r,X_r^{-1}]:=A[\Z^r]$, $\underline{X}^{\underline{n}}:=X_1^{n_1}\cdots X_r^{n_r}:=e_{\underline{n}}$, $\underline{n}\in \Z^r$ donc en particulier, $\underline{X}^{\underline{n}}\underline{X}^{-\underline{n}}= e_{\underline{n}}e_{-\underline{n}}=e_{\underline{n}-\underline{n}}=e_0=1$) des polynômes de Laurent à $r$ indéterminées. Comme se donner    un morphisme de monoïdes $\nu:(\Z^r,+)\rightarrow (B ,\cdot)$ revient à se donner les images $b_i \in B^\times $ des $(\delta_{i,j})_{1\leq j\leq r}\in \Z$, $i=1,\dots, r$ on peut reformuler \ref{Poly}.8 de la fa\c{c}on suivante.\\
 
   Pour toute $A$-algèbre $\phi:A\rightarrow B$, en notant  
 $$\frak{B}^\times_r:= \lbrace \underline{b}=(b_1,\dots, b_r)\in (B^\times)^r\;|\; b_ib_j=b_jb_i,\; 1\leq i,j\leq r\rbrace,$$  l'application canonique
$$\hbox{\rm Hom}_A(A[X_1,X_1^{-1},\cdots, X_r,X_r^{-1}],B)\rightarrow \frak{B}_r,\; f\rightarrow (f(X_1),\dots, f(X_r))$$
 est bijective. Son inverse est l'application qui à $\underline{b}=(b_1,\dots, b_r)\in \frak{B}^\times_r$ associe l'unique morphisme de $A$-algèbres $ev_{\underline{b}}^\phi:A[X_1,X_1^{-1},\cdots, X_r,X_r^{-1}]\rightarrow B$ tel que $ev_{\underline{b}}^\phi(X_i)=b_i$, $i=1,\dots r$ (donc $ev_{\underline{b}}^\phi(\sum_{\underline{n}\in \Z^r}a_{\underline{n}}\underline{X}^{\underline{n}})=\sum_{\underline{n}\in \Z^r}\phi(a_{\underline{n}})\underline{b}^{\underline{n}}$).\\

 

 \end{enumerate}

 
 
 
 

 
  (\ref{Poly}.9)  Soit $(N,\cdot)$ un monoïde et $\phi: A\rightarrow B$ un morphisme d'anneaux commutatifs. La propriété universelle de $\iota_A:A\rightarrow A[N]$ appliquée avec $A\stackrel{\phi}{\rightarrow} B\stackrel{\iota_B}{\rightarrow} B[N]$ donne un unique morphisme de $A$-algèbres $\tilde{\phi}:A[N]\rightarrow B[N]$ tel que $\nu_B=\tilde{\phi}\circ \nu_A$. Explicitement $\tilde{\phi}(\sum_{n\geq 0}a_ne_n)=\sum_{n\geq 0}\phi(a_n)e_n$.  Par construction, im$(\phi)=\hbox{\rm im}(\phi)[N]\subset B[N]$ et $\ker(\tilde{\phi})$ est l'ensemble des éléments de la forme $\sum_{n\geq 0}a_ne_n\in A[N]$ tels que $a_n\in \ker(\phi)$, $n\geq 0$. On notera $\ker(\phi)[N]:=\ker(\tilde{\phi})\subset A[N]$.\\

	$$ \xymatrix{ A\ar[r]^\Phi \ar[d]^{\iota_A} & B \ar[d]_{\iota_B} \\ A[N]\ar[r]_{\tilde{\Phi} } & B[N] } $$
 
 \ref{Poly}.10 \textbf{Exercice.} \\
 
 \begin{enumerate}[leftmargin=* ,parsep=0cm,itemsep=0cm,topsep=0cm]
 \item Montrer qu'on a un morphisme surjectif $A$-algèbres canonique $$A[X_1,Y_1,\dots, X_r,Y_r]\twoheadrightarrow A[X_1,X_1^{-1},\dots,X_r, X_r^{-1}].$$
 
   Correction. \textit{Plus généralement, on peut montrer qu'on a une application canonique injective $$\begin{tabular}[t]{cccc}
$\tilde{-}:$&$ \SHom_{Mono}(N_1,N_2)$&$\hookrightarrow$&$ \SHom_{A}(A[N_1], A[N_2])$\\
&$\nu:N_1\rightarrow N_2$&$\rightarrow$&$\tilde{\nu}:A[N_1]\rightarrow A[N_2]$
\end{tabular}$$ 
qui envoie morphismes de monoïdes injectifs (resp. surjectifs, resp. bijectifs) sur morphismes de $A$-algèbres injectifs (resp. surjectifs, resp. bijectifs). L'existence de $\tilde{-}: \SHom_{Mono}(N_1,N_2) \rightarrow  \SHom_{A}(A[N_1], A[N_2])$ est une conséquence formelle de la propriété universelle de la $A$-algèbre de monoïdes $\iota_A:A\rightarrow A[N_1]$ appliquée  avec la $A$-algèbre   $\iota_A:A\rightarrow A[N_2]$ et le morphisme de monoïdes $ N_1\stackrel{\nu}{\rightarrow} N_2\stackrel{\nu_A}{\rightarrow} A[N_2]$: il existe un unique morphisme de $A$-algèbre $\tilde{\nu}:A[N_1]\rightarrow A[N_2]$ tel que le diagramme suivant commute
$$\xymatrix{N_1\ar[r]^{\nu_A}\ar[d]_\nu&A[N_1]\ar[d]^{\tilde{\nu}}\\
N_2\ar[r]_{\nu_A}  &A[N_2]}$$
L'injectivité de $\tilde{-}: \SHom_{Mono}(N_1,N_2) \rightarrow  \SHom_{A}(A[N_1], A[N_2])$ résulte de l'injectivité des $\nu_A:N_i\rightarrow A[N_i]$, $i=1,2$. Enfin, le fait que $\tilde{-}: \SHom_{Mono}(N_1,N_2) \rightarrow  \SHom_{A}(A[N_1], A[N_2])$ envoie morphismes de monoïdes injectifs (resp. surjectifs, resp. bijectifs) sur morphismes de $A$-algèbres injectifs (resp. surjectifs, resp. bijectifs) résulte du fait que, par construction, tout élément de $A[N]$ s'écrit de fa\c{c}on unique sous la forme $\sum_{n\in N}ae_n$ (on verra dans le chapitre sur les modules que $A[N]$ est un $A$-module libre de base les $e_n$, $n\in N$) et que la condition $\nu_A\circ \nu=\tilde{\nu}\circ \nu_A$ impose $\tilde{\nu}(e_n)=e_{\nu(n)}$.\\
La question posée correspond au cas particulier du morphisme de monoïdes surjectif $\nu:(\N^2,+)\twoheadrightarrow (\Z,+)$ défini par $\nu(n_1,n_2)=n_1-n_2$ (le $\tilde{\nu}:A[X_1,Y_1,\dots, X_r,Y_r]\twoheadrightarrow A[X_1,X_1^{-1},\dots,X_r, X_r^{-1}]$ correspondant étant défini par $\tilde{\nu}(X_i)=Z_i$, $\tilde{\nu}(Y_i)=Z_i^{-1}$, $i=1,2$). }\\
 
\item Montrer qu'on a des isomorphismes de $A$-algèbres canonique $$  A[X_1,\dots,X_{i-1},X_{i+1},\dots, X_r][X_i]\tilde{\rightarrow} A[X_1,\dots,X_r],\; i=1,\dots, r.$$

   Correction. \textit{Observons d'abord que toute permutation $\sigma\in \mathcal{S}_r$ induit un automorphisme du monoïde $(\N^r,+)$ par permutation des coordonnées donc, d'après (1), un automorphisme de la $A$-algèbre $A[X_1,\dots, X_r]$. (Explicitement, $\sigma P(X_1,\dots, X_r)=P(X_{\sigma(1)},\dots, X_{\sigma(r)})$). Il suffit donc de montrer le résultat pour $i=r$. Par unicité  à unique isomorphisme près des objets universel, il suffit de montrer que 
$\iota_A:A\rightarrow A[X_1,\dots, X_r]$ et $A\stackrel{\iota_A}{\rightarrow} A[X_1,\dots, X_{r-1}]\stackrel{\iota_{A[X_1,\dots, X_{r-1}]}}{\rightarrow} A[X_1,\dots, X_{r-1}][X_r]$ vérifie la même propriété universelle. Notons que par hypothèse  $A[b_1,\dots ,b_r]$ est un anneau commutatif (\textit{cf.} \ref{SousAlg} pour la notation $A[b_1,\dots, b_{r-1}]$). Soit donc $\phi:A\rightarrow B$ une $A$-algèbre et $b_1,\dots, b_r\in B^r$ commutant deux à deux. Par la propriété universelle de $\iota_A:A\rightarrow A[X_1,\dots, X_{r-1}]$, il existe un unique morphisme de $A$-algèbre $ev_{(b_1,\dots, b_{r-1})}^\phi:A[X_1,\dots,X_{r-1}]\rightarrow B$ tel que $\phi_1(X_i):=ev_{(b_1,\dots, b_{r-1})}^\phi(X_i)=b_i$, $i=1,\dots, r-1$. Puis, par la propriété universelle de $\iota_{A[X_1,\dots, X_{r-1}]}:A[X_1,\dots, X_{r-1}]\rightarrow A[X_1,\dots, X_r]$, il existe un unique morphisme de $A$-algèbre $ev_{b_r}^{\phi_1}:A[X_1,\dots,X_{r-1}][X_r]\rightarrow A[b_1,\dots ,b_{r-1}][b_r]=A[b_1,\dots,b_r]$ tel que $ev_{b_r}^{\phi_1}(X_r)=b_r$... On laisse le soin au lecteur de généraliser ce genre d'exercice formel un tantinet fastidieux.} \\
 
  \end{enumerate}
 
 

 
  \section{Sous-anneau engendré par une partie} Soit  $A_i\subset A$, $i\in I$ une famille de sous-anneaux. On vérifie immédiatement que $\cap_{i\in I}A_i\subset A$ est un sous-anneau. Pour tout sous-ensemble $X\subset A$, il existe 
un unique sous-anneau $\langle X\rangle \subset A$, contenant $X$ et minimal pour $\subset$ \textit{i.e.} tel que pour  tout sous-anneau $A'\subset A$,   $X\subset A'$ implique  $\langle X\rangle\subset A'$. On dit que $\langle X\rangle\subset A$ est le sous-anneau de $A$ engendré par $X$.  Explicitement $\langle X\rangle$ est l'intersection de tous les sous-anneaux de $A$ contenant $X$. On peut également décrire $\langle X\rangle$ comme  l'ensemble des sommes finies de produits finis d'éléments de $X$. Si $A=\langle X\rangle$, on dit que $X$ est un système de générateurs de $A$ comme anneau (ou que $A$ est engendré par $X$ comme anneau). Si on peut prendre de plus $X$ fini, on dit que $A$ est un anneau de type fini.\\

 Lorsque les éléments de $X$ commutent deux à deux, on note en général $\Z[X]:=\langle X\rangle \subset A$ le sous-anneau de $A$ engendré par $X$. Si  $X=\lbrace x_1,\dots,x_r\rbrace $ est fini, on note plutôt $\Z[x_1,\dots,x_r]:=\Z[X]$ et \ref{Poly}.8  nous donne un unique morphisme d'anneaux - automatiquement  surjectif - $ev_{\underline{x}}:\Z[X_1,\cdots, X_r]\twoheadrightarrow \Z[x_1,\dots,x_r] $ tel que $ev_{\underline{x}}(X_i)=x_i$, $i=1,\dots, r$.   \\
 
  \section{Sous-$A$-algèbre engendrée par une partie}\label{SousAlg} Soit $\phi:A\rightarrow B$ une $A$-algèbre. Une sous-$A$-algèbre de $\phi:A\rightarrow B$ est un sous-anneau $B'\subset B$ tel que $\hbox{\rm im}(\phi)\subset B'$ (noter que $Z(B)\cap B'\subset Z(B')$); le morphisme $\phi|^{B'}:A\rightarrow B'$ munit alors $B'$ d'une structure de $A$-algèbre qui fait de l'inclusion $B'\subset B$ un morphisme de $A$-algèbres. Si    $B_i\subset B$, $i\in I$ est une famille de sous-$A$-algèbres, $\cap_{i\in I}B_i\subset B$  est encore une sous-$A$-algèbre. Pour tout sous-ensemble $X\subset B$, il existe 
une unique sous-$A$-algèbre $\langle X\rangle_A \subset B$, contenant $X$ et minimale pour $\subset$. On dit que $\langle X\rangle_A\subset B$ est la sous-$A$-algèbre de $B$ engendrée par $X$. Explicitement $\langle X\rangle_A$ est l'intersection de tous les sous-$A$-algèbres de $B$ contenant $X$. On peut également décrire $\langle X\rangle_A$ comme  le sous-anneau de $B$ engendré par $X\cup\hbox{\rm im}(\phi)$. Si $B=\langle X\rangle_A$, on dit que $X$ est un système de générateurs de $B$ comme $A$-algèbre (ou que $B$ est engendré par $X$ comme $A$-algèbre). Si on peut prendre $X$ fini, on dit que $B$ est une $A$-algèbre de type fini\index{de type fini (Algèbre)}.\\

 Lorsque les éléments de $X$ commutent deux à deux, on note en général $A[X]:=\langle X\rangle_A \subset B$ la sous-$A$-algèbre de $B$ engendré par $X$. Si  $X=\lbrace x_1,\dots,x_r\rbrace $ est fini, , on note plutôt $A[x_1,\dots,x_r]:=A[X]$ et  \ref{Poly}.8  nous donne un unique morphisme de $A$-algèbres - automatiquement  surjectif - $ev_{\underline{x}}^\phi:A[X_1,\cdots, X_r]\twoheadrightarrow A[x_1,\dots, x_r] $ tel que $ev^\phi_{\underline{x}}(X_i)=x_i$, $i=1,\dots, r$. \\

 
 
 \begin{center} **  Dans la suite,  sauf mention explicite du contraire, nous ne considérerons que des anneaux commutatifs **\\\end{center}
