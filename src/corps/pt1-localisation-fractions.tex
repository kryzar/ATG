\chapter{Localisation, anneaux de fractions.}\textit{}\\
 On va maintenant généraliser la construction du corps des fractions d'un anneau intègre à des anneaux non nécessairement intègre. Soit $A$ un anneau commutatif non réduit à $\{0\}$.

\section{Localisations}

\subsection{Parties multiplicatives}

\begin{definition}
	Une \textit{partie multiplicative} de $A$ est un sous-ensemble $S\subset A\setminus\lbrace 0\rbrace$ stable par multiplication et contenant $1$.
\end{definition}

% \label{LocEx1}\textbf{Exemples.}\\
% \ref{LocEx1}.1 $S:=A\setminus A_{\mathrm{tors}}$; en particulier, si $A$ est intègre, $S:=A\setminus\lbrace 0\rbrace$;\\

% \ref{LocEx1}.2 Pour $a\in A\setminus \sqrt{\lbrace 0\rbrace}$, $S_a:=\lbrace a^n\;|\; n\in\N\rbrace$;\\

% \ref{LocEx1}.3 Pour $\frak{p}\in \Spec(A)$, $S_{\frak{p}}:=A\setminus \frak{p}$.

\begin{exemples}
	\begin{itemize}
		\item $S:=A\setminus A_{\mathrm{tors}}$; en particulier, si $A$ est intègre, $S:=A\setminus\lbrace 0\rbrace$;
		\item Pour $a\in A\setminus \sqrt{\lbrace 0\rbrace}$, $S_a:=\lbrace a^n\;|\; n\in\N\rbrace$;
		\item Pour $\frak{p}\in \Spec(A)$, $S_{\frak{p}}:=A\setminus \frak{p}$.
	\end{itemize}
\end{exemples}

\subsection{Définition}
	\label{LocDef} Soit $S\subset A\setminus\lbrace 0\rbrace$ une partie multiplicative. On munit le produit ensembliste $S\times A$ de la relation $\sim $ définie par: pour tout $(s,a),(s',a')\in S\times A$, $(s,a)\sim (s',a')$ s'il existe $s''\in S$ tel que $s''(s'a-sa')=0$. \\

 On vérifie que $\sim$ est une relation d'équivalence. On remarquera que si $A$ est intègre, on peut, dans la définition de $\sim$, simplifier par $s''$ et la relation $\sim$ devient simplement $(s,a),(s',a')\in S\times A$, $(s,a)\sim (s',a')$ si $s'a-sa'=0$.  Mais on prendra garde que si $A$ n'est pas intègre,  la relation $(s,a)\sim (s',a')$ si $s'a-sa'=0$ n'est pas transitive donc ne définit pas une relation d'équivalence.\\


 On note $S^{-1}A:=S\times A/\sim$ et
$$ \begin{tabular}[t]{llll}
$-/-$ :&$S\times A$&$\rightarrow$&$S^{-1}A$\\
 &$(s,a)$&$\rightarrow$&$a/s$
 \end{tabular}$$
 la projection canonique.\\



  Considérons les applications $$\begin{tabular}[t]{lclc}
 $+$&$:(S\times A)\times (S\times A)$&$\rightarrow$&$S^{-1}A$\\
 &$((s,a),(t,b))$&$\rightarrow$&$(ta+sb)/(st)$
 \end{tabular} $$ et $$\begin{tabular}[t]{lclc}
 $\cdot $&$:(S\times A)\times (S\times A)$&$\rightarrow$&$S^{-1}A$\\
 &$((s,a),(t,b))$&$\rightarrow$&$(ab)/(st)$
 \end{tabular}$$
 Si $(s,a)\sim (s',a')$, $(t,b)\sim (t',b')$ \ie{} il existe $s'',t''\in S$ tels que $s''(s'a-sa')=0$, $t''(t'b-tb')=0$. Comme $s''t''\in S$ par multiplicativité, on a

 % align permet d'aligner les équations les unes après les autres
 % * permet de ne pas numéroter les équations de chaque ligne
 \begin{align*}
	s''t''(s't'(ta+sb)-st(t'a'+s'b')) &= s'' s'a'tt't''-t'bss's''-s''t''st(t'a'+s'b') \\
	&= s'' sa''tt't''-tb'ss's''-s''t''st(t'a'+s'b') \\
	&=0.
 \end{align*}
 et
 \begin{align*}
	s''t''(s't'ab-sta'b') &=s''s'at''t'b-s''sa't''tb' \\
	&=s''sa't''t'b-s''sa't''tb' \\
	&=s''sa't''(t'b-tb') \\
	&=0.
 \end{align*}

 Cela montre que les applications $+,\cdot :(S\times A)\times (S\times A) \rightarrow S^{-1}A$  se factorisent en
 $$\xymatrix{(S\times A)\times (S\times A) \ar[r]^{+,\cdot}\ar[d]_{-/-\times -/-}& S^{-1}A\\
 S^{-1}A\times S^{-1}A\ar[ur]_{+,\cdot}&} $$
 On laisse en exercice le soin de vérifier que $S^{-1}A$ muni des lois $+,\cdot :S^{-1}A\times S^{-1}A\rightarrow S^{-1}A$ ainsi définies vérifie bien les axiomes d'un anneau commutatif  de zéro $0/1$ et d'unité $1/1$ et que, pour cette structure d'anneau, l'application canonique $$\begin{tabular}[t]{lclc}
 $\iota_S:$&$A$&$\rightarrow$&$S^{-1}A$\\
 &$a$&$\rightarrow$&$a/1$
 \end{tabular}$$
 est un morphisme d'anneaux de noyau $\ker(\iota_S)=\lbrace a\in A\; |\; \exists s\in S, \; sa=0\rbrace$. En particulier, si $A$ est intègre (ou plus généralement si $S$ ne contient pas d'éléments de torsion), $ \iota_S:  A\rightarrow S^{-1}A$ est injectif. De plus, $\iota_S(S)\subset (S^{-1}A)^\times$ puisque $s/1\cdot 1/s=s/s=1/1$. \\

 \subsection{Propriété universelle}\label{LocUniv}
	\begin{lemme}[Propriété universelle de la localisation]
		Pour toute partie multiplicative $S\subset A\setminus\lbrace 0\rbrace$  il existe un anneau $F$ et morphisme d'anneaux $\iota_S:A\rightarrow F$ tel que $\iota_S(S)\subset F^\times$ et pour tout  morphisme d'anneaux $\phi:A\rightarrow B$ tel que $\phi(S)\subset B^\times $, il  existe un unique morphisme d'anneaux $ \phi_S:F\rightarrow B$ tel que $\phi=  \phi_S\circ \iota_S$.
	\end{lemme}


  Plus visuellement,
	$$\xymatrix{
		S\ar[r]^\phi\ar@{_{(}->}[d]&B^\times\ar@{_{(}->}[d] \\
		A\ar[r]^{\forall\phi}\ar@{_{(}->}[d]_{\iota_S}&B \\
		F\ar@{.>}[ur]_{\exists ! \phi_S}}
	$$

 \begin{proof} Montrons que  $S^{-1}A$ muni de la structure d'anneau ci-dessus et le morphisme canonique $\iota_S:A\rightarrow S^{-1}A$ conviennent. Soit $\phi:A\rightarrow B$  un morphisme d'anneaux tel que $\phi(S)\subset B^\times$.  Si $\phi_S:S^{-1}A\rightarrow B$ existe la relation $\phi=  \phi_S\circ \iota_S$ impose que  $\phi_S :S^{-1}A\rightarrow B$ est unique puisqu'on doit nécessairement avoir
 $$\phi_S(a/s)=\phi_S((a/1)(1/s))=\phi_S(a/1)\phi_S((s/1)^{-1})=\phi(a)\phi(s)^{-1},\; (s,a)\in S\times A.$$
 Considérons donc l'application  \begin{tabular}[t]{lclc}
	 $\tilde{\phi} : $&$ S\times A $&$\rightarrow$&$B$\\
 &$(a,s) $&$\rightarrow$&$ \phi(s)^{-1}\phi(a)$.
 \end{tabular}
 Si $(s,a)\sim (s',a')$  \ie{} il existe $s''\in S$ tels que $s''(s'a-sa')=0$, on a $\phi(s'')(\phi(s')\phi(a)-\phi(s)\phi(a'))=\phi(s''(s'a-sa'))=\phi(0)=0$. Mais comme $\phi(s),\phi(s'),\phi(s'')\in B^{\times}$, on peut réécrire cette égalité comme
 $$\phi_S(s,a)=\phi(s)^{-1}\phi(a)=\phi(s')^{-1}\phi(a')=\phi_S(s',a').$$
 Cela montre que l'application $\phi_S: S\times A  \rightarrow  B$ se factorise  en
 $$\xymatrix{S\times A \ar[r]^{\tilde{\phi}_S}\ar[d]_{  -/-}& B\\
 S^{-1}A \ar[ur]_{\phi_S}&} $$
Par construction $\phi=  \phi_S\circ \iota_S$ et on vérifie que $\phi_S:S^{-1}A\rightarrow B$ est bien un morphisme d'anneaux.
  \end{proof}

  Comme d'habitude, le morphisme d'anneaux  $\iota_S:A\rightarrow S^{-1}A$ est unique à unique isomorphisme près et on dit que  c'est `la' localisation\index{Localisation (Anneau)} de $A$ en $S$. Localiser $A$ en $S$ revient donc à inverser formellement les éléments de $S$.  \\

 \begin{exercices}
	 \begin{enumerate}
		\item  Montrer qu'on a un isomorphisme  d'anneaux canonique \\ $S^{-1}(A[X])\tilde{\rightarrow} (S^{-1}A)[X]$. \\
			Correction. \textit{On va utiliser les propriétés universelles de la localisation et de l'algèbre des polynômes à une indéterminée pour construire un morphisme dans les deux sens.  Considérons le morphisme d'anneaux $\phi:A\stackrel{\iota_S}{\rightarrow}S^{-1}A \stackrel{\iota_{S^{-1}A}}{\rightarrow}(S^{-1}A)[X]$. Par propriété universelle de $\iota_A:A\rightarrow A[X]$, il s'étend en un unique morphisme de $A$-algèbre $\phi:A[X]\rightarrow (S^{-1}A)[X]$ tel que $\phi(X)=X$. De plus, $\phi(S)=\iota_{S^{-1}A}(\iota_S(S))\subset \iota_{S^{-1}A}((S^{-1}A)^\times)=(S^{-1}A)[X]^\times$ donc par propriété universelle de $\iota_S:A[X]\rightarrow S^{-1}(A[X])$, $\phi:A[X]\rightarrow (S^{-1}A)[X]$ s'étend en un unique morphisme d'anneaux $\phi_S:S^{-1}(A[X])\rightarrow (S^{-1}A)[X]$ tel que $\phi_S\circ \iota_S=\phi$. Dans l'autre sens, considérons le morphisme d'anneaux $\psi:A\stackrel{\iota_A}{\rightarrow}A[X] \stackrel{\iota_{S}}{\rightarrow}S^{-1}A([X])$. On a $\psi(S)=\iota_S(\iota_A(S))\subset S^{-1}A([X])^\times$ donc par propriété universelle de $\iota_S:A\rightarrow S^{-1}A$ il existe un unique morphisme d'anneaux $\psi_S:S^{-1}A\rightarrow S^{-1}A([X])$ tel que $\psi_S\circ \iota_S=\psi$. Enfin, par la propriété universelle de $\iota_{S^{-1}A}:S^{-1}A\rightarrow S^{-1}A$, il existe un unique morphisme de $S^{-1}A$-algèbre $\psi_S:(S^{-1}A)[X]\rightarrow S^{-1}(A[X])$ tel que $\psi_S(X)=X$. On vérifie immédiatement sur les constructions que $\psi_S\circ \phi_S=Id$ et $\phi_S\circ \psi_S=Id$.} 
		 \item Soit $p,q$ deux premiers distincts. Déterminer les idéaux premiers $\frak{p}$ de $A:=\Z/pq$ et déterminer dans chaque cas le localisé $(A\setminus \frak{p})^{-1}A$. \\
			 Correction. \textit{Les idéaux de $A$ sont les images par la projection canonique $\Z\twoheadrightarrow A$ des idéaux de $\Z$ contenant l'idéal $pq\Z$. Or $\Z$ est principal donc ses idéaux sont tous de la forme $n\Z$ et la condition $pq\Z\subset n\Z$ est équivalente à $n|pq$. On n'a donc que quatre possibilités: $A$, $\lbrace 0\rbrace$, $\frak{p}:=p\Z/pq\Z$ et $\frak{q}:=q\Z/pq\Z$. On a $\Z/p\Z\tilde{\rightarrow}A/\frak{p}$ et  $\Z/q\Z\tilde{\rightarrow}A/\frak{q}$ donc $spec(A)=\lbrace \frak{p},\frak{q}\rbrace$. J'affirme qu'on a un isomorphisme  (nécessairement unique)   $\phi_\frak{p}:A/\frak{p}\tilde{\rightarrow} A_\frak{p}$ tel que le diagramme canonique suivant commute $$\xymatrix{&A\ar[dr]^{\iota_{A\setminus \frak{p}}}\ar[dl]_{p_{\frak{p}}}&\\
 A/\frak{p}\ar[rr]^\simeq_{ \phi_\frak{p}}&& A_\frak{p}.}$$
 (et idem pour $\frak{q}$). On va utiliser les propriétés universelles de $p_\frak{p}:A\rightarrow A/\frak{p}$ et $\iota_{A\setminus \frak{p}}:A\rightarrow A_\frak{p}$. Par le lemme chinois $A\tilde{\rightarrow}\Z/p\times \Z/q$  et $\frak{p}$ s'identifie à  l'idéal engendré par $e_2:=(0,1)$; notons aussi $e_1:=(1,0)$. Soit $\iota_\frak{p}:A\rightarrow A_\frak{p}$ le morphisme de localisation. On a pour tout $a\in \frak{p}$, $e_1a=0$. Or, comme $e_1\in A\setminus \frak{p}$, on a $\iota_\frak{p}(e_1)\in A_\frak{p}^\times$ donc $0=\iota_\frak{p}(e_1a)=\iota_\frak{p}(e_1)\iota_\frak{p}(a)$ implique en simplifiant par $\iota_\frak{p}(e_1)\in A_\frak{p}^\times$, $\iota_\frak{p}(a)=0$. Autrement dit, on vient de montrer que $\frak{p}\subset \ker(\iota_\frak{p})$; le morphisme canonique $ \iota_\frak{p}:A\rightarrow A_\frak{p}$ se factorise donc en un unique morphisme $\phi_\frak{p}:A/\frak{p}\rightarrow A_\frak{p}$ tel que $\phi_\frak{p}\circ p_\frak{p}=\iota_\frak{p}$.  Considérons maintenant la projection canonique $p_\frak{p}:A\rightarrow A/\frak{p}$. Puisque $\frak{p}$ est maximal, pour tout   $a\in A\setminus \frak{p}$ il existe $b\in A$ tel que $ab-1\in \frak{p}$ donc $1=p_\frak{p}(ab)=p_\frak{p}(a)p_\frak{p}(b)$. Ce qui montre que $p_\frak{p}(A\setminus \frak{p})\subset (A/\frak{p})^\times$. Donc il existe un unique morphisme d'anneaux $\psi_\frak{p}:A_\frak{p}\rightarrow A/\frak{p}$ tel que $\psi_\frak{p}\circ \iota_{A\setminus \frak{p}}=p_\frak{p}$. Mais on a alors par construction $\psi_\frak{p}\circ \phi_\frak{p}\circ p_\frak{p}=p_\frak{p}$, ce qui par unicité dans la propriété universelle de $p_\frak{p}:A\rightarrow A/\frak{p}$ (appliquée à $p_\frak{p}:A\rightarrow A/\frak{p}$!) impose $\psi_\frak{p}\circ \phi_\frak{p}=Id$. De m\^eme avec la localisation, $\phi_\frak{p}\circ \psi_\frak{p}\circ \iota_{A\setminus \frak{p}}=\iota_{A\setminus \frak{p}}$ impose $\phi_\frak{p}\circ \psi_\frak{p}=Id$.}
		 \item Montrer que si $A$ est intègre (resp. réduit, resp. intégralement clos, resp. factoriel) alors $S^{-1}A$ l'est aussi.\\
	 \end{enumerate}
 \end{exercices}


% \subsection{}\label{LocEx2}\textbf{Exemples.}\\

% \ref{LocEx2}.1 On dit que $(A\setminus A_{tors})^{-1}A$ est l'anneau des fractions de $A$. Si $A$ est un anneau intègre, on retrouve le corps  des fractions de $A$. Si $A$ n'est pas intègre, $(A\setminus A_{tors})^{-1}A$ n'est pas un corps (le vérifier sur un exemple).\\

% \ref{LocEx2}.2 Pour $a\in A\setminus \sqrt{\lbrace 0\rbrace}$ on note $A_a:=S_a^{-1}A $;\\

% \ref{LocEx2}.3 Pour $\frak{p}\in spec(A)$, on note $A_\frak{p}:=S_{\frak{p}}^{-1}A$. Noter que si $A$ est intègre  $\lbrace 0\rbrace \in spec(A)$ et, dans ce cas, $A_{\lbrace 0\rbrace}=Frac(A)$.\\

\begin{exemples}
	\begin{enumerate}
		\item On dit que $(A\setminus A_{tors})^{-1}A$ est l'anneau des fractions de $A$. Si $A$ est un anneau intègre, on retrouve le corps  des fractions de $A$. Si $A$ n'est pas intègre, $(A\setminus A_{tors})^{-1}A$ n'est pas un corps (le vérifier sur un exemple).
		\item Pour $a\in A\setminus \sqrt{\lbrace 0\rbrace}$ on note $A_a:=S_a^{-1}A $;
		\item Pour $\frak{p}\in \Spec(A)$, on note $A_\frak{p}:=S_{\frak{p}}^{-1}A$. Noter que si $A$ est intègre  $\lbrace 0\rbrace \in \Spec(A)$ et, dans ce cas, $A_{\lbrace 0\rbrace}=\Frac(A)$.
	\end{enumerate}
\end{exemples}

\subsection{Morphismes}
Soit $\phi:A\rightarrow B$ un morphisme d'anneaux et $S\subset A$, $T\subset B$ des parties multiplicatives telles que $\phi(S)\subset T$. On a en particulier $\iota_T\circ \phi(S)\subset \iota_T(T)\subset (T^{-1}B)^\times$ donc par propriété universelle de $\iota_S:A\rightarrow S^{-1}A$ il existe un unique morphisme d'anneaux $\phi_{S,T}:S^{-1}A\rightarrow T^{-1}B$ tel que $\iota_T\circ \phi=\phi_{S,T}\circ \iota_S$; explicitement $\phi_{S,T}(a/s)=\phi(a)/\phi(s)$ dans $T^{-1}B$. Si $\phi:A\rightarrow B$, $\psi:B\rightarrow C$ sont des morphismes d'anneaux et $S\subset A$, $T\subset B$, $U\subset C$ des parties multiplicatives telles que $\phi(S)\subset T$, $\psi(T)\subset U$, on a $(\psi\circ \phi)_{S,U}=\psi_{T,U}\circ \phi_{S,T}$. \\

	$$\xymatrix{
		A \ar[r]^\phi \ar@{^{(}->}[d]_{\iota_S} & B \ar@{^{(}->}[d]^{\iota_T} \\
		S^{-1}A \ar[r]^{\tilde{\phi}} & T^{-1}B
	}$$

\begin{exemples}
\begin{enumerate}
\item \label{LocMorphismes}Soit $\phi:A\rightarrow B$ un morphisme d'anneaux et $\frak{q}\subset \Spec(B)$. On a alors $\frak{p}:=\phi^{-1}(\frak{q})\in \Spec(A)$ et $\phi(A\setminus \frak{p})\subset B\setminus \frak{q}$ donc $\phi:A\rightarrow B$ induit un morphisme d'anneaux canonique $\phi_\frak{p}:A_{\frak{p}}\rightarrow B_{\frak{q}}$.
 \item Si $A$ est intègre,  $\lbrace 0\rbrace\in \Spec(A)$ et pour toute partie multiplicative $S\subset A\setminus \lbrace 0\rbrace$, en appliquant ce qui précède à $\phi=\Id:A\rightarrow A$, $S=S$, $T=A\setminus \lbrace 0\rbrace$, on obtient un morphisme canonique $\phi:S^{-1}A\rightarrow A_{\lbrace 0\rbrace}=\\Frac(A)$ dont on vérifie immédiatement qu'il est injectif.
\end{enumerate}

	$$ \xymatrix{
		A \ar@{^{(}->>}[r]^{\mathrm{Id} } \ar@{^{(}->}[d]_{\iota_S} & A \ar@{^{(}->}[d]^{\iota_T} \\
		S^{-1}A \ar@{^{(}->}[r]^{\tilde{\mathrm{Id} } } & \mathrm{\Frac}(A) }
	$$
\end{exemples}
\section{Idéaux}\label{LocIdeal}

 Soit $S\subset A$ une partie multiplicative. Pour un sous-ensemble $X\subset A$, notons $$S^{-1}X:=\left\{  \frac{a}{s}\;|\; a\in X,\; s\in S\right\} \subset S^{-1}A. $$
On vérifie immédiatement que si $I\subset A$ est un idéal alors $S^{-1}I\subset S^{-1}A$ est aussi un idéal. On a donc une application bien définie et croissante pour $\subset$
$$S^{-1}:(\mathcal{I}_A,\subset)\rightarrow (\mathcal{I}_{S^{-1}A},\subset).$$
Dans l'autre direction on a l'application $$\iota_S^{-1}:(\mathcal{I}_{S^{-1}A},\subset)\rightarrow (\mathcal{I}_A,\subset)$$ induite  par le morphisme de localisation $\iota_S:A\rightarrow S^{-1}A$.
\begin{itemize}[leftmargin=* ,parsep=0cm,itemsep=0cm,topsep=0cm]
\item Pour $I\subset A$ un idéal, on a
	$$\iota_S^{-1}S^{-1}I=\left\{ a\in A\;|\; \frac{a}{1}\in S^{-1}I\right\} = \lbrace a\in A\;|\;  Sa\cap I\not=\varnothing\rbrace=\bigcup_{s\in S}(s\cdot)^{-1}I.$$
Où $s\cdot$ est l'application de multiplication par $s$, pour tout $s$ dans $S$. En particulier, $S^{-1}I=S^{-1}A$ (si et seulement si $\iota_S^{-1}S^{-1}I=A$) si et seulement si $S\cap I\not=\varnothing$.\\
\item Pour $I\subset S^{-1}A$ un idéal, on a
	$$S^{-1}\iota_S^{-1}I=\left\{ \frac{a}{s} \in S^{-1}I\;|\; a\in \iota_S^{-1}I\right\}\supset I$$
et comme  pour tout $a/s\in I$ on a $a/1= (s/1)^{-1}(a/s)\in I$ donc $a\in \iota_S^{-1}I$, on a en fait $S^{-1}\iota_S^{-1}I=I$.\\
 \end{itemize}

 On a donc montré:

\label{LocIdeaux}\begin{lemme} L'application $S^{-1}:(\mathcal{I}_A,\subset)\rightarrow (\mathcal{I}_{S^{-1}A},\subset) $ est surjective, croissante pour $\subset$ et se restreint en une surjection    $$S^{-1}:\lbrace I\in\mathcal{I}_A\;|\; I\cap S=\varnothing\rbrace \twoheadrightarrow  \mathcal{I}_{S^{-1}A}\setminus \lbrace S^{-1}A\rbrace . $$
L'application $\iota_S^{-1}:(\mathcal{I}_{S^{-1}A},\subset) \rightarrow  (\mathcal{I}_A,\subset)$ est injective,  croissante pour $\subset$  et induit une bijection
$$\iota_S^{-1}:\mathcal{I}_{S^{-1}A} \tilde{\rightarrow} \lbrace I\in\mathcal{I}_A\;|\; I=\bigcup_{s\in S}(s\cdot)^{-1}I\rbrace .$$\end{lemme}


\begin{lemme}Les applications $S^{-1}: \mathcal{I}_A \rightarrow  \mathcal{I}_{S^{-1}A}  $ et  $\iota_S^{-1}: \mathcal{I}_{S^{-1}A} \rightarrow \mathcal{I}_A $ se restreignent en des bijections inverses l'une de l'autres
% ajout de tirets après les flèches du diagramme et avant le nom des fonctions
% pour centrer les items
$$\xymatrix{\Spec(S^{-1}A) \ar@<1ex>[r]^-{\iota_S^{-1}}& \lbrace \frak{p}\in \Spec(A)\;|\; \frak{p}\cap S=\varnothing\rbrace \ar@<1ex>[l]^-{S^{-1}} }$$\end{lemme}
\begin{proof} Si $\frak{p}\in \Spec(A)$ est tel que $S\cap \frak{p}=\varnothing$ alors $\frak{p}=\bigcup_{s\in S}(s\cdot)^{-1}\frak{p}$ (si $s\in S$, $a\in \frak{p}$, $sa\in\frak{p}$ implique $a\in\frak{p}$) donc   $\iota_S^{-1}S^{-1}\frak{p}=\frak{p}$. Comme on a toujours $S^{-1}\iota_S^{-1}=\Id$, et $\iota_S^{-1}\Spec(S^{-1}A)\subset \Spec(A)$, il reste seulement à montrer que si $\frak{p}\in \Spec(A)$ est tel que $S\cap \frak{p}=\varnothing$ alors  $S^{-1}\frak{p}\in \Spec(S^{-1}A)$. Soit donc $\frak{p}\in \Spec(A)$ et $a/s,b/t\in S^{-1}A$ tels que $(ab)/(st)\in S^{-1}\frak{p}$ \ie{} il existe $p\in \frak{p}$ et $u,v\in S$ tels que $v(uab-stp)=0$ ou encore $vuab=vstp\in\frak{p}$. Mais comme $\frak{p}\in \Spec(A)$ et $vu\notin\frak{p}$, on a $ab\in\frak{p}$ donc $a\in\frak{p}$ ou $b\in\frak{p}$.   \end{proof}

	\begin{exemple}[Corps résiduel] Si $\frak{p}\in \Spec(A)$,  $A_{\frak{p}}$ est local d'unique idéal maximal $\frak{p}A_{\frak{p}}$. Le corps $\kappa(\frak{p}):=A_{\frak{p}}/\frak{p}A_{\frak{p}}$ est appelé le corps résiduel de $\Spec(A)$ en $\frak{p}$. Si on reprend les notations de l'Exemple \ref{LocMorphismes}, le morphisme $\phi:A_{\frak{p}}\rightarrow B_{\frak{q}}$ envoie $\frak{p}$ dans $\frak{q}$ donc induit par passage au quotient un morphisme de corps --- nécessairement injectif --- $\kappa(\frak{p})\hookrightarrow \kappa(\frak{q})$.\end{exemple}


	\begin{corollaire}Si $A$ est noethérien (resp. principal) alors $S^{-1}A$ l'est aussi.\end{corollaire}


		\begin{exercices}
	\begin{enumerate}
	\item Soit $\frak{p}\in \Spec(A)$. Montrer qu'on a un morphisme d'anneaux canonique injectif $A/\frak{p}\rightarrow \kappa(\frak{p})$. Montrer que si $\frak{p}$ est maximal, ce morphisme est un isomorphisme.
	\item Montrer que les localisés d'un anneau principal en ses idéaux premiers sont des anneaux de valuation discrète.
	\item Si $I,J\subset A$ sont des idéaux, montrer que  $S^{-1}(I\cap J)=S^{-1}I\cap S^{-1}J$ et $S^{-1}(I+J)=S^{-1}I+S^{-1}J$.
	\item  Si $I\subset J$ sont des idéaux et si on note $\overline{S}\subset A/I$ l'image de $S$ \textit{via} la projection canonique $A\twoheadrightarrow A/I$, montrer qu'on a un isomorphisme canonique $$S^{-1}I/S^{-1}J\tilde{\rightarrow}\overline{S}^{-1}(I/J).$$
	\end{enumerate}
		\end{exercices}
