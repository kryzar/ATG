% \documentclass[a4paper, 12pt]{amsart}
\documentclass[a4paper,oneside,12pt]{book}
\usepackage[utf8]{inputenc} % permet d'écrire é et non é, par exemple
\usepackage[frenchb]{babel}
\usepackage[a4paper, left=1.5cm,right=1.5cm,bottom=2cm,top=3.5cm]{geometry}
\usepackage{graphicx,epsfig}
\usepackage{color}
\usepackage{amscd}
\usepackage{amssymb}
\usepackage{enumitem}

\usepackage{amsmath}
\usepackage{amsthm}
\usepackage{amsfonts}
\usepackage{mathrsfs}

\usepackage[hidelinks]{hyperref}

\newtheorem{theoreme}{Théorème}[paragraph]
\newtheorem{lemme}[theoreme]{Lemme}
\newtheorem{proposition}[theoreme]{Proposition}
\newtheorem{propositiondef}[theoreme]{Proposition-Définition}
\newtheorem{definition}[theoreme]{Définition}
\newtheorem{remarque}[theoreme]{Remarque}
\newtheorem{enonce}[theoreme]{Enoncé}
\newtheorem{corollaire}[theoreme]{Corollaire}
\newtheorem{exemple}[theoreme]{Exemple}
\newtheorem{probleme}[theoreme]{Problème}
\newtheorem{conjecture}[theoreme]{Conjecture}
\newtheorem{exercice}[theoreme]{Exercice}
\setcounter{secnumdepth}{6}

\input xy

\xyoption{all}

%\usepackage{makeidx}

\makeindex

% remplacement des \newcommand par des \DeclaraMathOperator
% c'est plus adapté
% voir https://tex.stackexchange.com/questions/67506/newcommand-vs-declaremathoperator

\DeclareMathOperator{\Spec}{Spec}
\DeclareMathOperator{\SH}{H}
\DeclareMathOperator{\SExt}{Ext}
\DeclareMathOperator{\SEnd}{End}
\DeclareMathOperator{\SGal}{Gal}
\DeclareMathOperator{\SHom}{Hom}
\DeclareMathOperator{\SGL}{GL}
\DeclareMathOperator{\SLie}{Lie}
\DeclareMathOperator{\Sres}{res}
\DeclareMathOperator{\SCH}{CH}
\DeclareMathOperator{\im}{im}
\DeclareMathOperator{\Scl}{cl}
\DeclareMathOperator{\Frac}{Frac}
\DeclareMathOperator{\spm}{spm}
\DeclareMathOperator{\pgcd}{pgcd}
\DeclareMathOperator{\ppcm}{ppcm}

\newcommand{\Q}{\mathbb{Q}}
\newcommand{\C}{\mathbb{C}}
\newcommand{\Z}{\mathbb{Z}}
\newcommand{\N}{\mathbb{N}}
\newcommand{\R}{\mathbb{R}}
\newcommand{\HH}{\mathbb{H}}
\newcommand{\F}{\mathbb{F}}
\newcommand{\G}{\mathbb{G}}
\newcommand{\Sdim}{\hbox{\rm dim}}


\renewcommand{\labelitemi}{-}

\title{Algèbre et théorie de Galois}
\author{Anna Cadoret\\
\textit{}\\
 Cours de Master 1 à Sorbonne Université - version 2019 (en cours d'actualisation)} 
\makeindex

\setlength{\parindent}{0cm}
  
\begin{document}
\maketitle  
 \tableofcontents
 
     \begin{thebibliography}{99}
  \bibitem[AM69]{AM} M.F. \textsc{Atiyah} et I.G. \textsc{MacDonald},
{\it Introduction to Commutative Algebra}, Addison-Wesley,  1969.
\bibitem[D18]{Dat} J.F. \textsc{Dat}, {\it Algèbre et théorie de Galois}, polycopié de cours disponible sur: https://webusers.imj-prg.fr/~jean-francois.dat/enseignement/AlgebreM1/ATG1718.pdf
 \bibitem[L02]{Lang} S. \textsc{Lang},
{\it Algebra (3rd ed.)}, G.T.M. \textbf{211}, Springer, 2002.
 \bibitem[S68]{CL} J.P. \textsc{Serre},
{\it Corps locaux}, Hermann, 1968.
 
\end{thebibliography}

 \textbf{Remerciements}: Sarah Wajsbrot (promotion 2018-19).\\

\textcolor{red}{Ne pas hésiter à me signaler les coquilles et, le cas échéant, me demander de clarifier  certains arguments/définitions. Tout  commentaire permettant d'améliorer l'exposition est le bienvenu.}\\

 On utilisera les notations $X\twoheadrightarrow Y$, $X\hookrightarrow Y$, $X\tilde{\rightarrow} Y$ (ou $X\stackrel{\simeq}{\rightarrow} Y$) pour une application ensembliste $X\rightarrow Y$ respectivement  surjective, injective, bijective.\\

 
  On aura parfois recours à l'axiome du choix sous l'une des formulations équivalentes suivantes:
 \begin{itemize}
 \item Un produit cartésien d'ensembles finis non vides est non vide.
 \item (Lemme de Zorn) tout ensemble non vide ordonné inductif admet un élément maximal. (On rappelle qu'un ensemble ordonné est dit inductif si toute suite croissante admet un majorant).
 \end{itemize}
\part{Anneaux - généralités}
\section{Premières définitions et constructions}
\subsection{Définitions}
\subsubsection{}On rappelle qu'un monoïde\index{Monoïde}  est un couple $(M,\times)$ formé d'un ensemble $M$ et d'une application $\times:M\times M\rightarrow M$ qui vérifient les axiomes suivants:
\begin{enumerate}
\item Associativité: $(l\cdot m)\cdot n=l\cdot (m\cdot n)$, $l,m,n\in M$;
\item Élément neutre: il existe $e_M\in M$ tel que $m\cdot e_M=m=e_M\cdot m$, $m\in M$;
\end{enumerate}
 Et on dit qu'un monoïde    $(M,\times)$ est un groupe\index{Groupe} si, de plus
\begin{enumerate}
\setcounter{enumi}{2}
\item Inverse: pour tout $m\in M$ il existe $n\in M$ tel  que $m\cdot n=e_M=n\cdot m$.\\
\end{enumerate}
 Etant donnés deux monoïdes $M,N$, un morphisme de monoïdes est une application $\phi:M\rightarrow N$   qui vérifie:
\begin{enumerate}
\item $\phi(m\cdot n)=\phi(m)\cdot\phi(n)$, $m,n\in M$;
\item $\phi(e_M )=e_N$.
\end{enumerate}
On remarquera que l'application identité $Id:M\rightarrow M$ est un morphisme de monoïdes et que si $\phi:L\rightarrow M$ et $\psi:M\rightarrow N$ sont des morphismes de monoïdes alors $\psi\circ \phi:L\rightarrow N$ est un morphisme de monoïdes. On notera $Hom_{Mono}(M,N)$ l'ensemble des morphismes de monoïdes $\phi:M\rightarrow N$ et, si $M=N$, $End_{Mono}(M):=Hom_{Mono}(M,M)$.  Etant donnés deux groupes $M,N$, un morphisme de groupes $\phi:M\rightarrow N$ est un morphisme entre les monoïdes sous-jacents. Dans ce cas, on notera plutôt $Hom_{Grp}(M,N)$ et $End_{Grp}(M)$ que $Hom_{Mono}(M,N)$, $End_{Mono}(M,N)$. \\

 On dit qu'un monoïde $(M,\cdot)$ est abélien ou commutatif si $m\cdot n=n\cdot m$, $m,n\in M$.

\subsubsection{}Un anneau\index{Anneau} est  un triplet $(A,+,\cdot)$ formé d'un ensemble $A$ et de deux applications $+,\cdot :A\times A\rightarrow A$ - appelées respectivement l'addition et la multiplication -  vérifiant les axiomes suivants:
\begin{enumerate}
\item $(A,+)$ est un groupe abélien; on note $0_A$ (ou simplement $0$) son élément neutre (appelé zéro) et $-a$ l'inverse d'un élément $a\in A$;
\item $(A ,\cdot)$ est un monoïde; on note $1_A$ (ou simplement $1$) son élément neutre (appelé unité). 
\item La multiplication est distributive par rapport à l'addition \textit{i.e.} $a\cdot (b+c)=a\cdot b+a\cdot c$ et $(b+c)\cdot a=b\cdot a+c\cdot a$, $a,b,c\in A$.\\
\end{enumerate}
 Dans la suite, on écrira presque toujours $ab$ au lieu de $a\cdot b$, $0:=0_A$, $1:=1_A$.\\ 

 On omet presque toujours les données $+,\cdot$ des notations.\\


 Un anneau $A$ est dit commutatif si    $a b=ba$, $a,b\in A$. \\


\subsubsection{}Le monoïde $(A ,\cdot)$ n'est pas un groupe en général; on note $A^\times\subset A $ le sous-ensemble des éléments inversibles \textit{i.e.} l'ensemble des $ a\in A$ tel qu'il existe $b\in A$ tel que $ab=1=ba$; c'est un groupe d'élément neutre $1$. On note alors $a^{-1}\in A^\times$ l'inverse d'un élément de $a\in A^\times$.\\


 On dit qu'un anneau $A$ est un anneau à division ou un corps gauche si $1\not=0$ et $A\setminus \lbrace 0\rbrace=A^\times$. Si $A$ est de plus commutatif, on dit simplement que $A$ est un corps.\\


\textbf{Exemples.}
\begin{itemize}[leftmargin=* ,parsep=0cm,itemsep=0cm,topsep=0cm]
\item L'anneau nul $A=\lbrace 0\rbrace$ (on n'a pas exclu $1\not=0$ dans la définition d'anneaux).
\item L'anneau  $\Z$ des entiers. Dans ce cas  $\Z^\times=\lbrace \pm 1\rbrace$.
\item Les corps commutatifs, par exemple $\Q$, $\R$, $\C$.  
\item Si $M$ est un groupe abélien, l'ensemble $End_{Grp}(M)$ des endomorphismes du groupe  abélien $M$ muni de $(\phi+\psi)(m)=\phi(m)+\psi(m)$ et $(\psi\cdot\phi)(m)=\psi\circ \phi(m)$ est  un anneau (non commutatif en général) de zéro l'application nulle et d'unité l'application identité. Dans ce cas,  $End_{Grp}(M)^\times=Aut_{Grp}(M)$.  
\item Si $M$ est un espace vectoriel sur un corps commutatif $k$, l'ensemble $End_k(M)$ des endomorphismes du $k$-espace vectoriel  $M$ muni de $(\phi+\psi)(m)=\phi(m)+\psi(m)$ et $(\psi\cdot\phi)(m)=\psi\circ \phi(m)$ est un anneau (non commutatif si $M$ est de $k$-dimension $\geq 2$) de zéro l'application nulle et d'unité l'application identité. Dans ce cas, $End_k(M)^\times=GL_k(M)$.
\item On rencontre aussi beaucoup d'anneaux en analyse: les anneaux $\mathcal{C}(X,\R)$ ou $\mathcal{C}(X,\C)$ de fonctions continues à valeurs réelles  ou complexes sur un espace topologique $X$, les anneaux $L^p(X,\mu)$ de fonctions intégrables  sur un espace mesuré $(X,\mu)$, les anneaux de séries entières \textit{etc}.\\
\end{itemize}




\subsubsection{}Etant donnés deux anneaux $A,B$, un morphisme d'anneaux est une application $\phi:A\rightarrow B$ qui induit à la fois un morphisme de groupes $\phi:(A,+)\rightarrow (B,+)$ et de monoides unitaires $\phi:(A,\cdot)\rightarrow (B,\cdot)$ \textit{i.e} qui vérifie:
\begin{enumerate}
\item $\phi(a+b)=\phi(a)+\phi(b)$, $a,b\in A$;q
\item $\phi(a  b)=\phi(a) \phi(b)$, $a,b\in A$ et $\phi(1)=1$;
\end{enumerate}
On remarquera que l'application identité $Id:A\rightarrow A$ est un morphisme d'anneaux et que si $\phi:A\rightarrow B$ et $\psi:B\rightarrow C$ sont des morphismes d'anneaux alors $\psi\circ \phi:A\rightarrow C$ est un morphisme d'anneaux. On notera $Hom(A,B)$ l'ensemble des morphismes d'anneaux $\phi:A\rightarrow B$ et, si $A=B$, $End(A):=Hom(A,A)$. \\

 On dit qu'un morphisme d'anneaux $\phi:A\rightarrow B$ est injectif, (resp. surjectif, resp. un isomorphisme) si l'application d'ensembles sous-jacente est injective (resp. surjective, resp. bijective). On vérifie que si $\phi:A\rightarrow B$ est un isomorphisme d'anneaux l'application inverse $\phi^{-1}:B\rightarrow A$ est automatiquement un morphisme d'anneaux. Comme un morphisme d'anneaux $\phi:A\rightarrow B$ est en particulier un morphisme de groupes, $\phi:A\rightarrow B$ est injectif si et seulement si $\ker(\phi):=\phi^{-1}(0_B)=\lbrace 0_A\rbrace$. On notera aussi $\im(\phi):=\phi(A)$. \\

 Si $\phi:A\rightarrow B$ est un morphisme d'anneaux, on vérifie que $\phi(A^\times)\subset B^\times$ et que $\phi:A\rightarrow B$ induit par restriction un morphisme de groupes $\phi:A^\times\rightarrow B^\times$.

\subsubsection{}Si $A$ est un anneau, un sous-anneau\index{Sous-anneau} de $A$ est un sous-ensemble $A'\subset A$ tel que $1_A\in A'$ et $a'-b'\in A'$, $a'\cdot b'\in A'$, $a',b'\in A'$. \\

\textbf{Exemples.}
\begin{itemize}[leftmargin=* ,parsep=0cm,itemsep=0cm,topsep=0cm]
\item $\Z$ est un sous anneau de $\Q$, $\Q$ est un sous-anneau de $\R$, $\R$ est un sous-anneau de $\C$. 
\item Si $M$ est un espace vectoriel sur un corps commutatif $k$, $End_k(M)$ est un sous-anneau de $End_{Grp}(M)$. 
\item $Z(A):=\lbrace a\in A\;|\; a\cdot b=b\cdot a,\; b\in A\rbrace\subset A$ est un sous-anneau de $A$, appelé le centre\index{Centre (Anneau)} de $A$. Par exemple $Z(End_k(M))=kId_M$ et $Z(A)=A$ si et seulement si $A$ est commutatif. 
\item Si $\phi:A\rightarrow B$ est un morphisme d'anneaux, et $A'\subset A$ (resp. $B'\subset B$) est un sous-anneau alors $\phi(A')\subset B$ (resp. $\phi^{-1}(B')\subset A$) est un sous-anneau. En particulier, $\im(\phi)\subset B$ est un sous-anneau mais  $\ker(\phi)\subset A$ n'est un sous-anneau que si $A$ ou $B$ est l'anneau nul, sinon il ne contient pas $1$ (on verra un peu plus loin que $\ker(\phi)$ est ce qu'on appelle un idéal).
\end{itemize}

\subsubsection{}Soit $A$ un anneau commutatif. Une $A$-algèbre\index{Algèbre sur un anneau commutatif} est un couple $(B,\phi)$ o\`u $B$ est  un anneau et  $\phi:A\rightarrow B$ est un morphisme d'anneaux tel que $\hbox{\rm im}(\phi)\subset Z(B)$.   On notera en général $\phi:A\rightarrow B$ ou simplement  (lorsque la donnée de $\phi:A\rightarrow B$ ne peut prêter à confusion) $B$  la $A$-algèbre $(B,\phi)$.
  Etant donnés deux $A$-algèbres $\phi_B:A\rightarrow B$, $\phi_C:A\rightarrow C$, un morphisme de $A$-algèbres est un morphisme d'anneaux $\phi:B\rightarrow C$ tel que $\phi\circ \phi_B=\phi_C$. On remarquera que l'application identité $Id:B\rightarrow B$ est un morphisme de $A$-algèbres et que si $\phi:B\rightarrow C$ et $\psi:C\rightarrow D$ sont des morphismes de $A$-algèbres alors $\psi\circ \phi:B\rightarrow D$ est un morphisme de $A$-algèbres. On notera $Hom_A(B,C)$ l'ensemble des morphismes de $A$-algèbres $\phi:B\rightarrow C$ et, si $B=C$, $End_A(B):=Hom_A(B,C)$. On dit encore qu'un morphisme de $A$-algèbres $\phi:B\rightarrow C$ est injectif, (resp. surjectif, resp. un isomorphisme) si l'application d'ensembles sous-jacente est  injective (resp. surjective, resp. bijective). On vérifie que si $\phi:B\rightarrow C$ est un isomorphisme de $A$-algèbres l'application inverse $\phi^{-1}:C\rightarrow B$ est automatiquement un morphisme de $A$-algèbres.\\

 \textbf{Remarque.} On verra dans la partie II du cours, qu'une $A$-algèbre $\phi:A\rightarrow B$ est aussi la même chose qu'un anneau $B$ muni d'une structure de $A$-module  et qu'avec cette terminologie un morphisme de $A$-algèbres est un morphisme d'anneaux qui est aussi un morphisme de $A$-modules.  \\  

\textbf{Exemples.}
\begin{itemize}[leftmargin=* ,parsep=0cm,itemsep=0cm,topsep=0cm]
\item Le morphisme caractéristique\index{Caractéristique (Morphisme)} $c_A:\Z\rightarrow A$, $1\rightarrow 1_A$ munit tout anneau $A$ d'une structure de $\Z$-algèbre canonique et tout morphisme d'anneaux $\phi:A\rightarrow B$ est automatiquement un morphisme de $\Z$-algèbres pour ces structures (\textit{i.e.} $\phi\circ c_A=c_B$).   
\item L'inclusion $\iota_A: Z(A)\hookrightarrow  A $  munit tout anneau $A$ d'une structure de $Z(A)$-algèbre canonique. 
\item Si $A,B$ sont des anneaux commutatifs, tout morphisme d'anneaux $\phi:A\rightarrow B$ munit $B$ d'une structure de $A$-algèbre.\\
\end{itemize}

 \textbf{Exercice.} (Quaternions) On considère le $\R$-espace vectoriel $\HH$ de base $1,i,j,k$ muni du produit $\HH\times \HH\rightarrow \HH$ définie par $i^2=j^2=k^2=-1$, $ij=-ji=k$, $jk=-kj=i$, $ki=-ik=j$. 
 \begin{enumerate}
 \item Montrer que $(\HH,+,\cdot)$ est un anneau à division, non commutatif. Déterminer son centre   et en déduire que c'est  une $\R$-algèbre. 
 \item On note $i$ une racine carré de $-1$ dans $\C$ et on considère les matrices $$I:=\left(\begin{tabular}[c]{ll}
 $i$&$0$\\
 $0$&$-i$
 \end{tabular}\right),\; J:=\left(\begin{tabular}[c]{ll}
 $0$&$1$\\
 $-1$&$0$
 \end{tabular}\right),K:=\left(\begin{tabular}[c]{ll}
 $0$&$i$\\
 $i$&$0$
 \end{tabular}\right).$$
 Montrer que le sous-$\R$-espace vectoriel de $M_2(\C)$ engendré par $Id$, $I$, $J$, $K$ est un sous-$\R$-algèbre de $(M_2(\C),+,\cdot)$ isomorphe à $\HH$.
 \end{enumerate}
 
\subsection{Produits}\label{Produit (Anneaux)}\label{Produit}\index{Produit} Si $A_i$, $i\in I$ est une famille d'anneaux, on peut munir le produit ensembliste $\prod_{i\in I}A_i$ d'une structure d'anneau en posant, pour $\underline{a}=(a_i)_{i\in I}$, $\underline{b}=(b_i)_{i\in I}\in \prod_{i\in I}A_i$
$$\underline{a}+\underline{b}=(a_i+b_i)_{i\in I},\;\; \underline{a}\cdot \underline{b}=(a_i\cdot b_i)_{i\in I}$$
On a alors $0 =(0_{A_i})_{i\in I}$, $1 =(1_{A_i})_{i\in I}$. De plus, les projections $p_i:\prod_{i\in I}A_i\rightarrow A_i$, $\underline{a}\rightarrow a_i$, $i\in I$ sont automatiquement des morphismes d'anneaux. \\

\ref{Produit}.1 \textbf{Lemme.} (Propriété universelle du produit) \textit{Pour toute famille d'anneaux $A_i$, $i\in I$ il existe un anneau $\Pi$ et une famille de morphisme d'anneaux $p_i:\Pi\rightarrow A_i$, $i\in I$ tels que pour tout anneau $A$ et famille de morphisme d'anneaux $\phi_i:A\rightarrow A_i$, $i\in I$, il existe un unique morphisme d'anneaux $\phi:A\rightarrow \Pi$ tel que $p_i\circ \phi=\phi_i$, $i\in I$. }

\begin{proof} Vérifions que $\Pi:= \prod_{i\in I}A_i$ et les $p_i:\prod_{i\in I}A_i\rightarrow A_i$, $\underline{a}\rightarrow a_i$, $i\in I$ conviennent. Si $\phi:A\rightarrow \prod_{i\in I}A_i$ existe, la condition $p_i\circ \phi =\phi_i$, $i\in I$ force  
$\phi(a)=(\phi_i(a))_{i\in I}$, $a\in A$. Cela montre l'unicité de $\phi$  sous réserve de son existence. Pour conclure, il faut vérifier que $\phi$ défini par  $\phi(a)=(\phi_i(a))_{i\in I}$, $a\in A$ est bien un morphisme d'anneaux, ce qui résulte immédiatement des définitions.
\end{proof}

 On peut aussi réécrire \ref{Produit} en disant que, pour tout anneau $A$  l'application  canonique
$$\hbox{\rm Hom}(A,\prod_{i\in I}A_{i})\rightarrow \prod_{i\in I}\hbox{\rm Hom} (A,A_{i}), \; \phi\rightarrow (p_i\circ \phi)_{i\in I}$$
 est bijective ou encore, plus visuellement:
$$\xymatrix{&&A_i \\
A\ar@{.>}[r]^{\exists ! \phi}\ar@/^1pc/[urr]^{\phi_i}\ar@/_1pc/[drr]_{\phi_j}&\prod_{i\in I}A_i\ar[ur]^{p_i}\ar[dr]_{p_j}\\
&&A_j& }$$

\ref{Produit}.2 \textbf{Remarque.} Supposons que l'on ait un autre anneau  $\Pi'$ et une famille de morphisme d'anneaux $p_i':\Pi'\rightarrow A_i$, $i\in I$ vérifiant aussi la propriété du Lemme \ref{Produit}.1. On a alors, formellement:
\begin{enumerate}[leftmargin=* ,parsep=0cm,itemsep=0cm,topsep=0cm]
\item un unique morphisme d'anneaux $\phi:\Pi\rightarrow \Pi'$ tel que $p_i'\circ \phi=p_i$, $i\in I$;
\item un unique morphisme d'anneaux $\phi':\Pi'\rightarrow \Pi$ tel que $p_i\circ \phi'=p_i'$, $i\in I$;
\item un unique morphisme d'anneaux $\psi:\Pi\rightarrow \Pi$ tel que $p_i\circ \psi=p_i$, $i\in I$;
\item un unique morphisme d'anneaux $\psi':\Pi'\rightarrow \Pi'$ tel que $p_i'\circ \psi'=p_i'$, $i\in I$.
\end{enumerate}
 Mais on voit que dans (3) $\psi=\phi'\circ \phi$ et $\psi=Id_\Pi$ conviennent. L'unicité de $\psi$ dans (3) impose donc $\phi'\circ \phi=Id_\Pi$. Le même argument dans (4) montre que $\phi\circ \phi'=Id_{\Pi'}$. Autrement dit, les morphismes d'anneaux $\phi:\Pi\rightarrow \Pi'$ de (1) et $\phi':\Pi'\rightarrow \Pi$ de (2) sont inverses l'un de l'autre. On dit de fa\c{c}on un peu informelle que l'anneau produit  $p_i:\prod_{i\in I}A_i\rightarrow A_i$, $i\in I$ est unique à unique isomorphisme près. On rencontrera beaucoup d'autres constructions de ce type dans la suite.\\

   Soit $\phi_i:A_i\rightarrow B_i$, $i\in I$ une famille de morphismes d'anneaux. En appliquant la propriété universelle des $p_j:\prod_{i\in I}B_i\rightarrow B_j$, $j\in I$ à la famille de morphismes d'anneaux
$$ \prod_{i\in I}A_i\stackrel{p_j}{\rightarrow}A_j\stackrel{\phi_j}{\rightarrow} B_j,\; j\in I$$
on obtient un unique morphisme d'anneaux $\phi:=\prod_{i\in I}\phi_i:\prod_{i\in I}A_i\rightarrow \prod_{i\in I} B_i$ tel que $p_i\circ \phi=\phi_i\circ p_i$, $i\in I$. \\


\ref{Produit}.3 Si $A_i=A$, $i\in I$, on note $\prod_{i\in I}A_i=A^I$. On peut  voir $A^I$ comme l'anneau des fonctions $a:I\rightarrow A$ muni de $(a+b)(i)=a(i)+b(i)$  et $(a\cdot b)(i)=a(i)\cdot b(i)$  de zéro l'application nulle et d'unité l'application constante de valeur $1_A$. On notera qu'on a un morphisme d'anneaux injectif canonique $\Delta_A:A\hookrightarrow A^I$, $a\rightarrow (i\rightarrow a(i)=a)$ appelé morphisme diagonal (et qui, si $A$ est commutatif,   fait de $A^I$  une $A$-algèbre de fa\c{c}on canonique).\\

  Pour tout $\underline{a}=(a_i)_{i\in I}\in A^I$ notons $supp(\underline{a}):=\lbrace i\in I\; |\; a_i\not= 0\rbrace\subset I$ le \textit{support} de $\underline{a}$. Notons $$A^{(I)}:=\lbrace \underline{a}\in A^I\; |\; |supp(\underline{a})|<+\infty\rbrace \subset A^I.$$
On observera que $A^{(I)}\subset A^I$ est stable par différence et produit mais que, si $I$ est infini, ce n'est pas un sous-anneau de $A^I$ car il ne contient pas $1_{A^I}$. 


\subsection{Algèbres de polynômes}\label{Poly}\index{Polynômes}Soit $A$ un anneau commutatif.   
Comme on vient de l'observer, le sous-ensemble $A^{(\N)}$ de $ A^\N$ est stable par différence et produit mais ce n'est pas un sous-anneau de $A^\N$ car il ne contient pas $1_{A^\N}$. En utilisant que $(\N,+)$ est un monoide on peut cependant    faire un anneau de $A^{(\N)}$, en le munissant d'une autre multiplication que celle héritée de $A^{\N}$. Notons  $e_n:=(\delta_{m,n}1_{A})_{m\in \N}$, $n\in \N$ et pour $a\in A$, $ae_n:=(\delta_{m,n}a)_{m\in \N}$, $n\in \N$ ; $A^{(\N)}$ contient les $ae_n$, $n\in \N$, $a\in A$ et, par définition,  tout élément $\underline{a}\in A^{(\N)}$ s'écrit de fa\c{c}on unique sous la forme $\underline{a}=\sum_{n\in \N}a_ne_n$. Munissons donc $A^{(\N)}$ de l'addition héritée de celle de $A^\N$ et du produit `de convolution' $*$ défini sur les éléments $e_n$, $n\in \N$ par 
$e_m*e_n=e_{m+n} $ et en général par
$$(\ref{Poly}.1)\;\; (\sum_{n\in\N}a_ne_n) *(\sum_{n\in \N}b_ne_n)=\sum_{n\in \N}(\sum_{i,j\in \N, i+j=n}a_ib_j)e_n.$$
On vérifie facilement que $(A^{(\N)},+,*)$ est un anneau commutatif ayant pour unité $e_0$. L'application canonique $\iota_A:A\rightarrow A^{(\N)}$, $a\rightarrow ae_0  $  est un morphisme d'anneaux.
 On note traditionnellement cet anneau $(A[X],+,\cdot)$ et on dit que $\iota:A\rightarrow A[X]$ est la $A$-algèbre des polynômes à une inderminée. On pose aussi $X^n:=e_n$, $n\in \N$ et $1:=X^0$ de sorte que (\ref{Poly}.1) se réécrit de fa\c{c}on plus intuitive sous la forme
 $$(\ref{Poly}.2)\;\; (\sum_{n\in\N}a_nX^n)(\sum_{n\in \N}b_nX^n)=\sum_{n\in \N}(\sum_{i,j\in \N, i+j=n}a_ib_j)X^n.$$
 
 \ref{Poly}.3 \textbf{Lemme.} (Propriété universelle de la $A$-algèbre des polynômes à une indéterminée) \textit{Pour tout anneau commutatif $A$, il existe une $A$-algèbre $\iota_A: A\rightarrow P$ munie d'un élément $p\in P$ tels que pour toute $A$-algèbre $\phi: A\rightarrow B$ et  $b\in B$, il existe un unique  morphisme de $A$-algèbres $ev^\phi_b:P\rightarrow B$  tel que $ ev^\phi_b(p)=b$. }

\begin{proof} Vérifions que $\iota_A:A\rightarrow A[X]$ munie de $X$ conviennent. Si $ev_b^\phi:A[X]\rightarrow B$ existe,  on a par définition d'un morphisme de $A$-algèbres:
$$ev^\phi_b(\sum_{n\geq 0}a_nX^n)=\sum_{n\geq 0}ev_b^\phi(a_n)ev_b^\phi(X)^n=\sum_{n\geq 0}\phi(a_n)b^n,$$
d'o\`u l'unicité de $ev_b^\phi$ sous réserve d'existence.  Pour conclure, il faut vérifier que $ev_b^\phi$ défini par  $ev^\phi_b(\sum_{n\geq 0}a_nX^n)= \sum_{n\geq 0}\phi(a_n)b^n,$ est bien un morphisme d'anneaux, ce qui là encore résulte immédiatement des définitions.
\end{proof}

  Le même argument  formel que celui utilisé dans \ref{Produit}.2 montre que la $A$-algèbre $\iota_A:A\rightarrow A[X]$ est unique à unique isomorphisme  près.\\
 
  On peut aussi réécrire \ref{Poly}.3 en disant que, pour toute $A$-algèbre $\phi:A\rightarrow B$  l'application canonique
$$\hbox{\rm Hom}_A(A[X],B)\rightarrow B,\; f\rightarrow f(X)$$
 est bijective. On adopte  en général la notation plus intuitive $ev_{b}^\phi(P)=P( b)$ et on dit que $ev_{b}^\phi$ est le morphisme d'évaluation en $b$.\\
\\
 Soit $\phi:A  \rightarrow B $ un morphisme  d'anneaux commutatifs. En appliquant la propriété universelle des $\iota_A:A\rightarrow A[X]$  à la $A$-algèbre
$$A\stackrel{\phi}{\rightarrow} B\stackrel{\iota_B}{\rightarrow}B[X]$$
on obtient un unique morphisme d'anneaux  $\tilde{\phi}:A[X]\rightarrow B[X]$ tel que $\iota_B\circ \phi=\phi\circ \iota_A$; explicitement $\tilde{\phi}(\sum_{n\geq 0}a_xX^n)=\sum_{n\geq 0}\phi(a_n)X^n$.\\ 
 
 
   \ref{Poly}.4 \textbf{Remarque.} Ce qui nous a permis de définir le produit $*$ sur $A^{(\N)}$ et le fait que $(\N,+)$ est un monoïde: on a utilisé l'addition pour définir $e_n*e_m=e_{n+m}$, l'associativité de $*$ résulte de celle de $+$ sur $\N$ et le fait que $e_0$ soit l'unité de $A^{(\N)}$ du fait que $0$ est l'unité de $\N$.  Pour un monoïde $(M,\cdot)$ quelconque, l'application
 $$\hbox{\rm Hom}_{Mono}(\N,M)\rightarrow M,\; f\rightarrow f(1)$$
 est bijective d'inverse l'application qui à $m\in M$ associe le morphisme de monoïdes $f_m:(\N,+)\rightarrow (M,\cdot)$, $n\rightarrow m^n(=m\cdots m$ $n$ fois). Dans \ref{Poly}.3, se donner $p\in P$ et $b\in B$ revient donc à se donner des morphismes de monoïdes $\nu_A:(\N,+)\rightarrow (P,\cdot)$, $n\rightarrow p^n$ et $\nu:(\N,+)\rightarrow (B,\cdot)$, $n\rightarrow b^n$ et la condition $ev^\phi_b(p)=b$ signifie que $ev^\phi_b\circ \nu_A=\nu$. Avec ce point de vue, on peut reformuler \ref{Poly}.3 comme suit. \\
 
  \ref{Poly}.3' \textbf{Lemme.} (Propriété universelle de la $A$-algèbre des polynômes à une indéterminée) \textit{Pour tout anneau commutatif $A$, il existe une $A$-algèbre $\iota_A: A\rightarrow P$ et un morphisme de monoïdes $\nu_A:(\N,+)\rightarrow (P,\cdot)$ tels que pour toute $A$-algèbre $\phi: A\rightarrow B$ et tout morphisme de monoïdes $\nu:(\N,+)\rightarrow (B,\cdot)$, il existe un unique  morphisme de $A$-algèbres $ev^\phi_b:P\rightarrow B$  tel que $ ev^\phi_b\circ \nu_A=\nu$. }\\

 Ou encore: pour toute $A$-algèbre $\phi:A\rightarrow B$  l'application canonique
$$\hbox{\rm Hom}_A(A[X],B)\rightarrow \hbox{\rm Hom}_{Mono}(\N,B),\; f\rightarrow f\circ \nu_A.$$
 est bijective. Explicitement, $\nu_A:(\N,+)\rightarrow (A[X],\cdot)$ est le morphisme qui envoie $n$ sur $X^n$ donc si $f:A[X]\rightarrow B$ est un morphisme de $A$-algèbres, $f\circ \nu_A:(\N,+)\rightarrow (B,\cdot)$  est le morphisme qui envoie $n$ sur $f(X)^n$.\\

 
 
  \ref{Poly}.5 Avec le point de vue développé dans la Remarque  \ref{Poly}.4, on peut faire la construction précédente en rempla\c{c}ant $(\N,+)$ par n'importe quel monoïde $(N,\cdot)$ (non nécessairement commutatif, non nécessairement dénombrable) d'unité $1_N$.   Notons toujours $e_{n}:=(\delta_{m,n}1_{A})_{m\in N}$, $n\in N$ et pour $a\in A$, $ae_{n}:=(\delta_{m,n}a)_{m\in  N}$, $n\in N$ ; $A^{(N)}$ contient les $ae_{n}$, $n\in N$, $a\in A$ et, par définition,  tout élément $\underline{a}\in A^{(N)}$ s'écrit de fa\c{c}on unique sous la forme $\underline{a}=\sum_{\underline{n}\in N^r}a_ne_{n}$. En munissant $A^{(N)}$ de l'addition héritée de celle de $A^{N}$ et du produit `de convolution' $*$ défini sur les éléments $e_{n}$, $n\in N$ par 
$e_{m}*e_{n}=e_{m\cdot n} $ et en général par
$$(\ref{Poly}.6)\;\; (\sum_{n\in N}a_{n}e_{n}) *(\sum_{n\in N}b_{n}e_{n})=\sum_{n\in N}(\sum_{i,j\in N,i\cdot j=n}a_{i}b_{j})e_{n}.$$
on obtient un anneau (commutatif si $(N,\cdot)$ est commutatif) $(A^{(N)},+,*)$  ayant pour unité $e_{\underline{1_N}}$. L'application canonique $\iota_A:A\rightarrow A^{(N)}$, $a\rightarrow ae_{1_N}  $  est un morphisme d'anneaux et l'application $\nu_A:N\rightarrow A^{(N)}$, $n\rightarrow e_n$ prend ses valeur dans $A^{(N)}\setminus\lbrace 0\rbrace$ et induit un morphisme de monoïdes $\nu_A:(N,\cdot)\rightarrow (A^{(N)}\setminus\lbrace 0\rbrace,*)$.
 On note traditionnellement cet anneau $(A[N],+,\cdot) $ et on dit que $\iota_A:A\rightarrow A[N]$ est la $A$-algèbre du monoïde $(N,\cdot)$. On pose aussi $n:=e_{n}$, $n\in N$    et $1:=1_N$ de sorte que (\ref{Poly}.5) se réécrit de fa\c{c}on plus intuitive sous la forme
 $$(\ref{Poly}.7)\;\; (\sum_{n\in N}a_{n}n) *(\sum_{n\in N}b_{n}n)=\sum_{n\in N}(\sum_{i,j\in N,i\cdot j=n}a_{i}b_{j})n.$$

 \ref{Poly}.8 \textbf{Lemme.} (Propriété universelle de la $A$-algèbre du monoïde $(N,\cdot)$) \textit{Pour tout anneau commutatif $A$, il existe une $A$-algèbre $\iota_A: A\rightarrow P$ et un  morphisme de monoïdes $\nu_A:(N,\cdot)\rightarrow (P  ,\cdot)$ tels que pour toute $A$-algèbre   $\phi: A\rightarrow B$ et tout morphisme de monoïdes $\nu:(N,\cdot)\rightarrow (B  ,\cdot)$  il existe un unique  morphisme de $A$-algèbres $\tilde{\nu}:P\rightarrow B$  tel que $ \tilde{\nu}\circ \nu_A=\nu$. }

\begin{proof} Similaire à celle de \ref{Poly}.3 en vérifiant que $\iota_A:A\rightarrow A[N]$ convient. \end{proof}

  Le même argument  formel que celui utilisé dans \ref{Produit}.2 montre que la $A$-algèbre $\iota_A: A\rightarrow A[N]$ est unique à unique isomorphisme  près.\\
 
  On peut aussi réécrire \ref{Poly}.8 en disant que, pour toute $A$-algèbre $\phi:A\rightarrow B$  l'application canonique
$$\hbox{\rm Hom}_A(A[N],B)\rightarrow\hbox{\rm Hom}_{Mono}(N,B),\; f\rightarrow f\circ\nu_A$$
 est bijective. Son inverse est l'application qui à $\nu:(N,\cdot)\rightarrow (B,\cdot) $ associe l'unique morphisme de $A$-algèbres $ \tilde{\nu}:A[N]\rightarrow B$ tel que $ \tilde{\nu}(n)=\nu(n)$ (donc $\tilde{\nu}(\sum_{n\in N}a_nn)=\sum_{n\in N}\phi(a_n)\nu(n)$).\\
 
  \textbf{Exemples.} Si on prend\\
  
  \begin{enumerate}[leftmargin=* ,parsep=0cm,itemsep=0cm,topsep=0cm]
 \item  $(N,\cdot)=(\N,+)$ on retrouve $A[\N]=A[X]$.\\
 % (observer que se donner un morphisme de monoïdes $\nu:(\N,+)\rightarrow (B ,\cdot)$ revient à se donner l'image $b\in B$ de $1\in \N$).\\
 \item $(N,\cdot)=(\N^r,+)$ o\`u $+$ est l'addition termes à termes (pour $\underline{m}=(m_1,\dots,m_r), \underline{n}:=(n_1,\dots, n_r)\in \N^r$,   $\underline{m}+\underline{n}=(m_1+n_1,\dots,m_r+n_r)\in \N^r$). Dans ce cas, on note $\underline{X}^{\underline{n}}:=X_1^{n_1}\cdots X_r^{n_r}:=e_{\underline{n}}$, $\underline{n}\in \N^r$ avec la convention $X_i^0=1$, $i=1,\dots, r$,  et $1:=\underline{X}^{\underline{0}}$ de sorte que (\ref{Poly}.5) se réécrit de fa\c{c}on plus intuitive sous la forme
 $$ (\sum_{\underline{n}\in\N^r}a_{\underline{n}}\underline{X}^{\underline{n}}) (\sum_{\underline{n}\in \N^r}b_{\underline{n}}\underline{X}^{\underline{n}})=\sum_{\underline{n}\in \N}(\sum_{\underline{i},\underline{j}\in \N^r, \underline{i}+\underline{j}=\underline{n}}a_{\underline{i}}b_{\underline{j}})\underline{X}^{\underline{n}}.$$
On note également $A[X_1,\dots, X_r]:=A[\N^r]$ et on dit que $\iota_A:A\rightarrow A[X_1,\dots, X_r]$ est la $A$-algèbre des polynômes à $r$ inderminées. Comme se donner un morphisme de monoïdes $\nu:(\N^r,+)\rightarrow (B ,\cdot)$ revient à se donner les images $b_i\in B $ de $(\delta_{i,j})_{1\leq j\leq r}\in \N^r$, on peut reformuler \ref{Poly}.7 de la fa\c{c}on suivante.\\

   Pour toute $A$-algèbre $\phi:A\rightarrow B$, en notant  
 $$\frak{B}_r:= \lbrace \underline{b}=(b_1,\dots, b_r)\in B^r\;|\; b_ib_j=b_jb_i,\; 1\leq i,j\leq r\rbrace,$$
$$\hbox{\rm Hom}_A(A[X_1,\dots,X_r],B)\rightarrow \frak{B}_r,\; f\rightarrow (f(X_1),\dots, f(X_r))$$
 est bijective. Son inverse est l'application qui à $\underline{b}=(b_1,\dots, b_r)\in \frak{B}_r$ associe l'unique morphisme de $A$-algèbres $ev_{\underline{b}}^\phi:A[X_1,\dots, X_r]\rightarrow B$ tel que $ev_{\underline{b}}^\phi(X_i)=b_i$, $i=1,\dots r$ (donc $ev_{\underline{b}}^\phi(\sum_{\underline{n}\in \N^r}a_{\underline{n}}\underline{X}^{\underline{n}})=\sum_{\underline{n}\in \N^r}\phi(a_{\underline{n}})\underline{b}^{\underline{n}}$). On adopte  en général la notation plus intuitive $ev_{\underline{b}}^\phi(P)=P(b_1,\dots, b_r)$ et on dit que $ev_{\underline{b}}^\phi$ est le morphisme d'évaluation en $\underline{b}$.\\
 
 
 \item Pour $(N,\cdot)$ un groupe, pour toute $A$-algèbre $\phi:A\rightarrow B$, tout morphisme de monoïdes $\nu:(N,\cdot)\rightarrow (B,\cdot)$ est automatiquement à valeur dans le groupe $(B^\times,\cdot)$. On dit dans ce cas que $ A[N]$ est la $A$-algèbre du groupe $(N,\cdot)$.\\
 
  Par exemple, pour $(N,\cdot)=(\Z,+)$, on obtient la $A$-algèbre (notations:  $A[X,X^{-1}]:=A[\Z]$, $X^n:=e_n$, $n\in \Z$ donc en particulier $X^nX^{-n}=e_ne_{-n}=e_{n-n}=e_0=1$) des polynômes de Laurent à une indéterminée. Comme se donner    un morphisme de monoïdes $\nu:(\Z,+)\rightarrow (B ,\cdot)$ revient à se donner l'image  $b \in B^\times $ de $1\in \Z$, on peut reformuler \ref{Poly}.7 de la fa\c{c}on suivante.\\

   Pour toute $A$-algèbre $\phi:A\rightarrow B$,  l'application canonique
$$\hbox{\rm Hom}_A(A[X,X^{-1}],B)\rightarrow B^\times,\; f\rightarrow f(X)$$
 est bijective. Son inverse est l'application qui à $b \in B^\times$ associe l'unique morphisme de $A$-algèbres $ev_{\underline{b}}^\phi:A[X,X^{-1}]\rightarrow B$ tel que $ev_{\underline{b}}^\phi(X)=b$ (donc $ev_{\underline{b}}^\phi(\sum_{n\in \Z}a_n\underline{X}^{\underline{n}})=\sum_{n\in \Z}\phi(a_n)b^n$).\\ 
 
  De même, pour $(N,\cdot)=(\Z^r,+)$, on obtient la $A$-algèbre (notations:  $A[X_1,X_1^{-1},\cdots, X_r,X_r^{-1}]:=A[\Z^r]$, $\underline{X}^{\underline{n}}:=X_1^{n_1}\cdots X_r^{n_r}:=e_{\underline{n}}$, $\underline{n}\in \Z^r$ donc en particulier, $\underline{X}^{\underline{n}}\underline{X}^{-\underline{n}}= e_{\underline{n}}e_{-\underline{n}}=e_{\underline{n}-\underline{n}}=e_0=1$) des polynômes de Laurent à $r$ indéterminées. Comme se donner    un morphisme de monoïdes $\nu:(\Z^r,+)\rightarrow (B ,\cdot)$ revient à se donner les images $b_i \in B^\times $ des $(\delta_{i,j})_{1\leq j\leq r}\in \Z$, $i=1,\dots, r$ on peut reformuler \ref{Poly}.8 de la fa\c{c}on suivante.\\
 
   Pour toute $A$-algèbre $\phi:A\rightarrow B$, en notant  
 $$\frak{B}^\times_r:= \lbrace \underline{b}=(b_1,\dots, b_r)\in (B^\times)^r\;|\; b_ib_j=b_jb_i,\; 1\leq i,j\leq r\rbrace,$$  l'application canonique
$$\hbox{\rm Hom}_A(A[X_1,X_1^{-1},\cdots, X_r,X_r^{-1}],B)\rightarrow \frak{B}_r,\; f\rightarrow (f(X_1),\dots, f(X_r))$$
 est bijective. Son inverse est l'application qui à $\underline{b}=(b_1,\dots, b_r)\in \frak{B}^\times_r$ associe l'unique morphisme de $A$-algèbres $ev_{\underline{b}}^\phi:A[X_1,X_1^{-1},\cdots, X_r,X_r^{-1}]\rightarrow B$ tel que $ev_{\underline{b}}^\phi(X_i)=b_i$, $i=1,\dots r$ (donc $ev_{\underline{b}}^\phi(\sum_{\underline{n}\in \Z^r}a_{\underline{n}}\underline{X}^{\underline{n}})=\sum_{\underline{n}\in \Z^r}\phi(a_{\underline{n}})\underline{b}^{\underline{n}}$).\\

 

 \end{enumerate}

 
 
 
 

 
  (\ref{Poly}.9)  Soit $(N,\cdot)$ un monoïde et $\phi: A\rightarrow B$ un morphisme d'anneaux commutatifs. La propriété universelle de $\iota_A:A\rightarrow A[N]$ appliquée avec $A\stackrel{\phi}{\rightarrow} B\stackrel{\iota_B}{\rightarrow} B[N]$ donne un unique morphisme de $A$-algèbres $\tilde{\phi}:A[N]\rightarrow B[N]$ tel que $\nu_B=\tilde{\phi}\circ \nu_A$. Explicitement $\tilde{\phi}(\sum_{n\geq 0}a_ne_n)=\sum_{n\geq 0}\phi(a_n)e_n$.  Par construction, im$(\phi)=\hbox{\rm im}(\phi)[N]\subset B[N]$ et $\ker(\tilde{\phi})$ est l'ensemble des éléments de la forme $\sum_{n\geq 0}a_ne_n\in A[N]$ tels que $a_n\in \ker(\phi)$, $n\geq 0$. On notera $\ker(\phi)[N]:=\ker(\tilde{\phi})\subset A[N]$.\\
 
 \ref{Poly}.10 \textbf{Exercice.} \\
 
 \begin{enumerate}[leftmargin=* ,parsep=0cm,itemsep=0cm,topsep=0cm]
 \item Montrer qu'on a un morphisme surjectif $A$-algèbres canonique $$A[X_1,Y_1,\dots, X_r,Y_r]\twoheadrightarrow A[X_1,X_1^{-1},\dots,X_r, X_r^{-1}].$$
 
   Correction. \textit{Plus généralement, on peut montrer qu'on a une application canonique injective $$\begin{tabular}[t]{cccc}
$\tilde{-}:$&$ \SHom_{Mono}(N_1,N_2)$&$\hookrightarrow$&$ \SHom_{A}(A[N_1], A[N_2])$\\
&$\nu:N_1\rightarrow N_2$&$\rightarrow$&$\tilde{\nu}:A[N_1]\rightarrow A[N_2]$
\end{tabular}$$ 
qui envoie morphismes de monoïdes injectifs (resp. surjectifs, resp. bijectifs) sur morphismes de $A$-algèbres injectifs (resp. surjectifs, resp. bijectifs). L'existence de $\tilde{-}: \SHom_{Mono}(N_1,N_2) \rightarrow  \SHom_{A}(A[N_1], A[N_2])$ est une conséquence formelle de la propriété universelle de la $A$-algèbre de monoïdes $\iota_A:A\rightarrow A[N_1]$ appliquée  avec la $A$-algèbre   $\iota_A:A\rightarrow A[N_2]$ et le morphisme de monoïdes $ N_1\stackrel{\nu}{\rightarrow} N_2\stackrel{\nu_A}{\rightarrow} A[N_2]$: il existe un unique morphisme de $A$-algèbre $\tilde{\nu}:A[N_1]\rightarrow A[N_2]$ tel que le diagramme suivant commute
$$\xymatrix{N_1\ar[r]^{\nu_A}\ar[d]_\nu&A[N_1]\ar[d]^{\tilde{\nu}}\\
N_2\ar[r]_{\nu_A}  &A[N_2]}$$
L'injectivité de $\tilde{-}: \SHom_{Mono}(N_1,N_2) \rightarrow  \SHom_{A}(A[N_1], A[N_2])$ résulte de l'injectivité des $\nu_A:N_i\rightarrow A[N_i]$, $i=1,2$. Enfin, le fait que $\tilde{-}: \SHom_{Mono}(N_1,N_2) \rightarrow  \SHom_{A}(A[N_1], A[N_2])$ envoie morphismes de monoïdes injectifs (resp. surjectifs, resp. bijectifs) sur morphismes de $A$-algèbres injectifs (resp. surjectifs, resp. bijectifs) résulte du fait que, par construction, tout élément de $A[N]$ s'écrit de fa\c{c}on unique sous la forme $\sum_{n\in N}ae_n$ (on verra dans le chapitre sur les modules que $A[N]$ est un $A$-module libre de base les $e_n$, $n\in N$) et que la condition $\nu_A\circ \nu=\tilde{\nu}\circ \nu_A$ impose $\tilde{\nu}(e_n)=e_{\nu(n)}$.\\
La question posée correspond au cas particulier du morphisme de monoïdes surjectif $\nu:(\N^2,+)\twoheadrightarrow (\Z,+)$ défini par $\nu(n_1,n_2)=n_1-n_2$ (le $\tilde{\nu}:A[X_1,Y_1,\dots, X_r,Y_r]\twoheadrightarrow A[X_1,X_1^{-1},\dots,X_r, X_r^{-1}]$ correspondant étant défini par $\tilde{\nu}(X_i)=Z_i$, $\tilde{\nu}(Y_i)=Z_i^{-1}$, $i=1,2$). }\\
 
\item Montrer qu'on a des isomorphismes de $A$-algèbres canonique $$  A[X_1,\dots,X_{i-1},X_{i+1},\dots, X_r][X_i]\tilde{\rightarrow} A[X_1,\dots,X_r],\; i=1,\dots, r.$$

   Correction. \textit{Observons d'abord que toute permutation $\sigma\in \mathcal{S}_r$ induit un automorphisme du monoïde $(\N^r,+)$ par permutation des coordonnées donc, d'après (1), un automorphisme de la $A$-algèbre $A[X_1,\dots, X_r]$. (Explicitement, $\sigma P(X_1,\dots, X_r)=P(X_{\sigma(1)},\dots, X_{\sigma(r)})$). Il suffit donc de montrer le résultat pour $i=r$. Par unicité  à unique isomorphisme près des objets universel, il suffit de montrer que 
$\iota_A:A\rightarrow A[X_1,\dots, X_r]$ et $A\stackrel{\iota_A}{\rightarrow} A[X_1,\dots, X_{r-1}]\stackrel{\iota_{A[X_1,\dots, X_{r-1}]}}{\rightarrow} A[X_1,\dots, X_{r-1}][X_r]$ vérifie la même propriété universelle. Notons que par hypothèse  $A[b_1,\dots ,b_r]$ est un anneau commutatif (\textit{cf.} \ref{SousAlg} pour la notation $A[b_1,\dots, b_{r-1}]$). Soit donc $\phi:A\rightarrow B$ une $A$-algèbre et $b_1,\dots, b_r\in B^r$ commutant deux à deux. Par la propriété universelle de $\iota_A:A\rightarrow A[X_1,\dots, X_{r-1}]$, il existe un unique morphisme de $A$-algèbre $ev_{(b_1,\dots, b_{r-1})}^\phi:A[X_1,\dots,X_{r-1}]\rightarrow B$ tel que $\phi_1(X_i):=ev_{(b_1,\dots, b_{r-1})}^\phi(X_i)=b_i$, $i=1,\dots, r-1$. Puis, par la propriété universelle de $\iota_{A[X_1,\dots, X_{r-1}]}:A[X_1,\dots, X_{r-1}]\rightarrow A[X_1,\dots, X_r]$, il existe un unique morphisme de $A$-algèbre $ev_{b_r}^{\phi_1}:A[X_1,\dots,X_{r-1}][X_r]\rightarrow A[b_1,\dots ,b_{r-1}][b_r]=A[b_1,\dots,b_r]$ tel que $ev_{b_r}^{\phi_1}(X_r)=b_r$... On laisse le soin au lecteur de généraliser ce genre d'exercice formel un tantinet fastidieux.} \\
 
  \end{enumerate}
 
 

 
  \subsection{Sous-anneau engendré par une partie} Soit  $A_i\subset A$, $i\in I$ une famille de sous-anneaux. On vérifie immédiatement que $\cap_{i\in I}A_i\subset A$ est un sous-anneau. Pour tout sous-ensemble $X\subset A$, il existe 
un unique sous-anneau $\langle X\rangle \subset A$, contenant $X$ et minimal pour $\subset$ \textit{i.e.} tel que pour  tout sous-anneau $A'\subset A$,   $X\subset A'$ implique  $\langle X\rangle\subset A'$. On dit que $\langle X\rangle\subset A$ est le sous-anneau de $A$ engendré par $X$.  Explicitement $\langle X\rangle$ est l'intersection de tous les sous-anneaux de $A$ contenant $X$. On peut également décrire $\langle X\rangle$ comme  l'ensemble des sommes finies de produits finis d'éléments de $X$. Si $A=\langle X\rangle$, on dit que $X$ est un système de générateurs de $A$ comme anneau (ou que $A$ est engendré par $X$ comme anneau). Si on peut prendre de plus $X$ fini, on dit que $A$ est un anneau de type fini.\\

 Lorsque les éléments de $X$ commutent deux à deux, on note en général $\Z[X]:=\langle X\rangle \subset A$ le sous-anneau de $A$ engendré par $X$. Si  $X=\lbrace x_1,\dots,x_r\rbrace $ est fini, on note plutôt $\Z[x_1,\dots,x_r]:=\Z[X]$ et \ref{Poly}.8  nous donne un unique morphisme d'anneaux - automatiquement  surjectif - $ev_{\underline{x}}:\Z[X_1,\cdots, X_r]\twoheadrightarrow \Z[x_1,\dots,x_r] $ tel que $ev_{\underline{x}}(X_i)=x_i$, $i=1,\dots, r$.   \\
 
  \subsection{Sous-$A$-algèbre engendrée par une partie}\label{SousAlg} Soit $\phi:A\rightarrow B$ une $A$-algèbre. Une sous-$A$-algèbre de $\phi:A\rightarrow B$ est un sous-anneau $B'\subset B$ tel que $\hbox{\rm im}(\phi)\subset B'$ (noter que $Z(B)\cap B'\subset Z(B')$); le morphisme $\phi|^{B'}:A\rightarrow B'$ munit alors $B'$ d'une structure de $A$-algèbre qui fait de l'inclusion $B'\subset B$ un morphisme de $A$-algèbres. Si    $B_i\subset B$, $i\in I$ est une famille de sous-$A$-algèbres, $\cap_{i\in I}B_i\subset B$  est encore une sous-$A$-algèbre. Pour tout sous-ensemble $X\subset B$, il existe 
une unique sous-$A$-algèbre $\langle X\rangle_A \subset B$, contenant $X$ et minimale pour $\subset$. On dit que $\langle X\rangle_A\subset B$ est la sous-$A$-algèbre de $B$ engendrée par $X$. Explicitement $\langle X\rangle_A$ est l'intersection de tous les sous-$A$-algèbres de $B$ contenant $X$. On peut également décrire $\langle X\rangle_A$ comme  le sous-anneau de $B$ engendré par $X\cup\hbox{\rm im}(\phi)$. Si $B=\langle X\rangle_A$, on dit que $X$ est un système de générateurs de $B$ comme $A$-algèbre (ou que $B$ est engendré par $X$ comme $A$-algèbre). Si on peut prendre $X$ fini, on dit que $B$ est une $A$-algèbre de type fini\index{de type fini (Algèbre)}.\\

 Lorsque les éléments de $X$ commutent deux à deux, on note en général $A[X]:=\langle X\rangle_A \subset B$ la sous-$A$-algèbre de $B$ engendré par $X$. Si  $X=\lbrace x_1,\dots,x_r\rbrace $ est fini, , on note plutôt $A[x_1,\dots,x_r]:=A[X]$ et  \ref{Poly}.8  nous donne un unique morphisme de $A$-algèbres - automatiquement  surjectif - $ev_{\underline{x}}^\phi:A[X_1,\cdots, X_r]\twoheadrightarrow A[x_1,\dots, x_r] $ tel que $ev^\phi_{\underline{x}}(X_i)=x_i$, $i=1,\dots, r$. \\

 
 
 \begin{center} **  Dans la suite,  sauf mention explicite du contraire, nous ne considérerons que des anneaux commutatifs **\\\end{center}
 

 
\section{Idéaux et quotients}\label{Ideaux}
\subsection{Définitions, premiers exemples}
\subsubsection{}Soit $A$ un anneau (commutatif, donc). Un idéal\index{Idéal} de  $A$ est un sous-ensemble $I\subset A$ tel que $a'-b'\in I$,  $a',b'\in I$ et $aa'\in I$, $a\in A$, $a'\in I$. On notera $\mathcal{I}_A$ l'ensemble des idéaux de $A$; l'inclusion ensembliste $\subset$ munit $\mathcal{I}_A$ d'un ordre partiel. Pour un idéal $I\subset A$, on notera $V^{tot}(I)\subset \mathcal{I}_A$ le sous-ensemble des idéaux de $A$ qui contiennent $I$\\

\textbf{Exemples.}
\begin{itemize}[leftmargin=* ,parsep=0cm,itemsep=0cm,topsep=0cm]
\item Le singleton  $\lbrace 0\rbrace$ et $A$ sont  des idéaux de $A$.
\item Si $k$ est un corps commutatif, les seuls idéaux de $k$ sont  $\lbrace 0\rbrace$ et $k$.
\item Un idéal $I\subset A$ est en particulier un sous-groupe de $(A,+)$. Par exemple, les seuls candidats possibles pour les idéaux de $\Z$ sont les $n\Z$, $n\geq 0$ (division euclidienne). On vérifie immédiatement que les $n\Z$ sont bien des idéaux de $\Z$. Donc les idéaux de $\Z$ sont exactement les $n\Z$, $n\geq 1$. On notera que $n\Z\subset m\Z$ si et seulement si $m|n$.  La $k$-algèbre $k[X]$ des polynômes à une indéterminée sur un corps est également munie d'une division euclidienne et on verra que dans ce cas aussi, tous les idéaux de $k[X]$ sont de la forme $Pk[X]$, $P\in k[X]$.
\item Pour tout $a\in A$, $Aa\subset A$ est un idéal. Les idéaux de cette forme sont appelés principaux. On dit qu'un anneau $A$ principal si tous ses idéaux sont principaux et s'il est intègre. Les anneaux $\Z$ et $k[X]$ sont principaux.  Par contre, $k[X,Y]$ n'est pas principal, par exemple l'ensemble $I:=\lbrace XP+YQ\;|\; P,Q\in k[X,Y]\rbrace\subset k[X,Y]$ est un idéal qui n'est pas principal.
\item Si $A_i$, $i\in I$ est une famille d'anneaux, et, pour chaque $i\in I$, $I_i\subset A_i$ est un idéal, $\prod_{i\in I}I_i\subset \prod_{i\in I}A_i$ est un idéal. Mais les idéaux de $\prod_{i\in I}A_i$ ne sont pas tous de cette forme. Par exemple, $A^{(I)}\subset A^I$ est un idéal de $A^I$ qui n'est pas un produit d'idéaux.
\item Si $I\subset A$ est un idéal, $I[X_1,\dots, X_r]:=\lbrace\sum_{\underline{n}\in \N^r}a_{\underline{n}}\underline{X}^{\underline{n}}\;|\; a_{\underline{n}}\in I, \; \underline{n}\in \N^r \rbrace \subset A[X_1,\dots,X_r]$ est un idéal.

\end{itemize}
 

\subsubsection{Idéal engendré par une partie, sommes d'idéaux}Soit $\mathcal{I}\subset \mathcal{I}_A$ une famille d'idéaux. On vérifie immédiatement que $\cap_{I\in \mathcal{I}}I\subset A$ est idéal. Pour tout sous-ensemble $X\subset A$, il existe 
un unique idéal $\langle\langle X\rangle\rangle_A \subset A$, contenant $X$ et minimal pour $\subset$ \textit{i.e.} tel que pour  tout idéal $I\subset A$,   $X\subset I$ implique  $\langle\langle X\rangle\rangle_A\subset I$. On dit que $\langle\langle X\rangle\rangle_A\subset A$ est l'idéal engendré par $X$. Explicitement $\langle\langle X\rangle\rangle_A$ est l'intersection de tous les idéaux de $A$ contenant $X$. On peut également décrire $\langle\langle X\rangle\rangle_A$ comme  
$$\langle\langle X\rangle\rangle_A=\lbrace \sum_{x\in X}a(x)x\;|\; a \in A^{(X)}\rbrace,$$
ce qui justifie la notation plus intuitive $\langle\langle X\rangle\rangle_A:=\sum_{x\in X}Ax\subset A$. Si  $\mathcal{I}\subset \mathcal{I}_A$ une famille d'idéaux, on note en particulier $$ \langle\langle\; \bigcup_{I\in \mathcal{I}}I\;  \rangle\rangle_A :=\sum_{I\in \mathcal{I}}I\subset A.$$ et on dit que $\sum_{I\in \mathcal{I}}I\subset A$ est la somme des $I$, $I\in \mathcal{I}$. Si  $I=\sum_{x\in X} Ax$, on dit que $X$ est un système de générateurs de $I$ et si on peut prendre $X$ fini, on dit que $I$ est un idéal de type fini\index{de type fini (Idéal)}.\\

\textbf{Exemples}  Les idéaux principaux d'un anneau $A$ sont les idéaux engendrés par les singletons $\lbrace a\rbrace$, $a\in A$. En particulier, dans un anneau principal comme $\Z$ ou $k[X]$, tout idéal est de type fini.  De fa\c{c}on plus surprenante, on verra que tous les idéaux de $k[X_1,\cdots, X_r]$ (et, partant, de toute $k$-algèbre de type fini) sont de type fini. Un anneau ayant cette propriété est dit noetherien. Les anneaux qui ne sont pas de type fini, par exemple $A^\N$, fournissent tautologiquement des idéaux qui ne sont pas de type fini. L'idéal $A^{(\N)}\subset A^\N$ n'est pas de type fini.\\
 

\subsubsection{Produits d'idéaux}Si $I_1,\dots, I_r\subset A$ est une famille finie d'idéaux, on note $I_1\cdots I_r\subset A$ l'idéal engendré par les éléments de la forme $a_1\cdots a_r$, $a_i\in I_i$, $i=1,\dots, r$. On a toujours $$(*)\;\; I_1\cdots I_r\subset \displaystyle{\bigcap_{1\leq i\leq r}I_i}\subset I_i\subset \sum_{1\leq i\leq r} I_i.$$ 

\textbf{Exemple.} Dans $\Z$, on a pour tout $m_1,\dots, m_r\in \Z$, $m_1\Z\cdots m_r\Z=(m_1\cdot m_r)\Z$, $m_1\Z\cap\cdots\cap  m_r\Z=ppcm(m_1,\dots, m_r)\Z$, $m_1\Z+\cdots+m_r\Z=pgcd(m_1,\cdots,m_r)\Z$. Les inclusions $(*)$ ci-dessus correspondent aux relations de divisibilité $$pgcd(m_1,\cdots,m_r)| m_i|ppcm(m_1,\cdots,m_r)| m_1\cdots m_r.$$

\subsubsection{}Si $\phi:A\rightarrow B$ un morphisme d'anneaux, et   $J\subset B$  un idéal alors $\phi^{-1}(J)\subset A$ est un idéal. En particulier,  $\ker(\phi)\subset A$ est un idéal. Si $\phi:A\twoheadrightarrow B$ est surjectif  et   $I\subset A$  est un idéal alors $\phi(I)\subset B$ est un idéal mais montrer par un contre-exemple que ce n'est plus vrai si on ne suppose pas $\phi:A\twoheadrightarrow B$ surjectif. \\

  \subsection{Quotient}\label{Quot}\index{Quotient (anneaux)}Le noyau d'un morphisme d'anneaux $\phi: A\rightarrow B$ est un idéal. Réciproquement, on va voir que tout idéal est le noyau d'un morphisme d'anneaux. En effet, si $A $ est un anneau, un idéal  $I\subset A$ est en particulier un sous-groupe de $(A,+)$. On dispose donc du groupe  quotient $A/I$, qui est un groupe abélien et de la projection canonique $p_I:=\overline{-}:A\twoheadrightarrow A/I$ qui est un morphisme surjectif de groupes, de noyau $I$.   Le groupe quotient $A/I$ est muni d'une unique structure d'anneau  telle que la projection canonique $p_I:=\overline{-}:A\twoheadrightarrow A/I$ est un morphisme d'anneaux. La condition que   $p_I:=\overline{-}:A\twoheadrightarrow A/I$ soit un morphisme d'anneaux impose que $\overline{ab}=\overline{a}\overline{b}$. Il faut donc vérifier que $\overline{ab}$ ne dépend pas du choix des représentants $a$, $b$ de $\overline{a}$, $\overline{b}$.
ou encore que  l'application
  $$\begin{tabular}[t]{lll}
  $A\times A$&$\rightarrow$&$A/I$\\
  $(a,b)$&$\rightarrow$&$\overline{ab}$
  \end{tabular}$$
se factorise en   $$\xymatrix{A\times A\ar[r]^{(a,b)\rightarrow \overline{ab}}\ar@{->>}[d]_{\overline{-}\times \overline{-}}&A/I\\
A/I\times A/I\ar[ur]_{(\overline{a},\overline{b})\rightarrow \overline{a}\cdot\overline{b}:=\overline{ab}}}.$$
Cela résulte de la relation $(a+I)(b+I)=ab+aI+Ib+I^2\subset ab+I$, $a,b\in I$. On vérifie ensuite facilement que $(A/I,+,\cdot)$ ainsi défini vérifie bien les axiomes d'un anneau commutatif de zéro $\overline{0}$ et d'unité $\overline{1}$.     \\
  
 \ref{Quot}.1 \textbf{Lemme.} (Propriété universelle du quotient) \textit{Pour tout idéal $I\subset A$ il existe un morphisme d'anneaux $p:A\rightarrow Q$ tel  que pour tout  morphisme d'anneaux $\phi:A\rightarrow B$ avec $I\subset \ker(\phi)$, il  existe un unique morphisme d'anneaux $\overline{\phi}:Q\rightarrow B$ tel que $\phi=  \overline{\phi}\circ p$.}
 
 \begin{proof} Montrons que  $A/I$ muni de la structure d'anneau ci-dessus et la projection canonique $\overline{-}:A\twoheadrightarrow A/I$ conviennent. Soit $\phi:A\rightarrow B$  un morphisme d'anneaux tel que $I\subset \ker(\phi)$. Si 
 $\overline{\phi}:A/I\rightarrow B$ existe, la condition $\phi=  \overline{\phi}\circ p$ force $\overline{\phi}(\overline{a})=\phi(a)$, $a\in A$. Cela montre l'unicité de $\overline{\phi}$ sous réserve de son existence.   Il reste à voir que  $\overline{\phi}:A/I\rightarrow B$ est automatiquement un morphisme d'anneaux. On sait déjà que c'est un morphisme de groupes additifs, donc il suffit de vérifier la compatibilité au produit. Cela résulte des définitions: $$\overline{\phi}(\overline{a}\overline{b})\stackrel{(1)}{=}\overline{\phi}(\overline{a b})\stackrel{(2)}{=} \phi(ab)\stackrel{(3)}{=} \phi(a)\phi(b)\stackrel{(4)}{=}\overline{\phi}(\overline{a})\overline{\phi}(\overline{b}),$$
 o\`u (1) est par construction du produit sur $A/I$, (2) et (4) est la relation  $\phi=  \overline{\phi}\circ \overline{-}$ et (3) est le fait que $\phi$ est un morphisme d'anneaux. \end{proof}
 
  Comme d'habitude, la $A$-algèbre quotient   $p_I:=\overline{-}:A\twoheadrightarrow A/I$ est unique à unique isomorphisme près.  Par construction $p_I: A\twoheadrightarrow A/I$ est surjectif de noyau $I$. \\
 
   On peut aussi réécrire \ref{Quot}.1 en disant que, pour tout anneau $B$ l'application  canonique 
$$\hbox{\rm Hom}(A/I,B)\rightarrow \lbrace A\stackrel{\phi}{\rightarrow}B\; |\; I\subset \ker(\phi)  \rbrace,\;\overline{\phi}\rightarrow \overline{\phi}\circ\overline{(-)}  $$
est bijective  ou encore, plus visuellement:

$$\xymatrix{I\ar[r]\ar@/^1.5pc/[rr]^{0}&A\ar[r]^{\phi}\ar[d]_{\overline{(-)}}&B\\
&A/I\ar@{.>}[ur]_{\exists ! \overline{\phi}}&}$$
  
   En particulier, tout  morphisme d'anneaux $\phi:A\rightarrow B$ se décompose de fa\c{c}on canonique sous la forme 
  $$\xymatrix{A\ar@{->>}[r]^{\phi|^{\hbox{\rm \small im}(\phi)}}\ar@{->>}[d]_{\overline{-}}&\im(\phi)\ar@{^{(}->}[r]&B\\
  A/\ker(\phi)\ar[ur]^\simeq_{\overline{\phi}}&&}$$
  
  \textbf{Exemples.}\index{Caractéristique} (Caractéristique d'un anneau) Le noyau du morphisme caractéristique $c_A:\Z\rightarrow A$ est un idéal de $\Z$ donc de la forme $\ker(c_A)=n\Z$ pour un unique entier $n\geq 0$, appelé la caractéristique de $A$. 
  \begin{itemize}[leftmargin=* ,parsep=0cm,itemsep=0cm,topsep=0cm]
\item $\Z,\Q,\R,\C$ sont de caractéristique $0$;
\item $\Z/n$ est de caractéristique $n$, $n\geq 0$;
\item Si $A'\subset A$ est un sous-anneau, $A$ et $A'$ ont même caractéristique. En particulier $A$, $A^I$, $A[X]$ ont même caractéristique. Si $\mathcal{P}$ est un ensemble infini de nombres premiers distincts, l'anneau produit $\prod_{p\in \mathcal{P}}\Z/p$ est de caractéristique $0$.
\item  Si $\phi : A\rightarrow B$ est une $A$-algèbre, la caractéristique de $B$ divise la caractéristique de $A$.\\
\end{itemize}
  
  
   \textbf{Exercices.}  
     \begin{enumerate}[leftmargin=* ,parsep=0cm,itemsep=0cm,topsep=0cm]
     \item Soit  $I,J\subset A$ des idéaux; notons $\overline{A}:=A/I$ et $\overline{J}:=P_I(J)$. Montrer que si $I\subset J$, on a  un isomorphisme d'anneaux canonique $A/J\tilde{\rightarrow} \overline{A}/\overline{J}$. En déduire qu'on a toujours un isomorphisme d'anneaux canonique $A/(I+J)\tilde{\rightarrow}  \overline{A}/\overline{J}$.\\    
\item Soit $I\subset A$ un idéal. Montrer qu'on a un isomorphisme de $A$-algèbres canonique $A[X]/I[X]\tilde{\rightarrow} (A/I)[X]$.\\
  
\item Soit $f_1,\dots f_s\in A[X_1,\dots, X_r]$. Montrer que la $A$-algèbre quotient $$A\rightarrow A[X_1,\dots, X_n]/\sum_{1\leq i\leq s}f_iA[X_1,\dots, X_r]$$ munie des images $\overline{X}_1,\cdots, \overline{X}_r$ de $X_1,\dots, X_r$ vérifie la propriété universelle suivante.\\

 (Propriété universelle de $A\rightarrow A[X_1,\dots, X_n]/\sum_{1\leq i\leq s}f_iA[X_1,\dots, X_r]$) Il existe une $A$-algèbre $ A\rightarrow \overline{P}$ munie d'éléments $\overline{p}_1,\dots, \overline{p}_r\in \overline{P}$ tels que pour tout $A$-algèbre $\phi:A\rightarrow B$ et $b_1,\dots, b_r\in B$ vérifiant $ev_{\underline{b}}^\phi(f_i)=0$, $i=1,\dots ,s$ il existe un unique morphisme de $A$-algèbre $\overline{ev}_{\underline{b}}^\phi:\overline{P}\rightarrow B$ tel que $\overline{ev}_{\underline{b}}^\phi(\overline{p}_i)=b_i$, $i=1,\dots, r$.\\
 
\item Montrer qu'on a un isomorphisme de $A$-algèbres canonique $$A[X_1,Y_1,\dots, X_r,Y_r]/\sum_{1\leq i\leq r}(X_iY_i-1)A[X_1,Y_1,\dots, X_r,Y_r]\tilde{\rightarrow} A[X_1,X_1^{-1},\dots,X_r,X_r^{-1}].$$
  
\end{enumerate}

   
    
   \ref{Quot}.2 \textbf{Lemme.} \textit{Soit $I\subset A$ un idéal. La projection canonique $p_I :A\twoheadrightarrow A/I$ induit une bijection d'ensembles ordonnés $p_I:(V^{tot}(I),\subset)\tilde{\rightarrow}(\mathcal{I}_{A/I} ,\subset)$.}
   
   \begin{proof} Le fait que $p_I: V^{tot}(I)  \rightarrow \mathcal{I}_{A/I} $ préserve l'inclusion est immédiat. Pour montrer que c'est une bijection, il suffit d'exhiber l'application inverse. Comme $\ker(p_I)=I$, $p_I^{-1}:\mathcal{I}_{A/I}\rightarrow \mathcal{I}_{A}$ est à valeur dans $V^{tot}(I)$ donc induit une application $p_I^{-1}:\mathcal{I}_{A/I}\rightarrow V^{tot}(I)$;  vérifions que celle-ci convient. Comme $p_I :A\twoheadrightarrow A/I$ est surjective, on a toujours $p_I\circ p_I^{-1}(\overline{J})=\overline{J}$, $\overline{J}\in \mathcal{I}_{A/I}$. Inversement, si $J\in \mathcal{I}_{A} $, on a $ p_I^{-1}\circ p_I(J)=I+J$ donc, si on suppose de plus $I\subset J$, on a $ p_I^{-1}\circ p_I(J)=I+J=J$.  \end{proof}
   
    Soit $I_1,\dots, I_r\subset A$ des idéaux et considérons le produit des projections canoniques $p:=\prod_{1\leq i\leq r}p_{I_i}:A\rightarrow \prod_{1\leq i\leq r}A/I_i$; c'est un morphisme d'anneaux de noyau $\cap_{1\leq i\leq r}I_i$. De plus\\

  
     \ref{Quot}.3 \textbf{Lemme.} (Restes chinois) \textit{Si $I_i+I_j=A$, $1\leq i\not=j\leq r$ alors $\cap_{1\leq i\leq r}I_i=I_1\cdots I_r$ et  $p :A\rightarrow \prod_{1\leq i\leq r}A/I_i$ est surjective. Inversement, si   $p :A\rightarrow \prod_{1\leq i\leq r}A/I_i$ est surjective alors $I_i+I_j=A$, $1\leq i\not=j\leq r$.}
   
   \begin{proof} Supposons d'abord que $I_i+I_j=A$, $1\leq i\not=j\leq r$.  On a toujours $\cap_{1\leq i\leq r}I_i\supset I_1\cdots I_r$. Pour l'inclusion inverse et la surjectivité de  $p:=\prod_{1\leq i\leq r}p_{I_i}:A\rightarrow \prod_{1\leq i\leq r}A/I_i$, on procède par récurrence sur $r$. Si $r=2$, il existe $a_i\in I_i$, $i=1,2$ tels que $1=a_1+ a_2$. En particulier, 
   \begin{itemize}[leftmargin=* ,parsep=0cm,itemsep=0cm,topsep=0cm]
   \item  Pour tout $x\in I_1\cap I_2$, $x=x 1=x  (a_1+ a_2)=xa_1+xa_2=a_1x+xa_2\in I_1\cdot I_2$.
   \item Soit $x_1,x_2\in A$ arbitraires. En posant $x=a_1x_2+a_2x_1$ on a bien $p_{I_1}(x)=p_{I_1}(a_2)p_{I_1}(x_1)= p_{I_1}(x_1)$ et $p_{I_2}(x)=p_{I_2}(a_1)p_{I_2}(x_2)= p_{I_2}(x_2)$.
   \end{itemize}
  Si $r\geq 3$,  on a par hypothèse de récurrence $I_2\cap\cdots \cap I_{r }=I_2\cdots I_{r}$ et   $A/(I_2\cap \cdots\cap I_r)\twoheadrightarrow \prod_{2\leq i\leq r}A/I_i$. Il suffit de montrer que $I_1+I_2\cdots I_r=A$. En effet, le cas $r=2$ (et l'hypothèse de récurrence) nous donnera alors
     \begin{itemize}
   \item  $I_1\cap (I_2\cap\cdots \cap I_{r })=I_1\cap (I_1\cdots I_r)=I_1\cdot (I_2\cdots I_r)= I_1\cdots I_r$.
   \item $A\twoheadrightarrow A/I_1\times A/(I_2\cap\cdots\cap  I_r)\twoheadrightarrow A/I_1\times\prod_{2\leq i\leq r}A/I_i\twoheadrightarrow  \prod_{1\leq i\leq r}A/I_i$
   \end{itemize}
    Mais pour $i=2,\dots, r$ il existe $a_i\in I_1$, $b_i\in I_i$ tels que $a_i+b_i=1$. On a donc $1=\prod_{2\leq i\leq r}(a_i+b_i)=\prod_{2\leq i\leq r}a_i+\cdots\in I_1+I_2\cdots I_r$.\\
     Inversement, si $p :A\rightarrow \prod_{1\leq i\leq r}A/I_i$ est surjective, pour tout $1\leq i\not=j\leq r$, il existe $x\in A$ tel que $p(x)=(\delta_{i,k})_{1\leq k\leq r}\in \prod_{1\leq i\leq r}A/I_i$ \textit{i.e.} $x\in 1+I_i$ et $x\in I_j$. Donc $1=(1-x)+x\in I_i+I_j$. \end{proof}
  


\subsection{Corps et idéaux maximaux} \label{Corps}Le singleton $\lbrace 0\rbrace$ et $A$ sont des ideaux de $A$. En général, un anneau contient beaucoup d'idéaux. L'ensemble des idéaux et leur `position' dans l'anneau mesure la complexité de celui-ci.  En ce sens, les anneaux les plus simples sont les corps. \\

\ref{Corps}.1 \textbf{Lemme.} \textit{Les PSSE:
\begin{tabular}[t]{ll}
(i)& $A$ est un corps;\\
(ii)& Les seuls idéaux de $A$ sont $\lbrace 0\rbrace$ et $A$.
\end{tabular}}
\begin{proof}Si $A$ est un corps,  tout idéal $\lbrace 0\rbrace\subsetneq I\subset A$ contient un élément $a\not= 0$ donc inversible. Mais alors $1=a^{-1}a\in AI=I$ donc $A=A1\subset AI=I$. Inversement, si les seuls idéaux de $A$ sont $\lbrace 0\rbrace$ et $A$, pour tout $a\not=0$, $\lbrace 0\rbrace\subsetneq Aa\subset A$ est un idéal donc $Aa=A$. En particulier $1\in Aa$ \textit{i.e.} il existe $a^{-1}\in A$ tel que $1=a^{-1}a$.\end{proof}

\ref{Corps}.2 \textbf{Lemme.} \textit{Soit $I\subsetneq A$ un idéal. Les PSSE
\begin{tabular}[t]{ll}
(i)&  $A/I$ est un corps;\\
(ii)& $I$ est maximal  dans  $(\mathcal{I}_A\setminus \lbrace 0\rbrace,\subset)$.  
\end{tabular}}
\begin{proof} Cela résulte de \ref{Quot}.2.  \end{proof}

 On dit qu'un idéal qui vérifie les propriétés (i), (ii) de \ref{Corps}.2 est \textit{maximal}\index{Maximal (Idéal)}.  \\

% Un ensemble ordonné est dit inductif si toute suite croissante admet un majorant. Le Lemme de Zorn (qui est équivalent à l'axiome du choix dénombrable) affirme que tout ensemble non vide ordonné iductif admet un élément maximal.\\ 

\ref{Corps}.3 \textbf{Lemme.} [Utilise le Lemme de Zorn] \textit{L'ensemble ordonné  $(\mathcal{I}_A\setminus \lbrace A\rbrace,\subset)$ est (non-vide; il contient $\lbrace 0\rbrace$) inductif. En particulier, tout idéal $I\subsetneq A$ est contenu dans un idéal maximal.}
\begin{proof} Il suffit d'observer que si $I_1\subset I_2\subset\cdots \subsetneq A$ est une suite d'ideaux de $A$ distincts de $A$ et croissante  pour $\subset$, $I:=\cup_{n\geq1}I_n\subsetneq A$ est encore un idéal de $A$ distincts de $A$. En effet, pour tout $a,b\in I$ il existe $n$ tel que $a,b\in I_n$ donc $a-b\in I_n\subset I$ et pour tout $\alpha\in A$, $\alpha a\in I_n\subset I$; cela montre déja que $I\subset A$ est un idéal. Dans ce cas, $I= A$ si et seulement si $1\in I$. Mais si $1\in I$, il existerait $n\geq 1$ tel que $1\in I_n$, ce qui n'est pas possible puisque par hypothèse $I_n\subsetneq A$. \end{proof}

 En particulier, pour tout $a\in A$, $a\notin A^\times$ $\Leftrightarrow$ $Aa\subsetneq A$  $\Leftrightarrow$ $a$ est contenu dans au moins un  idéal maximal de $A$.\\




 On notera $spm(A)$ l'ensemble des idéaux maximaux de $A$ et on dit que c'est le \textit{spectre maximal}\index{Spectre maximal} de $A$. D'après \ref{Produit}.1, les projections canoniques $p_{\frak{m}}:A\twoheadrightarrow A/\frak{m}$, $\frak{m}\in spm(A)$ induisent un morphisme d'anneaux canonique 

$$p_{max}:A\rightarrow \prod_{\frak{m}\in spm(A)}A/\frak{m}$$
dont le noyau  $\mathcal{J}_A:=\ker(p_{max})=\displaystyle{\bigcap_{\frak{m}\in spm(A)}}\frak{m}\subset A$ est un idéal  appelé \textit{radical de Jacobson}\index{Radical de Jacobson} de $A$.\\

\textbf{Exercice.} Soit $a\in A$. Montrer que $a\in \mathcal{J}_A$ si et seulement si $1-ab\in A^\times$, $b\in A$. \\

 


% Un anneau $A$  qui possède  un unique idéal maximal $\frak{m} (= \mathcal{J}_A)$ est dit \textit{local} et on dit que $A/\frak{m}$ est le corps résiduel de $A$. Un anneau $A$ est local si et seulement si  $A\setminus A^\times$ est un idéal, auquel cas $A\setminus A^\times$ est l'unique idéal maximal de $A$. \\


  \subsubsection{Anneaux intègres et idéaux premiers}\label{Integre} On dit qu'un élément $t\in A$ est de torsion (ou est un diviseur de zéro)\index{Diviseur de zéro}\index{Torsion} s'il existe $0\not= a\in A$ tel que $at=0$. On notera $A_{tors}\subset A$ l'ensemble des éléments de torsion de $A$. On dit qu'un anneau $A$ est \textit{intègre}\index{Intègre (Anneau)} si $A_{tors}=\lbrace 0\rbrace $.\\
  
 \textbf{Exemples.}
   \begin{itemize}[leftmargin=* ,parsep=0cm,itemsep=0cm,topsep=0cm]
     \item Les corps  sont intègres, $\Z$ est intègre.
     \item Tout sous-anneau d'un anneau intègre est intègre. Si $A$ est un anneau intègre, $A[X]$ est intègre. Par contre, le produit $A_1\times A_2$ de deux anneaux non nuls n'est jamais intègre.
     \item $\Z/n$ est intègre si et seulement si $n$ est un nombre premier. \\
\end{itemize}
 
 
  \textbf{Remarque.} Pour tout $a\in A\setminus A_{tors}$ et pour tout $b,c\in A$ on a $ab=ac\Leftrightarrow a(b-c)=\Leftrightarrow b-c=0$. Autrement dit, `on peut simplifier par $a$'. En particulier, si $A$ est intègre, on peut simplifier par tout élément $a\not=0$.\\
  
  \ref{Integre}.1 \textbf{Lemme.} \textit{Soit $I\subsetneq A$ un idéal. Les PSSE
\begin{tabular}[t]{ll}
(i)&  $A/I$ est intègre;\\
(ii)& Pour tout $a,b\in A$, $ab\in I$ $\Rightarrow$ $a\in I$ ou $b\in I$.  
\end{tabular}}
\begin{proof} (i) $\Rightarrow$ (ii): Si $ab\in I$ alors $\overline{a}\overline{b}=0$ dans $A/I$. Par (i), on a forcément $\overline{a}=0$ (\textit{i.e.} $a\in I$) ou $\overline{b}=0$ (\textit{i.e.} $b\in I$) dans $A/I$. (ii) $\Rightarrow$ (i): Pour tout $0\not= \overline{a}, \overline{b}\in A/I$, choisissons $a,b\in A$ relevant $\overline{a}, \overline{b}\in A/I$. On a forcément $a,b\notin I$ donc, par (ii), $an\notin I$ \textit{i.e.} $\overline{a}\overline{b}=\overline{ab}\not= 0$ in $A/I$.  \end{proof}


 On dit qu'un idéal qui vérifie les propriétés (i), (ii) de \ref{Corps}.2 est \textit{premier}\index{Premier (Idéal)}.  On notera $spec(A)$ l'ensemble des idéaux premiers de $A$ et on dit que c'est le \textit{spectre}\index{Spectre} de $A$. D'après \ref{Produit}.1, les projections canoniques $p_{\frak{p}}:A\twoheadrightarrow A/\frak{p}$, $\frak{p}\in spec(A)$ induisent un morphisme d'anneaux canonique 
$$p_{prem}:A\rightarrow \prod_{\frak{p}\in spec(A)}A/\frak{p}$$
dont le noyau $\mathcal{R}_A:=\ker(p_{prem})=\displaystyle{\bigcap_{\frak{p}\in spec(A)}}\frak{p}\subset A$ est un idéal appelé \textit{radical}\index{Radical} de $A$.\\

 On dit qu'un élément $a\in A$ est \textit{nilpotent}\index{Nilpotent} s'il existe un entier $n\geq 1$ tel que $a^n=0$ et, si $a\not=0$, on dit que le plus petit entier $n\geq 1$ tel que $a^{n-1}\not=0$ et $a^n=0$ est l'indice de nilpotence de $a$ (on dit parfois que $0$ est d'indice de nilpotence $1$). On note $\mathcal{N}_A\subset A$ l'ensemble des éléments nilpotents de $A$. On a évidemment $\mathcal{N}_A\subset A_{tors}$ donc, en particulier, si $A$ est un anneau intègre, $\mathcal{N}_A=\lbrace 0\rbrace$. \\

  \ref{Integre}.2 \textbf{Proposition.}  [Utilise le Lemme de Zorn]  \textit{$\mathcal{N}_A\subset A$ est un idéal et $\mathcal{N}_A=\mathcal{R}_A$.}

\begin{proof} Vérifions d'abord que $\mathcal{N}_A\subset A$ est un idéal. Pour tout $a,b \in \mathcal{N}_A$, il existe des entiers $m,n\geq 1$ tel que $a^m=b^n=0$. Donc, par la formule du binôme de Newton $$(a-b)^{m+n-1}=\sum_{0\leq k\leq m+n-1}\binom{k}{m+n-1}(-1)^{m+n-k-1}a^kb^{m+n-k}=0$$
puisque, si $k<m$, $m+n-k-1>n-1$ donc $m+n-k-1\geq n$.  On a aussi pour tout $\alpha\in A$ $(\alpha a)^m=\alpha^m a^m=0$.\\
\indent Pour tout morphime d'anneaux $\phi:A\rightarrow B$ on a $\phi(\mathcal{N}_A)\subset \mathcal{N}_B$. En particulier, si $B$ est un anneau intègre, $\mathcal{N}_A\subset \ker(\phi)$. En appliquant cette observation aux projections canoniques $p_\frak{p}:A\twoheadrightarrow A/\frak{p}$, $\frak{p}\in spec(A)$, on en déduit l'inclusion $\mathcal{N}_A\subset \mathcal{R}_A$. Inversement, soit $a\notin \mathcal{N}_A$; on veut montrer que $a\notin \mathcal{R}_A$ \textit{i.e.} il existe $\frak{p}\in spec(A)$ tel que $a\notin\frak{p}$ (ce qui équivaut aussi à $a^n\notin \frak{p}$ pour n'importe quel entier $n\geq 1$). Notons $X_a:=\lbrace a^n\;|\; n\in \Z_{\geq 1}\rbrace $ l'ensemble des puissances de $a$. On a par hypothèse $0\notin X_a$ donc l'ensemble $\Sigma_a\subset \mathcal{I}_A$ des idéaux $I\subset A$ tels que $X_a\cap I=\emptyset$ est non-vide puisqu'il contient $\lbrace 0\rbrace$. On vérifie immédiatement que $(\Sigma_a,\subset)$ est ordonné inductif donc, par le Lemme de Zorn, possède un élément maximal $I\in \Sigma_a$. Puisque  $a\notin I$, il suffit de montrer que $I$ est premier \textit{i.e.} que $A/I$ est intègre. Notons $\overline{a}$  l'image de $a$ dans $A/I$. Par définition de $I$, $0\notin X_{\overline{a}}$ mais  pour tout idéal $\lbrace 0\rbrace\subsetneq \overline{J}\subset A/I$, $X_{\overline{a}}\cap \overline{J}\not=\emptyset$. En particulier, pour tout $0\not=\overline{b}\in A/I$, il existe $n_b\geq 1$ tel que $\overline{a}^{n_b}\in (A/I)\overline{b}$ donc pour tout $0\not=\overline{b}, \overline{b'}\in A/I$, $\overline{a}^{n_bn_{b'}}\in (A/I)\overline{b} \overline{b'}$ donc  $\overline{b} \overline{b'}\not= 0$. \end{proof}

\textbf{Exercice.}
     \begin{itemize}[leftmargin=* ,parsep=0cm,itemsep=0cm,topsep=0cm]
     \item Montrer que si $a\in A$ est nilpotent, $1+a\in A^\times$. En déduire que la somme d'un élément nilpotent et d'un élément inversible est encore inversible.
     \item Montrer que $A[X]^\times$ est l'ensemble des polynômes $P=\sum_{n\geq 0}a_nX^n$ tels que $a_0\in A^\times$ et $a_n$ est nilpotent, $n\geq 1$. Déterminer $A[X_1,\dots, X_r]^\times$. \\
\end{itemize}
 

\ref{Integre}.3 \textbf{Exercice.}\begin{enumerate}[leftmargin=* ,parsep=0cm,itemsep=0cm,topsep=0cm]
\item Soit $\frak{p}_1,\dots,\frak{p}_r$ des idéaux premiers et $I\subset A$ un idéal. Si $I\subset \cup_{1\leq i\leq r}\frak{p}_i$ il existe $1\leq i\leq r$ tel que $I\subset \frak{p}_i$;
\item Soit $I_1,\dots, I_r$ des idéaux   et $\frak{p}\subset A$ un idéal premier. Si $\frak{p}\supset \cap_{1\leq i\leq r}I_i$ il existe $1\leq i\leq r$ tel que $\frak{p}\supset I_i$.\\
\end{enumerate}

 


%\textbf{Lemme.} \textit{\begin{enumerate}[leftmargin=* ,parsep=0cm,itemsep=0cm,topsep=0cm]
%\item Soit $\frak{p}_1,\dots,\frak{p}_r$ des idéaux premiers et $I\subset A$ un idéal. Si $I\subset \cup_{1\leq i\leq r}\frak{p}_i$ il existe $1\leq i\leq r$ tel que $I\subset \frak{p}_i$;
%\item Soit $I_1,\dots, I_r$ des idéaux   et $\frak{p}\subset A$ un idéal premier. Si $\frak{p}\supset \cap_{1\leq i\leq r}I_i$ il existe $1\leq i\leq r$ tel que $\frak{p}\supset I_i$;
%\end{enumerate}}
%\begin{proof} (1) On procède par réccurrence sur $r$. Si $ r=1$, c'est tautologique. Supposons $r\geq 2$ et   $I\not\subset \frak{p}_i$, $i=1,\dots ,r$. Par hypothèse de récurrence, pour $i=1,\dots, r$ on peut trouver $x_i\in I$, $x_i\notin \cup_{1\leq i\not=j\leq r}\frak{p}_j$. S'il existe $1\leq i\leq r$ tel que $x_i\notin\frak{p}_i$, on a gagné. Sinon, $$x:=\sum_{1\leq i\leq r}x_1\cdots x_{i-1}x_{i+1}\cdots x_r$$
%est bien dans $I$ par construction mais pas dans $\frak{p}_i$ car $\frak{p}_i$ est premier, $i=1\dots, r$.\\
% (2) Supposons $\frak{p}\not\supset I_i$, $i=1,\dots, r$. Pour   $i=1,\dots, r$ il existe $x_i\in I_i$, $x_i\not\in\frak{p}$. L'élément $x=x_1\cdots x_r$ est dans $I_1\cdots I_r\subset \cap_{1\leq i\leq r}I_i$ mais, comme $\frak{p}$ est premier, $x\notin\frak{p}$.\end{proof}

  \subsubsection{Anneaux réduits et idéaux radiciels}\label{Reduit} On dit qu'un anneau $A$ est \textit{réduit}\index{Réduit (Anneau)} si  $\mathcal{R}_A=\mathcal{N}_A=0$. \\
  
  \textbf{Exemples.} Les anneaux intègres sont réduits, l'anneau $\Z\times\Z$ est réduit non-intègre. Si $p$ est un nombre premier l'anneau  $\Z/p^n$ n'est  pas réduit et contient un élément d'indice de nilpotence $n$, $n\geq 1$. Si on note $p_n$ le nième nombre premier, l'anneau $\prod_{n\geq 1}\Z/p_n^n $    n'est  pas réduit et contient un élément d'indice de nilpotence $n$ pour tout $n\geq 1$.\\
  
  
  
   Pour un idéal $I\subset A$, on note $\sqrt{I}:=p_I^{-1}(\mathcal{N}_{A/I})$. Par définition, $$I\subset \sqrt{I}=\bigcup_{n\geq 1}\lbrace a\in A\;|\; a^n\in I\rbrace.$$
  On dit que $\sqrt{I}$ est la racine de $A$. Avec cette notation, $\mathcal{N}_A=\sqrt{\lbrace 0\rbrace}$.  Il résulte des définitions que pour un idéal $I\subsetneq A$ les PSSE
\begin{tabular}[t]{ll}
(i)&  $A/I$ est réduit;\\
(ii)& $I=\sqrt{I}$.  \\
\end{tabular} 

 On dit qu'un idéal $I\subsetneq A$ qui vérifie les propriétés (i), (ii) ci-dessus est \textit{radiciel}\index{Radiciel (Idéal)}. On notera $\mathcal{I}_A^{red}$ l'ensemble des idéaux radiciels de $A$.  \\

 En résumé on a 
$$\hbox{\rm Maximal}\; \Rightarrow\;\hbox{\rm Premier}\; \Rightarrow\;\hbox{\rm  Radicie\l};\;i.e.\;\; spm(A)\subset spec(A)\subset \mathcal{I}_A^{red}$$
et $$\begin{tabular}[t]{l|l}
$I$&$A/I$\\
\hline\\
Maximal&Corps\\
Premier&Intègre\\
Radiciel&Réduit
\end{tabular}$$
$$\hbox{\rm\small Classification grossière des idéaux}$$
\subsubsection{} Tout morphisme $\phi:A\rightarrow B$  d'anneaux commutatifs induit une application $\phi^{-1}: (\mathcal{I}_B,\subset)\rightarrow (\mathcal{I}_A,\subset)$ préservant $\subset$.  De plus, si $I\in \mathcal{I}_B$, le noyau de 
$A\stackrel{\phi}{\rightarrow}B\stackrel{p_I}{\twoheadrightarrow}B/I$ est $\phi^{-1}(I)$, d'o\`u un morphisme d'anneaux injectifs $A/\phi^{-1}(I)\hookrightarrow B/I$. Comme un sous-anneau d'un anneau intègre (resp. réduit) est intègre (resp. réduit), on en déduit que  $\phi^{-1}: (\mathcal{I}_B,\subset)\rightarrow (\mathcal{I}_A,\subset)$ se restreint en des applications  
 $$\begin{tabular}[t]{ccc}
 $(\mathcal{I}_B,\subset)$&$\stackrel{ \phi^{-1}}{\rightarrow}$&$ (\mathcal{I}_A,\subset)$\\
$\displaystyle{\bigcup}$ &&$\displaystyle{\bigcup}$\\
$(\mathcal{I}_B^{red},\subset)$&$\stackrel{ \phi^{-1}}{\rightarrow}$&$ (\mathcal{I}_A^{red},\subset)$\\
$\displaystyle{\bigcup}$ &&$\displaystyle{\bigcup}$\\
$(spec(B),\subset)$&$\stackrel{ \phi^{-1}}{\rightarrow}$&$ (spec(A),\subset)$\\
 \end{tabular}$$
 
  Il n'est par contre pas vrai qu'un sous-anneau d'un corps est un corps (\textit{e.g.} $\Z\subset \Q$) donc l'image inverse d'un idéal maximal par un morphisme d'anneau n'est, en général, pas maximal. 
   \section{Anneaux noetheriens}\label{AnneauNoetherien}
   \subsection{Lemme}\label{NoethDef} \textit{Soit $A$ un anneau. Les PSSE.
       \begin{enumerate}[leftmargin=* ,parsep=0cm,itemsep=0cm,topsep=0cm]
     \item  Tout idéal $I\subset A$ est de type fini.
     \item  Toute suite  d'idéaux de $A$ croissante pour $\subset$ est stationnaire à partir d'un certain rang.
     \item Tout sous-ensemble non vide d'idéaux de $A$ admet un élément maximal pour $\subset $.
\end{enumerate}}
\begin{proof} (1) $\Rightarrow$ (2). Supposons que tous les idéaux de $A$ sont de type fini. Soit $I_0\subset\cdots\subset I_n\subset I_{n+1}\subset \cdots\subset A$ une suite croissante d'idéaux pour $\subset $. L'ensemble $I:=\cup_{n\geq 0}I_n\subset A$ est un idéal; il existe donc un ensemble fini $X\subset A$ tels que $I=\sum_{x\in X}Ax$. Mais pour chaque $x\in X$, il existe $n_x\geq 0$ tel que $x\in I_{n_x}$. Donc avec $n:=\hbox{\rm max}\lbrace n_x\;|\; x\in X\rbrace$, on a $X\subset I_n$ donc $I\subset I_n$. \\
 (2) $\Rightarrow$ (3). Soit $\mathcal{I}\subset \mathcal{I}_A$ un sous-ensemble non-vide. Supposons que $\mathcal{I}$ n'admette pas d'élément maximal pour $\subset$. Soit $I_0\in \mathcal{I}$. Puisque $I_0$ n'est pas maximal pour $\subset$, on peut  trouver $I_1\in \mathcal{I}$ tel que $I_0\subsetneq I_1$. En réitérant l'argument on construit une suite strictement croissante $I_0\subsetneq I_1\subsetneq I_2\subsetneq\cdots\subsetneq I_n\subsetneq I_{n+1}\subsetneq \cdots$ d'élément de $\mathcal{I}$, ce qui contredit (1).\\
 (3) $\Rightarrow$ (1). Soit $I\subset A$ un idéal. Notons $\mathcal{I}\subset \mathcal{I}_A$ le sous-ensemble des idéaux de type fini de $A$ contenu dans $I$. $\mathcal{I}$ est non-vide puisqu'il contient $\lbrace 0\rbrace$. Par (3), il admet donc un élément $I^\circ$ maximal pour $\subset$. Si $I^\circ\subsetneq I$, il existe $a\in I$ tel que $I^\circ\subsetneq I^\circ+Aa\subset I$. Par construction $I^\circ+Aa$ est de type fini, ce qui contredit la maximalité de $I$.\end{proof}
    
     On dit qu'un anneau $A$ qui vérifie les propriétés équivalente du Lemme \ref{NoethDef} est \textit{noetherien}\index{Noetherien (Anneau)}.
    
    \subsection{Exemples}\label{NoethEx}
      \begin{enumerate}[leftmargin=* ,parsep=0cm,itemsep=0cm,topsep=0cm]
     \item  Les anneaux principaux (\textit{e.g.} $k$, $\Z$, $k[X]$, o\`u $k$ est un corps commutatif) sont noetheriens. 
     \item  Si $k$ est un corps commutatif, une  $k$-algèbre $\phi:k\rightarrow A$ est toujours munie d'une structure de $k$-espace vectoriel: $k\times A\rightarrow A$, $(\lambda,a)\rightarrow \phi(\lambda)a$. Avec cette structure de $k$-espace vectoriel, les idéaux de $A$ sont automatiquement des sous-$k$-espace vectoriel. Si $A$ est de dimension finie sur $k$, elle est donc noetherienne. Par exemple l'anneau $k[X]/X^nk[X]$ est un  noetherien. 
     \item  Tout quotient d'un anneau noetherien est noetherien. En effet, soit $A$ est un anneau noetherien et $I\subset A$  un idéal; notons $p_I:A\twoheadrightarrow A/I$ la projection canonique. Si $J\subset A/I$ est un idéal, $p_I^{-1}(J)\subset A$ est un idéal donc, en particulier, il est engendré par un nombre fini $a_1,\dots, a_r$ d'éléments. Mais alors, $J=p_Ip_I^{-1}(J)$ est engendré par les  $p_I(a_1),\dots, p_I(a_r)$.
     \item Par contre un sous-anneau d'un anneau noetherien n'est pas forcément noetherien. Par exemple, on va voir (\ref{NoethTransfert}) que si $k$ est un corps commutatif, l'anneau $k[X_1,X_2]$ est noetherien mais la sous-$k$-algèbre engendrée par les $  X_1X_2^n$, $n\geq 0$ n'est pas un anneau noetherien.\\
\end{enumerate}
    
 La proposition suivante et son corollaire fournissent un très grand nombre d'exemples d'anneaux noetheriens.   
    
\subsection{Proposition}\label{NoethTransfert} (Transfert de noetherianité) \textit{$A$ noetherien $\Rightarrow$ $A[X]$ noetherien.}
\begin{proof}Soit $I\subset A[X]$ un idéal. Pour chaque $n\geq 0$ définissons $\frak{I}_n\setminus \lbrace 0\rbrace\subset A$ comme l'ensemble des $a\in A$ qui apparaissent comme coefficient dominant  d'un polynôme de degré $n$ dand $I$ \textit{i.e.} $a\in \frak{I}_n$ si et seulement si il existe $a_0+a_1X+\cdots+a_{n-1}X^{n-1}+aX^n\in I$. Comme $I\subset A[X]$ est un idéal, les $\frak{I}_n\subset A$ sont automatiquement des idéaux. De plus,  $$a_0+a_1X+\cdots+a_{n-1}X^{n-1}+aX^n\in I\Rightarrow a_0X+a_1X^2+\cdots+a_{n-1}X^{n}+aX^{n+1}\in I$$ donc on a 
$$\frak{I}_0\subset \frak{I}_1\subset \cdots\subset \frak{I}_n\subset \frak{I}_{n+1}\subset \cdots$$
Comme $A$ est noetherien, cette suite devient stationnaire à partir d'un certain rang, disons $n$. De plus,  chaque $\frak{I}_k$ est de type fini; notons $a_{k,1},\dots, a_{k,r_k}\in \frak{I}_k$ un ensemble fini de générateurs de $\frak{I}_k$ . Enfin,  pour  $k=0,\dots, n$, $l=1,\dots, r_k$, fixons un polynôme $P_{k,l}\in I$ de degré $k$ et de coefficient dominant $a_{k,l}$. Il suffit de montrer que $I$ est engendré par les $P_{k,l}$,  $l=1,\dots, r_k$, $k=0,\dots, n$. Notons donc $I^\circ:=\sum AP_{k,l}\subset I$ et montrons par induction sur le degré $d$ de $P\in I$ que $P\in I^\circ$. Si $d=0$, on a par définition $  \frak{I}_0\subset I^\circ$. Supposons que $I^\circ$ contient tous les   éléments de $I$ de degré $\leq d$. Soit $P=a_0+\cdots+a_dX^d+a_{d+1}X^{d+1}\in I$ de degré $d+1$. Si $d+1\geq n$, on a $a_{d+1}\in \frak{I}_{d+1}=\frak{I}_n$ donc on peut écrire $a_{d+1}=\sum_{1\leq i\leq r_n}\alpha_ia_{n,i}$ et $P-\sum_{1\leq i\leq r_n}\alpha_iX^{d+1-n}P_{n,i}$ est encore dans $I$ mais de degré $\leq d$ donc, par hypothèse de récurrence, dans $I^\circ$. Si $d+1\leq n$, $a_{d+1}\in \frak{I}_{d+1}$ donc on peut écrire $a_{d+1}=\sum_{1\leq i\leq r_n}\alpha_ia_{d+1,i}$ et $P-\sum_{1\leq i\leq r_n}\alpha_i P_{d+1,i}$ est encore dans $I$ mais de degré $\leq d$ donc, par hypothèse de récurrence, dans $I^\circ$. 
\end{proof}


\subsection{Corollaire}\label{NoethTransfertCor} \textit{Si $A$ est un anneau noetherien, toute $A$-algèbre de type fini est un anneau noetherien.}
\begin{proof} Observons d'abord qu'en raisonnant par induction sur $n\geq 1$,   l'isomorphisme $$A[X_1,\dots, X_n]\tilde{\rightarrow} A[X_1,\cdots,X_{n-1}] [X_n]$$ et la Proposition \ref{NoethTransfert} impliquent que $A[X_1,\dots, X_n]$ est un anneau noetherien. On conclut oar l'Exemple \ref{NoethEx} (3) puisque toute $A$-algèbre de type fini  est quotient d'une $A$-algèbre de la forme $A[X_1,\dots, X_n]$. \end{proof}

\subsection{Exercices}\label{NoethExercices}
    \begin{enumerate}[leftmargin=* ,parsep=0cm,itemsep=0cm,topsep=0cm]
     \item  Soit $A$ un anneau noetherien. Montrer que pour tout idéal  $I\subsetneq A$, $\sqrt{I}$ est l'intersection d'un nombre fini d'idéaux premiers. En déduire que $A$ possède un nombre fini d'idéaux premiers minimaux pour $\subset$. \\
     \item   (Anneaux artiniens) Soit $A$ un anneau. Montrer que les PSSE\\
       \begin{enumerate} 
     \item  Toute suite  d'idéaux de $A$ croissante pour $\subset$ est stationnaire à partir d'un certain rang.
     \item Tout sous-ensemble non vide d'idéaux de $A$ admet un élément maximal pour $\subset $.\\
\end{enumerate}
 On dit qu'un anneau $A$ qui vérifie les propriétés équivalente ci-dessus est \textit{artinien}\index{Artinien (Anneau)}. En dépit de la similitude des définitions, les anneaux artiniens et noetheriens se comportent très différemment. Soit $A$ un anneau artinien. Montrer que\\
 \begin{enumerate} 
     \item  Tout idéal premier de $A $ est maximal.
     \item  $A$ ne possède qu'un nombre fini d'idéaux (premiers=) maximaux.
      \item  $A$ est noetherien.\\
\end{enumerate}
En fait, on peut montrer qu'un anneau est artinien si et seulement si il est noetherien et tous ses idéaux premiers sont maximaux. 
\end{enumerate}



  \section{Anneaux principaux, euclidiens } 

\subsection{ }On dit qu'un anneau commutatif intègre $A$ est \\

\begin{itemize}[leftmargin=* ,parsep=0cm,itemsep=0cm,topsep=0cm]
\item \textit{euclidien}\index{Euclidien (Anneau)} s'il est munit d'une application - appelée stathme euclidien - $\sigma:A\setminus \lbrace 0\rbrace\rightarrow \N$ vérifiant la propriété suivante (division euclidienne): pour tout $0\not=a,b\in A$ il existe $q,r\in A$ tels que 
 $$\begin{tabular}[t]{l}
$b=qa+r$\\
$r=0$ ou $r\not=0$ et $\sigma(r)<\sigma(a)$.
\end{tabular}$$
\item \textit{principal}\index{Principal (Anneau)} si tout idéal est de la forme $Aa$, $a\in A$.  
\end{itemize}

\subsection{Exemples}\label{Ex}\textbf{ } \\

    \begin{enumerate}[leftmargin=* ,parsep=0cm,itemsep=0cm,topsep=0cm]
     \item  La valeur absolue usuelle   $|-|:\Z\setminus\lbrace 0\rbrace\rightarrow \N$ sur $\Z$ est un stathme euclidien. En effet, pour tout $0\not= a,b\in \Z$ notons $R:=\lbrace b-qa\;|\; q\in \Z\rbrace $. On a évidemment $R\cap\N\not=\emptyset$  donc on peut poser $r:=\hbox{\rm min} R\cap\N$. Par définition de $R$, $b=qa+r$ et si $|a|\leq  r$ on aurait $r-|a|\in R$: contradiction.  \\

\item \textbf{Algèbres de polynômes sur un anneau intègre.} Soit $A$ un anneau commutatif et $r\geq 1$ un entier. La $A$-algèbre $A[X_1,\dots, X_r]$ n'est  euclidienne  que si $A$ est un corps et $r=1$ mais, lorsque $A$ est intègre, elle se comporte presque comme un anneau euclidien.  

\begin{itemize}[leftmargin=* ,parsep=0cm,itemsep=0cm,topsep=0cm] 
\item  $r=1$. On rappelle que tout $P\in A[X]$ s'écrit de fa\c{c}on unique sous la forme $f=\sum_{n\in\N}a_nX^n$ avec $\underline{a}:n\rightarrow a_n\in A^{(\N)}$. Cela permet de définir l'application degré:
$$ \begin{tabular}[t]{llll}
$deg:$&$A[X]\setminus\lbrace 0\rbrace $&$\rightarrow$&$\N$\\
&$f=\sum_{n\in\N}a_nX^n$&$\rightarrow$&$\hbox{\rm max}\lbrace n\in\N\; |\; a_n\not=0\rbrace$
\end{tabular}$$
et une application `coefficient dominant'
$$ \begin{tabular}[t]{llll}
$CD:$&$A[X]\setminus\lbrace 0\rbrace $&$\rightarrow$&$A\setminus \lbrace 0\rbrace$\\
&$f=\sum_{n\in\N}a_nX^n$&$\rightarrow$&$a_{deg(f)}$
\end{tabular}$$
La définition du produit dans $A[X]$ montre que $deg(fg)\leq deg(f)+deg(g)$ et que si  l'un au moins de $CD(f),CD(g)$ n'est  pas  de torsion dans $A$, $deg(fg)= deg(f)+deg(g)$, $CD(fg)=CD(f)CD(g)$. On a aussi toujours $deg(f+g)\leq \hbox{\rm max}\lbrace deg(f),deg(g)\rbrace$.\\
 
\textbf{Lemme} \textit{Soit $0\not=f,g\in A[X]$ et supposons que $CD(f)\in A^\times$. Alors il existe un unique couple $q,r\in A[X]$ tel que $g=fq+r$ et $r=0$ ou $deg(r)<deg(f)$.}\\

\begin{proof} Montrons l'existence par récurrence sur $deg(g)$. Ecrivons $f=\sum_{0\leq n\leq d_f}a_nX^n$, $g=\sum_{0\leq n\leq d_g}b_nX^n$, o\`u $d_f:=deg(f)$, $d_g:=deg(g)$. Si $d_g=0$ et $d_f>0$, $q=0$ et $r=g$ conviennent. Si $d_g=d_f=0$, $f=a_0=a_{d_f}\in A^\times\subset A[X]^\times$ donc $q=f^{-1}g$ et $r=0$ conviennet. Si $d_g\geq 1$ et $d_f>d_g$, $q=0$ et $r=g$ conviennent. Supposons donc $d_f\leq d_g$. Comme $a_{d_f}\in A^\times$ on peut écrire 
$$g=a_{d_g}a_{d_f}^{-1}X^{d_g-d_f}f+(g-a_{d_g}a_{d_f}^{-1}X^{d_g-d_f}f).$$
Par construction, $g_1:=(g-a_{d_g}a_{d_f}^{-1}X^{d_g-d_f}f)$ est de degré $\leq d_g-1$. Par hypothèse de récurrence il existe donc $q_1,r_1\in A[X]$ tels que $g_1=q_1f+r_1$ et $r_1=0$ ou $deg(r_1)<deg(f)$; $q:=a_{d_g}a_{d_f}^{-1}X^{d_g-d_f}+q_1$, $r:=r_1$ conviennent. Il reste à prouver l'unicité. Si $q',r'\in A[X]$ est un autre couple tel que $g=fq'+r'$ et $r'=0$ ou $deg(r')<deg(f)$, on a $f(q-q')=r'-r$. Si $r-r'\not= 0$, en prenant le degré $$deg(f)\geq deg(f)+deg(q-q')\stackrel{(1)}{=}deg(f(q-q')=deg(r-r')<deg(f),$$
o\`u $(1)$ utilise encore que $CF(f)\in A^\times$. On a donc forcément $r=r'$ donc $f(q-q')=0$ donc, toujours parce que $CD(f)\in A^\times$, $q=q'$.\\ \end{proof}

 En particulier, si $A=k$ est un corps, le degré $deg:k[X]\setminus\lbrace 0\rbrace\rightarrow \N$  est un stathme euclidien sur $k[X]$.\\

\item $r\geq 1$. En utilisant les isomorphismes canoniques $A[X_1,\dots, X_r]\tilde{\rightarrow} A[X_1,\dots,\hat{X}_i,\cdots , X_r][X_i]$, $i=1,\dots, r$, on peut encore appliquer le Lemme ci-dessus dans $A[X_1,\dots, X_r]$: les polynômes par lesquels on peut diviser sont ceux de la forme $aX_i^{d}+\sum_{\underline{n}\in \N^r,\;|\; n_i<d}a_{\underline{n}}\underline{X}^{\underline{n}}$, avec $a\in A[X_1,\dots,\hat{X}_i,\cdots , X_r]^\times=A^\times$ (car $A$ est intègre donc réduit).\\
\end{itemize}
 \item On peut montrer que le carré de la valeur absolue usuelle  $|-|^2:\Z[w]\rightarrow \N$ est un stathme euclidien sur certains sous-anneaux de $\C$ de la forme $\Z[w]\subset \C$; c'est par exemple le cas pour $w=\sqrt{-1},^3\sqrt{-1},\sqrt{-2}$.
 \end{enumerate}
 
 

\subsection{Lemme}\label{EuclIsPrinc} \textit{Euclien $\Rightarrow$ Principal.} 
 
 \begin{proof} Soit $A$ un anneau euclidien et soit $I\subset A$ un idéal. Fixons $a\in I$ tel que $\sigma(a)=\hbox{\rm min}\sigma(I)$. Puisque $a\in I$, on a $Aa\subset I$. Réciproquement, pour tout $b\in I$, effectuons la division euclidienne de $b$ par $a$: il existe $q,r\in A$ tels que $b=qa+r$ et $r=0$ ou $\sigma(r)<\sigma(a)$. Mais comme $r=b-qa\in I$, on ne peut pas avoir $\sigma(r)<\sigma(a)$, donc $r=0$.\\
 \end{proof}
 
 \textbf{(Contre-)Exemple.} Les anneaux principaux ne sont pas tous euclidiens. Par exemple  $A=\Z[\frac{1+\sqrt{-19}}{2}]$ est principal non euclidien.\\
 
 \subsection{Exercice} Montrer que $A[X]$ est principal si et seulement si $A$ est un corps. 
  \subsection{Lemme}\textit{Si $A$ est un anneau principal, $spm(A)=spec(A)\setminus \lbrace 0\rbrace$.}
  
  \begin{proof} On a toujours $spm(A)\subset spec(A)$. Soit $\frak{p}=Ap\in spec(A)$; on veut montrer que $A/Ap$ est un corps. Supposons le contraire. Alors $A/Ap$ contient un idéal maximal $\lbrace \bar{0}\rbrace\subsetneq\frak{m}\subsetneq A/Ap$. Ecrivons $p_{\frak{p}}^{-1}(\frak{m})=Am$. On a des inclusions strictes  $Ap\subsetneq Am\subsetneq A$ donc il existe $a\in A$ tel que $p=am$ donc $\bar{0}=\bar{a}\bar{m}$ dans $A/Ap$. Comme $A/Ap$ est intègre et $\bar{m}\not= 0$, on en déduit $a\in Ap$. Ecrivons donc $a=bp$; on a $p=am=pbm$. Comme $A$ est intègre, on peut simplifier par $p$ pour obtenir $1=bm$, ce qui contredit $ Am\subsetneq A$.   \end{proof}
 \section{Anneaux factoriels}\label{Factoriel} \textit{}\\
  Soit $A$ un anneau commutatif intègre. \\
 
  Pour tout $a,b\in A$ on a $Aa=Ab$ si et seulement si $A^\times a=A^\times b$. L'implication $A^\times a=A^\times b$ $\Rightarrow$ $Aa=Ab$ est toujours vraie (sans supposer $A$ intègre). Réciproquement,   si $a=0$ alors $Ab=0$ impose $b=0$ puisque $A$ est intègre. Supposons donc $a,b\not= 0$ et $Aa=Ab$. On peut écrire $a=\alpha b$ et $b=\beta a$ donc $a=\alpha\beta a$ et, comme $A$ est intègre, on peut simplifier par $a$, ce qui montre que $\alpha,\beta\in A^\times$. On note $a\sim b$ (et on dit que $a,b$ sont \textit{associés} dans $A$) la relation $Aa=Ab$ ($\Leftrightarrow$ $A^\times a=A^\times b$); c'est une relation d'équivalence sur $A$.
 

 
 
\subsection{Eléments irréductibles, éléments premiers}\label{IrrPrem}On dit que $0\not=p\in A\setminus A^\times$  est \textit{irréductible} si pour tout $a,b\in A$, $p=ab$ implique $a\in A^\times$ ou $b\in A^\times$. On notera $\mathcal{P}_A^\circ\subset A$ l'ensemble des éléments irréductibles de $A$.
  On munit  $\mathcal{P}_A^\circ$ de la relation d'équivalence $\sim$ définie par: pour tout $p,q\in \mathcal{P}_A^\circ$, $p\sim q$ si et seulement si $Ap=Aq$, ce qui est aussi équivalent à  $A^\times p=A^\times q$.\\
  
   On notera $\mathcal{P}_A\subset \mathcal{P}_A^\circ$ un système de représentants de $\mathcal{P}_A^\circ/\sim$.\\
  
  
  
  \textbf{Exemple.} On a $\Z^\times=\lbrace \pm 1\rbrace$ et les irr\'ductibles de $\Z$ sont les nombres premiers. Si l'on veut déterminer si un entier $n\in\Z_{\geq 1}$ est premier, on dispose d'un algorithme évident consistant à lister tous les premiers $\leq \sqrt{n}$ et vérifier s'ils divisent $n$ mais cet algorithme devient très vite inutilisable sur machine. Les arithméticiens ont beaucoup étudié et étudient encore  le problème de la construction et de la r\'partition des nombres premiers.  L'une de leurs motivations  est l'application des nombres premiers en cryptographie. Parmi les énoncés classiques  les plus spectaculaires, on trouve par exemple le théorème des nombres premiers, qui dit que si on note $\pi(n)$ le nombre de nombre premiers $0\leq  p\leq n$, on a $\pi(n)\sim_{n\to+\infty} \ln(n)/n$ ou le théorème de   la progression arithmétique, qui dit que pour tout entier $0\not= m,n$ premiers entre eux l'ensemble $m+\Z n$ contient une infinité de nombres premiers. Ces énoncés se démontrent souvent par des méthodes analytiques. \\
  
   \textbf{Exercice.}  Montrer directement le  théorème de   la progression arithmétique pour $(m,n)=(3,4)$.\\
  

  
   On dit que $0\not=p\in A\setminus A^\times$  est \textit{premier}\index{Premier (Elément)} si   $Ap\in spec(A)$. On notera $\mathcal{P}^\dag_A\subset A$ l'ensemble des éléments premiers de $A$. \\
  

  
\ref{IrrPrem}.1 \textbf{Lemme.} \textit{On a toujours $\mathcal{P}_A^\dag\subset \mathcal{P}_A^\circ$.}
  \begin{proof} En effet, si $Ap\in spec(A)$, pour tout $a,b\in A$, $p=ab$ implique $ab\in Ap$ donc comme $Ap$ est premier, $a\in Ap$ ou $b\in Ap$. Supposons $a\in Ap$ \textit{i.e.} $a=\alpha p$. On a alors $p=ab=\alpha bp$  et, comme $A$ est intègre, on peut simplifier par $p$ ce qui donne $\alpha b=1$ donc $b\in A^\times$.\end{proof}
  
  \textbf{(Contre-)exemple.} En général $\mathcal{P}_A^\dag\subsetneq \mathcal{P}_A^\circ$. Par exemple, dans $A=\Z[i\sqrt{5}]$, $2$ est irréductible mais pas premier. En effet, introduisons la norme $N:A\rightarrow \Z_{\geq 0}$, $a+ib\sqrt{5}\rightarrow |a+ib\sqrt{5}|^2=a^2+5b^2$. On vérifie immédiatement que $N(xy)=N(x)N(y)$, $N(x)=0$ $\Leftrightarrow$ $x=0$ et que
  $$ x\in A^\times\Leftrightarrow N(x)=1\Leftrightarrow x=\pm 1.$$
  Vérifions d'abord que $2\in \mathcal{P}_A^\circ$. Si on écrit $2=xy$ on doit avoir $4=N(2)=N(xy)=N(x)N(y)$. En particulier, $N(x)=N(y)=2$ ou $\lbrace N(x),N(y)\rbrace=\lbrace 1,4\rbrace$. Or $2\notin N(A)$ donc nécessairement $N(x)=1$ ou $N(y)=1$ \textit{i.e.} $x\in A^\times$ ou $y\in A^\times$. Montrons ensuite que $2$ n'est pas premier. Pour cela, observons que 
 $$2\cdot 3=(1+i\sqrt{5})\cdot (1-i\sqrt{5})=2\times 3\in 2A$$
  mais que  $1\pm i\sqrt{5}\notin 2A$ car $N(1\pm i\sqrt{5})=6\notin N(2A)=4N(A)\subset 4\Z_{\geq 0}$.    \\
  
\ref{IrrPrem}.2 On dit qu'un anneau commutatif intègre $A$ est \textit{factoriel}\index{Factoriel (Anneaux)} si pour tout système de représentants $\mathcal{P}_A$ de $\mathcal{P}_A^\circ$ l'application
$$(\ref{IrrPrem}.2.1)\;\; \begin{tabular}[t]{cll}
 $A^\times\times \N^{(\mathcal{P}_A)}$&$\rightarrow$&$A\setminus\lbrace 0\rbrace $\\
 $(u,\nu)$&$\rightarrow$&$u\displaystyle{\prod_{p\in\mathcal{P}_A}p^{\nu(p)}}$
 \end{tabular}$$ 
est bijective \textit{i.e.} pour tout $0\not=a\in A $ il existe une unique application $v_-(a):\mathcal{P}_A\rightarrow \N $ à support fini et un unique $u_a\in A^\times $ tels que $ a= u_a\prod_{p\in \mathcal{P}_A}p^{v_p(a)}$ (on parle de `la' décomposition en produit d'irréductibles de $a$).\\

 On prendra garde au fait que l'élément $u_a\in A^\times $  dépend  du choix du système de représentants $\mathcal{P}_A$ de $\mathcal{P}_A^\circ/\sim$ qu'on s'est fixé. Par contre, l'application  $v_{-}(a):\mathcal{P}_A\rightarrow \N $ n'en dépend pas; si on note $\frak{p}:=Ap$, on peut la définir intrinsèquement par $v_{\frak{p}}(a)=\hbox{\rm max}\lbrace n\in\N\;|\; a\in \frak{p}^n\rbrace$. On dit que $v_p(a)$ est la multiplicité ou l'ordre de $a$ en $p$ ou, encore, la valuation $p$-adique de $a$. \\

\ref{IrrPrem}.3 Soit $A$ un anneau factoriel. On prolonge les applications $v_p:A\setminus \lbrace 0\rbrace\rightarrow \N$ en $v_p:A \rightarrow \overline{\N}:=\N\cup\lbrace \infty\rbrace$ par $v_p(0)=\infty$. Avec les conventions $n+\infty=\infty$ et $n\leq \infty$, $n\in \overline{\N}$, il résulte immédiatement de l'unicité dans la définition d'anneaux factoriel que les applications $v_p:A \rightarrow \overline{\N}$ , $p\in \mathcal{P}_A$ vérifient les propriétés élémentaires suivantes.
\begin{enumerate}
\item $v_p(ab)=v_p(a)+v_p(b)$, $ a,b\in A $;\\
\item $v_p(a+b)\geq \hbox{\rm min}\lbrace v_p(a),v_p(b)$  et si $v_p(a)\not= v_p(b)$, $v_p(a+b)= \hbox{\rm min}\lbrace v_p(a),v_p(b)$, $ a,b\in A$, $a\not= p$. \\

  En effet, écrivons $a=p^{v_p(a)}a'$, $b=p^{v_p(b)}b'$ avec $v_p(a')=v_p(b')=0$. Si   $v_p(a)>v_p(b)$, on a $a+b=p^{v_p(b)}(a'p^{v_p(a)-v_p(b)}+b')$ avec $v(a'p^{v_p(a)-v_p(b)}+b')=0$ car $v_p(b')=0$ et $v_p(a'p^{v_p(a)-v_p(b)})= v_p(a)-v_p(b)>0$. Si $v_p(a)=v_p(b)=v$, on a $v_p(a+b)=v +v_p(a'+b')\geq v$.\\
\item $v_p^{-1}(0)=A\setminus Ap$, $v_p^{-1}(\overline{\N }\setminus\lbrace 0\rbrace)=Ap$.\\
\end{enumerate}

 On déduit  de (1) et (3) que\\
 
\ref{IrrPrem}.4 \textbf{Lemme.} \textit{$A$  factoriel $\Rightarrow$   $\mathcal{P}_A^\dag= \mathcal{P}_A^\circ$. } 
\begin{proof}On sait déjà que $\mathcal{P}_A^\dag\subset \mathcal{P}_A^\circ$. Inversement, soit $p\in\mathcal{P}_A^\circ$. Alors pour tout $a,b\in A$, on a $ab\in Ap$ $\Leftrightarrow$ $v_p(a)+v_p(b)=v_p(ab)\geq 1$ $\Leftrightarrow$ $v_p(a)\geq 1$ ou $v_p(b)\geq 1$ $\Leftrightarrow$ $a\in Ap$ ou $b\in Ap$.
\end{proof}


 
\subsection{Proposition}\label{Implications}\textit{
\begin{enumerate}[leftmargin=* ,parsep=0cm,itemsep=0cm,topsep=0cm] 
\item Principal $\Rightarrow$ (Noetherien intègre $+$ $\mathcal{P}_A^\dag=\mathcal{P}_A^\circ$) $\Rightarrow$ factoriel.
\item {[Utilise le Lemme de Zorn]} Factoriel $+$ $spm(A)=spec(A)\setminus \lbrace 0\rbrace$ $\Rightarrow$ Principal.\\
\end{enumerate}}


\ref{Implications}.1 Le lemme suivant montre que ce qui est 'profond' dans la définition d'anneau factoriel c'est surtout l'unicité de la décomposition en produit d'irréductibles. L'existence est vérifiée pour une classe d'anneaux beaucoup plus large.\\

 \textbf{Lemme.} \textit{Si $A$ est un anneau notherien intègre, l'application $$ \begin{tabular}[t]{cll}
 $A^\times\times \N^{(\mathcal{P}_A)}$&$\rightarrow$&$A\setminus\lbrace 0\rbrace $\\
 $(u,\nu)$&$\rightarrow$&$u\displaystyle{\prod_{p\in\mathcal{P}_A}p^{\nu(p)}}$
 \end{tabular}$$ est surjective.}
\begin{proof}Notons $\mathcal{F}\subset A$ l'image de $A^\times\times \N^{(\mathcal{P}_A)} \rightarrow A\setminus\lbrace 0\rbrace $. Observons que $\mathcal{F}$ est stable par produit et qu'il contient $\mathcal{P}_A^\circ$, $A^\times$. Si $a\notin \mathcal{F}$, $a\notin\mathcal{P}_A$ donc il existe $a_1,a_2\notin A^\times$ tels que $a=a_1a_2$. En particulier, $Aa\subsetneq Aa_1,Aa_2$. De plus, comme $\mathcal{F}$ est stable par produit, on a $a_1\notin\mathcal{F}$ ou $a_2\notin\mathcal{F}$. Supposons $a_1\notin\mathcal{F}$. En itérant, $ a_1 =a_{1,1}a_{1,2} $ avec $a_{1,1},a_{1,2}\notin A^\times$ - donc $Aa_1\subsetneq Aa_{1,1}, Aa_{1,2}$ - et $a_{1,1}\notin \mathcal{F}$ \textit{etc.} on construit ainsi une suite strictement croissante  $Aa\subsetneq Aa_{1}\subsetneq Aa_{1,1,1}\subsetneq Aa_{1,1,1,1}\subsetneq\cdots$ d'idéaux de A, ce qui contredit la noetherianité de $A$. 
\end{proof}

 
\begin{proof} \begin{enumerate}[leftmargin=* ,parsep=0cm,itemsep=0cm,topsep=0cm] 
\item  Principal $\Rightarrow$ (Noetherien intègre $+$ $\mathcal{P}_A^\dag=\mathcal{P}_A^\circ$). \\

 Soit $A$ un anneau principal. On sait déjà que  $A$ est  intègre (par définition) et  noetherien (puisque tous ses idéaux sont engendrés par un seul élément). Soit $p\in A $ irréductible; on veut montrer que $ Ap$ est premier. Il suffit de montrer qu'il est maximal. Considérons donc un idéal $Ap\subsetneq I $. Fixons $a\in I\setminus Ap$. Commme $A$ est principal, $Ap\subsetneq Ap+Aa=Ab$ donc $p=\alpha b$ avec $\alpha\in A\setminus A^\times$ (puisque $Ap\subsetneq Ab$). Mais puisque $p$ est irréductible, on a nécessairement $b\in A^\times$ \textit{i.e.} $Ab=A$. En particulier $A=Ap+Aa \subset I$.\\ 

 (Noetherien intègre $+$ $\mathcal{P}_A^\dag=\mathcal{P}_A^\circ$) $\Rightarrow$ factoriel.\\

 Par le Lemme \ref{Implications}.1, on sait déjà que l'application $A^\times\times \N^{(\mathcal{P}_A)} \rightarrow A\setminus\lbrace 0\rbrace $ est surjective. 
Supposons que 
l'on ait $$a:=u\prod_{p\in \mathcal{P}_A}p^{\mu(p)}=v\prod_{p\in \mathcal{P}_A}p^{\nu(p)}$$
et que, $\nu(p)>\mu(p)$ pour un certain $p\in \mathcal{P}_A$. Comme $A$ est intègre, on peut simplifier par $p^{\mu(p)}$; on peut donc supposer $\mu(p)=0$ et $\nu(p)>0$. Comme $\nu(p)>0$, $\overline{a}=0$  dans $A/p$. Comme $p\in \mathcal{P}_A^\dag$,  $A/p$ est intègre et comme $\overline{v}\in (A/p)^\times$, il existe forcément $q\in \mathcal{P}_A$ tel que $\overline{q}=0 $ dans $A/p$ \textit{i.e.} $q\in Ap$, ce qui force $q=p$ puisque $p,q$ sont irréductibles: contradiction.\\

\item Supposons   $A$ factoriel et $spm(A)= spec(A)\setminus \lbrace 0\rbrace$.\\

\begin{itemize}[leftmargin=* ,parsep=0cm,itemsep=0cm,topsep=0cm]
\item Montrons d'abord que tout idéal premier est principal: si $\lbrace 0\rbrace\subsetneq \frak{p}\subsetneq A$ est premier, il contient un élément $0\not=a\notin A^\times$. Comme $a$ est factoriel, on peut écrire $a=u_a\prod_{p\in \mathcal{P}_A}p^{v_p(a)}$. Comme $A/\frak{p}$ est intègre, il existe au moins un $p\in \mathcal{P}_A$ tel que $v_p(a)\geq 1$ et  $\bar{p}=0$ \textit{i.e.} $p\in \frak{p}$. En particulier $Ap\subset \frak{p}$. Mais comme $A$ est factoriel, $Ap \in spec(A)$ et comme $spm(A)=spec(A)$ par hypothèse, $Ap=\frak{p}$.\\
\item Soit maintenant $\mathcal{E}$ l'ensemble des idéaux de $A$ qui ne sont pas principaux. Supposons $\mathcal{E}\not=\emptyset$; comme $(\mathcal{E},\subset)$ est un ensemble ordonné inductif, le Lemme de Zorn assure qu'il possède un élément  $0\subsetneq I\subsetneq A$  maximal pour $\subset$. Toujours par le Lemme de Zorn , $I$  est contenu dans un idéal maximal $\frak{m}$, dont on sait qu'il est principal $\frak{m}=Ap$. Introduisons l'ensemble
$$J:=\lbrace a\in A\;|\; ap\in I\rbrace $$
Puisque $I$ est un idéal, $I\subset J$ et $J $ est un idéal de $A$. De plus $I=Jp$. Par définition de $J$ on a $Jp\subset I$ et, inversement, puisque $I\subset Ap$, tout $a\in I$ sécrit sous la forme $a=bp$ avec, par définition de $J$, $b\in J$. Si $I\subsetneq J$, par maximalité de $I$ on aurait $J=Aa$ donc $I=Aap$, ce qui contredit $I\in\mathcal{E}$. Donc $I=J$. Ce qui signifie que la multiplication par $p$ induit une bijection (rappelons que $A$ est intègre) $-\cdot p:I\tilde{\rightarrow} I$. Cela contredit la factorialité de $A$. En effet, si $0\not=a\in I$, on peut écrire $a=p^{v_p(a)}b$ avec $v_p(b)=0$. Mais   par définition de $J$, $p^{v_p(a)-1}b\in J=I=pI$ $\Rightarrow$ $p^{v_p(a)-2}b\in J=I=pI$ $\Rightarrow$ $...$ $\Rightarrow$ $b\in J=I=pI$ $\Rightarrow$ $v_p(b)\geq 1$.
\end{itemize}
\end{enumerate}
\end{proof}

\textbf{Remarque.} Si on suppose $A$ noetherien dans (2), on n'a pas besoin d'invoquer le Lemme de Zorn.\\

\ref{Implications}.2 \textbf{(Contre-)Exemples.} Les implications de \ref{Implications}  ne sont pas des équivalences. Par exemple,
\begin{itemize}[leftmargin=* ,parsep=0cm,itemsep=0cm,topsep=0cm] 
\item Anneau noetherien $+$ $\mathcal{P}_A^\dag=\mathcal{P}_A^\circ$ non principal: $k[X_1,X_2]$, o\`u $k$ est un corps commutatif;
\item Anneau factoriel  non noetherien: $k[X_1,\dots,X_n,\dots, X_{n+1},\dots]$, o\`u $k$ est un corps commutatif.
\end{itemize}



\subsection{Polynômes sur les anneaux factoriels}

\subsubsection{}\textbf{Corps des fractions d'un anneau intègre.}\label{FracInt} Nous allons d'abord construire le corps des fractions d'un anneau intègre. Il s'agit d'un cas particulier de localisation, construction que nous verrons en toute généralité un peu plus loin.\\

 Soit donc $A$ un anneau intègre. On munit le produit ensembliste $A\setminus\lbrace 0\rbrace\times A$ de la relation $\sim $ définie par: pour tout $(s,a),(s',a')\in A\setminus\lbrace 0\rbrace\times A$, $(s,a)\sim (s',a')$ si $ s'a-sa'=0$. \\

 On vérifie facilement que $\sim$ est une relation d'équivalence. On note $Frac(A):=A\setminus\lbrace 0\rbrace\times A/\sim$ et 
$$ \begin{tabular}[t]{llll}
$-/-$ :&$A\setminus\lbrace 0\rbrace\times A$&$\rightarrow$&$Frac(A)$\\
 &$(s,a)$&$\rightarrow$&$a/s=:s^{-1}a$
 \end{tabular}$$
 la projection canonique.   Considérons les applications $+,\cdot: (A\setminus\lbrace 0\rbrace\times A)\times (A\setminus\lbrace 0\rbrace\times A)\rightarrow Frac(A)$ définies par 
 $$(s,a)+(t,b)= (ta+sb)/(st),\;\; (s,a)\cdot (t,b)= (ab)/(st)$$
  Si $(s,a)\sim (s',a')$, $(t,b)\sim (t',b')$  on a
 $$s't'(ta+sb)-st(t'a'+s'b')=(s'a)(t't)+(ss')(t'b)-(sa')(tt')-(ss')(tb')=(s'a-sa')t't+(ss')(t'b-tb')=0$$ 
  $$(s't')(ab)-(st)(a'b')= (s'a)(t'b)-(sa')(tb')= 0$$
 Cela montre que les applications $+,\cdot :(A\setminus\lbrace 0\rbrace\times A)\times (A\setminus\lbrace 0\rbrace\times A) \rightarrow Frac(A)$  se factorisent en 
 $$\xymatrix{(A\setminus\lbrace 0\rbrace\times A)\times (A\setminus\lbrace 0\rbrace\times A) \ar[r]^{+,\cdot}\ar[d]_{-/-\times -/-}& Frac(A)\\
 Frac(A)\times Frac(A)\ar[ur]_{+,\cdot}&} $$
 On laisse en exercice le soin de vérifier que $Frac(A)$ muni des lois $+,\cdot :Frac(A)\times Frac(A)\rightarrow Frac(A)$ ainsi définies vérifie bien les axiomes d'un anneau commutatif  de zéro $0/1$ et d'unité $1/1$ et que, pour cette structure d'anneau, l'application canonique $$\begin{tabular}[t]{lclc}
 $\iota_A:$&$A$&$\rightarrow$&$Frac(A)$\\
 &$a$&$\rightarrow$&$a/1$
 \end{tabular}$$
 est un morphisme d'anneaux injectif.  De plus, tout élément non nul $a/b\in Frac(A)$ est inversible d'inverse $b/a$; $Frac(A)$ est donc un corps.\\
 
 \textbf{Lemme.} (Propriété universelle du corps des fractions) \textit{Pour tout anneau intègre $A$  il existe un morphisme d'anneaux $\iota :A\rightarrow F$ tel que $\iota(A\setminus \lbrace 0\rbrace)\subset F^\times$ et pour tout  morphisme d'anneaux $\phi:A\rightarrow B$ tel que $\phi(A\setminus \lbrace 0\rbrace)\subset B^\times $, il  existe un unique morphisme d'anneaux $ \tilde{\phi}:F\rightarrow B$ tel que $\phi=  \tilde{\phi}\circ \iota $.}\\
 
  Plus visuellement,
 $$\xymatrix{A\setminus \lbrace 0\rbrace\ar[r]^\phi\ar@{_{(}->}[d]&B^\times\ar@{_{(}->}[d]\\
 A\ar[r]^{\forall\phi}\ar@{_{(}->}[d]_{\iota_A}&B\\
 F\ar@{.>}[ur]_{\exists !\tilde{\phi}}}$$

 \begin{proof} Montrons que  $Frac(A)$ muni de la structure d'anneau ci-dessus et le morphisme canonique $\iota_A:A\rightarrow Frac(A)$ conviennent. Soit $\phi:A\rightarrow B$  un morphisme d'anneaux tel que $\phi(A\setminus \lbrace 0\rbrace)\subset B^\times $.  Si $\tilde{\phi}:Frac(A)\rightarrow B$ existe la relation $\phi=\tilde{\phi}\circ \iota_A$ impose que  $\tilde{\phi}:Frac(A)\rightarrow B$ est unique puisqu'on doit nécessairement avoir 
 $$\tilde{\phi}(a/s)=\tilde{\phi}((a/1)(1/s))=\tilde{\phi}(a/1) \tilde{\phi}((s/1)^{-1})=\phi(a)\phi(s)^{-1},\; (s,a)\in A\setminus\lbrace 0\rbrace\times A.$$
 Considérons donc l'application  \begin{tabular}[t]{lclc}
 $\tilde{\phi}:$&$ A\setminus\lbrace 0\rbrace\times A $&$\rightarrow$&$B$\\
 &$(s,a) $&$\rightarrow$&$ \phi(s)^{-1}\phi(a)$.
 \end{tabular} 
 Si $(s,a)\sim (s,a)'$  on a $ \phi(s')\phi(a)-\phi(s)\phi(a') =\phi( (s'a-sa'))=\phi(0)=0$. Mais comme $\phi(s),\phi(s') \in B^{\times}$, on peut réécrire cette égalité comme 
 $$\tilde{\phi}(s,a)=\phi(s)^{-1}\phi(a)=\phi(s')^{-1}\phi(a')=\tilde{\phi}( s',a').$$
 Cela montre que l'application $\tilde{\phi}:  A\setminus\lbrace 0\rbrace  \rightarrow  B$ se factorise  en 
 $$\xymatrix{ A\setminus\lbrace 0\rbrace\ar[r]^{\tilde{\phi}}\ar[d]_{-/-}& B\\
Frac(A) \ar[ur]_{\tilde{\phi}}&} $$
Par construction $\phi=  \tilde{\phi}\circ \iota_A$ et on vérifie que $\tilde{\phi}: Frac(A)\rightarrow B$ est bien un morphisme d'anneaux. 
  \end{proof}
 
  Comme d'habitude, le morphisme d'anneaux  $\iota_A:  A\rightarrow Frac(A)$ est unique à unique isomorphisme près; on dit que c'est le   \textit{corps des fractions}\index{Corps des fractions (Anneau intègre)} de $A$.\\
 
\textbf{Exercice.} On dit qu'un anneau $A$ intègre de corps des fraction $K$ est intégralement clos\index{Intégralement clos (Anneaux)} si 
$$A=\lbrace x\in K[X]\; |\; \exists \; P_x=T^d+\sum_{0\leq n\leq d-1}a_nT^n\in A[X]\; \hbox{\rm tel que}\; P_x(x)=0\rbrace.$$
Montrer qu'un anneau factoriel est intégralement clos.\\
 

\textbf{Exercice.} On note $\Q:=Frac(\Z)$ et si $K$ est un corps, on note $K(X_1,\dots , X_n):=Frac(K[X_1,\dots, X_n])$. Montrer que si $A$ est un anneau intègre de corps des fractions $K$ alors $Frac(A[X_1,\dots, X_n])=K(X_1,\dots, X_n)$.

\subsubsection{Valuations $p$-adiques}Soit $A$ un anneau factoriel (donc en particulier intègre) et $\iota_A:A\hookrightarrow K:=Frac(A)$ son corps des fractions. Pour chaque $p\in\mathcal{P}_A$, l'application  $$
\begin{tabular}[t]{cccc}
$v_p:$&$A\setminus\lbrace 0\rbrace\times A$&$\rightarrow$&$\overline{\Z}:= \Z\cup\lbrace\infty\rbrace$\\
&$(s,a)$&$\rightarrow$&$v_p(a)-v_p(s)$
\end{tabular}$$
 vérifie $(s,a)\sim (s',a')$ $\Rightarrow$ $v_p(a)-v_p(s)=v_p(a')-v_p(s')$  donc se factorise \textit{via} $$\xymatrix{A\setminus\lbrace 0\rbrace\times A\ar[r]^{v_p}\ar[d]_{-/-}& \overline{\Z}\\
K\ar[ur]_{v_p}&}$$
qui vérifie encore 
\begin{enumerate}
\item $v_p(xy)=v_p(x)+v_p(y)$, $x,y\in K$;
\item $v_p(x+y)\geq \hbox{\rm min}\lbrace v_p(x),v_p(y)\rbrace$, $x,y\in K$;
\end{enumerate}
De plus, $$A^\times=\bigcap_{p\in\mathcal{P}_A}v_p^{-1}(0),\;\; A=\bigcap_{p\in\mathcal{P}_A}v_p^{-1}(\overline{\Z}_{\geq 0} ).$$
La bijection (\ref{IrrPrem}.2.1) s'étend également en une bijection 
$$ \begin{tabular}[t]{cll}
 $A^\times\times \overline{\Z}^{(\mathcal{P}_A)}$&$\rightarrow$&$K $\\
 $(u,\nu)$&$\rightarrow$&$u\displaystyle{\prod_{p\in\mathcal{P}_A}p^{\nu(p)}}$
 \end{tabular}$$ 
 d'inverse 
 $$ \begin{tabular}[t]{cll}
 $K$&$\rightarrow$&$A^\times\times \overline{\Z}^{(\mathcal{P}_A)}$\\
 $x$&$\rightarrow$&$(x\displaystyle{\prod_{p\in\mathcal{P}_A}p^{-v_p(x)}},p\rightarrow v_p(x))$
 \end{tabular}$$ 
 
 

\subsubsection{Contenu}\label{Contenu}Supposons toujours $A$ factoriel. Pour tout $p\in\mathcal{P}_A$ on étend $v_p:K\rightarrow \overline{\Z} $ en   $v_p:K[X]\rightarrow \overline{\Z}$ par
$$v_p(P):=\hbox{\rm min}\lbrace v_p(a_n)\;|\; n\geq 0\rbrace,\;\; P=\sum_{n\geq 0} a_nX^n\in K[X]$$
  On définit l'application contenu\index{Contenu (Polynôme)} $C_A:K[X]\rightarrow K$ par 
$$C_A(P)=\prod_{p\in \mathcal{P}_A}p^{v_p(P)},\;\; P\in K[X].$$
Noter que comme $P$ n'a qu'un nombre fini de coefficients non nuls, les $v_p(P)$ sont nuls sauf pour un nombre fini de $p\in\mathcal{P}_A$. On a 
\begin{itemize}
\item $C_A(P)=0$ si et seulement si $P=0$;
\item $C_A(P)\in A$ si et seulement si $P\in A[X]$;
\item Pour tout $a\in K $, $C_A(aP)=aC_A(P)$. En particulier, pour tout $P\in K[X]$, $P=C_A(P)P_1$ avec $C_A(P_1)=1$.  \\
\end{itemize}

\textbf{Lemme.} \textit{Pour tout $P,Q\in K[X]$ on a $C_A(PQ)=C_A(P)C_A(Q)$.}


\begin{proof}Si $P\in K$ ou $Q\in K$, c'est clair. Supposons donc $P,Q\in K[X]\setminus K$. En écrivant $P=C_A(P)P_1$, $Q=C_A(Q)Q_1$ on a $C_A(PQ)=C_A(P)C_A(Q)C_A(P_1Q_1)$. Il suffit donc de montrer que si $C_A(P)=C_A(Q)=1$ alors $C_A(PQ)=1$. Observons que pour $F\in K[X]$ in $K[X]$ on a  $C_A(F)=1$ si et seulement si
\begin{enumerate}
\item $F\in A[X]$;
\item  Pour tout $p\in\mathcal{P}_A$, $\overline{F}\not=0$ in $A/pA[X]$, 
\end{enumerate}
o\`u $\overline{F}$ est l'image de $F$ par le morphisme canonique $A[X]\rightarrow A[X]/pA[X]\tilde{\rightarrow} (A/pA)[X]$. La propriété (1) est  stable par produit puisque $A[X]$ est un anneau et la propriété (2) est stable par produit car $(A/pA)[X]$ est aussi un anneau intègre; ici on utilise que $p$ est irréductible donc premier puisque $A$ est factoriel.\end{proof}
%Ecrivons
%$$\begin{tabular}[t]{ll}
%$P=a_0+\cdots+a_mX^m$&, $a_m\not=0$\\
%$Q=b_0+\cdots+b_nX^n$&, $b_n\not=0$\\
%\end{tabular}$$
%Soit $p\in\mathcal{A}$ et $$r:=\hbox{\rm max}\lbrace 0\leq i\leq m\;|\; v_p(a_i)=0\rbrace,\;\; s:=\hbox{\rm max}\lbrace 0\leq i\leq n\;|\; v_p(b_i)=0\rbrace.$$
%Le coefficient $c_{r+s}$ de $X^{r+s}$ dans $PQ$ peut s'écrire sous la forme 
%
%$$\begin{tabular}[t]{ll}
%$c_{r+s}=a_rb_s$&$+a_{r+1}b_{s-1}+a_{r+2}b_{s-2}+\cdots$\\
%&$+a_{r-1}b_{s+1}+a_{r-2}b_{s+2}+\cdots$\\
%\end{tabular}$$ 
%Les définitions de $r$, $s$ montrent que $v_p(a_rb_s)=1$ et $v_p(a_{r+1}b_{s-1}+a_{r+2}b_{s-2}+\cdots), v_p(a_{r-1}b_{s+1}+a_{r-2}b_{s+2}+\cdots)\geq 1$ donc, par la propriété (ii) de $v_p$, $v_p(c_{r+s})=0$.
%
%\end{proof}



\subsubsection{}\label{FactTransfert}\textbf{Proposition.} (Transfert de factorialité) \textit{$A$ factoriel $\Rightarrow$ $A[X]$ factoriel. De plus, les  irréductible de $A[X]$ sont les irréductibles de $A$ et les irréductible de $K[X]$ de contenu $1$.}\\

\begin{proof}L'idée est bien s\^ur d'exploiter que $K[X]$ est factoriel car euclidien. Fixons un système $\mathcal{P}_{K[X]}$ de représentants de $\mathcal{P}^\circ_{K[X]}$ de contenu $1$ (il suffit de remplacer un système de représentants $\mathcal{P}$ donné par les $P/C_A(P)$, $P\in \mathcal{P}$). Notons $\mathcal{P}_{A[X]}$ l'union de $\mathcal{P}_A$ et de  $ \mathcal{P}_{K[X]}$. Comme $A $ est intègre, on sait déjà  que $A[X]^\times=A^\times$. On procède en deux temps.\\


\begin{enumerate}[leftmargin=* ,parsep=0cm,itemsep=0cm,topsep=0cm] 
\item  Les éléments de $  \mathcal{P}_{A[X]}$ sont irréductibles.\\

 Il suffit de montrer que les éléments de $  \mathcal{P}_{A[X]}$ sont premiers.\\

\begin{itemize}[leftmargin=* ,parsep=0cm,itemsep=0cm,topsep=0cm] 
\item Si $p\in \mathcal{P}_A$ comme $A $ est factoriel et $p$ est irréductible, $p$ est premier donc $A/pA$ est intègre. Cela implique que $(A/pA)[X]$ est intègre et on conclut par l'isomorphisme d'anneaux canoniques  $A[X]/pA[X]\tilde{\rightarrow} (A/pA)[X]$.
\item Si $P\in \mathcal{P}_{K[X]}$, considérons le morphisme canonique $\phi:A[X] \hookrightarrow  K[X] \twoheadrightarrow K[X]/PK[X]$. Par construction $PA[X]\subset \ker(\phi)$. Inversement, si $F\in \ker(\phi)$ alors $F=PQ$ dans $K[X]$. Par le Lemme \ref{Contenu}, $C_A(F)=C_A(P)C_A(Q)=C_A(Q)$ donc $C_A(Q)\in A$ \textit{i.e.} $Q\in A[X]$. Donc $F\in PA[X]$ et le morphisme $\phi:A[X] \hookrightarrow  K[X] \twoheadrightarrow K[X]/PK[X]$ se factorise en un morphisme d'anneaux injectif $A/PA[X]\hookrightarrow K[X]/PK[X]$. Comme $K[X]$ est factoriel et $P$ est irréductible, $P$ est premier donc $K[X]/PK[X]$ est intègre. Comme un sous-anneau d'un anneau intègre est intègre, $A[X]/PA[X]$ est donc intègre.\\
\end{itemize}

\item L'application canonique $  A^\times\times \N^{(\mathcal{P}_A \cup \mathcal{P}_{K[X]})} \rightarrow A[X]\setminus\lbrace 0\rbrace $ est bijective.\\


 Comme $K[X]$ est factoriel, l'application $K\setminus\lbrace 0\rbrace \times \N^{(\mathcal{P}_{K[X]})}\rightarrow K[X]\setminus\lbrace 0\rbrace$ est bijective. Elle se restreint en une application (injective!) $A\setminus\lbrace 0\rbrace \times \N^{(\mathcal{P}_{K[X]})}\rightarrow A[X]\setminus\lbrace 0\rbrace$. Cette dernière est en fait bijective car si $F=x\prod_{p\in\mathcal{P}_{K[X]} }P^{v(P)}$ (ici $x\in K\setminus \lbrace 0\rbrace$) est dans $A[X]$, par multiplicativité du contenu, $C_A(F)=x\prod_{p\in\mathcal{P}_{K[X]}} C_A(P)^{v(P)}$ et comme par hypothèse $C_A(P)=1$, $x\in A$. Enfin, par factorialité de $A$, l'application $A^\times\times \N^{(\mathcal{P}_A)}\tilde{\rightarrow} A\setminus\lbrace 0\rbrace$ est bijective donc on obtient la bijection voulue comme 
$$A^\times\times \N^{(\mathcal{P}_A \cup \mathcal{P}_{K[X]})}\tilde{\rightarrow}A^\times\times \N^{(\mathcal{P}_A)}\times \N^{(\mathcal{P}_{K[X]})}\tilde{\rightarrow} A\setminus\lbrace 0\rbrace\times  \N^{(\mathcal{P}_{K[X]})}\tilde{\rightarrow}A[X]\setminus\lbrace 0\rbrace.$$

 
\end{enumerate}

\textbf{Remarque.} On a bien montré en passant que tout irréductible de $A[X]$ admet un représentant dans $ \mathcal{P}_{A[X]}$: si $F\in A[X]$ est irréductible, il s'écrit de fa\c{c}on unique sous la forme $$F= u\prod_{p\in   \mathcal{P}_{A[X]}}p^{v_p(F)}$$
avec $u\in A^\times$ et comme $F$ est par définition non inversible et ne peut s'écrire  comme produit de deux éléments non-inversibles, on doit forcément avoir $v_p(F)=1$ pour un certain $p\in \mathcal{P}_{A}\cup\mathcal{P}_{K[X]}$ et $v_q(F)=0$, pour tout $p\not=q\in  \mathcal{P}_{A}\cup\mathcal{P}_{K[X]}$
\end{proof}

\subsubsection{}\label{FactTransfert}\textbf{Corollaire.} \textit{Pour tout $n\geq 1$, $A$ factoriel $\Rightarrow$ $A[X_1,\dots, X_n]$ factoriel. } 
\begin{proof} Par induction sur $n$ et en utilisant l'isomorphisme  canonique  $$A[X_1,\dots,X_n]\tilde{\rightarrow}A[X_1,\dots,  \dots, X_{n-1}] [X_n].$$
\end{proof}


 \subsubsection{}\textbf{Exercice - critères d'irréductibilité pour les algèbres de polynômes sur les corps.} Comme dans $\Z$, déterminer  si un élément de $K[X]$ est irréductible est un problème délicat. Voici les deux critères d'irréductibilité les plus classiques pour les algèbres de polynômes.\\
  \begin{enumerate}[leftmargin=* ,parsep=0cm,itemsep=0cm,topsep=0cm] 
  \item \textbf{(Critère d'Eisenstein)} Soit $A$ un anneau factoriel de corps des fractions $ K$ et $P=\sum_{n\geq 0} a_nX^n\in A[X]$. Montrer que s'il existe un irréductible $p$ de $A$ tel que $v_p(a_0)\leq 1$, $v_p(a_n)\geq 1$, $0\leq n\leq deg(P)-1$ et $v_p(a_{deg(P)})=0$ alors $P$ est irréductible dans $K[X]$.\\
  
  \textbf{Application.} Montrer que $P\in K[X]$ est irréductible si et seulement si $P(X+1)\in K[X]$ est irréductible. En déduire que pour tout nombre premier $p$, le polynôme $X^p+X^{p-1}+\cdots+X+1$ est irréductible dans $\Q[X]$. \\
  \item \textbf{(Critère de réduction)} Soit $A,B$ des anneaux intègres et $L$ le corps des fractions de $B$. Soit $\phi:A\rightarrow B$ un morphisme d'anneaux. La propriété universelle de $\iota_A:A\rightarrow A[X]$ appliquée avec $A\stackrel{\phi}{\rightarrow}B\stackrel{\iota_B}{\hookrightarrow} B[X]$ donne un unique morphisme d'anneaux $\tilde{\phi}:A[X]\rightarrow B[X]$ tel que $\tilde{\phi}\circ \iota_A=\iota_B\circ\phi$ (explicitement $\tilde{\phi}(\sum_{n\geq 0}a_nX^n)=\sum_{n\geq 0}\phi(a_n)X^n$). Soit $P\in A[X]$. Montrer  que si $deg(\tilde{\phi}(P))=deg(P)$ et $\tilde{\phi}(P)$ est irréductible dans $L[X]$ alors $P$ ne peut s'écrire sous la forme $P=P_1P_2$ avec $P_1,P_2\in A[X]$ de degré $\geq 1$.\\
  
  
   Correction. \textit{Ecrivons  $P=P_1P_2$ avec   $P_1,P_2\in A[X]$ et $deg(P_1)\leq deg(P_2)$. On veut montrer que $P_1\in A$. Notons que par construction $deg(\tilde{\phi}(P))\leq deg(P)$. Puisque $\tilde{\phi}:A[X]\rightarrow B[X]$ est un morphisme d'anneaux, on a $\tilde{\phi}(P)=\tilde{\phi}(P_1)\tilde{\phi}(P_2)$ dans $L[X]$. Puisque  $\tilde{\phi}(P)\in L[X]$ est irréductible par hypothèse, on a $\tilde{\phi}(P_1)\in K$ ou $\tilde{\phi}(P_2)\in K$. Enfin, puisque $$deg(P_1)+deg(P_2)\geq deg(\tilde{\phi}(P_1))+deg(\tilde{\phi}(P_2))=deg(\tilde{\phi}(P))=deg(P)=deg(P_1)+deg(P_2),$$ 
  on a $deg(\tilde{\phi}(P_i))=deg(P_i)$, $i=1,2$. Donc (on a supposé $deg(P_1)\leq deg(P_2)$) $\tilde{\phi}(P_1)\in K$, ce qui implique $deg(P_1)=deg(\tilde{\phi}(P_1))=0$ donc $P_1\in A$ comme annoncé.}\\
  
  \textbf{Remarque.} La terminologie `critère de réduction' vient du fait qu'on applique en général ce critère avec les morphismes   $p_I: A\twoheadrightarrow A/I$ de réduction modulo un idéal $I\subset A$. En général, on prend même $I=\frak{m}$ maximal,  ce qui permet de se ramener au cas de l'algèbre de polynôme $(A/\frak{m})[X]$ qui est un anneau euclidien puisque $A/\frak{m}$ est un corps. Typiquement, si $A=\Z$, on peut chercher un `bon' nombre premier  $p$ tel que la réduction modulo $p$ de $P\in \Z[X]$ soit irréductible dans  $\Z/p[X]$. On verra dans la partie du cours sur la théorie de Galois, qu'on comprend plutôt bien les   irréductibles de $\Z/p[X]$.\\
  
  \textbf{Application.} Montrer que $P=X^5-5X^3-6X-1$ est irréductible dans $\Q[X]$.\\
  
  Correction. \textit{En considérant $\phi:\Z\twoheadrightarrow \Z/2$, on a $\tilde{\phi}(P)=:\overline{P}=X^5+X^3+1$ dans $\F_2[X]$. Clairement $\overline{P}$ n'a pas de racine dans $\F_2$. Donc si $\overline{P}$ n'est pas irréductible, il s'écrit comme produit d'un polynôme  de degré $2$ et d'un polynôme de degré $3$:
 $$\overline{P}=(X^3+aX^2+bX+c)(X^2+dX+e).$$
 En développant et en identifiant les coefficients, on obtient le systèmes d'équations dans $\F_2$
 $$\begin{tabular}[t]{l}
 $d+a=0$\\
 $e+ad+b=1$\\
 $ae+bd+c=0$\\
 $be+cd=0$
 $ce=1$
 \end{tabular}$$
 Mais dans $\F_2$, $d+a=0$ implique $a=d$. Si $a=d=0$, $c=0$: contradiction. Si $a=d=1$,    $e+b=0$, $e+b+c=0$ donc $c=0$: contradiction.  Cela montre que $\overline{P}$ est irréductible dans $\F_2[X]$. Donc si $P=P_1P_2$ dans $\Z[X]$ avec $deg(P_1)\leq deg(P_2)$, on a forcément $P_1\in \Z$ (et en fait $P_1=\pm 1$ car $C_\Z(P)=1=C_\Z(P_1)C_\Z(P_2)=P_1C_\Z(P_2)$). Si $P=P_1P_2$ dans $\Q[X]$  avec $deg(P_1)\leq deg(P_2)$, on a $C_\Z(P_1)C_\Z(P_2)=C_\Z(P)=1$ donc $P=P_1P_2=\frac{P_1}{C_\Z(P_1)}\frac{P_2}{C_\Z(P_2)}$ avec, cette fois-ci, $\frac{P_1}{C_\Z(P_1)},\frac{P_2}{C_\Z(P_2)}\in \Z[T]$. Donc $P_1=C_\Z(P_1)\in \Q$. Cela montre bien que $P$ est irréductible dans $\Q[X]$.} \\
   \end{enumerate}
 

\subsection{Valuations et anneaux factoriels}\label{Val}Soit $K$ un corps.

\subsubsection{}\label{ValDef}Une \textit{valuation}\index{Valuation discrète} (de rang $1$) sur $K$ est une application surjective\footnote{On fait cette hypothèse par commodité. Il suffit en fait de supposer que $v:K\rightarrow \overline{\Z}$ est non nulle; on peut alors se ramener au cas surjectif en utilisant que  tout sou groupe non-nul de $\Z$ est isomorphe à $\Z$.}
 $v:K\rightarrow \overline{\Z}$  qui vérifie
\begin{enumerate}
\item $v(xy)=v (x)+v (b)$, $x,y\in K$;
\item $v(x+y)\geq \hbox{\rm min}\lbrace v(x),v(y)\rbrace$, $x,y\in K$;
\item $v(x)=\infty$ $\Leftrightarrow$ $x=0$.
\end{enumerate}

\textbf{Remarque.}   La propriété (1) peut se réécrire en disant que $v:(K^\times,\cdot)\rightarrow (\Z,+)$ est un morphisme de groupes.\\

 Notons $A_v:=v^{-1}([0,\infty])\subset K$. On dit qu'un anneau   est \textit{local}\index{Local (Anneau)} s'il possède un unique idéal maximal.
 
 \subsubsection{}\label{AVD}\textbf{Lemme.} \textit{L'ensemble $A_v \subset K$ est un sous-anneau de $K$, de corps des fractions $K$ et tel que $A_v^\times=v^{-1}(0)$ et $\frak{m}_v:= A_v\setminus A_v^\times \subset A_v$ est un idéal. En particulier, $A_v$ est local d'unique idéal maximal $\frak{m}_v$. De plus les seuls idéaux de $A_v$ sont les $\pi^nA_v$, $n\in\Z_{\geq 0}$, o\`u $\pi\in A$ est tel que $v(\pi)=1$.}
 \begin{proof} Montrons d'abord que $A_v\subset K$ est un sous-anneau. D'après la propriété (1) d'une valuation, $1\in A$ (utiliser $1^2=1$) et $a,b\in A_v$ implique $ab\in A$. De plus, pour tout $x\in K^\times$  la relation $(-x)^2=x^2$ et la propriété (1) d'une valuation montrent que $v(x)=v(-x)$ ce qui, combiné à la propriété (2) d'une valuation montre que $a,b\in A_v$ implique $a-b\in A_v$. Observons également que la propriété (1) d'une valuation implique $$A_v^\times=\lbrace x\in K^\times\;|\; x,x^{-1}\in A_v\rbrace =  v^{-1}( 0  ).$$ 
 Les propriétés (1) (respectivement (2)) assurent également que $\frak{m}_v$ est stable par multiplication par les éléments de $A$ (respectivement par différence) donc que $\frak{m}_v\subset A_v$ est un idéal. C'est automatiquement l'unique idéal maximal de $A_v$ puisque $A_v\setminus \frak{m}_v=A_v^\times$. 
 Soit $\pi\in A$ tel que $v(\pi)=1$ (on utilise ici la surjectivité de $v$). Pour un idéal $I\subset A_v$ arbitraire, notons $n:=\hbox{\rm min}v(I)$. On a alors pour tout $a\in I$, $v(\pi^{-n}a)\geq 0$ donc $a\in A_v\pi^n$. Cela montre que $I\subset A\pi^n$. Inversement, soit $a\in I$ tel que $v(a)=n$. On a alors $v(\pi^{-n}a)=0$ \textit{i.e.} $A^\times a=A^\times \pi^n$ donc $A\pi^n=Aa\subset I$. Il reste à voir que $K$ est le corps des fractions de $A_v$; cela résulte du fait que tout $x\in A$ s'écrit sous la forme $x=(x\pi^{-v(x)})\pi^{v(x)}$ avec $ x\pi^{-v(x)}\in A_v^\times$.
 \end{proof}
 
  On dit qu'un anneau de la forme $A_v$ est un \textit{anneau de valuation discrète}\index{Valuation discrète (Anneau)}. Ces anneaux jouent un rôle fondamental en géométrie arithmétique. Ils possèdent plusieurs caractérisations équivalentes. En voici quelques unes. \\
 
 \textbf{Exercice.} [Difficile - \textit{cf.} \cite[I,\S 2]{CL}] Soit $A$ un anneau commutatif. Montrer que les propriétés suivantes sont équivalentes.
 \begin{enumerate}
 \item $A$ est un anneau de valuation discrète. 
 \item $A$ est local, noetherien et son idéal maximal est principal, engendré par un élément non nilpotent.
 \item $A$ est intégralement clos et possède un unique idéal premier non nul.\\
 \end{enumerate}
 
 

 \subsubsection{}Si $A$ est factoriel de corps des fractions $\iota_A:A\hookrightarrow K:=Frac(A)$, les applications $v_p:K\rightarrow \overline{\Z}$, $p\in\mathcal{P}_A$ sont donc des valuations sur $K$. Et la famille de valuations
$$\mathcal{V}:=\lbrace v_p:Frac(A)\rightarrow \overline{\Z}\; |\; p\in\mathcal{P}_A\rbrace $$
 vérifie les propriétés suivantes:
\begin{itemize}
\item (\ref{ValDef}.1) Pour tout $0\not=x\in K$, $$|\lbrace v\in\mathcal{V} \;|\; v (x)\not=0\rbrace|<+\infty;$$
\item (\ref{ValDef}.2) Il existe une famille d'élements $(p_v)_{v\in\mathcal{V}}\in K$ telle que $v(p_w)=\delta_{v,w}$, $v,w\in\mathcal{V}$;
\item (\ref{ValDef}.3) $A=\displaystyle{\bigcap_{v\in\mathcal{V}}}A_v$
\end{itemize}

 Inversement, on a  \\

\subsubsection{}\label{ValProp}\textbf{Proposition.} \textit{Soit $K$ un corps muni d'une famille $\mathcal{V}$ de valuation $v:K\rightarrow \overline{\Z}$ vérifiant les propriétés (\ref{ValDef}.1), (\ref{ValDef}.2). Alors $$A:=\displaystyle{\bigcap_{v\in\mathcal{V}}}v^{-1}(\overline{\N})\subset K$$ est un sous-anneau qui est factoriel et les $p_v$, $v\in\mathcal{V}$ forme un système de représentants de $\mathcal{P}_A^\circ$.}\\

\begin{proof}Observons d'abord que  $A\subset K$ est un sous-anneau comme intersection de sous-anneaux (Lemme \ref{AVD}). La propriété (1) d'une valuation implique également que $$A^\times=\lbrace x\in K^\times\;|\; x,x^{-1}\in A\rbrace =\displaystyle{\bigcap_{v\in\mathcal{V}}v^{-1}(\lbrace 0\rbrace)}.$$  
 Montrons ensuite que les $p_v$, $v\in\mathcal{A}$ sont irréductibles.  Soit donc $v\in \mathcal{V}$. La condition $v(p_v)=1$ assure déjà que $p\notin A^\times$. Ecrivons $p_v=ab$, $a,b\in A$. On doit avoir $v(p_v)=1=v(a)+v(b)$ et $w(p_v)=0=w(a)+w(b)$, $v\not=w\in\mathcal{V}$. Comme par définition de $A$, $w(a),w(b)\geq 0$, $w\in\mathcal{V}$, ces relations impliquent $v(a)=1$ et $v(b)=0$ ou $v(a)=0$ et $v(b)=1$ et $w(a)=w(b)=0$, $v\not=w\in\mathcal{V}$. Donc $a\in A^\times$ ou $b\in A^\times$.\\
 Soit maintenant $0\not=a\in A$. Par (\ref{ValDef}.1), on peut définir  $$u_a:=a\prod_{v\in\mathcal{V}}p_v^{-v(a)}\in K^\times,$$
qui vérifie par construction et la propriété (1) d'une valuation $v(u_a)=0$, $v\in\mathcal{V}$ \textit{i.e.} $u_a\in A^\times$. L'écriture  $a=u_a\prod_{v\in\mathcal{V}}p_v^{v(a)}$ montre déjà que les $p_v$, $v\in\mathcal{V}$ forment  un système de représentants des classes d'irrréductibles de $A$. De plus, l'écriture $a=u_a\prod_{v\in\mathcal{V}}p_v^{v(a)}$ est unique.  Si on a une écriture $a=u\prod_{v\in\mathcal{V}}p_v^{v'(a)}$
avec $u'\in A^\times$, $v_-'(a):\mathcal{V}\rightarrow \N\in\N^{(\mathcal{V})}$, l'égalité
 $$u^{'-1} u_a=\prod_{v\in\mathcal{V}}p_v^{v'(a)-v(a)}\in A^\times$$
implique, par évaluation en chacune des $v\in\mathcal{V}$ et en utilisant (\ref{ValDef}.2) que $v'(a)=v(a)$, $v\in\mathcal{V}$ et donc $u'=u_a$.
\end{proof}

\subsubsection{}\textbf{Exercice.} (ppcm, pgcd) Soit $A$ un anneau factoriel. 
\begin{enumerate}
\item Montrer que $Aa\cap Ab$ est un idéal principal engendré par 
$$ppcm(a, b):=\prod_{p\in\mathcal{P}_A}p^{\hbox{\rm \footnotesize max}\lbrace v_p(a),v_p(b)\rbrace}.$$
On dit que les éléments de $A^\times ppcm(a, b) $ sont les plus petits communs multiples de $a$ et $b$.
\item Montrer que l'ensemble des idéaux principaux de $A$ qui contiennent $Aa+ Ab$ admet un plus petit élément, engendré par
$$pgcd(a, b):=\prod_{p\in\mathcal{P}_A}p^{\hbox{\rm \footnotesize min}\lbrace v_p(a),v_p(b)\rbrace}.$$
On dit que les éléments de $A^\times pgcd(a, b) $ sont les plus grands communs diviseurs de $a$ et $b$. Montrer sur un exemple qu'en général l'inclusion $Aa+Ab\subsetneq Apgcd(a,b)$ est stricte.
\item Généraliser  (1) et (2) à un nombre fini $a_1,\dots, a_r$ d'éléments de $A$.
\item (Bézout) Supposons $A$ principal. Montrer que $pgcd(a_1,\dots, a_r)A^\times= A^\times$ si et seulement si il existe $u_1,\dots, u_r\in A$ tels que $u_1a_1+\cdots+u_ra_r=1$.
\end{enumerate}

 

 
\section{Localisation, anneaux de fractions.}\textit{}\\
 On va maintenant généraliser la construction du corps des fractions d'un anneau intègre à des anneaux non nécessairement intègre. Soit $A$ un anneau commutatif. 

\subsection{}Une \textit{partie multiplicative}   de $A$ est un sous-ensemble $S\subset A\setminus\lbrace 0\rbrace$ stable par multiplication et contenant $1$. \\

\subsubsection{}\label{LocEx1}\textbf{Exemples.}\\

\ref{LocEx1}.1 $S:=A\setminus A_{tors}$; en particulier, si $A$ est intègre, $S:=A\setminus\lbrace 0\rbrace$;\\

\ref{LocEx1}.2 Pour $a\in A\setminus \sqrt{\lbrace 0\rbrace}$, $S_a:=\lbrace a^n\;|\; n\in\N\rbrace$;\\

\ref{LocEx1}.3 Pour $\frak{p}\in spec(A)$, $S_{\frak{p}}:=A\setminus \frak{p}$.


\subsubsection{}\label{LocDef} Soit $S\subset A\setminus\lbrace 0\rbrace$ une partie multiplicative. On munit le produit ensembliste $S\times A$ de la relation $\sim $ définie par: pour tout $(s,a),(s',a')\in S\times A$, $(s,a)\sim (s',a')$ s'il existe $s''\in S$ tel que $s''(s'a-sa')=0$. \\

 On vérifie que $\sim$ est une relation d'équivalence. On remarquera que si $A$ est intègre, on peut, dans la définition de $\sim$, simplifier par $s''$ et la relation $\sim$ devient simplement $(s,a),(s',a')\in S\times A$, $(s,a)\sim (s',a')$ si $s'a-sa'=0$.  Mais on prendra garde que si $A$ n'est pas intègre,  la relation $(s,a)\sim (s',a')$ si $s'a-sa'=0$ n'est pas transitive donc ne définit pas une relation d'équivalence.\\


 On note $S^{-1}A:=S\times A/\sim$ et 
$$ \begin{tabular}[t]{llll}
$-/-$ :&$S\times A$&$\rightarrow$&$S^{-1}A$\\
 &$(s,a)$&$\rightarrow$&$a/s$
 \end{tabular}$$
 la projection canonique.\\
 
 
 
  Considérons les applications $$\begin{tabular}[t]{lclc}
 $+$&$:(S\times A)\times (S\times A)$&$\rightarrow$&$S^{-1}A$\\
 &$((s,a),(t,b))$&$\rightarrow$&$(ta+sb)/(st)$
 \end{tabular},\;\; \begin{tabular}[t]{lclc}
 $\cdot $&$:(S\times A)\times (S\times A)$&$\rightarrow$&$S^{-1}A$\\
 &$((s,a),(t,b))$&$\rightarrow$&$(ab)/(st)$
 \end{tabular}$$
 Si $(s,a)\sim (s',a')$, $(t,b)\sim (t',b')$ \textit{i.e.} il existe $s'',t''\in S$ tels que $s''(s'a-sa')=0$, $t''(t'b-tb')=0$. Comme $s''t''\in S$ par multiplicativité, on a 
  
 $$s''t''(s't'(ta+sb)-st(t'a'+s'b'))=s'' s'a'tt't''-t'bss's''-s''t''st(t'a'+s'b')=s'' sa''tt't''-tb'ss's''-s''t''st(t'a'+s'b')=0$$
 et
 $$s''t''(s't'ab-sta'b')=s''s'at''t'b-s''sa't''tb'=s''sa't''t'b-s''sa't''tb'=s''sa't''(t'b-tb')=0.$$
 Cela montre que les applications $+,\cdot :(S\times A)\times (S\times A) \rightarrow S^{-1}A$  se factorisent en 
 $$\xymatrix{(S\times A)\times (S\times A) \ar[r]^{+,\cdot}\ar[d]_{-/-\times -/-}& S^{-1}A\\
 S^{-1}A\times S^{-1}A\ar[ur]_{+,\cdot}&} $$
 On laisse en exercice le soin de vérifier que $S^{-1}A$ muni des lois $+,\cdot :S^{-1}A\times S^{-1}A\rightarrow S^{-1}A$ ainsi définies vérifie bien les axiomes d'un anneau commutatif  de zéro $0/1$ et d'unité $1/1$ et que, pour cette structure d'anneau, l'application canonique $$\begin{tabular}[t]{lclc}
 $\iota_S:$&$A$&$\rightarrow$&$S^{-1}A$\\
 &$a$&$\rightarrow$&$a/1$
 \end{tabular}$$
 est un morphisme d'anneaux de noyau $\ker(\iota_S)=\lbrace a\in A\; |\; \exists s\in S, \; sa=0\rbrace$. En particulier, si $A$ est intègre (ou plus généralement si $S$ ne contient pas d'éléments de torsion), $ \iota_S:  A\rightarrow S^{-1}A$ est injectif. De plus, $\iota_S(S)\subset (S^{-1}A)^\times$ puisque $s/1\cdot 1/s=s/s=1/1$. \\
 
 \subsubsection{}\label{LocUniv}\textbf{Lemme.} (Propriété universelle de la localisation) \textit{Pour toute partie multiplicative $S\subset A\setminus\lbrace 0\rbrace$  il existe un morphisme d'anneaux $\iota_S:A\rightarrow F$ tel que $\iota(S)\subset F^\times$ et pour tout  morphisme d'anneaux $\phi:A\rightarrow B$ tel que $\phi(S)\subset B^\times $, il  existe un unique morphisme d'anneaux $ phi_S:F\rightarrow B$ tel que $\phi=  \phi_S\circ \iota_S$.}\\
 
 
  Plus visuellement,
 $$\xymatrix{S\ar[r]^\phi\ar@{_{(}->}[d]&B^\times\ar@{_{(}->}[d]\\
 A\ar[r]^{\forall\phi}\ar@{_{(}->}[d]_{\iota_S}&B\\
 F\ar@{.>}[ur]_{\exists ! \phi_S}}$$

 \begin{proof} Montrons que  $S^{-1}A$ muni de la structure d'anneau ci-dessus et le morphisme canonique $\iota_S:A\rightarrow S^{-1}A$ conviennent. Soit $\phi:A\rightarrow B$  un morphisme d'anneaux tel que $\phi(S)\subset B^\times$.  Si $\phi_S:S^{-1}A\rightarrow B$ existe la relation $\phi=  \phi_S\circ \iota_S$ impose que  $\tilde{\phi}:S^{-1}A\rightarrow B$ est unique puisqu'on doit nécessairement avoir 
 $$\phi_S(A/s)=\phi_S((a/1)(1/s))=\phi_S(a/1)\phi_S((s/1)^{-1})=\phi(a)\phi(s)^{-1},\; (s,a)\in S\times A.$$
 Considérons donc l'application  \begin{tabular}[t]{lclc}
 $\phi_S:$&$ S\times A $&$\rightarrow$&$B$\\
 &$(a,s) $&$\rightarrow$&$ \phi(s)^{-1}\phi(a)$.
 \end{tabular} 
 Si $(s,a)\sim (s',a')$  \textit{i.e.} il existe $s''\in S$ tels que $s''(s'a-sa')=0$, on a $\phi(s'')(\phi(s')\phi(a)-\phi(s)\phi(a'))=\phi(s''(s'a-sa'))=\phi(0)=0$. Mais comme $\phi(s),\phi(s'),\phi(s'')\in B^{\times}$, on peut réécrire cette égalité comme 
 $$\phi_S(s,a)=\phi(s)^{-1}\phi(a)=\phi(s')^{-1}\phi(a')=\phi_S(s',a').$$
 Cela montre que l'application $\phi_S: S\times A  \rightarrow  B$ se factorise  en 
 $$\xymatrix{S\times A \ar[r]^{\tilde{\phi}}\ar[d]_{  -/-}& B\\
 S^{-1}A \ar[ur]_{\phi_S}&} $$
Par construction $\phi=  \phi_S\circ \iota_S$ et on vérifie que $\phi_S:S^{-1}A\rightarrow B$ est bien un morphisme d'anneaux. 
  \end{proof}
 
  Comme d'habitude, le morphisme d'anneaux  $\iota_S:A\rightarrow S^{-1}A$ est unique à unique isomorphisme près et on dit que  c'est `la' localisation\index{Localisation (Anneau)} de $A$ en $S$. Localiser $A$ en $S$ revient donc à inverser formellement les éléments de $S$.  \\
 
 \subsubsection{}\label{LocExo1}\textbf{Exercice.} 
 \begin{enumerate}
 \item  Montrer qu'on a un isomorphisme  d'anneaux canonique $S^{-1}(A[X])\tilde{\rightarrow} (S^{-1}A)[X]$.
 \item Soit $p,q$ deux premiers distincts. Déterminer les idéaux premiers $\frak{p}$ de $A:=\Z/pq$ et déterminer dans chaque cas le localisé $(A\setminus \frak{p})^{-1}A$.
 \item Montrer que si $A$ est intègre (resp. réduit, resp. intégralement clos, resp. factoriel) alors $S^{-1}A$ l'est aussi.\\
 \end{enumerate}
 
 
 
 \subsubsection{}\label{LocEx2}\textbf{Exemples.}\\

\ref{LocEx2}.1 On dit que $(A\setminus A_{tors})^{-1}A$ est l'anneau des fractions de $A$. Si $A$ est un anneau intègre, on retrouve le corps  des fractions de $A$. Si $A$ n'est pas intègre, $(A\setminus A_{tors})^{-1}A$ n'est pas un corps (le vérifier sur un exemple).\\

\ref{LocEx2}.2 Pour $a\in A\setminus \sqrt{\lbrace 0\rbrace}$ on note $A_a:=S_a^{-1}A $;\\

\ref{LocEx2}.3 Pour $\frak{p}\in spec(A)$, on note $A_\frak{p}:=S_{\frak{p}}^{-1}A$. Noter que si $A$ est intègre  $\lbrace 0\rbrace \in spec(A)$ et, dans ce cas, $A_{\lbrace 0\rbrace}=Frac(A)$.\\

 \subsubsection{}\label{LocMorphismes}Soit $\phi:A\rightarrow B$ un morphisme d'anneaux et $S\subset A$, $T\subset B$ des parties multiplicatives telles que $\phi(S)\subset T$. On a en particulier $\iota_T\circ \phi(S)\subset \iota_T(T)\subset (T^{-1}B)^\times$ donc par propriété universelle de $\iota_S:A\rightarrow S^{-1}A$ il existe un unique morphisme d'anneaux $\tilde{\phi}:S^{-1}A\rightarrow T^{-1}B$ tel que $\iota_T\circ \phi=\tilde{\phi}\circ \iota_S$; explicitement $\tilde{\phi}(a/s)=\phi(a)/\phi(s)$ dans $T^{-1}B$. Si $\phi:A\rightarrow B$, $\psi:B\rightarrow C$ sont des morphismes d'anneaux et $S\subset A$, $T\subset B$, $U\subset C$ des parties multiplicatives telles que $\phi(S)\subset T$, $\psi(T)\subset U$, on a $S^{-1}(\psi\circ \phi)=(\tilde{\phi})\circ (T^{-1}\psi)$. \\

\textbf{Exemple.} 

\begin{enumerate}
\item Soit $\phi:A\rightarrow B$ un morphisme d'anneaux et $\frak{q}\subset spec(B)$. On a alors $\frak{p}:=\phi^{-1}(\frak{q})\in spec(A)$ et $\phi(A\setminus \frak{p})\subset B\setminus \frak{q}$ donc $\phi:A\rightarrow B$ induit un morphisme d'anneaux canonique $\phi_\frak{p}:A_{\frak{p}}\rightarrow B_{\frak{q}}$. 
 \item Si $A$ est intègre,  $\lbrace 0\rbrace\in spec(A)$ et pour toute partie multiplicative $S\subset A\setminus \lbrace 0\rbrace$, en appliquant ce qui précède à $\phi=Id:A\rightarrow A$, $S=S$, $T=A\setminus \lbrace 0\rbrace$, on obtient un morphisme canonique $\phi:S^{-1}A\rightarrow A_{\lbrace 0\rbrace}=Frac(A)$ dont on vérifie immédiatement qu'il est injectif. 
\end{enumerate}
\subsection{Idéaux}\label{LocIdeal} 

 Soit $S\subset A$ une partie multiplicative. Pour un sous-ensemble $X\subset A$, notons $$S^{-1}X:=\lbrace  a/s\;|\; a\in X,\; s\in S\rbrace \subset S^{-1}A. $$
On vérifie immédiatement que si $I\subset A$ est un idéal alors $S^{-1}I\subset S^{-1}A$ est aussi un idéal. On a donc une application bien définie et croissante pour $\subset$
$$S^{-1}:(\mathcal{I}_A,\subset)\rightarrow (\mathcal{I}_{S^{-1}A},\subset).$$
Dans l'autre direction on a l'application $$\iota_S^{-1}:(\mathcal{I}_{S^{-1}A},\subset)\rightarrow (\mathcal{I}_A,\subset)$$ induite  par le morphisme de localisation $\iota_S:A\rightarrow S^{-1}A$. 
\begin{itemize}[leftmargin=* ,parsep=0cm,itemsep=0cm,topsep=0cm] 
\item Pour $I\subset A$ un idéal, on a 
$$\iota_S^{-1}S^{-1}I=\lbrace a\in A\;|\; a/1\in S^{-1}I\rbrace= \lbrace a\in A\;|\;  Sa\cap I\not=\emptyset\rbrace=\bigcup_{s\in S}(s\cdot)^{-1}I.$$
En particulier, $S^{-1}I=S^{-1}A$ (si et seulement si $\iota_S^{-1}S^{-1}I=A$) si et seulement si $S\cap I\not=\emptyset$.\\
\item Pour $I\subset S^{-1}A$ un idéal, on a 
$$S^{-1}\iota_S^{-1}I=\lbrace a/s\in S^{-1}I\;|\; a\in \iota_S^{-1}I\rbrace\supset I$$
et comme  pour tout $a/s\in I$ on a $a/1= (s/1)^{-1}(a/s)\in I$ donc $a\in \iota_S^{-1}I$, on a en fait $S^{-1}\iota_S^{-1}I=I$.\\
 \end{itemize}
 
 On a donc montré: 
 
\subsubsection{}\label{LocIdeaux}\textbf{Lemme.} \textit{L'application $S^{-1}:(\mathcal{I}_A,\subset)\rightarrow (\mathcal{I}_{S^{-1}A},\subset) $ est surjective, croissante pour $\subset$ et se restreint en une surjection    $$S^{-1}:\lbrace I\in\mathcal{I}_A\;|\; I\cap S=\emptyset\rbrace \twoheadrightarrow  \mathcal{I}_{S^{-1}A}\setminus \lbrace S^{-1}A\rbrace . $$
L'application $\iota_S^{-1}:(\mathcal{I}_{S^{-1}A},\subset) \rightarrow  (\mathcal{I}_A,\subset)$ est injective,  croissante pour $\subset$  et induit une bijection  
$$\iota_S^{-1}:\mathcal{I}_{S^{-1}A} \tilde{\rightarrow} \lbrace I\in\mathcal{I}_A\;|\; I=\bigcup_{s\in S}(s\cdot)^{-1}I\rbrace .$$}


\subsubsection{}\textbf{Lemme.} \textit{Les applications $S^{-1}: \mathcal{I}_A \rightarrow  \mathcal{I}_{S^{-1}A}  $ et  $\iota_S^{-1}: \mathcal{I}_{S^{-1}A} \rightarrow \mathcal{I}_A $ se restreignent en des bijections inverses l'une de l'autres
$$\xymatrix{spec(S^{-1}A) \ar@<1ex>[r]^{\iota_S^{-1}}& \lbrace \frak{p}\in spec(A)\;|\; \frak{p}\cap S=\emptyset\rbrace \ar@<1ex>[l]^{S^{-1}} }$$} 
\begin{proof} Si $\frak{p}\in spec(A)$ est tel que $S\cap \frak{p}=\emptyset$ alors $\frak{p}=\bigcup_{s\in S}(s\cdot)^{-1}\frak{p}$ (si $s\in S$, $a\in \frak{p}$, $sa\in\frak{p}$ implique $a\in\frak{p}$) donc   $\iota_S^{-1}S^{-1}\frak{p}=\frak{p}$. Comme on a toujours $S^{-1}\iota_S^{-1}=Id$, et $\iota_S^{-1}spec(S^{-1}A)\subset spec(A)$, il reste seulement à montrer que si $\frak{p}\in spec(A)$ est tel que $S\cap \frak{p}=\emptyset$ alors  $S^{-1}\frak{p}\in spec(S^{-1}A)$. Soit donc $\frak{p}\in spec(A)$ et $a/s,b/t\in S^{-1}A$ tels que $(ab)/(st)\in S^{-1}\frak{p}$ \textit{i.e.} il existe $p\in \frak{p}$ et $u,v\in S$ tels que $v(uab-stp)=0$ ou encore $vuab=vstp\in\frak{p}$. Mais comme $\frak{p}\in spec(A)$ et $vu\notin\frak{p}$, on a $ab\in\frak{p}$ donc $a\in\frak{p}$ ou $b\in\frak{p}$.   \end{proof}

\textbf{Exemple.} Si $\frak{p}\in spec(A)$,  $A_{\frak{p}}$ est local d'unique idéal maximal $\frak{p}A_{\frak{p}}$. Le corps $\kappa(\frak{p}):=A_{\frak{p}}/\frak{p}A_{\frak{p}}$ est appelé le corps résiduel de $spec(A)$ en $\frak{p}$. Si on reprend les notations de l'Exemple \ref{LocMorphismes}, le morphisme $\phi:A_{\frak{p}}\rightarrow B_{\frak{q}}$ envoie $\frak{p}$ dans $\frak{q}$ donc induit par passage au quotient un morphisme de corps - nécessairement injectif $\kappa(\frak{p})\hookrightarrow \kappa(\frak{q})$. \\

 
\subsubsection{}\textbf{Corollaire.} \textit{Si $A$ est noetherien (resp. principal) alors $S^{-1}A$ l'est aussi.}
 

\subsubsection{}\textbf{Exercice.} 
\begin{enumerate} 
\item Soit $\frak{p}\in spec(A)$. Montrer qu'on a un morphisme d'anneaux canonique injectif $A/\frak{p}\rightarrow \kappa(\frak{p})$. Montrer que si $\frak{p}$ est maximal, ce morphisme est un isomorphisme.
\item Montrer que les localisés d'un anneau principal en ses idéaux premiers sont des anneaux de valuation discrète.
\item Si $I,J\subset A$ sont des idéaux, montrer que  $S^{-1}(I\cap J)=S^{-1}I\cap S^{-1}J$ et $S^{-1}(I+J)=S^{-1}I+S^{-1}J$. 
\item  Si $I\subset J$ sont des idéaux et si on note $\overline{S}\subset A/I$ l'image de $S$ \textit{via} la projection canonique $A\twoheadrightarrow A/I$, montrer qu'on a un isomorphisme canonique $$S^{-1}I/S^{-1}J\tilde{\rightarrow}\overline{S}^{-1}(I/J).$$
 \end{enumerate} 
 \section{Completion (Hors programme)}
 \subsection{Limites projectives}
 Un système projectif \index{Système Projectif (Ensembles)} d'ensembles est une suite d'applications ensemblistes $$(X_\bullet,\phi_\bullet)\;\; \cdots X_{n+1}\stackrel{\pi_{n+1}}{\rightarrow} X_n\stackrel{\pi_{n }}{\rightarrow}X_{n-1}\stackrel{\pi_{n-1}}{\rightarrow}\cdots \stackrel{\pi_{1}}{\rightarrow}X_0.$$
  Etant donné un système projectif $(X_\bullet,\phi_\bullet)$ d'ensembles, on note
  $$\lim X_n :=\lbrace \underline{x}=(x_n)_{n\geq 0}\in\prod_{n\geq 0}X_n\; |\; \pi_{n+1}(x_{n+1})=x_n,\; n\geq 0\rbrace\subset \prod_{n\geq 0}X_n$$
  et pour chaque $m\geq 0$, on note $p_m:\lim X_n\rightarrow X_m$ la  restriction à $\lim X_n$ de la $m$ième projection $p_m:\prod_{n\geq 0}X_n\rightarrow X_m$.
  
 \subsubsection{}\label{LimProj}\textbf{Lemme.} (Propriété universelle de la limite projective) \textit{Pour tout système projectif $(X_\bullet,\phi_\bullet)$ d'ensembles  il existe des applications ensemblistes $p_m:P\rightarrow X_m$, $m\geq 0$ telles que pour toute famille  d'applications ensemblistes $\psi_m:Y\rightarrow X_m$, $m\geq 0$ telles que $\phi_{m+1}\circ \psi_{m+1}=\psi_m$, il existe une unique application ensembliste $\psi:Y\rightarrow \lim X_n$ telle que $p_m\circ \psi=\psi_m$, $m\geq 0$.}\\
 
  Plus visuellement
 $$\xymatrix{
 &&&X_{m+1}\ar[dd]^{\phi_{m+1}}\\
 Y\ar@{.>}[rr]^{\exists ! \psi}\ar[urrr]^{\psi_{m+1}}\ar[drrr]_{\psi_m}&&\lim X_n\ar[ur]^{p_{m+1}}\ar[dr]_{p_m}&\\
 &&&X_m
 }$$
 \begin{proof} Comme d'habitude, on montre que les $p_m:\lim X_n\rightarrow X_m$, $m\geq 0$ vérifie la propriété universelle. La condition  $\phi_{m+1}\circ \psi_{m+1}=\psi_m$, $m\geq 0$ impose que si $\psi:Y\rightarrow  \lim X_n$ existe, elle est unique, définie par 
  $$\psi:Y\rightarrow\prod_{n\geq 0}X_n,\; y\rightarrow (\psi_n(y))_{n\geq 0}.$$
  On vérifie ensuite immédiatement que $\psi(Y)\subset  \lim X_n$ et que $p_m\circ \psi=\psi_m$, $m\geq 0.$
 \end{proof}
  Comme d'habitude, la suite d'applications  $p_m:\lim X_n\rightarrow X_m$, $m\geq 0$ est unique à unique isomorphisme près et on dit que  c'est `la' limite projective \index{Limite Projective (Ensemble)}  de $(X_\bullet,\phi_\bullet)$.  \\
 
 
  Si les applications $\phi_{n+1}:X_{n+1}\rightarrow X_n$, $n\geq 0$ sont des morphismes de monoïdes (resp. de groupes, resp. d'anneaux), on verifie immédiatement que   $ \lim X_n\subset \prod_{n\geq 0} X_n$ est un sous-monoïde  (resp. unesous- groupe, resp. un sous-anneaux) et que les projections  $p_m:\prod_{n\geq 0}X_n\rightarrow X_m$, $m\geq 0$ sont des morphismes de monoïdes (resp. de groupes, resp. d'anneaux). Le Lemme \ref{LimProj} admet la variante suivante dont on laisse la preuve en exercice au lecteur.
 
  \subsubsection{}\label{LimProj}\textbf{Lemme.}   \textit{Pour tout système projectif $(X_\bullet,\phi_\bullet)$ de monoïdes (resp. de groupes, resp. d'anneaux),  il existe des morphismes de monoïdes (resp. de groupes, resp. d'anneaux) $p_m:P\rightarrow X_m$, $m\geq 0$ telles que pour toute famille  de morphismes de monoïdes (resp. de groupes, resp. d'anneaux) $\psi_m:Y\rightarrow X_m$, $m\geq 0$ telles que $\phi_{m+1}\circ \psi_{m+1}=\psi_m$, il existe un  unique morphismes de monoïdes (resp. de groupes, resp. d'anneaux) $\psi:Y\rightarrow \lim X_n$ tel  que $p_m\circ \psi=\psi_m$, $m\geq 0$.}\\
  
  
 \subsection{}Soit $A$ un anneau commutatif et  $$A:=I_0\supset  I_1\supset I_2\supset \cdots \supset I_n\supset I_{n+1}\supset \cdots$$
une suite décroissante d'idéaux  tels que $I_m I_n\subset I_{m+n}$. 
Par définition, la projection canonique $p_n:A\rightarrow A/I_n$ se factorise en 
$$\xymatrix{A\ar[r]^{p_n}\ar[d]^{p_{n+1}}&A/I_n\\
A/I_{n+1}\ar[ur]_{\pi_{n+1}}&}$$
d'o\`u un système projectif de morphismes d'anneaux 
$$\cdots  A/I_{n+1}\stackrel{\pi_{n+1}}{\rightarrow} A/I_n\stackrel{\pi_{n }}{\rightarrow}A/I_{n-1}\stackrel{\pi_{n-1}}{\rightarrow}\cdots \stackrel{\pi_1}{\rightarrow}A/I$$
et, par propriété universelle de la limite projective, un unique morphisme d'anneaux $$c_I:A\rightarrow \widehat{A}:=\lim A/I_n.$$
 On note $$\widehat{I}_n:=\lbrace \underline{a}\in \widehat{A}\; |\; a_m=0,\; m\leq n\rbrace $$
 \subsubsection{}Toute suite décroissante d'idéaux 
  $$A:=I_0\supset  I_1\supset I_2\supset \cdots \supset I_n\supset I_{n+1}\supset \cdots$$
  tels que $I_m I_n\subset I_{m+n}$
 munit $A$ d'une   topologie définie par les systèmes fondamentaux de voisinages $a+I_n$, $n\geq 0$. Pour cette topologie, $+,\cdot: A\times A\rightarrow A$ sont continues. Une suite de Cauchy dans $A$ est alors une suite $\underline{a}\in A^\N$ telle que pour tout $N\geq 0$ il  existe $n\geq 0$ tel que $a_{n+p}-a_n\in I_N$, $p\geq 0$. Si toute suite de Cauchy est convergente dans $A$, on dit que $A$ est complet. On laisse la preuve du lemme suivant en exercice.\\
 
 
 \textbf{Lemme.} \textit{Avec les notations ci-dessus,  le morphisme canonique d'anneaux $c_I:A\rightarrow \widehat{A}$ est continu (pour les topologies défines par les suites $I_n$, $n\geq 0$ et $\widehat{I}_n$, $n\geq 0$). De plus, $\widehat{A}$ est complet, séparé et $c_I:A\rightarrow \widehat{A}$ induit des isomorphismes canoniques $A/I_n\tilde{\rightarrow} \widehat{A}/\widehat{I}_n$.}\\
 
    On dit que   $c_I:A\rightarrow \widehat{A}$ est `la' completion de $A$ pour la topologie définie par la suites  $I_n$, $n\geq 0$ (ce morphisme vérifie une propriété universelle que le lecteur devrait à peu près deviner mais que nous ne formulerons pas).\\

 \subsubsection{}Le cas le plus fréquent d'application de la construction ci-dessus est pour $I_n=I^n $, $n\geq 0$ et $I\subset A$ un  idéal. On parle alors de topologie $I$-adique et de completion $I$-adique. Voici deux exemples importants.
 
 
 \begin{enumerate}[leftmargin=* ,parsep=0cm,itemsep=0cm,topsep=0cm] 
 \item $A=\Z$, $I=p\Z$ pour $p$ un nombre premier. Dans ce cas on note $\widehat{\Z}:= \Z_p$ et on dit que $\Z\rightarrow  \Z_p$ est  la complétion $p$-adique de $\Z$ (ou l'anneau des entiers $p$-adiques). Si on munit $\Z$ de la valeur absolue $p$-adique définie par  $|n|=p^{-v_p(n)}$, on peut vérifier que    $\Z\rightarrow  \Z_p$ est la complétion de $\Z$ (au sens usuel des espaces métriques) pour la distance $d_p(m,n)=|m-n|_p$).\\
 
 \textbf{Remarque.} On peut montrer (théorème d'Ostrowski) que les seules valeurs absolues sur $\Q$ sont (à équivalence près) la valeur absolue usuelle est les valeurs absolues $p$-adiques.\\
 
 
 
   \textbf{Exercice.} Montrer que si $n\in \Z$ est premier à $p$ alors $c_{p\Z}(n)\in \Z_p^{\times}$. En déduire qu'on a un isomorphisme canonique $\widehat{\Z_{p\Z}}\tilde{\rightarrow} \Z_p$, o\`u $\widehat{\Z_{p\Z}}\rightarrow \widehat{\Z_{p\Z}}$ est la complétion $p\Z_{p\Z}$-adique de $\Z_{p\Z}$.\\
 
 \item Soit $A$ un anneau commutatif intègre. $A=A[X]Z$, $I=XA[X]$. Plus précisément, reprenons les notations du paragraphe \ref{Poly}.   On munit $A^\N$ des lois $+,\cdot: A^\N\times A^\N\rightarrow A^\N$ définies par 
 $$\underline{a}+\underline{b}=(a_n+b_n)_{n\geq 0},\;\; \underline{a}\cdot\underline{b}=(\sum_{0\leq k\leq n}a_kb_{n-k}).$$
 On vérifie facilement que $(A^\N,+,\cdot)$ est un anneau de zéro la suite nulle et d'unité la suite $e_0$. On note cet anneau  $A[[X]]$  et  ses éléments  $\underline{a}=(a_n)_{n\geq 0} \sum_{n\geq 0}a_n X^n$. L'inclusion naturelle $A^{(\N)}\hookrightarrow A^\N$ induit un morphisme d'anneaux $A[X]\hookrightarrow A[[X]]$, dont on vérifie facilement que c'est la complétion de $A[X]$ par rapport à l'idéal $XA[X]$. On dit que $A[[X]]$ est l'anneau des séries formelles de $A$ en l'indéterminée $X$.\\
 
 
 
   \textbf{Exercice.} Montrer que si $P\in A[X]$ est premier à $X$ alors $c_{XA[X]}(P)\in A[[X]]^{\times}$. En déduire qu'on a un isomorphisme canonique $\widehat{A[X]_{XA[X]}}\tilde{\rightarrow} A[[X]]$, o\`u $A[X]\rightarrow  A[[X]]$ est la complétion $XA[X]_{XA[X]}$-adique de $A[X]_{XA[X]}$.\\

 \end{enumerate}
 \section{Un peu de géométrie (Hors programme)}
 

 
 


 
\part{Modules sur un anneaux}
\textit{}\\
 On rappelle que sauf mention explicite du contraire tous les anneaux sont commutatifs.
\section{Premières définitions et constructions}
\subsection{Définitions}
\subsubsection{}Soit $A$ un anneau, un $A$-module\index{Module} (à gauche) est un couple$((M,+),\cdot)$ formé d'un groupe abélien  $(M,+)$ (on notera $0$ son élément neutre et $-m$ l'inverse d'un élément $m\in M$)) et d'une application $ \cdot :A\times A\rightarrow A$ - appelées la multiplication extérieure -  vérifiant les axiomes suivants:
\begin{enumerate} 
\item $a\cdot (m+n)=a\cdot m+a\cdot n $, $a\in A$, $m,n\in M$;
\item $(a+b)\cdot m=a\cdot m+b\cdot m$, $a,b\in A$, $m\in M$;
\item $(a\cdot b)\cdot m=a\cdot (b\cdot m)$, $a,b\in A$, $m\in M$;
\item $1\cdot m=m$, $m\in M$. \\
\end{enumerate}
De fa\c{c}on équivalente, l'application $A\rightarrow End_{Grp}(M)$ est un morphisme d'anneaux.\\

 Etant donnés deux  $A$-modules $M,N$, un morphisme de $A$-modules est un morphisme de groupes $f:(M,+)\rightarrow (N,+)$   $A$-linéaire \textit{i.e} qui vérifie:
  $$f(a\cdot m)=a\cdot f(m),\; a\in A,\; m\in M.$$ 
On remarquera que l'application identité $Id:M\rightarrow M$ est un morphisme de $A$-modules et que si $f:M\rightarrow N$ et $g:N\rightarrow P$ sont des morphismes de $A$-modules alors $g\circ f:M\rightarrow P$ est un morphisme de $A$-modules. On notera $Hom_A(M,N)$ l'ensemble des morphismes de $A$-modules $\phi:M\rightarrow N$ et, si $M=N$, $End_A(M):=Hom_A(M,M)$. \\

  

 On dit qu'un morphisme de $A$-modules $f:M\rightarrow N$ est injectif, (resp. surjectif, resp. un isomorphisme) si l'application d'ensemble sous-jacente est injective (resp. surjective, resp. bijective). On vérifie que si   $f:M\rightarrow N$ est un isomorphisme de $A$-modules l'application inverse $f^{-1}:N\rightarrow M$ est automatiquement un morphisme de $A$-modules.

 \subsubsection{Exemples}
\begin{itemize}[leftmargin=* ,parsep=0cm,itemsep=0cm,topsep=0cm]
\item  Si $A=\Z$, les $\Z$-modules sont les groupes abéliens.\\
\item  Si $A=k$ est un corps commutatif, les $k$-modules sont les $k$-espaces vectoriels.\\
\item On peut toujours voir un anneau $A$ comme un $A$-module sur lui-même en prenant pour multiplication extérieure le produit $\cdot:A\times A\rightarrow A$. Cet exemple qui semble tautologique est en fait fondamental! On va s'en rendre compte rapidement. Plus généralement, tout idéal $I\subset A$ muni de $\cdot:A\times I\rightarrow I$ induite par le produit de $A$ est un $A$-module.
\item Si $N,N$ sont deux $A$-modules, $Hom_A(M,N)$ est naturellement muni d'une structure de $A$-module pour les lois $(f+g)(m)=f(m)+g(m)$, $(a\cdot f)(m)=a\cdot (f(m))$.  
\item Si $\phi:A\rightarrow B$ est un morphisme d'anneaux tout $B$-module $M$ est naturellement un $A$-module pour la multiplication extérieure $ A\times M\rightarrow M$, $(a,n)\rightarrow \phi(a)\cdot n$. On notera $\phi^*M$ ou $M|_A$ lorsqu'il n'y a pas d'ambiguité sur $\phi:A\rightarrow B$ le $A$-module ainsi obtenu à partir du $B$-module $N$. On notera que tout morphisme de $B$-modules $f:M\rightarrow N$ est automatiquement un morphisme de $A$-modules $f|_A=f:M|_A\rightarrow N_A$. En particulier,  une structure de $A$-algèbre $\phi:A\rightarrow B$ sur un anneau $B$ détermine une structure de $A$-module $\phi^*B$ sur $B$. Inversement, une structure de $A$-module $\cdot:A\times B\rightarrow B$ sur le   groupe abélien sous-jacent $(B,+)$ d'un anneau  $B$ détermine une structure de $A$-algèbre $\phi:A\rightarrow B$ sur $B$ en posant $\phi(a)=a\cdot 1_B$. En particulier, si $M$ est un $A$-module, $End_A(M)$ est naturellement muni d'une structure de $A$-algèbre.  
\item Soit $A$ un anneau commutatif. Par la propriété universelle de $\iota_A:A\rightarrow A[X_1,\dots, X_n]$, la donnée d'un $A[X_1,\dots,X_1]$-module est équivalente à la donnée d'un couple $(M,\underline{\phi})$, o\`u $M$ est un $A$-module et $\underline{\phi}:=(\phi_{1},\dots,\phi_{n})$ est un $n$-uplet d'endomorphismes $A$-linéaires de $M$ qui commutent deux à deux. Par exemple, si $V$ est un $k$-espace vectoriel de dimension finie, et $u\in \hbox{\rm End}_k(V)$, on peut munir $V$ de la structure $V_u$ de $k[X]$-module définie par $P(X)\cdot v=P(u)(v)$. Si $u,u'\in\hbox{\rm End}_k(V) $, on a 
$$\hbox{\rm Hom}_{k[X]}(V_u,V_{u'})=\lbrace \varphi:V\rightarrow V\; |\; \varphi\circ u=u'\circ \varphi\rbrace.$$
Un certain nombre de résultats d'algèbre lin\'raire s'interprètent (et deviennent bien plus naturels!) en termes de $k[X]$-modules.\\
\end{itemize} 

 \subsubsection{}Si $M$ est un $A$-module, un \textit{sous $A$-module} de $M$ est un sous-ensemble $M'\subset M$ tel que $am'+bn'\in M'$, $a,b\in A $, $m',n'\in M'$. \\
 
 \textbf{Exemple.}
 \begin{itemize}[leftmargin=* ,parsep=0cm,itemsep=0cm,topsep=0cm] 
 \item Les sous-$A$-modules du $A$-module régulier $A$ sont les idéaux de $A$.\\
\item Si $f:M\rightarrow N$ est un morphisme de $A$-module et $M'\subset M$ (resp. $N'\subset N$) est un sous-$A$-module alors $f(M')\subset N$ (resp. $f^{-1}(N')\subset M$) est un sous-$A$-module. En particulier, $\hbox{\rm im}(f)\subset N$ et $\ker(f)\subset M$ sont des sous-$A$-modules.\\
\item Si $I\subset A$ est un idéal et $M$ un $A$-module, $IM:=\lbrace am\; |\; a\in I,\; m\in M\rbrace\subset M$ est un sous-$A$-module.

\end{itemize} 

 

\subsection{Produits et sommes directes}\index{Somme directe (Modules)}\index{Produit (Modules)}
 Soit $M_{i}$, $i\in I$ une famille de $A$-modules.\\

  On munit le groupe abélien produit $\prod_{i\in I}M_{i}$ de la structure de $A$-module
$$\begin{tabular}[t]{cll}
$A\times \prod_{i\in I}M_{i}$&$\rightarrow$&$\prod_{i\in I}M_{i}$\\
$(a,\underline{m}=(m_{i})_{i\in I})$&$\rightarrow$&$a\cdot\underline{m}=(a\cdot m_{i})_{i\in I}$.
\end{tabular}$$
Avec cette structure de $A$-module, les projections canoniques $p_{j}:\prod_{i\in I}M_{i}\rightarrow M_{j}$, $j\in I$ deviennent des morphismes de $A$-modules.\\

  On note $\bigoplus_{i\in I}M_{i}\subset \prod_{i\in I}M_{i}$ le sous $A$-module des $\underline{m}=(m_{i})_{i\in I}$ tels que 
 $$|\lbrace i\in I\; |\; m_{i}\not=0\rbrace|<+\infty.$$
 Les injections canoniques $\iota_{j}:M_{j}\hookrightarrow \bigoplus_{i\in I}M_{i}$, $j\in I$ sont des morphismes de $A$-modules. Si $I$ est fini, on a tautologiquement $\bigoplus_{i\in I}M_{i}= \prod_{i\in I}M_i$.\\
 
 \textbf{Lemme.} (Propriété universelle du produit et de la somme directe) \textit{Pour toute famille   $M_{i}$, $i\in I$  de $A$-modules, il existe des morphismes de $A$-modules $p_i:\Pi\rightarrow M_i$, $i\in I$ et $\iota_i:M_i\rightarrow \Sigma$, $i\in I$ tels que
 \begin{enumerate}
 \item Pour toute famille de morphismes de $A$-modules $f_{i}:M\rightarrow M_{i}$, $i\in I$ il existe un unique morphisme de $A$-modules $f:M\rightarrow \Pi$ tel que $p_{i}\circ f=f_{i}$, $i\in I$. 
 \item  Pour toute famille de morphismes de $A$-modules $f_{i}:M_{i}\rightarrow M$, $i\in I$ il existe un unique morphisme de $A$-modules $f:\Sigma \rightarrow M$ tel que $f\circ\iota_{i}=f_{i}$, $i\in I$. 
 \end{enumerate}}
 
\begin{proof} On vérifie comme d'habitude que les morphismes de $A$-modules $p_{j}:\prod_{i\in I}M_{i}\rightarrow M_{j}$, $j\in I$ et  $\iota_{j}:M_{j}\hookrightarrow \bigoplus_{i\in I}M_{i}$, $j\in I$ construits ci-dessus conviennent.  \end{proof}
 
 On peut aussi réécrire \ref{ProduitSommeDirecte} en disant que, pour tout $A$-module $M$ les morphismes canoniques
$$\hbox{\rm Hom}_{A}(M,\prod_{i\in I}M_{i})\rightarrow \prod_{i\in I}\hbox{\rm Hom}_{A}(M,M_{i}), \; f\rightarrow (p_i\circ f)_{i\in I}$$
$$\hbox{\rm Hom}_{A}(\oplus_{i\in I}M_{i},M)\rightarrow \prod_{i\in I}\hbox{\rm Hom}_{A}(,M_{i}M), \; f\rightarrow ( f\circ \iota_i)_{i\in I}$$
sont des isomorphismes ou encore, plus visuellement:
$$\xymatrix{&&M_i&M_i\ar[dr]^{\iota_i}\ar@/^1pc/[drr]^{f_i}&&\\
M\ar@{.>}[r]^{\exists ! f}\ar@/^1pc/[urr]^{f_i}\ar@/_1pc/[drr]_{f_j}&\prod_{i\in I}M_i\ar[ur]^{p_i}\ar[dr]_{p_j}&&&\oplus_{i\in I}M_i \ar@{.>}[r]^{\exists ! f}&M\\
&&M_j&M_j\ar[ur]_{\iota_j}\ar@/_1pc/[urr]_{f_j}&&}$$ 

  Comme d'habitude, le produit $p_{j}:\prod_{i\in I}M_{i}\rightarrow M_{j}$, $j\in I$ et la somme directe $\iota_{j}:M_{j}\hookrightarrow \bigoplus_{i\in I}M_{i}$, $j\in I$ sont uniques à unique isomorphisme près. \\

 


 Si $M_{i}=M$ pour tout $i\in I$, on notera $M^{I}:=\prod_{i\in I}M_{i}$ et $M^{(I)}:=\oplus_{i\in I}M_{i}$. Par construction, on a des isomorphismes   canoniques
$$ \hbox{\rm Hom}(A^{(I)},-)\simeq \prod_{i\in I}\hbox{\rm Hom}(A,-)\simeq (-)^{I}$$
et on dit que $A^{(I)}$ est le \textit{$A$-module libre de base $I$}\index{Libre (Module)}.\\


 Soit $f_i:M_i\rightarrow N_i$, $i\in I$ une famille de morphismes de $A$-modules. En appliquant la propriété universelle des $p_j:\prod_{i\in I}N_i\rightarrow N_j$, $j\in I$ à la famille de morphismes de $A$-modules 
$$ \prod_{i\in I}M_i\stackrel{p_j}{\rightarrow}M_j\stackrel{f_j}{\rightarrow} N_j,\; j\in I$$
on obtient un unique morphisme de $A$-modules $f:=\prod_{i\in I}f_i:\prod_{i\in I}M_i\rightarrow \prod_{i\in I}N_i$ tel que $p_i\circ f=f\circ p_i$, $i\in I$.   De même,  en appliquant la propriété universelle des $\iota_j:M_j\rightarrow \oplus_{i\in I}M_i$, $j\in I$ à la famille de morphismes de $A$-modules 
$$ M_j\stackrel{f_j}{\rightarrow}N_j\stackrel{\iota_j}{\rightarrow} \oplus_{i\in I}M_i,\; j\in I$$
on obtient un unique morphisme de $A$-modules $f:=\oplus_{i\in I}f_i:\oplus_{i\in I}M_i\rightarrow \oplus_{i\in I}N_i$ tel que $  f\circ \iota_i=  \iota_i\circ f$, $i\in I$.  

\subsection{Sous-module engendré par une partie, sommes}  Si $M_{i}\subset M$, $i\in I$ est une famille de sous $A$-modules de $M$, on vérifie immédiatement que l'intersection $$\bigcap_{i\in I}M_{i}\subset M$$ est encore un sous-$A$-module de $M$.\\

 Si $X\subset M$ est un sous-ensemble, on note
$\langle X\rangle$ l'intersection de tous les sous $A$-modules $M'\subset M$ contenant $X$. D'après ce qui précède, c'est encore un sous $A$-module de $M$
et, par construction, c'est le plus petit sous $A$-module de $M$ contenant $X$. On dit que $\langle X\rangle$ est le \textit{sous $A$-module} engendré par $X$ et on vérifie qu'il coincide avec l'ensemble des éléments de la forme $\sum_{x\in X}a(x)x$, o\`u $a:X\rightarrow A$ est une application à support fini. La propriété universelle de $\iota_x:A\hookrightarrow A^{(X)}$, $x\in X$ appliquée aux morphismes  de $A$-modules $-\dot x: A\rightarrow M$, $x\in X$ nous donne un unique morphisme de $A$-modules  $p_X:A^{(X)}\rightarrow M$ tel que $p\circ \iota_x(a)=ax$, $x\in X$. On vérifie immédiatement que les propri\ 'etés suivantes sont équivalentes:
\begin{enumerate}
\item  $M=\langle X\rangle$;
\item Le morphsime de $A$-modules $p:A^{(X)}\rightarrow M$ est surjectif.
\end{enumerate}
On dit alors que $X$ est un système de générateurs de $M$ comme $A$-modules (ou que $M$ est engendré par $X$ comme $A$-module).   Si on peut prendre $X$ fini, on dit que $M$ est un $A$-module \textit{de type fini}\index{Type fini (module)}.\\


 Si $M_{i}\subset M$, $i\in I$ est une famille de sous $A$-modules de $M$, on note $$\sum_{i\in I}M_{i}=\langle \bigcup_{i\in I}M_{i}\rangle\subset M.$$
Là encore la propriété universelle de $\iota_i:M_i\hookrightarrow \oplus_{i\in I}M_i$, $i\in I$ appliquée aux morphismes  de $A$-modules $M_i\subset \sum_{i\in I}M_i$ (inclusion), $i\in I$ nous donne un unique morphisme de $A$-modules - automatiquement surjectif - $p:\oplus_{i\in I}M_i\twoheadrightarrow \sum_{i\in I}M_i(\subset M)$ tel que $p\circ \iota_i(m_i)=m_i$, $m_i\in M_i$, $i\in I$. 

\subsection{Quotients}\index{Noyau (module)}\index{Conoyau (module)}\index{Quotient (module)}
 Soit $M'\subset M$ un sous $A$-module. C'est en particulier un sous groupe abélien et on dispose donc du quotient $p_M:=\overline{(-)}:M\rightarrow M/M'$ comme groupe  abélien . On peut munir $M/M'$ d'une structure de $A$-module comme suit. Pour tout $a\in A$, l'application
$$\begin{tabular}[t]{llll}
$\mu_{a}$:&$M$&$\rightarrow$&$M/M'$\\
&$m$&$\rightarrow$&$\overline{a\cdot m}$
\end{tabular}$$
est un morphisme de groupes ab\' eliens tel que $M'\subset \ker(\mu_{a})$; il se factorise donc en
$$\xymatrix{M\ar[r]^{\mu_{a}}\ar[d]_{\overline{(-)}}&M/M'\\
M/M'\ar[ur]_{\overline{\mu}_{a}}&}$$
On pose alors $$\begin{tabular}[t]{cll}
$A\times M/M'$&$\rightarrow$&$M/M'$\\
$(a,\overline{m})$&$\rightarrow$&$a\cdot\overline{m}:=\overline{\mu}_{a}(m)(=\overline{a\cdot m})$.
\end{tabular}$$
 On vérifie immédiatement que cela définit bien une structure de $A$-module sur $M/M'$ et que c'est   l'unique structure de $A$-module sur $M/M'$ qui fait de  $\overline{(-)}:M\rightarrow M/M'$ un morphisme de $A$-modules. De plus,

\subsubsection{}\label{Quotient}\textbf{Lemme.} (Propriété universelle du quotient) \textit{Pour tout sous-$A$-module $M'\subset M$ il existe un morphisme de $A$-modules $p:M\rightarrow Q$ tel que 
pour tout morphisme de $A$-modules $f:M\rightarrow N$ tels que $M'\subset \ker(f)$, il existe unique morphisme de $A$-modules $\overline{f}:Q\rightarrow N'$ tel que $\overline{f}\circ p=f$.}\\

\begin{proof} On vérifie comme d'habitude que le morphisme de $A$-modules $p_M:M\twoheadrightarrow M/M'$ construit ci-dessus convient. \end{proof}
 

 On peut aussi réécrire \ref{Quotient} en disant que, pour tout $A$-module $N$ le  morphisme  canonique 
$$\hbox{\rm Hom}_{A}(M/M',N)\rightarrow \lbrace M\stackrel{f}{\rightarrow}N\; |\; M'\subset \ker(f)  \rbrace,\;\overline{f}\rightarrow \overline{f}\circ\overline{(-)}  $$
est un isomorphisme ou encore, plus visuellement:

$$\xymatrix{M'\ar[r]\ar@/^1.5pc/[rr]^{0}&M\ar[r]^{f}\ar[d]_{\overline{(-)}}&N\\
&M/M'\ar@{.>}[ur]_{\exists ! \overline{f}}&}$$


 

 On observera que $M'=\ker(\overline{(-)})$ et $M/M'=\hbox{\rm im}(\overline{(-)})$. Inversement, si $f:M\rightarrow N$ est un morphisme de  $A$-modules, on  a un diagramme commutatif canonique de morphismes de $A$-modules
$$\xymatrix{&\ker(f)\ar@{_{(}->}[d]\ar[dl]^\simeq&&&\\
\ker(f)\ar@{^{(}->}[r]&M\ar@{->>}[d]_{\overline{(-)}}\ar@{->>}[r]^{f|^{\hbox{\rm\tiny im}(f)}}&\hbox{\rm im}(f)\ar@{^{(}->}[r]&N\ar@{->>}[r]&N/\hbox{\rm im}(f)=:\hbox{\rm coker}(f)\\
&M/\ker(f)=:\hbox{\rm coim}(f)\ar[ur]_{\simeq}&&}$$
 On a donc une correspondance bijective entre sous $A$-modules et noyaux de morphismes de $A$-modules d'une part et $A$-modules quotients et images de morphismes de $A$-modules d'autre part. Même si les $A$-modules im$(f)$ et $M/\ker(f)$ sont isomorphes, on notera parfois $\hbox{\rm coim}(f):=M/\ker(f)$ (coimage). On note $\hbox{\rm coker}(f):=M'/\hbox{\rm im}(f)$ (conoyaux). \\

\subsubsection{}\textbf{Suites exactes, lemme du serpent et lemme des cinq}\label{SuiteExacte}\index{Suite exacte (Module)}\index{Suite exacte courte (Module)}\index{Suite exacte courte scindée (Module)@Suite exacte courte scindée (Module)}


 On dit qu'une suite de morphismes de $A$-modules
$$M_{0}\stackrel{u_{0}}{\rightarrow} M_{1}\stackrel{u_{1}}{\rightarrow} M_{2}\stackrel{u_{2}}{\rightarrow}   \cdots\stackrel{u_{n}}{\rightarrow} M_{n+1}$$
est exacte\index{Exacte (Suite de morphismes)} si $\hbox{\rm im}(u_{i})=\ker(u_{i+1})$ pour tout $0\leq i\leq n-1$.  Une suite exacte courte est une suite exacte de la forme:
$$0\rightarrow M'\rightarrow M\rightarrow M''\rightarrow 0.$$
La notion de suite exacte est au coeur de l'étude de la structure des $A$-module. La raison première est que c'est l'outil qui permet de 'dévisser' un $A$-module compliqué ($M$) en deux $A$-modules plus simples ($M'$ et $M''$).  \\

\ref{SuiteExacte}.1 \textbf{Lemme.} \textit{Soit $$0\rightarrow M'\stackrel{u}{\rightarrow} M\stackrel{v}{\rightarrow}  M''\rightarrow 0$$
une suite exacte courte de $A$-modules. Montrer que les propriétés suivantes sont équivalentes:
\begin{enumerate}
\item il existe un morphisme de $A$-modules $s:M''\rightarrow M$ tel que $v\circ s=Id_{M''}$;
\item il existe un morphisme de $A$-modules $s:M'\rightarrow M'$ tel que $s\circ u=Id_{M'}$;
\item il existe un isomorphisme de $A$ modules $f:M\tilde{\rightarrow}M'\oplus M''$ tel que $\iota_{M'}=f\circ u$ et $p_{M''}\circ f=v$.\\
\end{enumerate} }
 On dit qu'une suite exacte courte vérifiant les conditions équivalentes ci-dessus est \textit{scindée}.

 \begin{proof} On peut par exemple montrer $(1)\Rightarrow (2)\Rightarrow (3)\Rightarrow (1)$. \\

  $(1)\Rightarrow (2)$: Si $s:M''\rightarrow M$ est un morphisme de $A$-modules tel que $vs=Id_{M''}$ on vérifie que le morphisme de $A$-modules $Id-sv:M\rightarrow M$ a son image contenue dans $\ker(v)=u(M')$ et que  $t:=(u|^{u(M')})^{-1}\circ (Id-sv):M\rightarrow M'$ vérifie bien $tu=Id_{M'}$.\\
 
  $(2)\Rightarrow (3)$: Si $s:M\rightarrow M'$ est un morphisme de $A$-modules tel que $su=Id_{M'}$, on peut considérer $f:=s\oplus v:M\rightarrow M'\oplus M''$. Par construction, $p_{M''}\circ f=v$ et $f\circ u(s(m))=s(m)=\iota_{M'}(s(m))$ donc, comme  $s:M\rightarrow M'$ est surjective, $f\circ u=\iota_{M'}$. Enfin, $f:M\rightarrow M'\oplus M''$ est un isomorphisme. Il est injectif car si $f(m)=0$ alors $v(m)=0$ \textit{i.e.} $m\in \ker(v)=u(M')$ donc $m=u(m')$ et $m'=su(m')=0$. Donc, en fait $m=0$. Il est surjectif car pour tout $m'\in M'$, $m''\in M''$, on peut écrire $m''=v(m)=v(m-us(m)+u(m'))$ et $m'=su(m')=s(m-us(m)+u(m'))$.\\
  
 $(3)\Rightarrow (1)$: Si $f:M\tilde{\rightarrow} M'\oplus M''$ est un isomorphisme de $A$-modules tel que $p_{M''}\circ f=v$ et   $f\circ u=\iota_{M'}$, on peut   considérer $s:= f^{-1}\circ \iota_{M''}:M''\rightarrow M $. Par construction $vs(m)=vf^{-1}  \iota_{M''}=p_{M''}\iota_{M''}=Id_{M''}$.\end{proof}

\ref{SuiteExacte}.2 \textbf{Exemple.} 
\begin{enumerate}
\item Si $n\geq 2$ est un entier, la suite de $\Z$-modules $0\rightarrow \Z\stackrel{n\cdot}{\rightarrow} \Z\rightarrow \Z/n\rightarrow 0$ n'est pas scindée.
\item On considère les structure de   $\Z[X]$-modules  suivantes sur $\Z^2$  
\begin{enumerate}
\item $X\cdot (a,b)=(b,a)$
\item $X\cdot (a,b)=(a+b,b)$
\end{enumerate}
Dans le cas (a), la suite exacte courte de $\Z[X]$-modules
$$0\rightarrow \Z\stackrel{a\rightarrow (a,a)}{\rightarrow} \Z^2\rightarrow \Z\rightarrow 0$$
est-elle scindée? Même question avec dans le cas (b), la suite exacte courte de $\Z[X]$-modules
$$0\rightarrow \Z\stackrel{a\rightarrow (a,0)}{\rightarrow} \Z^2\rightarrow \Z\rightarrow 0.$$
\end{enumerate}
 

 

\ref{SuiteExacte}.3 \textbf{Exercice.} (Lemme du serpent) \index{Serpent (lemme du)}
\begin{enumerate}
\item Soit $$\xymatrix{M'\ar[r]^{u'}\ar[d]_{\alpha'}&M\ar[d]^{\alpha}\\
N'\ar[r]_{v'}&N}$$
un diagramme commutatif de morphismes de $A$-modules. Montrer que $u':M'\rightarrow M$ induit un morphisme canonique $\ker(\alpha')\rightarrow \ker(\alpha)$ et que $v':N'\rightarrow N$ induit un morphisme canonique $\hbox{\rm coker}(\alpha')\rightarrow\hbox{\rm coker}(\alpha)$.
\item Soit  
$$\xymatrix{&M'\ar[r]^{u'}\ar[d]_{\alpha'}&M\ar[d]^{\alpha}\ar[r]^{u}&M''\ar[r]\ar[d]^{\alpha''}&0\\
0\ar[r]&N'\ar[r]_{v'}&N\ar[r]_{v}&N''}$$
un diagramme commutatif de morphismes de $A$-modules dont les lignes horizontales sont exactes. 
\begin{enumerate}
\item Construire un morphisme 'naturel' $\delta:\ker(\alpha'')\rightarrow\hbox{\rm coker}(\alpha')$;
\item Montrer que la suite de morphismes
$$\ker(\alpha')\rightarrow\ker(\alpha)\rightarrow\ker(\alpha'')\stackrel{\delta}{\rightarrow} \hbox{\rm coker}(\alpha')\rightarrow\hbox{\rm coker}(\alpha)\rightarrow\hbox{\rm coker}(\alpha'') $$
est exacte.
\item Montrer que si $\alpha'$, $\alpha''$ sont injectives (resp. surjectives) alors $\alpha$ est injective (resp. surjective).
\item On suppose de plus que $u':M'\rightarrow M$ est injective et $v:N\rightarrow N''$ est surjective. Montrer que si deux des trois morphismes $\alpha$, $\alpha'$, $\alpha''$ sont des isomorphismes alors le troisième l'est aussi. 
\item Soit $0\rightarrow M'\rightarrow M\rightarrow M''\rightarrow 0$ une suite exacte courte de groupes abéliens et soit $p$ un nombre premier. Montrer qu'on a une suite exacte longue canonique de groupes abéliens
$$M'[p]\rightarrow M[p]\rightarrow M''[p]\rightarrow\rightarrow M'/p\rightarrow M/p\rightarrow M''/p\rightarrow 0,$$
(o\`u on a noté $M[p]:=\lbrace m\in M\; |\; pm=0\rbrace$ et $M/p:=M/(pM)$).
\end{enumerate}
\item  Soit  
$$\xymatrix{M_{1}\ar[r]\ar[d]^{\alpha_{1}}&M_{2}\ar[r]\ar[d]^{\alpha_{2}}&M_{3}\ar[r]\ar[d]^{\alpha_{3}}&M_{4}\ar[r]\ar[d]^{\alpha_{4}}&M_{5}\ar[d]^{\alpha_{5}}\\
N_{1}\ar[r]&N_{2}\ar[r]&N_{3}\ar[r]&N_{4}\ar[r] &N_{5}}$$
un diagramme commutatif de morphismes de $A$-modules dont les lignes horizontales sont exactes. 
\begin{enumerate}
\item Montrer que si $\alpha_{1}$ est surjective et $\alpha_{2}$, $\alpha_{4}$ sont injectives alors $\alpha_{3}$ est injective.
\item  Montrer que si $\alpha_{5}$ est injective et $\alpha_{2}$, $\alpha_{4}$ sont surjectives alors $\alpha_{3}$ est surjective. 
\end{enumerate}
\end{enumerate}


 
 

\section{Conditions de finitude}

 Soit $A$ un anneau commutatif. \\
  
\subsection{Lemme}\label{Noetherien}\textit{
Soit $M$ un $A$-module. Les conditions suivantes sont équivalentes.
\begin{enumerate}[leftmargin=* ,parsep=0cm,itemsep=0cm,topsep=0cm] 
\item Toute suite croissante de sous $A$-modules $$M_{0}\subset M_{1}\subset\dots\subset M_{n}\subset M_{n+1}\subset\dots\subset M$$
est stationnaire à partir d'un certain rang;
\item Tout ensemble non vide de sous $A$-modules de $M$ possède un élément maximal pour l'inclusion;
\item Tout sous $A$-module de $M$ est de type fini.\\
\end{enumerate}}

 Un $A$-module $M$ vérifiant les conditions équivalentes du Lemme \ref{Noetherien} est dit \textit{noetherien}\index{Noetherien (Module)}.\\

\begin{proof} (1) $\Rightarrow$ (2): Si (2) n'était pas vrai, il existerait un ensemble non vide $\mathcal{E}$ de sous $A$-modules de $M$ ne contenant aucun élément maximal pour l'inclusion. Soit $M_{0}\in \mathcal{E}$. Comme $M_{0}$ n'est pas maximal pour l'inclusion, il existe $M_{1}\in\mathcal{E}$ tel que $M_{0}\subsetneq M_{1}$. On itère l'argument avec $M_{1}$ et on construit ainsi une suite strictement croissante infinie de sous $A$-modules de $M$, ce qui contredit (1).\\
 (2) $\Rightarrow$ (3): Soit $M'\subset M$ un sous $A$-module et $\mathcal{E}$ l'ensemble des sous $A$-modules de type fini de $M'$. Comme le module trivial $\lbrace 0\rbrace$ est dans $\mathcal{E}$, $\mathcal{E}$ est non-vide donc admet un élément $M''$ maximal pour l'inclusion. Pour tout $m\in M'$, le $A$-module $M''+Am$ est dans $\mathcal{E}$ et contient $M''$. Par maximalité de $M''$, on a $M''+Am=M''$ donc $m\in M''$.\\
 (3) $\Rightarrow$ (1): Soit $$M_{0}\subset M_{1}\subset\dots\subset M_{n}\subset M_{n+1}\subset\dots\subset M$$
une suite croissante de sous $A$-modules. La réunion $$U:=\bigcup_{n\geq 0} M_{n}\subset M$$
est un sous $A$-module. Soit $m_{1},\dots, m_{r}$ une famille de générateurs de $U$.  Chaque $m_{i}$ est dans $M_{n_{i}}$ pour un certain $n_{i}\geq 0$. Avec 
$$N:=\hbox{\rm max}\lbrace n_{i}\; |\; i=1,\dots, r\rbrace$$
on a $M_{n}=M_{N}$, $n\geq N$. \end{proof}

\textbf{Remarque.} Un anneau $A$ est en particulier noetherien au sens de \ref{AnneauNoetherien} s'il l'est comme $A$-module sur lui-même.


\subsection{Lemme}\label{Artinien}\textit{Soit $M$ un $A$-module. Les conditions suivantes sont équivalentes.
\begin{enumerate}
\item Toute suite décroissante de sous $A$-modules $$M\supset\dots\supset M_{0}\supset M_{1}\supset\dots\supset M_{n}\supset M_{n+1}\supset\dots $$
est stationnaire à partir d'un certain rang;
\item Tout ensemble non vide de sous $A$-modules de $M$ possède un élément minimal pour l'inclusion.
\end{enumerate}}

 Un $A$-module $M$ vérifiant les conditions équivalentes du Lemme \ref{Artinien} est dit \textit{artinien}\index{Artinien (module)}. On laisse en exercice la preuve du Lemme \ref{Artinien}, qui est exactement similaire à celle du Lemme \ref{Noetherien}\\
 

  

\subsection{Exemple}
\begin{enumerate}
\item Le $\mathbb{Z}$-module $\Q$ n'est ni noetherien ni artinien.\\
\item Le $\mathbb{Z}$-module régulier est noetherien mais pas artinien.\\
\item Le $\mathbb{Z}$-module $\mathbb{Z}[\frac{1}{p}]/\mathbb{Z}\subset \Q/\Z$ est artinien mais pas noetherien (observer que les sous $\mathbb{Z}$-modules de $\mathbb{Z}[\frac{1}{p}]/\mathbb{Z}$ sont les $(\mathbb{Z}\frac{1}{p^{n}}+\mathbb{Z})/\mathbb{Z}$, $n\geq 0$).\\
\item Tout $\mathbb{Z}$-module fini est à la fois noetherien et artinien. Si $A$ est une algèbre sur un corps $k$, tout $A$-module de $k$-dimension finie est à la fois noetherien et artinien.
\end{enumerate} 

\subsection{Lemme}\label{Exo1}\textit{
\begin{enumerate}[leftmargin=* ,parsep=0cm,itemsep=0cm,topsep=0cm] 
\item Soit $0\rightarrow M'\rightarrow M\rightarrow M''\rightarrow 0$ une suite exacte courte de $A$-modules. Alors $M$ est noetherien (resp. artinien) si et seulement si $M'$ et $M''$ sont noetheriens (resp. artininens). 
\item Une somme directe finie de $A$-modules noetheriens (resp. artiniens) est encore noetherien (resp. artinien).
\item Tout module de type fini sur un anneau noethérien (resp. artinien) est noetherien (resp. artinien). Montrer que tout module de type fini sur un anneau noethérien est de présentation finie.
\end{enumerate} }


 \begin{proof}
  \begin{enumerate}[leftmargin=* ,parsep=0cm,itemsep=0cm,topsep=0cm] 
\item  Supposons $M$ noetherien (resp. artinien). Toute suite croissante (resp. décroissante) de sous-$A$ modules de $M'$ est une suite de  sous-$A$ modules de $M$ donc stationne à partir d'un certain rang. De même, l'image inverse dans $M$ de toute suite croissante (resp. décroissante) de sous-$A$ modules de $M''$ est une suite de  sous-$A$ modules de $M$ donc stationne à partir d'un certain rang. Supposons $M'$ et $M''$ noetheriens (resp. artiniens). Soit $M_1\subset \dots\subset M_n\subset M_{n+1}\subset \dots M$ une suite croissante de sous-$A$ modules de $M$. Il existe un entier $N$ tel que   $M_N\cap M'=M_n\cap M'$ et $(M_N+M')/M'=(M_n+M')/M'$n $n\geq N$. La conclusion résulte du lemme du serpent appliqué à 
$$\xymatrix{0\ar[r]& M_N\cap M'\ar[r]\ar@{=}[d]&M_N\ar[r]\ar@{_{(}->}[d]&(M_N+M')/M'\ar[r]\ar@{=}[d]& 0\\
0\ar[r]& M_n\cap M'\ar[r]&M_n\ar[r]&(M_n+M')/M'\ar[r]& 0\\}$$
L'assertion pour `artinien' se montre de la même fa\c{c}on.
\item  On procède par induction sur $n$ en utilisant  1.3.4 (1) et la suite exacte courte de $A$-modules
$$0\rightarrow \oplus_{1\leq i\leq n}M_i\rightarrow  \oplus_{1\leq i\leq n+1}M_i\rightarrow M_{n+1}\rightarrow 0.$$
\item D'après 1.3.4 (2) $A^{\oplus n}$ est noetherien (resp. artinien) et, par définition, tout $A$-module de type fini est quotient d'un $A$-module de la forme $A^{\oplus n}$. Donc la conclusion résulte de 1.3.4 (1).
\end{enumerate}
\end{proof}

  La propriété d'être noetherien et artinien est la bonne généralisation de la notion de dimension finie lorsque $A=k$ est un corps. Les points (1) et (2) du lemme suivant, par exemple, servent de substitut au Lemme du rang.
 


\subsection{Lemme} (Fitting)\label{ExoFitting} \textit{Soit $f:M\rightarrow M$ un endomorphisme de $A$-module.  
\begin{enumerate}
\item Si $M$ est noetherien et $f$ surjectif alors $f$ est un isomorphisme.
\item Si $M$ est artinien et $f$ injectif alors $f$ est un isomorphisme.
\item (Lemme de 'Fitting') Si $M$ est artinien et noetherien alors il existe une décomposition $M=f^{\infty}(M)\oplus f^{-\infty}(0)$ en somme directe de deux sous $A$-modules $f$-stables tels que la restriction de $f$ à $f^{\infty}(M)$ soit un automorphisme et la restriction de $f$ à $f^{-\infty}(0)$ soit nilpotente.
\end{enumerate}}

\begin{proof} 
 \begin{enumerate}
\item   Il existe un entier $N\geq 1$ tel que $\ker(f^N)=\ker(f^n)$, $n\geq N$ et on applique le lemme du serpent à 
$$\xymatrix{0\ar[r]& \ker(f^N)\ar[r]\ar[d]^\simeq&M \ar[r]^{f^N}\ar[d]^{Id}_\simeq&M\ar[r]\ar[d]^f& 0\\
0\ar[r]& \ker(f^{N+1})\ar[r]&M \ar[r]_{f^{N+1}}&M\ar[r]& 0\\}$$
\item   Il existe un entier $N\geq 1$ tel que $\hbox{\rm im}(f^N)=\hbox{\rm im}(f^n)$, $n\geq N$ et on applique le lemme du serpent à 
$$\xymatrix{0\ar[r]& M\ar[r]^{f^{N+1}}\ar[d]^f&M \ar[r] \ar[d]^{Id}_\simeq&M/\hbox{\rm im}(f^{N+1})\ar[r]\ar[d]^\simeq & 0\\
0\ar[r]&M\ar[r]_{f^N}&M \ar[r] &M/\hbox{\rm im}(f^{N})\ar[r]& 0\\}$$
\item (3) Comme $M $est artinien et noetherien, il existe un entier $N\geq 1$ tel que $$f^{\infty}(M):=\bigcap_{n\geq 0}\hbox{\rm im}(f^n)=\hbox{\rm im}(f^N),\; \; f^{-\infty}(M):=\bigcup_{n\geq 0}\ker(f^n)=\ker(f^N).$$
On vérifie que $f^{\infty}(M)$, $f^{-\infty}(M)$ ainsi définis conviennent. Le seul point un peu astucieux est $M=f^{\infty}(M)+f^{-\infty}(M)$. On a envie d'écrire $m=f^N(m)+m-f^N(m)$ mais \c{c}a ne marche pas. Il faut ajuster en utilisant que $\hbox{\rm im}(f^N)=\hbox{\rm im}(f^{2N})$ et donc qu'il existe $\mu\in M$ tel que $f^N(m)=f^{2N}(\mu)$. La décomposition $m=f^N(\mu)+m-f^N(\mu)$ elle, convient. 
  
\end{enumerate}
\end{proof}


   $$***$$ 
 
 Syllabus prochaines séances:\\
$$ \begin{tabular}[t]{l}
 
Modules indécomposables, Krull-Schmidt\\
Modules de type fini sur les anneaux principaux\\
\end{tabular}$$
  \begin{tabular}[t]{l}
\textit{anna.cadoret@imj-prg.fr}\\
 IMJ-PRG, Sorbonne Université\\
 Paris, FRANCE
\end{tabular}
 \printindex
\end{document}
\section{Modules indécomposables, Krull-schmidt}
 

\subsection{Modules indécomposables}  Un $A$-module $M$ est dit \textit{indécomposable} s'il est non nul et ne peut s'écrire sous la forme $M=M'\oplus M''$ avec $M',M''\subset M$ deux sous $A$-modules non nuls. Un $A$-module $M$ est dit \textit{totalement décomposable}\index{Indecomposable (Module)@Indécomposable (Module)} s'il peut s'écrire sous la forme $M=M_{1}\oplus \dots\oplus M_{r}$ avec $M_{1},\dots,M_{r}\subset M$ des sous $A$-modules indécomposables.\\

 Un anneau $E$ est dit \textit{local} \index{Local (Anneau)} si  $E\setminus E^\times$ est un idéal; auquel cas, $E\setminus E^\times$ est l'unique idéal  bilatère maximal de $E$.\\

\subsection{}\textbf{Lemme.}\label{EndoIndecomp} \textit{Soit $M$ un $A$-module. Si $E: =\hbox{\rm End}_{A}(M)$ est local, $M$ est indécomposable. Réciproquement, si $M$ est  artinien et noetherien,  $E$ est local.}
\begin{proof} Supposons $E$ local et qu'on puisse écrire  $M=M'\oplus M''$ avec $M',M''\subset M$ deux sous $A$-modules non nuls. Notons $e:=\iota_{M'}\circ p_{M'}\in E$ la projection de $M$ sur $M'$ parallèlement à $M''$. 
 On a $e,1-e\in E\setminus E^\times$. Mais si $E$ est local, $E\setminus E^\times$ est un idéal donc $1=e+(1-e)\in E\setminus E^\times$: contradiction.  Supposons maintenant que $M$ est un $A$-module artinien et noetherien indécomposable, d'après l'Lemme \ref{ExoFitting} (3), tout élément non nul de $E$ est soit inversible soit nilpotent. En particulier $J:=E\setminus E^{\times}$ est l'ensemble des éléments nilpotents de $E$. Il suffit de montrer que $J$ est un idéal bilatère. Soit donc $j\in J$ et $e\in E$. Comme $j$ est nilpotent on a $\ker(j)\not=0$ et im$(j)\not= M$ (Lemme \ref{ExoFitting} (1), (2)). Donc aussi $\ker(ej)\not=0$ et im$(je)\not=M$, ce qui montre que $ej,je\in E\setminus E^{\times}=J$. Donc $EJ=JE=J$. Il reste à voir que $J$ est stable par addition. Soit $j,j'\in J$, si $j+j'\in E^{\times}$ il existerait $e\in E$ tel que $ej=1-ej'$. Comme $ej'\in J$, on a forcément $1-ej'\in E^{\times}$ (d'inverse $\sum_{n\geq 0} (ej')^n$), ce qui contredit le fait que $j\in J$. \end{proof}
 
 
 
\subsection{Théorème de Krull-Schmidt} Notons $Ind(A)$ l'ensemble des classes d'isomorphismes de $A$-modules indécomposables.

\subsubsection{}\textbf{Théorème.}  \label{KS}\index{Krull-Schmidt (theoreme, module)@Krull-Schmidt (Théorème, Module)} (Krull-Schmidt) \textit{ Soit $M$ un $A$-module artinien ou noetherien. Alors il existe une application à support finie $\kappa:Ind(A)\rightarrow \mathbb{Z}_{\geq 0}$ telle que 
$$M=\bigoplus_{N\in Ind(A)}N^{\oplus \kappa(N)}.$$
Si $M$ est à la fois  artinien et noetherien alors $\kappa:Ind(A)\rightarrow \mathbb{Z}_{\geq 0}$ est unique; on la notera $\kappa_{M}:Ind(A)\rightarrow \mathbb{Z}_{\geq 0}$.}
\begin{proof} Commen\c{c}ons par montrer l'existence de la décomposition. Raisonnons par l'absurde. Si $M$ n'est pas totalement décomposable, $M$ n'est en particulier pas indécomposable donc $$M=M_1^{(0)}\oplus M_2^{(0)}$$
avec $0\not= M_1^{(0)}, M_2^{(0)}\subset M$ deux sous $A$-modules dont l'un au moins des deux - disons $M_1^{(0)}$ n'est pas totalement décomposable. On itère l'argument pour obtenir une suite de décompositions en sommes directes de sous $A$-modules non nuls
$$M=M_1^{(1)}\oplus M_2^{(1)}\oplus M_2^{(0)}$$
$$\cdots$$
$$M=M_1^{(n+1)}\oplus M_2^{(n+1)}\oplus M_2^{(n)}\oplus M_2^{(n-1)}\oplus\cdots\oplus M_2^{(1)}\oplus  M_2^{(0)}  $$
avec, à chaque fois, $M_1^{(n)}$ qui n'est pas totalement décomposable. On obtient en particulier une suite strictement croissante de sous $A$-modules
$$\lbrace 0\rbrace \subset M_2^{(0)} \subset M_2^{(1)}\oplus M_2^{(0)}\subset\cdots\subset   M_2^{(n)}\oplus\cdots M_2^{(1)}\oplus M_2^{(0)}\subset\cdots $$
et une suite  strictement décroissante de sous $A$-modules
$$M\supset M_1^{(0)}\supset M_1^{(0)}\supset\cdots\supset M_1^{(n)}\supset M_1^{(n+1)}\supset\cdots $$
  Supposons maintenant que $M$ est artinien et noetherien et montrons l'unicité de la décomposition. D'après le Lemme \ref{EndoIndecomp} et par récurrence, il suffit de montrer que  si on a un isomorphisme de $A$-modules noetherien et artinien
$$ M\oplus M'\simeq  N_{1}\oplus\dots\oplus N_{s}=:N$$
avec  $E:=\hbox{\rm End}_{A}(M)$ local et les $N_{1},\dots, N_{s}$ indécomposables 
alors  il existe $1\leq i\leq s$ tel que $M \simeq N_{i}$ et $M'\simeq \oplus_{  j\not=i}N_{j}$. Soit $\Phi=(\phi \; \phi'):M\oplus M'\tilde{\rightarrow}N$ un isomorphisme de $A$-modules d'inverse $$\Psi= \left(\begin{tabular}[c]{l}
$\psi$\\
$\psi'$\\
\end{tabular}
\right):N\tilde{\rightarrow} M\oplus M'.$$
 Par le lemme \ref{EndoIndecomp},  $E\setminus E^{\times}$ est  un idéal bilatère et l'égalité
$$Id_{M}=\psi\circ\phi=\sum_{1\leq i\leq s}\psi\circ \iota_{i}\circ p_{i}\circ \phi$$
implique que $\chi_{i}:=\psi\circ \iota_{i}\circ p_{i}\circ \phi\in E^{\times}$ pour au moins un $i=1,\dots, s$. On a alors $p_{i}\circ \phi:M\hookrightarrow N_{i}$ injectif, $\psi\circ \iota_{i}:N_{i}\twoheadrightarrow M$ surjectif et 
$$\xymatrix{0\ar[r]& \ker(\psi\circ \iota_{i})\ar[r]&N_{i}\ar[r]^{\psi\circ \iota_{i}}&\hbox{\rm im}(\psi\circ \iota_{i})\ar[r]\ar@/^1pc/[l]^{p_{i}\circ\phi\circ \chi_i^{-1}}& 0}$$
donc
$$N_{i}=\ker(\psi\circ \iota_{i})\oplus\hbox{\rm im}(p_{i}\circ \phi).$$    
Comme par hypothèse $N_{i}$ est indécomposable on a forcément $\ker(\psi\circ \iota_{i})=0$ et $\hbox{\rm im}(p_{i}\circ \phi)=N_{i}$.
Donc $p_{i}\circ \phi:M\tilde{\rightarrow} N_{i}$ et $\psi\circ \iota_{i}:N_{i}\tilde{\rightarrow}  M$ sont des isomorphismes. Il reste \` a voir que $M'\simeq \oplus_{  j\not=i}N_{j}$. Pour cela, considérons les suites exactes courtes de $A$-modules:
$$0\rightarrow M\stackrel{\iota}{\rightarrow}M\oplus M'\stackrel{p}{\rightarrow}M'\rightarrow 0$$
$$0\rightarrow\bigoplus_{ i\not= j}N_{j} \stackrel{\Psi\circ\iota_{i}'}{\rightarrow}M\oplus M'\stackrel{p_{i}\circ \Phi}{\rightarrow}N_{i}\rightarrow 0.$$
On sait que $p_{i}\circ \Phi\circ \iota=p_{i}\circ \phi:M\tilde{\rightarrow} N_{i}$ est un isomorphisme et on voudrait montrer que $p\circ\Psi\circ\iota_{i}':\bigoplus_{ i\not= j}N_{j}\rightarrow M'$ en est un aussi. Cela découle du petit lemme suivant, dont on laisse la preuve en exercice au lecteur. \end{proof}

\subsubsection{}\textbf{Lemme.}\textit{
\begin{enumerate}
\item Soit 
 $0\rightarrow K \stackrel{\alpha}{\rightarrow}M\stackrel{\beta}{\rightarrow}Q $
et $0\rightarrow K' \stackrel{\alpha'}{\rightarrow}M\stackrel{\beta'}{\rightarrow}Q' $
deux suites exactes de $A$-modules. Alors $\beta'\alpha$ est injectif si et seulement si $\beta\alpha'$ est injectif.
\item Soit 
 $  K \stackrel{\alpha}{\rightarrow}M\stackrel{\beta}{\rightarrow}Q\rightarrow 0 $
et $K' \stackrel{\alpha'}{\rightarrow}M\stackrel{\beta'}{\rightarrow}Q' \rightarrow 0$
deux suites exactes de $A$-modules. Alors $\beta'\alpha$ est surjectif si et seulement si $\beta\alpha'$ est surjectif.
\end{enumerate}}
\section{Modules de type fini sur les anneaux principaux}
 

\subsection{}\label{Strategy}Soit $M$ un $A$-module. Un élément $m\in M$ est dit \textit{de torsion}\index{Torsion (module)} s'il existe $0\not=a \in A$ tel que $am=0$. On note $T_{M}\subset M$ l'ensemble des éléments de torsion de $M$. On vérifie immédiatement que c'est un sous $A$-module et que le $A$-module $M/T_{M}$ est sans torsion. Le $A$-module $M$ s'insère donc dans la suite exacte courte
$$(*)\;\; 0\rightarrow T_{M}\rightarrow M\rightarrow M/T_{M}\rightarrow 0,$$
o\`u $T_{M}$ est de torsion et $M/T_{M}$ est sans torsion. Cela indique la voie pour classifier les $A$-modules de type fini: montrer que la suite exacte courte $(*)$ se scinde, ce qui par  le Lemme \ref{SuiteExacte}.1 impliquera automatiquement que $$M\tilde{\rightarrow }T_{M}\oplus M/T_{M}$$ et réduit donc le problème de la classification des $A$-modules de type fini à 
\begin{itemize}
\item la classification des $A$-modules de type fini sans torsion;
\item la classification des $A$-modules de type fini de torsion.\\
\end{itemize}
 En fait, on va plutôt procéder dans l'ordre suivant. Notons que comme $A$ est noterien et $M$ de type fini, $M$ est noetherien. Donc $T_M$ et $M/T_M$ sont aussi noetheriens donc de type fini.\\
\begin{enumerate}
\item Tout d'abord, la raison pour laquelle on se restreint aux $A$-modules de type fini provient du lemme suivant.\\

\ref{Strategy}.1 \textbf{Lemme.} \textit{Un $A$-module de type fini est noetherien. Un $A$-module de type fini et de torsion est noetherien et artinien.}
 
\begin{proof}   Comme $A$ est principal, tous ses sous $A$-modules (=idéaux) sont de type fini donc $A$ est noetherien; la première partie de l'énoncé résulte donc du Lemme \ref{Exo1}.  Supposons $M$ de torsion. Soit $m_{1},\dots,m_{r}\in M$ un système de générateurs. Pour chaque $i=1,\dots, r$ on peut trouver un élément $0\not=a_{i}\in A$ tel que $a_{i}m_{i}=0$. On a donc une factorisation  
$$\xymatrix{A^{r}\ar@{->>}[r]^{(m_{1},\dots,m_{r})}\ar@{->>}[d]&M\\
A/Aa_{1}\times\cdots \times A/Aa_{r}\ar@{->>}[ur]&}$$
D'après le Lemme \ref{Exo1}, il suffit donc de montrer que les $A$-module de la forme $A/Aa$ avec $0\not= a\in A$ sont artiniens. Soit $$A/Aa=:M_{0}\supset M_{1}\supset\cdots\supset M_{n}\supset M_{n+1}\supset \cdots$$
une suite décroissante de sous $A$-modules. Notons $\pi:A\rightarrow A/Aa$ la projection canonique et posons:
$$I_{n}:=\pi^{-1}(M_{n}),\; n\geq 0.$$
Par construction on obtient une suite décroissante d'idéaux 
$$A=I_{0}\supset I_{1}\supset\cdots\supset I_{n}\supset I_{n+1}\supset \cdots Aa.$$
Chacun de ces idéaux est de la forme $I_{n}=Aa_{n}$ avec $0\not= a_{n}\in A$ et $a_{n}| a$. Mais comme un anneau principal est factoriel, $a$ n'a qu'un nombre fini de diviseurs deux à deux non associés. Il n'y a donc qu'un nombre fini d'idéaux dans la suite $I_{n}$, $n\geq 0$.\end{proof}
\item On va ensuite montrer que tout  $A$-modules   libre (sur un anneau intègre) est classifié par son rang et qu'un $A$-module de type fini sans torsion sur un anneau principal   est   libre de rang fini. Cela permettra aussi d'appliquer l'observation suivante.\\


\ref{Strategy}.2 \textbf{Lemme.} \textit{Si $M''$ est un $A$-module libre alors toute suite exacte courte de $A$-modules $0\rightarrow M'\stackrel{u}{\rightarrow} M\stackrel{v}{\rightarrow} M''\rightarrow 0$ est scindée.}
\begin{proof}On construit une section en utilisant la propriété universelle de la somme directe. Plus précisément, quitte à composer $v$ par un isomorphisme,  on peut supposer que $M''=A^{(I)}$. Pour chaque $i\in I$ notons $e_i=(\delta_{i,j})_{j\in I}\in A^{(I)}$ et choisissons   $m_i\in I$ tel que $v(m_i)=e_i$. Le choix de $m_i$ définit un   morphisme de $A$-module $s_i:Ae_i\stackrel{e_i\rightarrow m_i}{\rightarrow}Am_i\hookrightarrow M $. Par proprièté universelle des $\iota_i:Ae_i\rightarrow A^{(I)}$, $i\in I$, on en déduit un unique morphisme $s:A^{(I)}\rightarrow M$ tel que $s\circ \iota_i=s_i$, $i\in I$. Par construction $v\circ s=Id$. On conclut par l'Exercice \ref{SuiteExacte}.1. 
\end{proof}
\item D'après  le Lemme \ref{Strategy}.1 et le Théorème de Krull-Schmidt \ref{KS}, $T_M$ est totalement décomposable; le point sera donc de classifier les modules indécomposables de torsion sur un anneau principal $A$. On montrera que ce sont exactement les $A$-modules de la forme $A/\frak{p}^n$, o\`u $\frak{p}$ est un idéal premier (=maximal) de $A$ et $n\geq 0$.
 
\end{enumerate}


 
\subsection{Classification des $A$-modules de type fini sans torsion} Supposons d'abord que $A$ est seulement un anneau commutatif intègre.\\

\subsubsection{}\textbf{Lemme.}\label{Free1}\textit{Un $A$-module de type fini sans torsion est isomorphe à un sous $A$-module d'un $A$-module libre de type fini. }
\begin{proof} Soit $m_{1},\dots m_{r}\in M$ un système de générateur. L'ensemble 
$$\mathcal{S}:=\lbrace I\subset \lbrace 1,\dots, r\rbrace\; |\; A^{I}\stackrel{(m_{i})_{i\in I}}{\hookrightarrow} M\rbrace$$
est non vide puisque $M$ est sans torsion donc contient un élément $I\subset \lbrace 1,\dots, r\rbrace$ maximal  pour l'inclusion. Notons $$N:=\sum_{i\in I}Am_{i}\simeq A^{I}.$$ Par maximalité de $I$, pour chaque $j\in I^{c}:=\lbrace 1,\dots, r\rbrace\setminus I $ il existe $0\not= a_{j}\in A$ tel que $a_{j}m_{j}\in N$. Notons $a:=\prod_{j\in I^{c}}a_{j}\in A$; c'est un élément non nul de $A$ puisque $A$ est intègre. On en déduit que le morphisme de $A$-module
$$\begin{tabular}[t]{lll}
$M$&$\rightarrow$&$N$\\
$m$&$\rightarrow$&$am$
\end{tabular}$$
est injectif, puisque $M$ est sans torsion. \end{proof}

\subsubsection{}\textbf{Lemme.} (Classification des $A$-modules libres de type fini par le rang)\label{Rang} \textit{ 
\begin{enumerate}
\item Le $A$-module libre $A^{(I)}$ est de type fini si et seulement si $|I|<+\infty$.
\item Soit $I, J $ deux ensembles finis. Alors $A^{(I)}$ et $A^{(J)}$ sont isomorphes comme $A$-modules si et seulement si $|I|=|J|$.
\end{enumerate} }
\begin{proof}L'idée est de se ramener au cas des espaces vectoriels sur un corps pour lesquels le lemme est connu. Soit donc $M$ un $A$-module libre de type fini et $\frak{m}\subset A$ un idéal maximal. Comme $M$ est de type fini, le $k:=A/\frak{m}$ espace vectoriel $M/\frak{m}M$ est de dimension finie - disons $r$ - sur $k$. Soit $I$ un ensemble pour lequel on a un isomorphisme de $A$-modules 
$$f:A^{(I)}\tilde{\rightarrow}M.$$
Posons $m_{i}:=f(e_{i})$, o\`u $e_{i}$ est le '$i$-ème vecteur de la base canonique', $i\in I$. On va montrer que $|I|=r$. Pour cela, il suffit de montrer que les images $\overline{m}_{i}$, $i\in I$ des $m_{i}$, $i\in I$ dans $M/\frak{m}M$ forment une $k$-base de $M/\frak{m}M$. Puisque $f$ est surjective, les $\overline{m}_{i}$, $i\in I$ forment une famille génératrice. Montrons qu'elle est libre. Soit $a:I\rightarrow A$ à support fini telle que 
$$\sum_{i\in I}a(i)m_{i}\in \frak{m}M.$$
Comme $M=\oplus_{i\in I}Am_{i}$ et $A\tilde{\rightarrow} Am_i$, $a\rightarrow am_i$, $i\in I$, cela implique $a(i)\in \frak{m}$ donc $\overline{a}_{i}=0$, $i\in I$. \end{proof}

 Le Lemme \ref{Rang} montre en particulier que si $M$ est un $A$-module libre de type fini il existe un unique entier $r\geq 1$ tel que $M\simeq A^{\oplus r}$. On appelle cet entier le \textit{rang} du $A$-module libre $M$. C'est également la dimension du $A/\frak{m}$-espace vectoriel $M/\frak{m}M$, pour $\frak{m}$ un idéal maximal de $1$.\\

 Supposons maintenant que \textit{$A$ est principal}.

\subsubsection{}\textbf{Lemme.}\label{Free2} \textit{Un sous $A$-module d'un $A$-module libre de rang fini $r$ est un $A$-module libre de rang $\leq r$.}

\begin{proof} On procède par récurrence sur  $r$. Si $r=1$, cela résulte du fait que $A$ est principal. Supposons que l'énoncé du Lemme \ref{Free2} est vérifié pour tout $A$-module libre de rang $\leq r$. Soit $M\subset A^{\oplus (r+1)}$ un sous $A$-module. Notons $p_{r+1}:A^{\oplus (r+1)}\twoheadrightarrow A$ la $r+1$-ième projection canonique. Comme $\ker(p_{r+1})\simeq A^{\oplus r}\subset A^{\oplus (r+1)}$ est un $A$-module libre de rang $r$, par hypothèse de récurrence, le sous $A$-module $M\cap \ker(p_{r+1})\subset \ker(p_{r+1})$ est un $A$-module libre de rang $s\leq r$. Comme $p_{r+1}(M)\subset A$ est un idéal et que $A$ est principal, il existe $d_0\in A$ et $m_{0}\in M$ tel que $p_{r+1}(M)=Ad_0\stackrel{\cdot d_0}{\tilde{\leftarrow}}A$ et on conclut par le Lemme \ \ref{Strategy}.2. \end{proof}
  

 On vient donc de montrer

 \subsubsection{}\textbf{Corollaire.} \textit{Un $A$-module de type fini sans torsion est libre de rang fini. Plus précisément, l'application $\Z_{\geq 0}\rightarrow \hbox{\rm Mod}_{/A}$, $r\rightarrow A^{\oplus r}$ induit une bijection de  $\Z_{\geq 0}$ sur l'ensemble des classes d'isomorphismes de $A$-modules de type fini  sans torsion.}\\

 En particulier, $M/T_{M}$ est un $A$-module libre de rang fini - disons $r$ - donc, par le Lemme \ref{Strategy}.2 on a $$M\simeq T_{M}\oplus M/T_{M}\simeq T_{M}\oplus A^{\oplus r}.$$
 Il reste  à classifier les $A$-modules  de type fini qui sont de torsion.

\subsection{Classification des $A$-modules de type fini de torsion}
 
 Soit $A$ un anneau principal.

\subsubsection{}\textbf{Théorème.}\label{IndecompPrinc}\textit{Les $A$-modules de type fini de torsion qui sont indécomposables sont exactement les $A$-modules de la forme
$A/\frak{p}^{n}$,
o\`u $\frak{p}\subset A$ est un idéal premier non nul et $n\in\mathbb{Z}_{\geq 0}$.}
\begin{proof} Vérifions d'abord qu'un $A$-module de la forme $A/\frak{p}^{n}$ est indécomposable. Observons que 
$$\hbox{\rm End}_{A}(A/\frak{p}^{n})\simeq \hbox{\rm End}_{A/\frak{p}^{n}}(A/\frak{p}^{n})\simeq A/\frak{p}^{n}$$
a un unique idéal maximal  (c'est par exemple la factorialité de $A$) - $\frak{p}/\frak{p}^{n}$, donc est local (ici $A/\frak{p}^n$ est commutatif). Le fait que $A/\frak{p}^{n}$ est indécomposable résulte alors du lemme \ref{EndoIndecomp}.\\
 Montrons maintenant que tout $A$-module indécomposable est de cette forme.  Soit $M$ un $A$-module. Pour tout $m\in M$, on note
$$Ann_{A}(m):=\lbrace a\in A\; |\; am=0\rbrace\subset A$$
l'idéal annulateur de $m$ et on se fixe un générateur $a_{m}\in Ann_{A}(m)$. On note également 
$$Ann_{A}(M):=\bigcap_{m\in M}Ann_{A}(m)\subset A$$
l'idéal annulateur de $M$.\\

 \ref{IndecompPrinc}.1 \textbf{Lemme.}\textit{
Il existe $m\in M$ tel que $Ann_{A}(m)=Ann_{A}(M)$. }\\

 

 Notons $B:=A/Ann_{A}(m)=A/Ann_{A}(M)$ et considérons la suite exacte courte
$$0\rightarrow B\stackrel{\cdot m}{\rightarrow} M\rightarrow M/Am\rightarrow 0. $$
On notera que comme $Ann_{A}(M)$ annnule $M$, cette suite est également une suite de $B$-modules. \\

 \ref{IndecompPrinc}.2 \textbf{Lemme.} \textit{La suite exacte courte de $B$-modules $$0\rightarrow B\stackrel{\cdot m}{\rightarrow} M\rightarrow M/Am\rightarrow 0$$
est scindée. }\\


 Elle est donc \textit{a fortiori} scindée comme suite exacte courte de $A$-modules \textit{i.e.}
$$M\simeq A/Ann_{A}(M)\oplus M/Am$$
comme $A$-module. Mais comme $M$ est indécomposable (et non nul), on en déduit $M=Am\simeq A/Ann_{A}(M)=A/Aa_{m}$. On conclut par la factorialité de $A$, le Lemme  des restes Chinois et l'indécomposabilité de $M$. \end{proof}

\textit{Preuve du lemme \ref{IndecompPrinc}.1.} Soit $m_{1},\dots, m_{r}$ un système de générateurs de $M$ comme $A$-module. On a $$Ann_{A}(M)=\cap_{1\leq i\leq r}Ann_{A}(m_{i}).$$
 Il suffit donc de montrer que pour tout $m_{1},m_{2}\in M$ il existe $m_{3}\in M$ tel que $$Ann_{A}(m_{1})\cap Ann_{A}(m_{2})=Ann_{A}(m_{3}).$$
Ecrivons $Ann_{A}(m_{i})=Aa_{i}$, $i=1,2$. Comme $A$ est factoriel, en utilisant la décomposition en produit de facteurs irréductibles de $a_{1}$ et $a_{2}$, on peut écrire $a_{1}=\alpha_{1}\beta_{1}$ et $a_{2}=\alpha_{2}\beta_{2}$ avec $\alpha_{1}$, $\alpha_{2}$ premier entre eux de produit 'le' plus petit commun multiple de $a_{1}$ et $a_{2}$. Posons $m_{3}:=\beta_{1}m_{1}+\beta_{2}m_{2}$ et vérifions que $m_{3}$ convient. On a clairement $Ann_{A}(m_{1})\cap Ann_{A}(m_{2})\subset Ann_{A}(m_{3})$. Pour l'inclusion réciproque, soit $a\in Ann_{A}(m_{3})$. On a $a\beta_{1}m_{1}=-a\beta_{2}m_{2}$. Par Bézout, il existe $u,v\in A$ tels que $u\alpha_{1}+v\alpha_{2}=1$. On a donc
$$a\beta_{1}m_{1}=(u\alpha_{1}+v\alpha_{2})a\beta_{1}m_{1}=au\underbrace{a_{1}m_{1}}_{=0}+v\alpha_{2}a\beta_{1}m_{1}= -av\underbrace{a_{2}m_{2}}_{=0}=0.$$
Donc $a\beta_{1}\in Ann_{A}(m_{1})=Aa_{1}$ et $a\beta_{2}\in Ann_{A}(m_{2})=Aa_{2}$ en particulier $a$ est un multiple commun de $\alpha_{1}$ et $\alpha_{2}$ donc de $\alpha_{1}\alpha_{2}=ppcm(a_{1},a_{2})$. Donc $a\in Ann_{A}(m_{1})\cap Ann_{A}(m_{2})$. $\square$\\

\textit{Preuve du Lemme \ref{IndecompPrinc}.2.}  Introduisons l'ensemble $\mathcal{E}$ des couples $(u,N)$ o\`u $m\in N\subset M$ est un sous-$B$-module et $u:N\rightarrow B$ un morphisme de $B$-modules tel que $u(m)=1$. On munit $\mathcal{E}$ de la relation d'ordre $\leq$ définie par $(u_1,N_1)\leq (u_2,N_2)$ si $N_1\subset N_2$ et $u_2|_{N_1}=u_1$. $\mathcal{E}$ est non-vide: par définition $B=A/Ann_A(m)$ donc on a un isomorphisme $v : B\tilde{\rightarrow} Bm$ et $(Am, v^{-1})\in \mathcal{E}$. Par définition, $\mathcal{E}$ est un
ensemble ordonné inductif donc admet un élément maximal $(u,N)$ (en fait, ici, on peut invoquer le fait que $M$ est noetherien, ce qui permet d'éviter le Lemme de Zorn). Montrons que $N=M$. Sinon, soit $\mu\in M\setminus N$ et montrons qu'on peut étendre $u:N\rightarrow B$ en $u_1:N+B\mu\rightarrow B$. Pour cela, il faut `deviner' la bonne valeur de $u_1(\mu)$. Introduisons l'idéal 
$$\frak{i}:=\lbrace b\in B\; |\; b\mu\in N\rbrace\subset B.$$
Ecrivons $Ann_A(M)=Aa$. Comme $B$ est quotient de l'anneau principal $A$, $\frak{i}=Ab/Aa$ avec $Aa\subset Ab $ \textit{i.e.}
 $a=\alpha   b$ pour un certain $\alpha\in A$.
Notons $u(b\mu)=\overline{c}$ (on note $\overline{-}$ les classes modulo $Aa$). On a $u(a\mu)=0=\alpha \overline{c}$ donc $\alpha c=qa=q\alpha b$ dans $A$. Mais comme $A$ est intègre $c=q  b$. On a donc envie de poser $u_1(\mu)=\overline{q}$. Définissons $u_0:N\oplus B\rightarrow B$ par $u_0(n\oplus \lambda )=u(n)+   \lambda\overline{q}$. On a $$\ker(N\oplus B \twoheadrightarrow N+B\mu, n\oplus \lambda\rightarrow n+\lambda \mu)=\lbrace \beta b\mu\oplus -\beta b\; |\; \beta\in B\rbrace\subset \ker(u_0)$$
En effet,  $u_0( \beta b\mu\oplus -\beta b)=u(\beta b\mu)-\beta b\overline{q}=\beta u(b\mu)-\beta b\overline{q}=\beta \overline{c}-\beta b\overline{q}=0$.
Donc $u_0:N\oplus B\rightarrow B$ passe au quotient en $u_1:N+B\mu\rightarrow B$ avec  $u_1|_N=u$. Cela contredit la maximalité de $(u,N)$. $\square$\\
 
 
\subsubsection{}\textbf{Corollaire.}\label{StructureTors} \textit{Soit $M$ un $A$-module de type fini de torsion. Il existe une unique suite décroissante d'idéaux
$$A\supsetneq I_{1}\supset I_{2}\supset\dots\supset I_{r}\supsetneq 0$$
telle que $$M\simeq A/I_{1}\oplus\dots\oplus A/I_{r}.$$}
\begin{proof} Comme $M$ est artinien et noetherien, d'après le Théorème de Krull-Schmidt   \ref{KS}, $M$ se décompose de fa\c{c}on unique comme somme directe de modules indécomposables. D'après le Théorème \ref{IndecompPrinc}, cette décomposition s'écrit

$$M\simeq \bigoplus_{\frak{p}}\bigoplus_{n\geq 0}A/\frak{p}^{\alpha_{M,\frak{p}}(n)},$$
o\`u la première somme est indexée par l'ensemble spec$(A)$ des idéaux premiers non nuls de $A$ et $$\alpha_{M,-}:\hbox{\rm spec}(A)\rightarrow \mathbb{Z}_{\geq 0}^{(\mathbb{Z}_{\geq 0})}$$ est une application à support fini telle que $ \alpha_{M,\frak{p}}=(\alpha_{M,\frak{p}}(n))_{n\geq 0}$ est une suite décroissante dont les termes sont nuls pour $n\gg 0$. Pour chaque $\frak{p}\in \hbox{\rm spec}(A)$ choisissons un générateur $p$ de $\frak{p}$ comme $A$-module. Soit $n\geq 0$ le plus grand des entiers tels qu'il existe $\frak{p}\in$spec$(A)$ pour lequel $\alpha_{M,\frak{p}}(n)\not=0$ et posons $$a_{n+1-j}:=\prod_{\frak{p}}p^{\alpha_{M,\frak{p}}(j)},\; j=1,\dots, n.$$
La suite d'idéaux $I_{i}:=Aa_{j}$, $j=1,\dots, n$ vérifie alors la propriété de l'énoncé. Leur unicité résulte de l'unicité dans le théorème de Krull-Schmidt. \end{proof}

 On dit que la suite $A\supsetneq  I_{1}\supset I_{2}\supset\dots\supset I_{r}\supsetneq 0$ est la \textit{suite des invariants}\index{Invariants (Modules)} du $A$-module $M$.\\


\subsection{Applications}
\subsubsection{Classification des groupes abéliens de type fini} On peut appliquer la classification des $A$-modules de type fini sur un anneau principal \` a l'anneau $\mathbb{Z}$ pour obtenir le classique théorème de classification des groupes finis.\\

\paragraph{}\textbf{Corollaire.}\textit{ Soit $M$ un groupe abélien de type fini. Il existe un unique $r\in\mathbb{Z}_{\geq 0}$ et une unique suite d'entiers positifs $d_{1}|d_{2}|\dots |d_{s}$ tels que} $$M\simeq \mathbb{Z}^{r}\oplus ( \oplus_{1\leq i\leq s}\mathbb{Z}/d_{i}).$$ 
\paragraph{}\textbf{Exercice.} Donner la liste des groupes abéliens d'ordre $6$, $18$, $24$ et $36$.
 

\subsubsection{}\textbf{Algèbre linéaire} On peut également appliquer la classification à l'anneau $k[T]$ des polynômes à une indéterminée sur le corps $k$  pour obtenir la classification des classes de conjugaison des endomorphisme d'un $k$-espace vectoriel de dimension finie par les invariants de similitude. Plus précisément, si $V$ est un $k$-espace vectoriel de dimension finie tout endomorphisme $u:V\rightarrow V$ définit une structure de $k[T]$ module $V_{u}$ sur $V$ par $P(T)v=P(u)(v)$, $P\in k[T]$, $v\in V$. Le $k[T]$-module $V_{u}$ est évidemment de type fini et de torsion. Il existe donc une unique suite de polynômes $P_{u,1}|P_{u,2}|\cdots |P_{u,r_{u}}$ telle que 
$$V_{u}\simeq k[T]/P_{u,1}\oplus\cdots\oplus k[T]/P_{u,r_{u}}.$$
On dit que la suite $P_{u,1}|P_{u,2}|\cdots |P_{u,r_{u}}$ est la suite des \textit{invariants de similitude} de l'endomorphisme $u$.\\

\paragraph{}\textbf{Exercice.} (Classification des classes de conjugaison par les invariants de similitude)
\begin{enumerate}
\item Soit $u,u':V\rightarrow V$ deux endomorphismes. Montrer qu'il existe $\phi\in \hbox{\rm Aut}_{k}(V)$ tel que $u=\phi\circ u'\circ \phi^{-1}$ si et seulement si $u$ et $u'$ ont mêmes invariants de similitude. 
\item Calculer le polynôme minimal et le polynôme caractéristique de $u$ en fonction de sa suite d'invariants de similitude. Montrer plus précisément qu'il existe une base du $k$-espace vectoriel $V$ dans laquelle $u$ a pour matrice la matrice diagonale par blocs dont les blocs diagonaux sont les matrices compagnons des $P_{u,i}$.
\item Calculer le nombre de classes de conjugaison (sous  $\hbox{\rm GL}_n(\F_q)$) dans $M_n(\F_q)$, dans $\hbox{\rm GL}_n(\F_q)$.\\
\end{enumerate}

 Au lieu d'appliquer le Corollaire \ref{StructureTors} sous la forme énoncée, on peut l'appliquer avec la décomposition donnée par Krull-Schmidt (\textit{cf.} preuve) \textit{i.e.} il existe une unique famille de polynômes irréductibles $P_1,\dots, P_s$ et des familles d'entiers 
 $n_{i,1}\geq \cdots\geq n_{i,r_i}>0$, $i=1,\dots ,s$
tels que $$V_u\simeq \bigoplus_{1\leq i\leq s}\bigoplus_{1\leq j\leq r_i}k[T]/P_i^{n_j}.$$
On retrouve alors la décomposition de Jordan en concaténant les bases $X^jP_i^l$, $0\leq j\leq d_i)-1$, $0\leq l\leq n_i-1$ de $k[T]/P_i^{n_j}$, $i=1,\dots, s$. Dans cette base, la matrice de $u$ est diagonale par blocs avec $s$ blocs $D_1,\dots, D_s$ et chaque bloc $D_i$  de la forme 
$$ \left(\begin{tabular}[c]{ccccc}
$C(P_i)$&$0$&$\cdots$&$0$&$0$\\
$U$&$C(P_i)$&&$0$&$0$\\
$0$&&$\cdots$&&\\
$0$&&&$U$&$C(P_i)$\\
\end{tabular}\right),$$
o\`u   $U$ est ma matrice carrée de taille $d_i\times d_i$ avec $u_{1,d_i}=1$ et $u_{i,j}=0$ sinon.\\  
 
\subsubsection{}\textbf{Base adaptée}\label{BA} La forme suivante du théorème de structure est aussi très utile en pratique. \\


\paragraph{}\textbf{Théorème. } (Base adaptée) \textit{Soit $A$ un anneau principal, $M$ un $A$-module libre de rang $r$ et $N\subset M$ un sous-$A$-module. Il existe un unique entier $0\leq s\leq r$, une unique suite $Ad_1\supset Ad_2\supset\cdots \supset Ad_s$ d'idéaux de $A$ et $m_1,\dots,m_r\in M$ tels que }
$$N=\bigoplus_{1\leq i\leq s}Ad_i m_i\subset \bigoplus_{1\leq i\leq r}Am_i=M.$$

\begin{proof} L'unicité de $s$ et de la suite  $Ad_1\supset Ad_2\supset\cdots\supset Ad_s$ résulte du Corollaire \ref{StructureTors} car $r-s$ est le rang de la partie libre de $M/N$ et $Ad_1\supset Ad_2\supset\cdots\supset Ad_s$ est la suite des invariants de la partie de torsion de  $M/N$. L'existence est un peu plus délicate. On procède par reccurence sur $r$. Si $r=1$, c'est la traduction du fait que $A$ est principal. Si $r\geq 1$, l'idée est de construire $d_1, e_1$ à partir de l'inclusion $N\hookrightarrow M$. Pour cela, on introduit l'ensemble $\mathcal{E}$ des idéaux de la forme $f(N)\subset A$, o\`u $f:M\rightarrow A$ est un morphisme de $A$-module. Comme $A$ est noetherien, $\mathcal{E}$ contient au moins un élément maximal $f(N)=Ad=Af(n)$. 
\begin{enumerate}
\item En fait, pour tout $g:M\rightarrow A$ on a $g(N)\subset f(N)$. En effet, si $\delta$ est le pgcd de $d$ et $g(n)$ il existe $u,v\in A$ tels que $ud+vg(n)=\delta$. Donc $$f(N)=Ad\subset A\delta=A(uf+vg)(n)\subset (uf+vg)(N)$$
Par maximalité de $f(N)$, cela implique $f(N)= Ad=A\delta=(uf+vg)(N)$. En particulier, pour tout $n'\in N$, $d$ divise $ (uf+vg)(n')$. Mais $f(N)=Ad$, donc $d$ divise aussi $f(n')$. On en déduit que $d$ divise $vg(n')$ et comme $d$ est premier avec $v$, que $d$ divise $g(n')$. \textit{In fine}, on a $g(N)\subset Ad=f(N)$ comme annoncé.\\
\item Il existe $\mu\in M$ tel que $dm=n$. Choisissons une  $A$-base quelconque $e_1,\dots, e_r$ de $M$ et notons $p_i:M\twoheadrightarrow Am_i\simeq A$ la projection correspondante sur la $i$-ème coordonnée. On a, dans cette base, $n=\sum_{a\leq i\leq r}a_ie_i$ et en appliquant (1) aux $p_i$, on obtient que $d$ divise $a_i$, $i=1,\dots, r$. Donc en écrivant $a_i=db_i$ pour un certain $b_i\in A$, $i=a,\dots, r$, on peut prendre $\mu=\sum_{1\leq i\leq r} b_ie_i$.\\
\item De $de=m$, on déduit $f(d\mu)=df(\mu)=f(n)=d$ donc comme $A$ est intègre, $f(\mu)=1$. Cela donne une décomposition $M\simeq \ker(f)\oplus A\mu$ ($m=(m-f(m)\mu)+f(m)\mu $ telle que $
N=\ker(f)\cap N\oplus Ad\mu$. On peut donc appliquer l'hypothèse de recurrence à $\ker(f)\cap N\subset \ker(f)$ puisqu'on sait que $\ker(f)$ est un $A$-module libre de rang $r$ pour obtenir une suite $A\supsetneq Ad_2\supset\cdots\supset Ad_s\supsetneq 0 $ d'idéaux de $A$ et $m_2,\dots, m_r\in \ker(f)$ tels que 
$$\ker(f)\cap N=\bigoplus_{2\leq i\leq s}Ad_im_i\subset \bigoplus_{2\leq i\leq r}Am_i=\ker(f).$$
Enfin, en appliquant à nouveau (1) à la projection $M=A\mu\oplus   \displaystyle{\bigoplus_{2\leq i\leq r}}Am_i\twoheadrightarrow Am_2\simeq A$, on voit que $d$ divise $d_2$.\\
\end{enumerate}
\end{proof}

\paragraph{}\textbf{Corollaire. } (Classes d'équivalence) \textit{On considère l'action de $\SGL_n(A)\times \SGL_m(A)$ sur $M_{n,m}(A)$ donnée par $(P,Q)\cdot M=PMQ^{-1}$. L'ensemble des classes d'équivalence $M_{n,m}(A)/\SGL_n(A)\times \SGL_m(A)$ est canoniquement en bijection avec les suites $Ad_1\supset Ad_2\supset \dots\supset Ad_n$ d'idéaux de $A$.}
\begin{proof} On suppose $m\geq n$. Notons $M:=A^m$, $N:=A^n$ et soit $f:M\rightarrow N\in \hbox{\rm Hom}_A(M,N)$. Par le théorème de la base adaptée pour $f(M)\subset N$ il existe un unique $0\leq r\leq n$, une unique suite d'idéaux $A\supsetneq Ad_1\supset Ad_2\supset\dots\supset Ad_r$ et des éléments $\nu_1,\dots, \nu_n\in N$ tels que 
$$f(M)=\oplus_{1\leq i\leq r}Ad_i\nu_i\subset \oplus_{1\leq i\leq n}A\nu_i=N. $$
Comme $f(M)$ est un $A$-module libre, la suite exacte courte 
$$0\rightarrow \ker(f)\rightarrow M\stackrel{f}{\rightarrow}f(M)\rightarrow 0$$
est scindée. Notons $s:f(M)\rightarrow M$ un scindage. On a alors $M\simeq \ker(f)\oplus s(f(N))$. Comme $A$ est principal et $M$ est un $A$-module libre, $\ker(f)\subset M$ est encore un $A$-module libre. En concaténant une $A$-base de $\ker(f)$ et la $A$-base $s(\nu_1),\dots, s(\nu_n)$ de $f(N)$, on obtient une $A$-base $\mu_1,\dots, \mu_n$ de $M$. La matrice de $f$ dans les bases $\mu_1,\dots, \mu_m$ et $\nu_1,\dots,\nu_n$ est de la forme 
$$D(d_1,\dots, d_n):=\left(\begin{tabular}[c]{ccccc}
$d_1$&$0$&$\cdots$&$0$&$0$\\
$0$&$d_2$&&$0$&$0$\\
$0$&&$\cdots$&&\\
$0$&&&$d_n$&$0$\\
\end{tabular}\right).$$
On a donc montré que si $f,g:M\rightarrow N$ sont des morphismes de $A$-modules tels que $N/f(M)\simeq N/g(M)$ alors $f,g$ sont équivalents. La réciproque est presque immédiate car s'il existe des automorphismes $\phi\in Aut_A(M)$, $\psi\in Aut_A(N)$ tels que $f\circ \phi=\psi\circ g $ alors $\psi:N\tilde{\rightarrow}N$ se restreint en un isomorphisme de $A$-modules $\psi:g(M)\tilde{\rightarrow}f(\phi(M))=f(M)$ donc induit un isomorphisme de $A$-modules $\overline{\psi}:N/g(M)\tilde{\rightarrow} N/f(M)$. \end{proof}

\textbf{Remarque.} Dans le cas o\`u $A=k$ est un corps commutatif, on retrouve le thèorème de classification des classes d'équivalence par le rang de la matrice. \\
 


\paragraph{}\textbf{Exercice.} Soit $M$ un $\Z$-module libre de rang fini $m$ et $\phi\in \hbox{\rm End}_{\Z}(M) $ tel que $\phi\otimes\Q\in \hbox{\rm Aut}_{\Q}(M\otimes \Q)$ est inversible. Montrer que $\phi(M)\subset M$ est d'indice fini et calculer $[M:\Phi(M)$.\\
 
 
 \section{Produit tensoriel}\index{Produit tensoriel (Modules)} 

\subsection{Construction}Soit $M_{1} ,\dots, M_{r}$ et $M$ des $A$-modules.  Notons $$L_{r,A}(M_{1}\times\cdots\times M_{r},M)$$ l'ensemble des applications $f:M_{1}\times\cdots\times M_{r}\rightarrow M$ qui sont $r$-$A$-linéaires \textit{i.e.} telles que $f\circ \iota_{i}:M_{i}\rightarrow M$ est un morphisme de $A$-modules, $i=1,\dots,r$. \\



 Notons $$\Sigma:= A^{(M_{1}\times\cdots\times M_{r})},$$
le $A$-module libre engendré par $M_{1}\times\cdots\times M_{r}$, $(m_1,\dots, m_r)\in M_{0}$ l'élément correspondant au terme avec des $0$ partout sauf en l'indice $(m_1,\dots, m_r)$  
 et $R \subset \Sigma$ le sous $A$-module engendré par les éléments de la forme 
$$(m_{1},\dots, a_{i}m_{i}+a_{i}'m_{i}',\dots, m_{r})-a_{i}(m_{1},\dots, m_{i},\dots, m_{r})-a_{i}'(m_{1},\dots, m_{i}',\dots, m_{r}).$$
En posant $M_1\otimes_{A}\cdots \otimes_{A} M_r:=\Sigma/R$ et $$\begin{tabular}[t]{llllll}
$p:$&$M_{1}\times\cdots\times M_{r}$&$\rightarrow$&$A^{(M_{1}\times\cdots\times M_{r})}$&$\rightarrow$&$M_1\otimes_{A}\cdots \otimes_{A} M_r$\\
&$(m_{1},\dots,m_{r})$&$\rightarrow$&$1\cdot (m_{1},\dots,m_{r})$&$\rightarrow$&$(m_{1},\dots,m_{r})$mod$M_{00}=:m_{1}\otimes\cdots\otimes m_{r}$
\end{tabular}$$
on vérifie facilement que $p:M_{1}\times\cdots\times M_{r}\rightarrow M_1\otimes_{A}\cdots \otimes_{A} M_r$ est une application $A$-$r$-linéaire. On prendra garde que $p:M_{1}\times\cdots\times M_{r}\rightarrow M_1\otimes_{A}\cdots \otimes _{A}M_r$  n'est pas surjective en général mais que, $M_1\otimes_{A}\cdots \otimes_{A} M_r$ est engendré comme $A$-module par les éléments de la forme $m_{1}\otimes\cdots\otimes m_{r}$.


\subsubsection{}\label{PTUniv}\textbf{Lemme.} (Propriété universelle du produit tensoriel) \textit{Pour toute famille $M_{1} ,\dots, M_{r}$ de  $A$-modules, il existe un $A$-module $T$ et une application $r$-$A$-linéaire $p:M_1\times\cdots\times M_r\rightarrow T$ tels que pour tout $A$-module $M$ et pour toute application  $r$-$A$-linéaire $f:M_1\times\cdots\times M_r\rightarrow M$ il existe un unique morphisme de $A$-modules $\overline{f}:T\rightarrow M$ tel que $\overline{f}\circ p=f$.}
% Notons $$L_{r,A}(M_{1}\times\cdots\times M_{r},M)$$ l'ensemble des applications $f:M_{1}\times\cdots\times M_{r}\rightarrow M$ qui sont $r$-$A$-linéaires \textit{i.e.} telles que $f\circ \iota_{i}:M_{i}\rightarrow M$ est un morphisme de $A$-modules, $i=1,\dots,r$. La structure de $A$-module sur $M$ induit une structure de $A$-module sur  $L_{r,A}(M_{1}\times\cdots\times M_{r},M)$ et on vérifie immédiatement que 
%$$L_{r,A}(M_{1}\times\cdots\times M_{r},-):Mod_{/A}\rightarrow Mod_{/A}$$
%est un foncteur.
\begin{proof}Vérifions  que $T:=M_1\otimes_{A}\cdots \otimes_{A} M_r$ et  l'application $r$-$A$-linéaire $p:M_1\times\cdots\times M_r\rightarrow M_1\otimes_{A}\cdots \otimes_{A} M_r$ conviennent. Si $\overline{f}:M_1\otimes\cdots \otimes M_r\rightarrow M$, la condition $p\circ \overline{f}=f$ impose $\overline{f}(m_1\otimes\cdots\otimes m_r)=f(m_1,\dots, m_r)$. Comme $M_1\otimes_{A}\cdots \otimes_{A} M_r$  est engendré, comme $A$-module, par les éléments de la forme $m_1\otimes\cdots\otimes m_r$, cela montre l'unicité de $\overline{f}$ sous réserve de son existence. Par propriété universelle des $\iota_{\underline{m}}:A\hookrightarrow A^{(M_{1}\times\cdots\times M_{r})}$, $\underline{m}=(m_1,\dots, m_r)\in M_1\times\cdots\times M_r$, il existe un unique morphisme de $A$-modules $F:M_{0}=A^{(M_{1}\times\cdots\times M_{r})}\rightarrow M$ tel que $F\circ \iota_{\underline{m}}:A\rightarrow M$ est le morphisme qui envoie $1$ sur $f(m_1,\dots, m_r)$. Comme $f:M_{1}\times\cdots\times M_{r}\rightarrow M$ est $r$-$A$-linéaire,  $M_{00}\subset \ker(F)$ donc $F:M_{0} \rightarrow M $ se factorise en un morphisme de $A$-modules $\overline{f}:M_1\otimes_{A} \cdots\otimes_{A} M_r\rightarrow M$ tel que $p\circ \overline{f}=F$; en particulier  $$p\circ \overline{f}(m_1,\dots,m_r)=F(m_1,\dots, m_r)=f(m_1,\dots, m_r),\; (m_1,\dots, m_r)\in M_{1}\times\cdots\times M_{r}.$$
\end{proof}

 Comme d'habitude, $p:M_{1}\times\cdots\times M_{r}\rightarrow M_1\otimes_{A}\cdots \otimes_{A} M_r$ est unique à unique isomorphisme près.\\

 
 On peut aussi réécrire \ref{PTUniv} en disant que pour tout $A$-module $M$ l'application canonique 
$$\SHom_A(M_{1}\times\cdots\times M_{r},M)\rightarrow L_{r,A}(M_{1}\times\cdots\times M_{r},M),\; f\rightarrow p\circ f$$
est bijective ou encore, plus visuellement, 
$$\xymatrix{M_{1}\times\cdots\times M_{r}\ar[r]^{\forall f}\ar[d]_{p}&M\\
M_{1}\otimes_{A}\cdots\otimes_{A}M_{r}\ar@{.>}[ur]_{\exists ! \overline{f}}}$$
 

 Si $f_{i}:M_{i}\rightarrow N_{i}$, $i=1,\dots ,r $ sont $r$ morphismes de $A$-modules, l'application
$$\begin{tabular}[t]{llll}
$(f_{1},\dots, f_{r})$&$M_{1}\times\dots\times M_{r}$&$\rightarrow$&$N_{1}\otimes_{A}\dots\otimes_{A}N_{r}$\\
&$(m_{1},\dots,m_{r})$&$\rightarrow$&$f_{1}(m_{1})\otimes\dots\otimes f_{r}(m_{r})$
\end{tabular}$$
est $r$-$A$-linéaire donc se factorise en un morphisme de $A$-modules $f_{1}\otimes_{A}\dots\otimes_{A}f_{r}:M_{1}\otimes_{A}\dots\otimes_{A}M_{r}\rightarrow N_{1}\otimes_{A}\dots\otimes_{A}N_{r}$ tel que $f_{1}\otimes_{A}\dots\otimes_{A}f_{r}(m_{1}\otimes\dots m_{r})=(f_{1},\dots, f_{r})(m_{1},\dots,m_{r})=f_{1}(m_{1})\otimes\dots\otimes f_{r}(m_{r})$.  

 
\subsection{Propriétés élémentaires}\label{Ad1} 


\subsubsection{}\label{PTSum}\textbf{Lemme.} (Le produit tensoriel 'commute' aux sommes directes) \textit{Soit $M_i$, $i\in I$ et $M$ des $A$-modules. On a un isomorphisme canonique}
$$\begin{tabular}[t]{lll}
$M\otimes_A(\bigoplus_{i\in I}M_{i})$&$\tilde{\rightarrow}$&$\bigoplus_{i\in I}(M\otimes_AM_{i})$\\
$m\otimes(m_i)_{i\in I}$&$\rightarrow$&$(m\otimes m_i)_{i\in I}$\\
\end{tabular}$$
\begin{proof}Vérifions d'abord que $\phi:M\otimes_A(\bigoplus_{i\in I}M_{i}) \rightarrow  \bigoplus_{i\in I}(M\otimes_AM_{i})$ est bien définie. L'application $\Phi:M\times \bigoplus_{i\in I}M_{i}\rightarrow  \bigoplus_{i\in I}(M\otimes_AM_{i})$, $(m,(m_i)_{i\in I})\rightarrow (m\otimes m_i)_{i\in I}$ est $2$-$A$-linéaire donc par propriété universelle de $p:M\times (\bigoplus_{i\in I}M_{i})\rightarrow M\otimes_A(\bigoplus_{i\in I}M_{i})$ se factorise effectivement en un morphisme de $A$-module $\phi:M\otimes_A(\bigoplus_{i\in I}M_{i}) \rightarrow  \bigoplus_{i\in I}(M\otimes_AM_{i})$ tel que $p\circ \phi=\Phi$. Inversement, pour tout $i\in I$ l'application $\Psi_i:M\times M_i\rightarrow M\otimes_A(\bigoplus_{i\in I}M_{i})$, $(m,m_i)\rightarrow m\otimes \iota_i(m_i)$ est $2$-$A$-linéaire donc  par propriété universelle de $p:M\times M_{i}\rightarrow M\otimes_A M_i$ se factorise  en un morphisme de $A$-module $\psi_i:M\otimes_A M_{i} \rightarrow  M\otimes_A(\bigoplus_{i\in I}M_{i})$ tel que $p\circ \psi_i=\Psi_i$. Puis, par propriété universelle de $\iota_i:M\otimes_A M_i\rightarrow \bigoplus_{i\in I}(M\otimes_AM_{i}) $, $i\in I$ on obtient un unique morphisme de $A$-module $\psi:\bigoplus_{i\in I}(M\otimes_AM_{i})\rightarrow M\otimes_A(\bigoplus_{i\in I}M_{i})$ tel que $\psi\circ \iota_i=\Psi_i$, $i\in I$. On vérifie sur les constructions que $\phi$, $\psi$ sont inverses l'un de l'autre.\end{proof}



 Les preuves des lemmes suivant sont du même acabit \textit{i.e.} purement formelles et laissées en exercice au lecteur.

\subsubsection{}\label{PTComAss}\textbf{Lemme.} (Commutativité et associativité) \textit{Soit $L,M,N$ des $A$-modules. On a des isomorphismes (de $A$-modules) canoniques}
$$\begin{tabular}[t]{ccc}
$L\otimes_{A}(M\otimes_A N)$&$\tilde{\rightarrow}$&$(L\otimes_A M)\otimes_A N$\\
$l \otimes(m\otimes n)$&$\rightarrow$&$(l\otimes m )\otimes n$\\
&&\\
$M\otimes_{A}N$&$\tilde{\rightarrow}$&$N\otimes_A M $\\
$m \otimes n$&$\rightarrow$&$n\otimes m  $\\
\end{tabular}$$

\subsubsection{}\label{PTTaut}\textbf{Lemme.}  \textit{Soit  $M$ un $A$-module. On a un isomorphisme canonique}
$$\begin{tabular}[t]{ccc}
$A\otimes_{A}M$&$\tilde{\rightarrow}$&$M$\\
$a\otimes m$&$\rightarrow$&$am$
\end{tabular}$$


\subsubsection{}Soit $I$ un ensemble. Pour $i\in I$ on rappelle qu'on note $e_i:=(\delta_{i,j})_{j\in I}\in A^{(I)}$ le $i$-ème élément de la base canonique de $A^{(I)}$.\\


\textbf{Lemme.} \textit{Soit $I_1,\dots, I_r$ des ensembles. On a un isomorphisme de $A$-modules canonique} 
$$A^{(I_1)}\otimes_{A}\cdots\otimes_{A}A^{(I_r)}\tilde{\rightarrow}A^{(I_1\times\dots\times  I_r)}$$
\textit{qui envoie $e_{i_1}\otimes\cdots\otimes e_{i_r}$ sur $e_{(i_1,\dots, i_r)}$, $(i_1,\dots, i_r)\in I_1\times\cdots\times I_r$. }
\begin{proof}On peut le déduire formellement des Lemmes \ref{PTSum}, \ref{PTComAss}, \ref{PTTaut}: $$\begin{tabular}[t]{ll}
$A^{(I_1)}\otimes_{A}\cdots\otimes_{A}A^{(I_r)}$ &$(\tilde{\rightarrow}A^{(I_1)}\otimes_{A}\cdots\otimes_A A^{(I_{r-1})})\otimes_{A}A^{(I_r)}$\\
&$\tilde{\rightarrow} ((A^{(I_1)}\otimes_{A}\cdots\otimes_{A}A^{(I_{r-1}})\otimes_AA)^{(I_r)}$\\
&$\tilde{\rightarrow} (A^{(I_1)}\otimes_{A}\cdots\otimes_{A}A^{(I_{r-1}}) )^{(I_r)}$\\
&$\tilde{\rightarrow}\cdots  \tilde{\rightarrow}A^{(I_1\times\dots\times  I_r)}$.
\end{tabular}$$
On peut aussi donner un argument direct. En effet, l'application  $A^{(I_1)}\times \cdots\times A^{(I_r)}\tilde{\rightarrow}A^{(I_1\times\dots\times  I_r)}$, $((a_{i_1})_{i_1\in I_1}, \cdots,(a_{i_r})_{i_r\in I_r})\rightarrow (a_{i_1}\cdots a_{i_r})_{(i_1,\dots, i_r)\in I_1\times\cdots\times I_r}$ est $r$-$A$-linéaire donc se factorise en un morphisme de $A$-modules $\phi:A^{(I_1)}\otimes_{A}\cdots\otimes_{A}A^{(I_r)} \rightarrow A^{(I_1\times\dots\times  I_r)}$ tel que $\phi((a_{i_1})_{i_1\in I_1}\otimes \cdots \otimes (a_{i_r})_{i_r\in I_r})= (a_{i_1}\cdots a_{i_r})_{(i_1,\dots, i_r)\in I_1\times\cdots\times I_r})$. Inversement, pour chaque $(i_1,\dots, i_r)\in I_1\times\cdots\times I_r$ on dispose du morphisme de $A$-modules $\psi_{(i_1,\dots, i_r)}:A\rightarrow A^{(I_1)}\otimes_{A}\cdots\otimes_{A}A^{(I_r)}\tilde{\rightarrow}A^{(I_1\times\dots\times  I_r)}$, $a\rightarrow a e_{i_1}\otimes\cdots\otimes e_{i_r}$ donc, par propriété universelle de la somme directe, d'un morphisme de $A$-modules $\psi=\oplus_{(i_1,\dots, i_r)\in I_1\times\cdots\times I_r}\psi_{(i_1,\dots, i_r)}: A^{(I_1\times\dots\times  I_r)}\rightarrow A^{(I_1)}\otimes_{A}\cdots\otimes_{A}A^{(I_r)}$ tel que $\psi\circ \iota_{(i_1,\dots, i_r)}=\psi_{(i_1,\dots, i_r)}$, $(i_1,\dots, i_r)\in I_1\times\cdots\times I_r$. On vérifie sur les définitions que $\phi$ et $\psi$ sont inverses l'une de l'autre. \end{proof}

 En particulier, si $M_i$ est un $A$-module  libre  de rang fini $d_i$, $i=1,\dots, r$, $M_{1}\otimes_{A}\cdots\otimes_{A}M_{r}$ est un $A$-module libre de rang $d_1\dots d_r$.  
  
\subsubsection{}\textbf{Lemme.} \textit{Soit $M,N$ des $A$-modules. On a des morphismes de $A$-modules canoniques}
$$ \begin{tabular}[t]{cll}
$M^{\vee}\otimes_AN$&$ \rightarrow $&$\hbox{\rm Hom}_A(M,N)$\\
$f\otimes n $&$\rightarrow$&$f(-)n$;\\
\end{tabular},\;\; 
  \begin{tabular}[t]{cll}
$M^{\vee}\otimes_AN^{\vee}$&$ \rightarrow $&$(M\otimes_AN)^{\vee}$\\
$f\otimes g $&$\rightarrow$&$m\otimes n\rightarrow f(m)g(n)$.\\
\end{tabular}$$
et $$\begin{tabular}[t]{cll}
$\hbox{\rm End}_A(M)\otimes_A\hbox{\rm End}_A(N)$&$ \rightarrow $&$\hbox{\rm End}_A(M\otimes_AN)$\\
$f\otimes g $&$\rightarrow$&$m\otimes n\rightarrow f(m)\otimes g(n)$;\\
\end{tabular}$$
\textit{Si de plus $M$ et $N$ sont libres de rang fini, ces trois morphismes sont des isomorphismes.}
 
\begin{proof}Les morphismes se construisent en utilisant la propriété universelle du produit tensoriel. En général, il n'y a par contre pas de fa\c{c}on canonique de construire des inverses de ces morphismes (et d'ailleurs, ce ne sont pas toujours des isomorphismes). Mais si  $M$ et $N$ sont libres de rang fini, on peut vérifier que ces trois morphismes envoient à chaque fois une base sur une base. \end{proof}
\subsection{Adjonctions}


\subsubsection{}\textbf{$-\otimes_A M$ versus $\hbox{\rm Hom}_A(M,-)$}\\


\textbf{Lemme.}\label{Ad1Prop}(Adjonction-1) \textit{Soit $L,M,N$ des $A$-modules. On a des isomorphismes (de $A$-modules) canoniques}
$$
 \begin{tabular}[t]{ccccc}
$\hbox{\rm Hom}_{A}(L,\hbox{\rm Hom}_{A}(M,N))$&$\tilde{\rightarrow}$&$L_{r,A}(L \times M,N)$&$\tilde{\rightarrow}$&$\hbox{\rm Hom}_{A}(L\otimes_{A}M,N)$\\
$f$&$\rightarrow$&$(l,m)\rightarrow f(l)(m)$&&\\
$l\rightarrow \beta(l,-)$&$\leftarrow$&$\beta$&&
\end{tabular}$$
\textit{(Le deuxième isomorphisme est simplement la propriété universelle du produit tensoriel).}\\
 

\textbf{Exercice} Soit $M$ un $A$-module. 
Montrer  que pour toute suite exacte courte de $A$-modules $$0\rightarrow N'\stackrel{u}{\rightarrow} N\stackrel{v}{\rightarrow} N'' \rightarrow 0$$
\begin{enumerate}[leftmargin=* ,parsep=0cm,itemsep=0cm,topsep=0cm]  
\item La suite $$0\rightarrow \hbox{\rm Hom}_{A}(M,N')\stackrel{u\circ }{\rightarrow} \hbox{\rm Hom}_{A}(M,N)\stackrel{v\circ }{\rightarrow} \hbox{\rm Hom}_{A}(M,N'')$$
est exacte.  
\item La suite $$M\otimes_AN'\stackrel{Id\otimes u}{\rightarrow} M\otimes_AN\stackrel{Id\otimes v}{\rightarrow} M\otimes_AN'' \rightarrow 0$$
est exacte.  
\end{enumerate}
 
\subsubsection{}\textbf{Extension/Restriction des scalaires.}\label{Ad2} Soit $\phi:A\rightarrow B$ une $A$-algèbre. A tout $B$-module $M$ on peut associer un $A$-module noté $M|_A$ (ou $\phi_*M$) dont le groupe abélien sous-jacent est encore $M$ et dont la structure de $A$-module est définie par $a\cdot m=\phi(a)m$, $a\in A$, $m\in M$. Tout morphisme de $B$-modules $f:M\rightarrow N$ induit alors  tautologiquement un morphisme de $A$-modules $f|_A:M|_A\rightarrow N|_A$. On voudrait, inversement, associer à tout $A$-module $M$ un $B$-module $\phi^*M$ et à tout morphisme de $A$-modules $f:M\rightarrow N$ un morphisme  de  $B$-modules $\phi^*f:\phi^*M\rightarrow \phi^*N$. \\

 Soit $M$ un   $B$-module et $N$ un $A$-module. Pour tout $b_0\in B$, l'application 
$$\begin{tabular}[t]{lll}
$ M\times N$&$\rightarrow$&$M\otimes_{A}N(:=(M|_A)\otimes_AN)$\\
$(m,n)$&$\rightarrow$&$(b_{0}m)\otimes n$
\end{tabular}$$
est $2$-$A$-linéaire donc  se factorise en un morphisme de $A$-module $ b_0\cdot : M\otimes_A N\rightarrow  M\otimes_A N$, $m\otimes n\rightarrow b_0\cdot (m\otimes n):=(b_0m)\otimes n$. On vérifie que cela définit une structure de $B$-module sur $M\otimes_AN$. Tout morphisme de $A$-modules $f:N\rightarrow N'$ induit alors un morphisme de $B$-modules $Id_{M}\otimes f:M\otimes_{A}N\rightarrow M\otimes_{A} N'$. Si $M=B$   muni  de la structure de $A$-module donnée par  $\phi:A\rightarrow B$ ($a\cdot b=\phi(a)b$, $a\in A$, $b\in B$),  on note parfois $\phi^*N:=B\otimes_AN$ et $\phi^*f:=Id_B\otimes f:\phi^*N\rightarrow \phi^*N'$. \\
 

 Les  constructions $\phi_*$, $\phi^*$ sont liées par le lemme suivant.\\

\paragraph{}\label{Ad2Lemma}\textbf{Lemme.} (Adjonction-2) \textit{Soit $M$ un $A$-module et $N$ un $B$-module. On a un isomorphisme canonique (de $\mathbb{Z}$-modules)}
$$  \begin{tabular}[t]{cll}
$\hbox{\rm Hom}_{A}(M, N|_A)$&$\tilde{\rightarrow}$&$ \hbox{\rm Hom}_{B}(B\otimes_AM,N)$\\
$f$&$\rightarrow$&$b\otimes m\rightarrow bf(m)$\\
$f(1\otimes -)$&$\leftarrow$&$f$
\end{tabular}$$
\paragraph{}\textbf{Exercice (Transitivité de l'extension des scalaires).} 

\begin{enumerate}
\item Soit $M,M' $ des $B$-modules et $N$ un $A$-module. Montrer qu'on a un isomorphisme canonique de $B$-modules
$$M'\otimes_B(M\otimes_AN)\tilde{\rightarrow} (M'\otimes_B M)\otimes_A.$$
En déduire qu'on a un isomorphisme canonique de $B$-modules $M\otimes_B(B\otimes_AN)\tilde{\rightarrow} M\otimes_AN$;
\item Soit $A\rightarrow B\rightarrow C$ des morphismes d'anneaux et $M$ un $A$-module. Montrer qu'on a un isomorphisme canonique
$$C\otimes_B(B\otimes_AM)\tilde{\rightarrow} C\otimes_AN.$$
\end{enumerate}

\paragraph{}\textbf{Extension des scalaires par un quotient.} Soit $M$ un $A$-module et $I\subset A$ un idéal. Notons $IM\subset M$ le sous-$A$-module engendré par les éléments de la forme $am$, $a\in I$, $m\in M$. Par propriété universelle du quotient, l'application $$\begin{tabular}[t]{clc}
 $A\times M/IM$&$\rightarrow$&$M/IM$\\
 $(a,\overline{m})$&$\rightarrow$&$a\overline{m}=\overline{am}$
 \end{tabular}$$
 donnée par la structure de $A$-module sur $M/IM$ se factorise en une application $A/I\times M/IM\rightarrow M/IM$, qui fait de $M/IM$ un $A/I$-module. Le l\\
 
 \textbf{Lemme.} (Propriété universelle de  $M\rightarrow M/IM$) \textit{Pour tout  $A$-module $M$ et idéal $I\subset A$, il existe un $A/I$-module $Q$ et un morphisme de $A$-modules $p:M\rightarrow (p_I)_*Q$ tel que pour tout $A$ pour tout $A/I$-module $N$ et tout morphisme de $A$-module $\phi:M\rightarrow (p_I)_*N$ il existe un unique morphisme de $A/I$-module $\overline{\phi}:Q\rightarrow N$ tel que $  \overline{\phi}\circ p=\phi$. }\\
 
\begin{proof}On vérifie que $p_{IM}:M\rightarrow M/IM$ convient. Si $\overline{\phi}:M/IM\rightarrow N$ existe la condition  $  \overline{\phi}\circ p_{IM}=\phi$  impose $  \overline{\phi}(\overline{m})=\phi(m)$, $m\in M$, d'o\`u l'unicité de $\overline{\phi}$ sous réserve de son existence. Par ailleurs, pour tout $a\in I$, $m\in M$, $\phi(am)=p_I(a)\phi(m)=0$ donc $IM\subset \ker(\phi)$ et $\phi:M\rightarrow N$ se factorise en un morphisme de $A$-modules $\overline{\phi}:M/IM\rightarrow (p_I)_*M$ qui induit tautologiquement un morphisme $ \overline{\phi}:M/IM\rightarrow  N$ de $A/I$-modules. \end{proof} 

 On peut réécrire le Lemme en disant que pour tout $A/I$-module $N$ l'application canonique
$$\SHom_{A/I}(M/IM,N)\rightarrow \SHom_{ A}(M ,N) ,\; \phi\rightarrow (\phi\circ p_{IM})$$
est bijective. Or \ref{Ad2Lemma} dit que le morphisme de $A$-module $p:M\rightarrow A/I\otimes_A M$, $m\rightarrow \overline{1}\otimes m$ vérifie la même propriété. Par unicité des objets universels, on a donc un unique morphisme de $A/I$-modules $\phi:A/I\otimes M\rightarrow M/IM$ tel que $\phi\circ p=p_{IM}$. On peut aussi démontrer cela `à la main', comme suit.\\

 L'application canonique
$$\begin{tabular}[t]{lll}
$A\times  M$&$\rightarrow$&$M/IM$\\
$(a, m)$&$\rightarrow$&$\overline{am}$
\end{tabular}$$
est $2$-$A$-linéaire et passe au quotient en une application  $2$-$A$-linéaire $A/I\times M\rightarrow M/IM$ donc se factorise en un morphisme de $A$-modules $f:(A/I)\otimes_A M\rightarrow M/IM$. Inversement, l'application $M\rightarrow (A/I)\otimes_A M$, $m\rightarrow 1\otimes m$ est un morphisme de $A$-modules dont le noyau contient $IM$  donc se factorise en un morphisme de  $A$-modules $M/IM\rightarrow (A/I)\otimes_A M$. Par construction, $f$ et $g$ sont inverses l'une de l'autre. On a donc montré qu'on avait un isomophisme de $A$-modules canoniques 
$$(A/I)\otimes_A M\tilde{\rightarrow} M/IM.$$

\textbf{Exemple.} Soit $A$ un anneau principal, $ a,b\in A$ des éléments premiers entre eux et $M$ un $A$-module tel que $aM=0$. Par Bézout on a alors $bM=M$ donc
$(A/b)\otimes_AM=0$. Par exemple si $p\not= q$ sont deux nombres premiers, $\Z/p\otimes_\Z \Z/q =0$.\\

\paragraph{}\textbf{Extension des scalaires par une localisation.} Soit $S\subset A$ une partie multiplicative et $M$ un $A$-module.  Munissons le produit cartésien $S\times M$ de la relation  $\sim $ définie par $(s,m)\sim (s',m')$ s'il existe $s''\in S$ tel que $s''(s'm-sm')=0$. \\

 On vérifie que $\sim$ est une relation d'équivalence. On remarquera que si $M$ est sans $S$-torsion, on peut, dans la définition de $\sim$, simplifier par $s''$ et la relation $\sim$ devient simplement $(s,m),(s',m')\in S\times M$, $(s,m)\sim (s',m')$ si $s'm-sm'=0$.  Mais on prendra garde que si $S$ a de la $S$-torsion,  la relation $(s,m)\sim (s',m')$ si $s'm-sm'=0$ n'est pas transitive donc ne définit pas une relation d'équivalence.\\


 On note $S^{-1}M:=S\times M/\sim$ et 
$$ \begin{tabular}[t]{llll}
$-/-$ :&$S\times M$&$\rightarrow$&$S^{-1}M$\\
 &$(s,m)$&$\rightarrow$&$m/s$
 \end{tabular}$$
 la projection canonique.\\
 
 
 
  On vérifie que les applications $$\begin{tabular}[t]{lclc}
 $+$&$:S^{-1}M\times S^{-1}M$&$\rightarrow$&$S^{-1}M$\\
 &$(m/s,n/t)$&$\rightarrow$&$(tm+sn)/(st)$
 \end{tabular},\;\; \begin{tabular}[t]{lclc}
 $\cdot $&$:S^{-1}A\times S^{-1}M$&$\rightarrow$&$S^{-1}A$\\
 &$(a/s,n/t)$&$\rightarrow$&$(an)/(st)$
 \end{tabular}$$
munissent $S^{-1}A$ d'une structure de $S^{-1}A$-module et que l'application canonique $\iota_S:=-/1:M\rightarrow (\iota_S)_*S^{-1}M$ est un morphisme de $A$-modules de noyau 
$\ker(\iota_S)=\lbrace m\in M\; |\; \exists s\in S\; \hbox{\rm tel que}\; sm=0\rbrace$.\\


\textbf{Lemme.} (Propriété universelle de la localisation des $A$-modules) \textit{Pour toute partie multiplicative $S\subset A\setminus\lbrace 0\rbrace$ et pour tout $A$-module $M$ il existe un $S^{-1}A$-module $L$ et un morphisme de $A$-modules $\iota_S:M\rightarrow (\iota_S)_*L$ tel que   pour tout $S^{-1}A$-module $N$ et pour tout  morphisme de $A$-module $f:M\rightarrow N$, il  existe un unique morphisme de $S^{-1}A$-modules $\tilde{f}:L\rightarrow N$ tel que $f=  \tilde{f}\circ \iota_S$.}\\

\begin{proof} On vérifie que $\iota_S:=-/1:M\rightarrow (\iota_S)_*S^{-1}M$ convient... \end{proof}

 En particulier,  pour tout morphisme de $A$-modules $f:M\rightarrow N$, en appliquant la propriété universelle de $\iota_S:M\rightarrow S^{-1}M$ au morphisme de $A$-modules $M\stackrel{f}{\rightarrow} N\stackrel{\iota_S}{\rightarrow} S^{-1}N$ on obtient  un morphisme de $S^{-1}A$-modules $S^{-1}f:S^{-1}M\rightarrow S^{-1}N$  donné explicitement par $f(m/s)=f(m)/s$, $s\in S$, $m\in M$. \\

 On peut réécrire le Lemme en disant que pour tout $S^{-1}A$-module $N$ l'application canonique
$$\SHom_{S^{-1}A}(S^{-1}M,(\iota_S)_*N)\rightarrow \SHom_{ A}(M ,N) ,\; \phi\rightarrow (\phi\circ \iota_S)$$
est bijective. Or \ref{Ad2Lemma} dit que le morphisme de $A$-modules $\iota:M\rightarrow (\iota_S)_*(S^{-1}A\otimes_A M)$, $m\rightarrow 1/1\otimes m$ vérifie la même propriété. Par unicité des objets universels, on a donc un unique morphisme de $S^{-1}A$-modules $\phi:S^{-1}A\otimes M\rightarrow S^{-1}M$ tel que $\phi\circ \iota=\iota_S$. Là encore, on peut    aussi démontrer cela `à la main'.\\

 L'application canonique
$$\begin{tabular}[t]{lll}
$S^{-1}A\times  M$&$\rightarrow$&$S^{-1}M$\\
$(a/s, m)$&$\rightarrow$&$ (am)/s(=a(m/s)$
\end{tabular}$$
est bien définie et $2$-$A$-linéaire donc se factorise en un morphisme de $A$-modules $f:S^{-1}A\otimes_A  M \rightarrow S^{-1}M$ qui est, automatiquement, un morphisme de $S^{-1}A$-modules. Inversement, l'application $S\times M\rightarrow S^{-1}A\otimes_A  M$, $(s,m)\rightarrow (1/s)\otimes m$ se factorise en un morphisme de  $S^{-1}A$-modules $g:S^{-1}M\rightarrow S^{-1}A\otimes_A  M $. Par construction, $f$ et $g$ sont inverses l'une de l'autre. On a donc montré qu'on avait un isomophisme de $S^{-1}A$-modules canonique 
$$S^{-1}A\otimes_A  M\tilde{\rightarrow} S^{-1}M.$$

  
 \textbf{Exemple.} Si pour tout $m\in M$ il existe $s\in S$ tel que $sm=0$, $S^{-1}M=0$. Si pour tout $s\in S$ l'application $s\cdot -:M\rightarrow M$ de multiplication par $s$
est bijective, $\iota_S:M\rightarrow S^{-1}M$ est un isomorphisme de $A$-modules. En particulier, si $A$ est un anneau principal de corps des fractions $K$ et $M $ est un $A$-module de type fini,   $K\otimes_AM=K^{\oplus r}$ et pour tout $\frak{p}\in spec(A)\setminus \lbrace 0\rbrace$, $A_{\frak{p}}\otimes_AM=A_{\frak{p}}^{\oplus r}\oplus M(\frak{p})$, o\`u on a noté $M(\frak{p})$  la $\frak{p}$-partie de $M$ et $r$ le rang de $M$. Par exemple 
$\Q\otimes_\Z (\Z/12\times \Z/6\times\Z/3)=0$, $\Z_{2\Z}\otimes_\Z (\Z/12\times \Z/6\times\Z/3)=\Z/4\times \Z/2\times \Z/2$,  $\Z_{3\Z}\otimes_\Z (\Z/12\times \Z/6\times\Z/3)=\Z/3\times \Z/3$, $\Z_{p\Z}\otimes_\Z (\Z/12\times \Z/6\times\Z/3)=0$ pour $p\not=2,3$.

\subsection{Produit tensoriel de $A$-algèbres} Soit $\phi:A\rightarrow B$ et $\psi: A\rightarrow C$ deux $A$-algèbres. Les applications produits $B\times B\rightarrow B$ et $C\times C\rightarrow C$ sont $2$-$A$-bilinéaires donc se factorisent en des morphismes de $A$-modules $\mu_B:B\otimes_A B\rightarrow B$ et $\mu_C:C\otimes_A C\rightarrow C$. On en déduit une application 
$$(B\otimes_A C)\otimes_A (B\otimes_A C)\tilde{\rightarrow} (B\otimes_A B)\otimes_A(C\otimes_A C)\stackrel{\mu_B\otimes\mu_C}{\rightarrow}B\otimes_A C$$
dont on vérifie qu'elle munit le $A$-module $B\otimes_AC$ d'une structure de $A$-algèbre telle que les  applications $\iota_B:B\rightarrow B\otimes_AC$, $b\rightarrow b\otimes 1$ et $\iota_C:C\rightarrow B\otimes_AC$, $c\rightarrow 1\otimes c$ sont des morphismes de $A$-algèbres. \\

\subsubsection{}\label{PTAlgUniv}\textbf{Lemme.} (Propriété universelle du produit tensoriel de $A$-algèbres) \textit{Pour toutes $A$-algèbres $ A\rightarrow B$ et $ A\rightarrow C$, il existe une $A$-algèbre $T$ et des morphismes de $A$-algèbres $\iota_B:B\rightarrow T$, $\iota_C:C\rightarrow T$ tels que pour toute $A$-algèbre $A\rightarrow D$ et morphismes de $A$-algèbres $\phi_B:B\rightarrow D$, $\phi_C:C\rightarrow D$ il existe un unique morphisme de $A$-algèbres $\phi:T\rightarrow D$ tel que $\phi\circ \iota_B=\phi_B$ et $\phi\circ \iota_C=\phi_C$}
\begin{proof}On  vérifie comme d'habitude que  $B\otimes_AC$ et $\iota_B:B\rightarrow B\otimes_AC$, $\iota_C:C\rightarrow B\otimes_AC$ conviennent. Si $\phi:B\otimes_AC\rightarrow D$ existe les conditions $\phi\circ \iota_B=\phi_B$ et $\phi\circ \iota_C=\phi_C$ forcent $\phi(b\otimes c)=\phi_B(b)\phi_C(c)$, d'o\`u l'unicité de $\phi$ sous réserve de son existence. Considérons   l'application $ B\times C\rightarrow D$, $(b,c)\rightarrow \phi_B(b)\phi_C(c)$. Elle est $2$-$A$-bilinéaire donc se factorise en un morphisme de $A$-modules $\phi:B\otimes_A C\rightarrow D$ tel que $\phi(b\otimes c)=\phi_B(b)\phi_C(c)$ et on vérifie sur la construction que c'est automatiquement un morphisme de $A$-algèbres.  \end{proof}


 
 On peut aussi réécrire \ref{PTAlgUniv} en disant que pour toutes $A$-algèbres $ A\rightarrow B$, $ A\rightarrow C$  et $A\rightarrow C$ l'application canonique 
$$\SHom_{Alg/_A}(B\otimes_AC,D)\rightarrow \SHom_{Alg/_A}(B ,D)\times \SHom_{Alg/_A}( C,D),\; \phi\rightarrow (\phi\circ \iota_B,\phi\circ \iota_C)$$
est bijective ou encore, plus visuellement, 
$$\xymatrix{&&D\\
C\ar[r]\ar@/^1pc/[urr]^{\phi_C}&B\otimes_AC\ar@{.>}[ur]{\exists ! \phi}&\\
A\ar[u]\ar[r]\ar@{}[ur]|{\square}&B\ar[u]\ar@/_1pc/[uur]_{\phi_B}&}$$
 
 
 
\subsubsection{}\textbf{Exercice.} 
\begin{enumerate} 
\item Soit $I,J\subset A$ deux idéaux. Montrer qu'on a un isomorphisme canonique de $A$-algèbres
$$(A/I)\otimes_A (A/J)\tilde{\rightarrow}A/(I+J).$$
Si $A$ est un anneau principal et $a,b\in A$, calculer  
$(A/a)\otimes_A (A/b)$.
\item Montrer qu'on a un isomorphisme canonique de $A$-algèbres $A[X_1]\otimes_A\cdots\otimes_AA[X_n]\tilde{\rightarrow} A[X_1,\dots, X_n]$.
\item Si $\varphi: A\rightarrow B$ est un morphisme d'anneaux et $P\in A[X]$, montrer qu'on a un isomorphisme canonique de $B$-algèbres $$B\otimes_A (A[X]/P)\tilde{\rightarrow} B[X]/\varphi(P)$$ (on note encore $\varphi: A[X]\rightarrow B[X]$ le morphisme obtenu en appliquant $\varphi$ aux coefficients). \\
Calculer $\C \otimes_{\R}\C$. Est-ce un corps? Même question avec $\Q(i)\otimes_{\Q}\Q(\sqrt{2})$.
\end{enumerate}
 
\textbf{Remarque.} On notera les similitudes suivantes au niveau des propriétés universelles.
$$\begin{tabular}[t]{l|l|l}
&$A$-modules&$A$-algèbres\\
\hline\\
Objets libres de rang fini&$A^{\oplus n}$&$A[X_1,\dots, X_n]$\\
Coproduits finis&$\oplus_{1\leq i\leq n}M_i$&$A_1\otimes_A\cdots\otimes_AA_n$\\
Produit&$\prod_{i\in I}M_i$&$\prod_{i\in I}A_i$\\
\end{tabular}$$


 
\part{Extensions de corps et théorie de Galois}
 Dans cette partie du cours, sauf mention explicite du contraire, tous les anneaux considérés sont commutatifs. On rappelle qu'un corps (commutatif) est un anneau (commutatif) dont tous les éléments non nuls sont inversibles. On a vu dans les chapitres précédents un certain nombre de techniques pour construire des corps intéressants. Par exemple, 
\begin{itemize}[leftmargin=* ,parsep=0cm,itemsep=0cm,topsep=0cm] 
\item Si $A$ est un anneau intègre,  $Frac(A)$ est un corps.\\
Ex.: $\Q=Frac(\Z)$, si $k$ est un corps, $k(T_1,\dots, T_r)=Frac(k[T_1,\dots, T_r])$, \textit{etc.}
\item Si $k$ est un corps muni d'une valeur absolue \textit{i.e.} d'une application $|-|:k\rightarrow \R_{\geq 0}$ telle que $|x|=0$ si et seulement si $x=0$, $|xy|=|x||y|$ et $|x+y|\leq |x|+ly|$, l'ensemble $Cau(k)\subset k^\N$ des suites de Cauchy de $(k,|-|)$ est une sous-$k$-algèbre et on vérifie facilement que le sous-ensemble $Cau_0(k)\subset Cau(k)$ des suites de Cauchy qui convergent vers $0$ est un idéal maximal. On dit que $\widehat{k}:=\widehat{k^{|-|}}:=Cau(k)/Cau_0(k)$ est le complété de $k$ par rapport à la valeur absolue $|-|$. Par exemple, sur $\Q$, on peut considérer 
\begin{itemize}[leftmargin=* ,parsep=0cm,itemsep=0cm,topsep=0cm] 
\item la valeur absolue usuelle $|x|=x$ si $x\geq 0$, $|x|=-x$ si $x<0$. On obtient ainsi le corps des réels $\R=\widehat{\Q^{|-|}}$;
\item pour chaque premier $0\not=p$, la valeur absolue $p$-adique $|x|_p=p^{-v_p(x)}$. On obtient ainsi  On obtient ainsi le corps des $p$-adiques $\Q_p=\widehat{\Q^{|-|_p}}$;
\end{itemize}
\item Si $\frak{m}\subset A$ est un idéal maximal, $A/\frak{m}$ est un corps.  \\
Ex.: si $A$ est un anneau principal et $p$ un élément irréductible, $A/p$ est un corps: $\F_p=\Z/p$, $\C=\R[T]/(T^2+1)$, $\Q[T]/T^p+T^{p-1}+\cdots+1$, $p$: premier, \textit{etc.}
\item Si $\frak{p}\subset A$ est un idéal premier, $A_\frak{p}$ est un anneau local d'unique idéal maximal $\frak{p}A_\frak{p}$ donc $\kappa(\frak{p})=A_\frak{p}/\frak{p}A_\frak{p}$ est un corps, \textit{etc.}
\end{itemize}


\section{Extensions algébriques, extensions transcendantes}
 Soit $k,K$ deux corps. On dit que $K$ est une extension de $k$ ou une $k$-extension ou que $k$ est un sous-corps de $K$ si  $K$ est une $k$-algèbre \textit{i.e.} s'il existe un morphisme  d'anneaux $\phi:k\rightarrow K$; ce morphisme est alors automatiquement injectif, ce qui justifie la terminologie `extension / sous-corps' et le fait que, dans la suite, on notera presque toujours $k\subset K$ ou $K/k$ au lieu de  $\phi:k\rightarrow K$, identifiant implicitement $k$ et son image $\phi(k)\subset K$.  Un morphisme de $k$-extensions $K/k\rightarrow K'/k$ est, par définition, un morphisme de $k$-algèbres. Un morphisme de $k$-extensions est automatiquement injectif et on dit parfois que c'est un $k$-plongement.\\

\subsection{}\label{Dim}Si $K/k$ est une extension de corps, on notera $[K:k]$ la dimension de $K$ comme espace vectoriel sur $k$ (avec la convention que si $K$ n'est pas de dimension finie sur $k$, $[K:k]=+\infty$). Avec la convention $+\infty\cdot +\infty=+\infty$, on a \\

\textbf{Lemme.} \textit{Si $K_3/K_2$ et $K_2/K_1$ sont des extensions de corps, $[K_3:K_1]=[K_3:K_2][K_2:K_1]$.}
\begin{proof}Il suffit d'observer que si $e^3_i$, $i\in I_3$ est une $K_2$-base de $K_3$ et $e^2_i$, $i\in I_2$ est une $K_1$-base de $K_2$ alors $e_{i_2}^3e_{i_3}^2$, $i_2\in I_2$, $i_3\in I_3$ est une $K_1$-base de $K_3$.\end{proof}


\subsection{Eléments algébriques, éléments transcendants} Soit $k\subset K$ un sous-corps. On rappelle que si $X\subset K$ est un sous-ensemble, on a noté  $k[X]\subset K$  la plus petite sous-$k$-algèbre de $K$ contenant $X$ et que   c'est aussi l'image de l'unique morphisme de $k$-algèbre $ k[T_x,\; x\in X ]\rightarrow K$, $T_x\mapsto x $.\\

\subsubsection{}Si $k\subset K_i\subset K$, $i\in I$ sont des  sous-corps contenant $k$, on vérifie immédiatement que $k\subset \cap_{i\in I}K_i\subset K $ est encore un sous-corps contenant $k$. il existe donc une unique sous-corps $k\subset k(X)\subset K$ contenant $k$, $X$ et minimal pour l'inclusion. On dit que $k\subset k(X)\subset K$ est la sous-$k$-extension de $k\subset K$ engendrée par $X$. Explicitement, $k(X)$ est l'intersection de toutes les sous-corps $k\subset K'\subset K$ tels que $X\subset K'$.  Si $K=k(X)$ on dit que $X$ est un système de générateurs de $K$ comme extension de corps de $k$ (ou que $K$ est engendré par $X$ comme extension de corps de $k$). Si on peut prendre $X$ fini, on dit que $K$ est une extension de corps de type fini de $k$. Comme $k[X]\subset k(X)$ et $k(X)$ est un corps, la propriété universelle du corps des fractions d'un anneau intègre montre qu'on a $k\subset k[X]\subset Frac(k[X])\subset k(X)\subset K$ et, comme $X\subset Frac(k[X])$, la minimalité de $k(X)$ assure que $Frac(k[X])=k(X)$.   \\

\subsubsection{}\textbf{Alternative algébrique/transcendant}\label{AlgTr} Soit $K/k$ une extension de corps, $x\in K$ et $ev_x:k[X]\twoheadrightarrow k[x]$ l'unique morphisme de $k$-algèbres tel que $ev_x(X)=x$\\

\ref{AlgTr}.1 \textbf{Lemme/définition.}\textit{On a l'alternative suivante.
\begin{enumerate}[leftmargin=* ,parsep=0cm,itemsep=0cm,topsep=0cm] 
\item Soit $ev_x:k[X]\tilde{\rightarrow} k[x]$ est un isomorphisme de $k$-algèbres et le morphisme $k[X]\stackrel{ev_x}{\tilde{\rightarrow}}k[x]\hookrightarrow k(x)$ se localise en un isomorphisme de corps $k(X)\tilde{\rightarrow} k(x)$. En particulier $k[x]$ (donc a fortiori $k(x)$) est de dimension infinie sur $k$.
\item Soit il existe un unique polynôme irréductible unitaire $P_x\in k[X]$ tel que $\ker(ev_x)=k[X]P_x$ et le morphisme de $k$-algèbres $ev_x:k[X]\twoheadrightarrow k[x]$ se factorise en un isomorphisme $k[X]/P_x\tilde{\rightarrow} k[x]$. En particulier $k[x]=k(x)$
 et $k(x)$ est de dimension finie sur $k$, égale au degré de $P_x$.\\
 \end{enumerate}}
 
  Dans le cas (1), on dit que $x\in K$ est \textit{transcendant sur $k$}\index{Transcendant (Elément)} et dans le cas (2) que $x\in K$ est \textit{algébrique}\index{Algébrique (Elément)} sur $k$ de degré $[k(x):k]=\hbox{\rm deg}(P_x)$ et que $P_x$ est le polynôme irréductible (unitaire) de $x$ sur $k$.\\

 
\begin{proof} Si $\ker(ev_x)=0$ on est dans le cas (1) et les assertions sont immédiates. Si  $\ker(ev_x)\not=0$, comme $k[X]$ est principal, il existe un unique polynôme unitaire $P_x$ tel que $\ker(ev_x)=k[X]P_x$ et  le morphisme de $k$-algèbres $ev_x:k[X]\twoheadrightarrow k[x]$ se factorise en un isomorphisme $k[X]/P_x\tilde{\rightarrow} k[x]$. Comme $k[x]\subset K$ est un sous-anneau d'un corps, il est intègre donc  $P_x$ est premier. Comme $k[X]$ est principal, tout idéal premier est maximal, donc $k[x]\subset K$ est un sous-corps de $K$. Comme $k[x]$ contient  $k$ et $x$, c'est nécessairement $k(x)$. Il reste à voir que $k[X]/P_x $ est de dimension le degré $d$ de $P_x$ sur $k$. Mais en utilisant la division euclidienne de   $k[X]$, on voit immédiatement que les classes de  $1,X,\dots, X^{d-1}$ forment une $k$-base de $k[X]/P_x$.  
\end{proof}

 On retiendra en particulier que si $x\in K$

$$\begin{tabular}[t]{ccccc}
$x$ est algébrique sur $k$&$\Leftrightarrow$&$[k(x):k]<+\infty$&$\Leftrightarrow$&$k[x]=k(x)$;\\
$x$ est transcendant sur $k$&$\Leftrightarrow$&$[k(x):k]=+\infty$&$\Leftrightarrow$&$k[x]\subsetneq k(x)$;\\
\end{tabular}$$

\ref{AlgTr}.2 \textbf{Exemple.} Considérons l'extension $\C/\Q$. Les nombres $i,\sqrt{2}$ \textit{etc} sont algébriques par définition. 
\begin{itemize}
\item Si $a\in \C$ est algébrique, $\exp(a)$ est transcendant (Lindemann, $\sim 1880$); en particulier, $e$ ($a=1$) et $\pi$ ($a=i\pi$) sont transcendants sur $\Q$.
\item  Si $a,b\in \C$ sont algébriques sur $\Q$ et $a\not=0,1$, $b\notin \Q$, $a^b=\exp(b\log a)$ est transcendant  sur $\Q$ ($7$ème problème de Hilbert, Gelfond, $\sim$ 1930); en particulier,  $e^\pi$ ($a=e^\pi$, $b=i$) est transcendant sur $\Q$. 
\item On ne sait pas si $e+\pi$ est transcendant. Il est par contre possible que d'ici la fin de l'année (2018), on sache que si $a,b\in \C$ sont algébriques sur $\Q$, $\exp(a)\log(b)$ soit transcendant  sur $\Q$.
\end{itemize}
En fait, sauf cas très particulier, on ne sait pas dire si un nombre complexe pris `au hasard' est transcendant sur  $\Q$ alors que moralement, `presque tous' les nombres complexes sont transcendants sur $\Q$ puisque 
  $\Q$ est dénombrable alors que   $\R$ (donc $\C) $ n'est pas dénombrable.  \\
 
 

\subsubsection{}Soit $K/k$ une extension de corps. On dit que $a_i\in K$, $i\in I$ sont algébriquement indépendants sur $k$ si pour tout sous-ensemble fini $J\subset I$, l'unique morphisme de $k$-algèbre $k[X_j,\; j\in J]\rightarrow K$, $X_j\rightarrow a_j$, $j\in J$ est injectif. Sinon, on dit que  $a_i\in K$, $i\in I$ sont algébriquement  liés sur $k$.\\

 Pour $|I|=1$, on retrouve la notion d'élément transcendant sur $k$.\\

 \textbf{Exemple.} Considérons encore l'extension $  \C/\Q$. On en sait encore moins sur l'indépendance algébrique que sur l'alternative algébrique/transcendant. Le seul résultat un peu général dont on dispose est que si $a_1,\dots, a_n\in \C$ sont algébriques  et linéairement indépendants sur $\Q$,   les $\exp(a_1),\dots, \exp(a_n)$ sont algébriquement indépendants sur $\Q$ (Lindeman-Weierstrass, $\sim$ 1885).\\

\subsubsection{}\label{TrPure}On dit qu'une extension de corps  $K/k$  est \textit{transcendante pure}\index{Transcendante pure (Extension de corps)}  s'il existe $a_i\in K$, $i\in I$   algébriquement indépendants sur $k$ tels que $K=k(a_i,\; i\in I)$. Le lemme suivant résulte immédiatement de la définition. \\

 
 
\textbf{Lemme.} \textit{Soit $K_3/K_2$ et $K_2/K_1$  des extensions de corps. Si $K_3/K_2$ et $K_2/K_1$   sont transcendantes pures alors $K_3/K_1$ est transcendante pure.}\\



\textbf{Remarque.} La réciproque n'est pas vraie en général mais ce n'est pas évident. En fait, toute sous-extension de $ k(X)/k$ est encore transcendante pure (Luroth $\sim$ 1875). Si $k$ est de caractéristique $0$, toute sous-extension de $  k(X_1,X_2/k$ est  encore transcendante pure  (Castelnuovo $\sim$ 1940) mais ce n'est plus  vrai si $k$ est de caractéristique $p>0$ (Zariski $\sim$ 1958) ou si $n\geq 3$ (Clemens-Griffith $\sim$ 1972). \\

\textbf{Exemple.} L'extension $k(T)[X]/\langle X^2-T\rangle/k$ est transcendante pure par contre, l'extension $k[X,Y]/\langle Y^2-X^3-X-1\rangle/k$ ne l'est pas.\\ 
 


\subsubsection{}On dit qu'une extension de corps  $K/k$ est  \textit{finie}\index{Finie (Extension de corps)} si $[K:k] <+\infty$ et que   $K/k$ est
  algébrique si tout $a\in K$ est algébrique sur $k$.\\
  
\paragraph{}\label{AlgTF}\textbf{Lemme.} \textit{Soit $K/k$ une extension de corps. Alors $K/k$ est finie si et seulement si $K/k$ est algébrique et de type fini.} 
 
 \begin{proof}L'implication $\Rightarrow$ est immédiate. Pour l'implication $\Leftarrow$, écrivons $K=k(a_1,\dots, a_n)$. Pour chaque $i=1,\dots, n$, $a_i\in K$ est algébrique sur $k$ donc \textit{a fortiori} sur $k(a_1,\dots, a_{i-1})$; en particulier, $[ k(a_1,\dots, a_{i-1})(a_i):k(a_1,\dots, a_{i-1}]<+\infty$. Et donc, par   \ref{Dim} on a 
 $$[K:k]=\prod_{1\leq i\leq n}[k(a_1,\dots, a_i):k(a_1,\dots, a_{i-1})]<+\infty. $$
  \end{proof}

\paragraph{}\label{AlgExt}\textbf{Lemme.} \textit{Soit $K_3/K_2$ et $K_2/K_1$  des extensions de corps.
\begin{enumerate}[leftmargin=* ,parsep=0cm,itemsep=0cm,topsep=0cm] 
\item $K_3/K_1$ est finie si et seulement si $K_2/K_1$ et $K_1/K_3$ sont finies, auquel cas on a $[K_3:K_1]=[K_3:K_2][K_2:K_1]$. 
\item $K_3/K_1$ est algébrique si et seulement si $K_2/K_1$ et $K_1/K_3$ sont algébriques. 
\end{enumerate}}

\begin{proof} Les implications $\Rightarrow$   sont immédiates. L'implication $\Leftarrow$  de (1) résulte de \ref{Dim}.  Pour l'implication $\Leftarrow$ de (2), fixons  $a\in K_3$ et montrons que $a$ est algébrique sur $K_1$. Comme $a$ est algébrique sur $K_2$, en écrivant son polynôme minimal sous la forme $P_a=T^d+\sum_{0\leq i\leq d-1}a_iTî\in K_2[T]$, on voit que  $a$ est aussi algébrique sur le sous-corps  $K_1(a_0,\dots, a_d)  $ de $K_2$ \textit{i.e} $[K_1(a_0,\dots, a_d,a):K_1(a_0,\dots, a_d)]<+\infty$. Mais comme $K_1(a_0,\dots, a_d)/K_1$ est algébrique de type fini, elle est finie par \ref{AlgTF} donc 
$$[K_1(a):K_1]\leq [K_1(a_0,\dots, a_d,a):K_1]=[K_1(a_0,\dots, a_d,a):K_1(a_0,\dots, a_d)][K_1(a_0,\dots, a_d):K_1]<+\infty.$$ \end{proof}


\textbf{Corollaire.} \textit{Soit $K/k$ une extension de corps. Alors l'ensemble $\overline{k^K}\subset K$ des $a\in K$ algébrique sur $k$ est un sous-corps de $K$ contenant $k$.}

\begin{proof} La partie non triviale de l'énoncé est l'affirmation que $\overline{k^K}\subset K$ est un sous-corps. Soit donc $a \in \overline{k^K}$, $0\not=b\in \overline{k^K}$. On a $a-b, ab^{-1}\in k(a,b)\subset K$ donc 
$$[k(a-b):k], [k(ab^{-1}):k]\leq [k(a,b):k]\leq [k(a)(b):k(a)][k(a):k]\leq [k(b):k][k(a):k]<+\infty$$
 \textit{i.e.} $a-b, ab^{-1}\in \overline{k}\cap K$ (de degré sur $k$ $\leq  [k(b):k][k(a):k]$). \end{proof}
 
 \textbf{Exercice.} Notons $\overline{\Q}:=\overline{\Q^\C}\subset  \C$. Montrer que $\overline{\Q}\subsetneq \C$ et que tout élément de $\C\setminus \overline{\Q}$ est transcendant sur $\overline{\Q}$ mais que  $\overline{\Q}\subset  \C$ n'est pas une extension transcendante pure.\\

 D'après  \ref{AlgTF}, une extension de corps est finie si et seulement si elle est algébrique et  de type  fini. Si $k\subset K$ est finie  et   si $a_1,\dots, a_n$$K=k(a_1,\dots, a_n)$ est un système de générateurs de $K$ comme extension de corps de $k$, \ref{AlgTr} montre que  $K=k(a_1,\dots, a_n)=k(a_1)\cdots(a_n)=k[a_1]\cdots [a_n]=k[a_1,\dots, a_n]$ donc $K$ est aussi une $k$-algèbre de type finie. Cette dernière propriété suffit en fait  à caractériser les extensions finies; c'est le Théorème des zéros de Hilbert ou Nullstellensatz. \\
 
\subsection{Nullstellensatz} 

\subsubsection{}\label{Hilbert1}\textbf{Proposition.} (Nullstellensatz) \textit{Soit $  K/k$ une extension de corps. Si $K$ est une $k$-algèbre de type finie alors $K$ est une extension finie de $k$.}

\begin{proof} Commen\c{c}ons par observer que le corps des fractions $k(X)$ de l'anneau des polynômes à une indéterminée sur $k$ n'est pas une $k$-algèbre de type fini. Sinon, on aurait   $k(X)=k[a_1,\dots, a_n]$ o\`u $a_i=P_i/Q_i$ avec $0\not= P_i,Q_i\in k[X]$, $i=1,\dots, n$. Posons  $Q:=Q_1,\dots Q_n\in k[X]$. Par construction $k(X)\subset k[X]_{Q}$. Mais si $P\in k[X]$ est premier avec $Q$, $1/P\notin k[X]_Q$: contradiction.\\
\indent On procède par récurrence sur le nombre $n$ de générateurs de $K$ comme $k$-algèbre. Plus précisément, pour tout $n\geq 0$ considérons l'assertion suivante
$$\begin{tabular}[t]{ll}
 H(n)&Pour tout corps $k$, toute $k$-algèbre  $K=k[a_1,\dots, a_n]$ qui est un corps est une extension finie de $k$
 \end{tabular}$$ 
\begin{itemize}[leftmargin=* ,parsep=0cm,itemsep=0cm,topsep=0cm] 
\item H(0) est trivialement vraie. 
\item  Supposons maintenant $n\geq 2$ et soit   $K=k[a_1,\dots, a_n]$ un corps. Comme $K$ est un corps, on a  $K=k[a_1,\dots, a_n]=k(a_1)[a_2,\dots, a_n]$ et H(n-1) assure que  $[K:k(a_1)]<+\infty$. Il suffit donc de montrer que $[k(a_1):k]<+\infty$ \textit{i.e.} \ref{AlgTr} que $a_1\in K$ est algébrique sur $k$. Choisissons une $k(a_1)$-base $b_1,\dots, b_r$ de $K$. Ecrivons $a_i=\sum_{1\leq j\leq r}x_{i,j}b_j$ avec $x_{i,j}\in k(a_1)$, $1\leq i\leq n$, $1\leq j \leq r$, $b_ib_j=\sum_{1\leq k\leq r}x_{i,j,k}b_k$ avec $x_{i,j,k}\in k(a_1)$, $1\leq i,j,k\leq r$ et introduisons la sous-$k$-algèbre $$A:=k[x_{i,j}, 1\leq i\leq n ,\; 1\leq j \leq r, x_{i,j,k},\;1\leq i,j,k\leq r]\subset K.$$ Comme $K=k[a_1,\dots, a_n]$ on a  $K= \oplus_{1\leq i\leq r}Ab_i$ donc  $$K= \oplus_{1\leq i\leq r}k(a_1)b_i\subset \oplus_{1\leq i\leq r}Ab_i\subset  \oplus_{1\leq i\leq r}k(a_1)b_i,$$
ce qui impose $k(a_1)=A$ et contredit notre observation préliminaire.
\end{itemize}
\end{proof} 
\subsubsection{}\label{Hilbert2}\textbf{Corollaire.} \textit{Soit $A$ une $k$-algèbre de type fini. Pour tout $\frak{m}\in spm(A)$, $A/\frak{m}$ est une extension finie de $k$.}\\

\textbf{Exemple.} En particulier, si $k=\C$,  \ref{AlgTr}.3 (3) montre que pour tout $\frak{m}\in spm(A)$, $A/\frak{m}=\C$. Rappelons qu'une sous-variété algébrique affine de $\C^n$ est par définition un sous-ensemble $V$ de $\C^n$ de la forme 
$$V=V(I)=\lbrace \underline{x}\in \C^n\; |\; P(\underline{x})=0, \; P\in I\rbrace.$$ Notons $\C[V]:=C[X_1,\dots, X_n]/I$. Les propriétés universelles de la $\C$-algèbre des polynômes à $n$ indéterminées et du quotient donnent une bijection canonique 
$$V\tilde{\rightarrow}\hbox{\rm Hom}_{Alg/\C}(\C[V],\C)$$
et le Corollaire \ref{Hilbert2}, une bijection canonique  $$\hbox{\rm Hom}_{Alg/\C}(\C[V],\C)\tilde{\rightarrow} spm(\C[V]).$$
La composée $V\tilde{\rightarrow} spm(\C[V])$ est donnée explicitement par $\underline{x}\rightarrow \ker(ev_{\underline{x}}:\C[X_1,\dots,X_n]\rightarrow \C)/I$. En particulier, comme $\C[V]$ est noetherien,  $spm(\C[V])\not=\emptyset$. Cela montre qu'une sous-variété algébrique affine de $\C^n$ a  toujours un point. En particulier, les idéaux maximaux de $\C[X_1,\dots, X_n]$ sont exactement les $\sum_{1\leq i\leq n}\C[X_1,\dots, X_n](X_i-x_i)$, $\underline{x}\in \C^n$.

 \subsubsection{}\label{Hilbert3}\textbf{Corollaire.} \textit{Soit $A$ une $k$-algèbre de type fini. Alors pour tout idéal $I\subset A$, $\sqrt{I}=\displaystyle{\bigcap_{\frak{m}\in spm(A),\; I\subset \frak{m}}\frak{m}}$.}
 
 \begin{proof}On peut réécrire le lemme sous la forme $\sqrt{\lbrace 0\rbrace}=\mathcal{J}_{A/I}$. Il suffit donc de montrer que si $A$ est une $k$-algèbre de type fini alors $\mathcal{J}_{A }\subset \sqrt{\lbrace 0\rbrace}$ ou encore $A\setminus \sqrt{\lbrace 0\rbrace}\subset A\setminus \mathcal{J}_A$.  Soit donc  $a\in\setminus \sqrt{\lbrace 0\rbrace}$. Il faut montrer qu'il existe un idéal maximal de $A $ qui ne contient pas $a$. Un tel idéal induit un idéal premier $\frak{m}A_a$. Cela amène naturellement à considérer  le morphisme de localisation  $c_a:A\rightarrow A_a$. \\
 
\begin{itemize}[leftmargin=* ,parsep=0cm,itemsep=0cm,topsep=0cm] 
\item\textbf{Lemme 1.} \textit{$A_a$ est une $k$-algèbre de type fini.}
 
 \begin{proof} Par la propriété universelle de $ A\rightarrow A[T]$, on a un unique morphisme de $A$-algèbres $ev_{1/a}:A[T]\rightarrow A_a$ tel que $ev_{1/a}(T)=1/a$. Par construction,  $(Ta-1)A[T]\subset \ker(ev_{1/a})$, d'o\`u une factorisation $\phi :=\overline{ev}_{1/a}:A[T]/(Ta-1)\rightarrow A_a$. Inversement, par la propriété universelle de $ A\rightarrow A_a$, le morphisme canonique $ A\stackrel{\iota_A}{\rightarrow}A[T]\twoheadrightarrow A[T]/(Ta-1)$ se factorise en un morphisme $\psi:A_a\rightarrow A[T]/(Ta-1)$ et on vérifie sur les constructions que $\phi: A[T]/(Ta-1)\rightarrow A_a$, $\psi:A_a\rightarrow A[T]/(Ta-1)$ sont inverses l'un de l'autre.\end{proof}
 
 \item\textbf{Lemme 2.}  \textit{Toute $k$-agèbre intègre de $k$-dimension finie est un corps.}
 
 \begin{proof} Soit $A$ une $k$-agèbre intègre de $k$-dimension finie et $0\not=a\in A$. La multiplication par $a$ induit un morphisme $\mu_a:A\rightarrow A$ de $k$-espaces vectoriels qui est est injectif puisque $A$ est intègre donc bijectif puisque $A$ est de $k$-dimension finie. En particulier, il existe $b\in A$ tel que $ab=\mu_a(b)=1$. \end{proof} 
 \item Fin de la preuve. D'après le Lemme 1, $A_a$ est encore une $k$-algèbre de type fini. Soit $\frak{m}\in spm(A_a)$. Alors $\frak{p}:=c_a^{-1}(\frak{m})\in spec(A)$ et $a\notin \frak{p}$. De plus, on a des morphismes d'anneaux injectifs $k\hookrightarrow A/\frak{p}\hookrightarrow A_a/\frak{m}$. Par \ref{Hilbert2}, $A_a/\frak{m}$ est une extension finie de $k$. Donc par le Lemme 2, $A/\frak{p}$ est   un corps \textit{i.e.} $\frak{p}\in spm(A)$.
 \end{itemize}
 \end{proof}
 
 \textbf{Exemple.} Reprenons les notations de l'Exemple \ref{Hilbert2}. Considérons le morphisme de $\C$-algèbres canonique $-|_V:\C[X_1,\dots,X_n]\rightarrow \C^V$ qui envoie $P\in \C[X_1,\dots, X_n]$ sur l'application $ev_-(P)|_V:V\rightarrow \C$, $\underline{x}\rightarrow ev_{\underline{x}}(P)=P(\underline{x})$. On a clairement $I\subset I(V):=\ker(-|_V)$ et le Corollaire \ref{Hilbert3} montre qu'en fait $I= \ker(-|_V)$ (rappelons qu'on a supposé $I=\sqrt{I}$). Le morphisme de $\C$-algèbre  $-|_V:\C[X_1,\dots,X_n]\rightarrow \C^V$ se factorise donc en un morphisme injectif $\C[V]\hookrightarrow \C^V$; on dit que $\C[V]$ est la $\C$-algèbre des applications polynômiales sur $V$.  On en déduit aussi que les applications $I\rightarrow V(I)  $ et $V\rightarrow I(V)$ sont des bijections  inverses l'une de l'autres entre l'ensemble des idéaux radiciels de $\C[X_1,\dots, X_n]$ et les sous-variétés algébriques affines de $\C^n$.
 
 \subsection{Bases de transcendance, degré de transcendance}Soit $K/k$ une extension de corps. Une \textit{base de transcendance}\index{Base de transcendance (Extension de corps)} de $K/k$ est une famille $a_i\in K$, $i\in I$ d'éléments algébriquement indépendants sur $k$ tels que $K/k(a_i,i\in I) $ est algébrique.
 
 \subsubsection{}\label{BaseTranscendance} [Utilise le Lemme de Zorn] \textbf{Proposition.} \textit{Les bases de transcendance existent et ont même cardinal.}\\
 
  On rappelle que deux ensembles $A,B$ ont même cardinal (notation: $|A|=|B|$) s'il existe une application bijective $A\tilde{\rightarrow}B$. On note $|A|\leq |B|$ s'il existe une application injective $A\hookrightarrow B$ et $|A|<|B|$ si $|A|\leq |B|$ et $|A|\not=|B|$. En utilisant l'axiome du choix, on peut montrer que si $A$ et $B$ sont deux ensembles alors on a toujours $|A|\leq |B|$ ou $|B|\leq |A|$. En fait on a même soit $|A|<|B|$ ou $|A|=|B|$ ou $|B|<|A|$ (trichotomie). Cela résulte de ce qu'on vient de dire combiné au lemme de Schr\"{o}der-Bernstein.\\
 
 \textbf{Exercice.} (Schr\"{o}der-Bernstein) \textit{Montrer que si $|A|\leq |B|$ et $|B|\leq |A|$ alors $|A|= |B|$.}
 
 
 
 \begin{proof}Montrons d'abord que si $\mathcal{L}\subset K$ est algébriquement indépendant sur $k$ et $\mathcal{G}\subset K$ est un système de générateurs de $k\subset K$ tel que $\mathcal{L}\subset \mathcal{G}\subset K$, il existe une base de transcendance $\mathcal{L}\subset \mathcal{B}\subset \mathcal{G}$. En effet, l'ensemble $\mathcal{E}$ des sous-ensembles $\mathcal{L}\subset \mathcal{U}\subset \mathcal{G}$, algébriquement indépendants sur $k$ est non vide (il contient $\mathcal{L}$) et ordonné inductif pour $\subset $. Par le Lemme de Zorn, il admet donc un élément $\mathcal{B}$ maximal pour $\subset $. La maximalité de $\mathcal{B}$ impose que $K/k(\mathcal{B})$ est algébrique. En effet, s'il existait $x\in K$ transcendant sur $k(\mathcal{B})$, en écrivant $x=y/z$ avec $y,z\in k[\mathcal{G}]$, on voit qu'il existerait forcément un élément $g\in \mathcal{G}$ transcendant sur  $k(\mathcal{B})$ donc tel que $\mathcal{B}\cup\lbrace g\rbrace \in \mathcal{E}$: contradiction.\\
  Soit maintenant $\mathcal{B},\mathcal{B}'$ deux bases de transcendance. Supposons $|\mathcal{B}'|\leq |\mathcal{B}|$. Pour chaque $b'\in \mathcal{B}'$ il existe une sous-ensemble fini $\mathcal{B}_{b'}\subset \mathcal{B}$ tel que $b'$ est algébrique sur $k(\mathcal{B}_{b'})$. Notons $$\mathcal{B}'':=\bigcup_{b'\in \mathcal{B}'}\mathcal{B}_{b'}\subset \mathcal{B}.$$
 On a en fait $\mathcal{B}''=\mathcal{B}$. En effet, s'il existait $b\in \mathcal{B}\setminus \mathcal{B}''$, comme $b$ est algébrique sur $k(\mathcal{B}')$ et $k(\mathcal{B}')$ est algébrique sur $k(\mathcal{B}'')$, $b$ est algébrique sur $k(\mathcal{B}'')$ \ref{AlgExt} (2):  contradiction. Cela montre déjà que $\mathcal{B}'$ est fini si et seulement si $\mathcal{B}$ est fini. Supposons d'abord $\mathcal{B},\mathcal{B}'$ infinis. On a alors $|\mathcal{B}|\leq |\mathcal{B}'|$ (quitte à remplacer $\mathcal{B}'$  et les $\mathcal{B}_{b'}$ par des sous-ensembles, on peut supposer que les $\mathcal{B}_{b'}$ sont tous disjoints et l'assertion est alors immédiate) et la conclusion résulte de Schr\"{o}der-Bernstein. Supposons donc $\mathcal{B},\mathcal{B}'$ finis et procédons par réccurence sur $m:=|\mathcal{B}'|\leq n:=|\mathcal{B}|$. Si $m=0$, $k\subset K$ est algébrique donc $n=0$. Si $m>0$, écrivons $\mathcal{B}'=\lbrace b'_1,\dots, b'_m\rbrace$, $\mathcal{B}=\lbrace b_1,\dots, b_n\rbrace$. Comme $b_1'$ est algébrique sur $k(\mathcal{B})$ et transcendant sur $k$, il existe $P\in k[X,Y_1,\dots, Y_n]$ irréductible et vérifiant $P(b_1',b_1,\dots, b_n)=0$ avec   $X$ et au moins l'un des $Y_i$ - disons $Y_1$ - qui apparaissent dans l'expression de $P$; en particulier, $b_1$ est algébrique sur $k(b_1',b_2,\dots, b_n)$. Considérons $\mathcal{B}'':=\lbrace b_1',b_2,\dots, b_n\rbrace$ (on a échangé $b_1$ et $b_1'$) et montrons que c'est encore une base de transcendance de $k\subset K$. En effet,  d'une part $K/k(\mathcal{B}'',b_1)$ et $k(\mathcal{B}'',b_1)/k(\mathcal{B}'')$ sont algébrique donc $K/k(\mathcal{B}'')$ est algébrique \ref{AlgExt} (2). D'autre par, s'il existait $P\in k[X,Y_2,\dots, Y_n]$ irréductible tel que $P(b_1',b_2,\dots, b_n)=0$, comme $ b_2,\dots, b_n$ sont algébriquement indépendants sur $k$, $X$  apparait dans l'expression de $P$ donc $b_1' $ est algébrique sur $k(b_2,\dots, b_n)$ et donc $b_1$ aussi: contradiction. On a donc deux bases de transcendance $\mathcal{B}'$ et $\mathcal{B}''$ de $k\subset K$ qui contiennent $b_1'$; on vérifie immédiatement sur la définition que $\mathcal{B}'\setminus \lbrace b_1'\rbrace$, $\mathcal{B}''\setminus \lbrace b_1'\rbrace$ sont alors des bases de transcendance de $k(b_1')\subset K$. Par hypothèse de récurrence, $m-1=n-1$.  \end{proof}
 
 
  On dit que le cardinal d'une base de transcendance est le \textit{degré de transcendance}\index{Degré de transcendance (Extension de corps)} de $k\subset K$; on le notera $trdeg_k(K)$.\\ 
 
  \textbf{Exemples.} 
 \begin{enumerate}[leftmargin=* ,parsep=0cm,itemsep=0cm,topsep=0cm] 
 \item La première partie de la preuve montre que si $k\subset K$ est de type fini alors $trdeg_k(K)$ est fini et $\leq$ au nombre minimal de générateurs de $k\subset K$. Dans ce cas,  si on note $n:=trdeg_k(K)$ on a  $X_1,\dots, X_n\in K$ algébriquement indépendants sur $k$ tels que $K/k(X_1,\dots, X_n)$ est algébrique. Mais comme $K/k$ est de type fini,  $K/k(X_1,\dots, X_n)$ l'est \textit{a fortiori} donc, en fait, $K/k(X_1,\dots, X_n)$ est même finie. 
\item  Comme $\Q(X_1,\dots, X_n)$ est dénombrable et $\C$ ne l'est pas, on voit par contre que $\C/\Q$ est de degré de transcendance infini.
\item On dit qu'une variété algébrique affine $V=V(I)\subset \C^n$ est irréductible si $\C[V]:=\C[X_1,\dots, X_n]/\sqrt{I}$ est intègre. Dans ce cas, on peut introduire le corps des fractions $\C(V)$ de $\C[V]$ et définir la dimension  de la variété algébrique affine $V$ comme étant le degré de transcendance de $\C(V)$ sur $\C$. Par exemple, l'extension de corps $\C[X,Y]/\langle Y^2-X^3-X-1\rangle/k$   est de degré de transcendance $1$ - donc $ V(Y^2-X^3-X-1)\subset \C^2$ est une courbe - mais ce n'est pas une extension transcendante pure, ce qui se traduit par le fait qu'on ne peut pas donner une paramétrisation rationnelle de $ V(Y^2-X^3-X-1)$.
 \end{enumerate}

 \subsubsection{}\textbf{Lemme.} \textit{Soit $K_1\subset K_2\subset K_3$ des extensions de corps. On a} 
 $$trdeg_{K_1}(K_3)=trdeg_{K_1}(K_2)+trdeg_{K_2}(K_3).$$
 \begin{proof} Si $\mathcal{E}_1$ est une base de transcendance de $K_1\subset K_2$ et $\mathcal{E}_2$ est une base de transcendance de $K_2\subset K_3$, il faut vérifier que $\mathcal{E}:=\mathcal{E}_1\cup \mathcal{E}_2$ est une base de transcendance de $K_1\subset K_3$. Les éléments de $\mathcal{E}$ sont clairement algébriquement indépendants sur $K_1$. Soit $x\in K_3$. Comme $\mathcal{E}_2$ est une base de transcendance de $K_2\subset K_3$, il existe un sous-ensemble fini $\mathcal{E}_{2,x}\subset \mathcal{E}_2$ et $P_x=T^d+\sum_{0\leq i\leq d-1}a_iT^i\in K_2(\mathcal{E}_{2,x})[T]$ tel que $ev_x(P)=0$. En écrivant  $a_i=\frac{b_i}{c_i}$ avec $b_i ,c_i\in K_2[\mathcal{E}_{2,x}]$, on voit qu'il existe un sous-ensemble  fini $A_i\subset K_2$ tel que $a_i\in K_1(A_i)(\mathcal{E}_{2,x})$, $i=0,\dots, d-1$. On a donc $$[K_1(A_0,\dots, A_{d-1}, \mathcal{E}_{2,x})(x):K_1(A_0,\dots, A_{d-1}, \mathcal{E}_{2,x})]<+\infty.$$ 
 Comme $\mathcal{E}_1$ est une base de transcendance de $K_1\subset K_2$, on a $[K_1(\mathcal{E}_1)(A_0,\dots, A_{d-1}):K_1(\mathcal{E}_1)]<+\infty$. On en déduit 
 $$\begin{tabular}[t]{l}
 $[K_1(\mathcal{E}_1, \mathcal{E}_{2,x})(x):K_1(\mathcal{E}_1, \mathcal{E}_{2,x})]$\\
 $\leq  [K_1(\mathcal{E}_1,A_0,\dots, A_{d-1},\mathcal{E}_{2,x})(x):K_1(\mathcal{E}_1, \mathcal{E}_{2,x})]$\\
 $=[K_1(\mathcal{E}_1,A_0,\dots, A_{d-1},\mathcal{E}_{2,x})(x):K_1(\mathcal{E}_1,A_0,\dots, A_{d-1}, \mathcal{E}_{2,x})][K_1(\mathcal{E}_1,A_0,\dots, A_{d-1}, \mathcal{E}_{2,x}):K_1(\mathcal{E}_1, \mathcal{E}_{2,x})]$\\
$=[K_1(\mathcal{E}_1,A_0,\dots, A_{d-1},\mathcal{E}_{2,x})(x):K_1(\mathcal{E}_1,A_0,\dots, A_{d-1}, \mathcal{E}_{2,x})][K_1( \mathcal{E}_{2,x},\mathcal{E}_1)(A_0,\dots, A_{d-1}):K_1( \mathcal{E}_{2,x},\mathcal{E}_1)] <+\infty.$
 \end{tabular}$$
   \end{proof}
 
  \subsubsection{}\textbf{Lemme.} \textit{Soit $K/k$ une extension de corps. Si  $K$ est de type fini sur $k$ alors $[\overline{k}^K:k]<+\infty$.} 
\begin{proof} Si $\mathcal{B}=\lbrace b_1,\dots, b_n\rbrace $ est  une base de transcendance de $K/k$,   $K/k(\mathcal{B})$ est algébrique de type fini donc finie. Pour toute sous-extension $k\subset k'\subset K$ finie sur $k$ on a 
 $[k':k]= [k'(\mathcal{B}):k(\mathcal{B})]\leq [K:k(\mathcal{B})]<+\infty.$ 
\end{proof}


 On peut donc toujours décomposer une extension de corps $K/k$ en trois parties 
$$k\subset \overline{k}^K\subset \overline{k}^K(\mathcal{B})\subset K$$
avec $ \overline{k}^K/k$ algébrique, $ \overline{k}^K(\mathcal{B})/ \overline{k}^K$ transcendante pure et $K/ \overline{k}^K(\mathcal{B})$ algébrique. Si, de plus, $K/k$ est de type fini alors $ \overline{k}^K/k$   et $K/ \overline{k}^K(\mathcal{B})$ sont finies et   $ \overline{k}^K(\mathcal{B})/ \overline{k}^K$ est de degré de transcendance finie. Les extensions de corps de type fini apparaissent naturellement en (et leur étude est motivée par) géométrie algébrique comme dans l'Exemple (3) de \ref{BaseTranscendance}. Il s'agit d'un sujet très vaste et encore largement ouvert. \\

 Dans la suite du cours, nous allons nous intéresser au cas de degré de transcendance $0$, qui est déjà très riche  dès lors que le corps de base $k$ est suffisamment 'compliqué' (on verra   ce que compliqué veut dire plus loin...). Il s'agit donc de comprendre les variétés algébriques de dimension $0$ sur un corps $k$ \textit{i.e.} les solutions d'une equation polynômiale à une indéterminée et à coefficients dans $k$. Ce problème concret remonte au 19ème siècle. A cette époque, on savait déjà depuis longtemps résoudre par radicaux (\textit{i.e.} en n'utilisant que les opérations $+,-,\times, -/-,^n\sqrt{-}$) des équations de degré $\leq 4$ (les formules de Cardan en degré $3$ et Ferrari en degré 4 remontent au milieu du 16ème siècle) mais pas au-delà.  Ruffini et Abel, au tout début du 19ème siècle, ont annoncé  qu'il `n'existait pas de formule universelle pour résoudre une équation de degré donné $\geq 5$'. Mais c'est Galois, dans son \textit{Mémoire sur les conditions de résolubilité des équations par radicaux} (écrit en 1831; Galois avait $20$ ans et devait mourir tué en duel l'année d'après), qui a eu l'intuition prodigieuse pour l'époque de relier le problème de la résolubilité d'une équation polynomiale aux symétries de ses zéros. Plus précisément, si $P\in \Q[T]$ est de degré $n$ et se factorise en $n$ facteurs de degré $1$ distincts  (ce que l'on appelera un polynôme séparable) dans $\C$
$$P=a_n\prod_{1\leq i\leq n}(T-\alpha_i)\in \C[T]$$
 on peut considérer le groupe   $\mathcal{S}_n$ des permutations de $\lbrace \alpha_1,\dots, \alpha_n\rbrace$.  On peut alors attacher à $P$ un sous-groupe $G_P\subset \mathcal{S}_n$ qui reflète les symétries de $P$ (et qu'on appelle maintenant le groupe de Galois de $P$). Plus ce groupe est compliqué, plus les racines de $P$ seront difficiles à calculer. Par exemple, l'introduction du groupe $G_P$ permet de résoudre de fa\c{c}on limpide le problème de la résolution par radicaux: `l'équation $P(x)=0$ est résoluble par radicaux dans $\C$ si et seulement si le groupe $G_P$ est résoluble (\textit{i.e.} si $D^nG_P=1$ pour $n\gg 0$ o\`u les $D^nG_P$  sont  les sous-groupes de $G_P $ définis inductivement par $D^0G_P=G_P$, $D^1G_P=[G_P,G_P]$, $D^{n+1} G_P= [D^nG_P,D^nG_P]$)'. En particulier, puisque les groupes $\mathcal{S}_n$ ne sont pas résolubles pour $n\geq 5$ (et qu'il existe des polynômes $P\in \Q[T]$ de degré $n$ tels que $G_P=\mathcal{S}_n$ - c'est en fait le cas `générique'), il n'y a en effet aucune chance de trouver un formule universelle de résolution par radicaux des équations polynômiales de degrés $\geq 5$.\\  
 
 Mais encore faut-il pouvoir calculer ces groupes $G_P$. Sur des exemples simples, on peut deviner intuitivement ce à quoi ressemble ces groupes. Par exemple, les solutions de  $P=(T^2+1)(T^2-2)=0$ sont $\pm i$, $\pm \sqrt{2}$ et si on peut échanger $i$ avec $-i$ (ce sont les racines de $T^2+1$) et $\sqrt{2}$ avec $-\sqrt{2}$ (ce sont les racines de $T^2-2$), on ne peut pas échanger $i$ et $\sqrt{2}$. Dans ce cas, $G_P$ est le sous-groupe de $\mathcal{S}_4$ engendré par les transpositions $(1,2)$ et $(3,4)$ \textit{i.e.}  $\Z/2\times \Z/2$. Cependant, très vite  cette approche empirique ne suffit plus. La résolution conceptuelle du problème est donnée par ce qu'on appelle maintenant la correspondance de Galois. Dans le cas particulier considéré ici - $P\in \Q[T]$ est de degré $n$ et a $n$ zéros distincts $\alpha_1,\dots,\alpha_n$ dans  $\C$ - $G_P$ est isomorphe au groupe $\SAut(K_P/\Q)$  des $\Q$-automorphismes de l'extension de corps $K_P:=\Q(\alpha_1,\dots,\alpha_n)/\Q$ et les sous-groupes    $H\subset \SAut(K_P/\Q)$ correspondent bijectivement aux  sous-extensions $\Q\subset L\subset K_P $ par $H\rightarrow K_P^H:=\lbrace x\in K_P\;|\; \sigma (x)=x;\; \sigma\in H\rbrace$ et $L\rightarrow \SAut(K_P/L)$. Par exemple, 
\begin{itemize}
\item Pour $P=X^5+10X^3-10X^2+35X-18$, $G_P=\mathcal{A}_5$;
\item Pour $P=X^5+10X^3-15$, $G_P=\mathcal{S}_5$.
\end{itemize}


 


 




 \section{Extensions algébriques}\textit{}\\
 
  On rappelle que si $K/k$ est une extension de corps et $x\in K$, on note $ev_x:k[X]\rightarrow K$ l'unique morphisme de $k$-algèbres qui envoie  $X$ sur $x$ et qu'on écrit en général $ev_x(P)=:P(x)$, $P\in k[X]$.\\
 
 \subsection{Corps algébriquement clos, clôture algébrique}Pour tout $P\in k[X]$ on dit que  $x\in K$ est une \textit{racine}\index{Racine (Polynôme)} de $P$ si $P(x)=0$ (\textit{i.e.} $P\in \ker(ev_x)=k[X](X-1)$).
 
 
  \subsubsection{}\label{AlgClos}\textbf{Lemme/définition.} \textit{Soit $ k$ un  corps. Les propriétés suivantes sont équivalentes.
 \begin{enumerate}[leftmargin=* ,parsep=0cm,itemsep=0cm,topsep=0cm] 
\item Tout $P\in k[X]\setminus k$ admet une racine sur $k$;
\item Les éléments irréductibles de $k[X]$ sont de degré $1$;
\item La seule extension algébrique $k\subset K$ est $k=K$. \\
\end{enumerate}}

 Un corps $k$ qui vérifie les propriétés équivalentes du Lemme \ref{AlgClos} est dit \textit{algébriquement clos}\index{Algébriquement clos (Corps)}. \\

\textbf{Exemple.} 
\begin{enumerate}
\item $\C$ est algébriquement clos. \\


 Utilisons la caractérisation (1). Soit $P=X^n+a_{n-1}X^{n-1}+\dots+a_0\in \C[X]$.  Dire que $P$ a une racine dans $\C$ revient à dire que la fonction  continue $x\rightarrow |P(x)|$ atteint son minimum sur $\C$ et que celui-ci vaut $0$. Observons déjà qu'elle atteint bien son minimum. En effet,   puisque  $\displaystyle{\lim_{|x|\rightarrow +\infty}|P(x)|}=+\infty$, il existe $R>0$ tel que $|x|>r$ implique $|P(x)|>|a_0|=|P(0)|$ donc  la fonction continue $x\rightarrow |P(x)|$ atteint son minimum   sur le compact $B(0,R):=\lbrace x\in \C\; |\; |x|\leq R\rbrace $ (et ce minimum est $\leq |a_0|$). Quitte à faire un changement de variable de la forme $X\rightarrow X-x$, on peut supposer que $x\rightarrow |P(x)|$ atteint son minimum en $0$ donc que ce minimum vaut $|a_0|$. Si $a_0=0$, on a gagné. Sinon, quitte à remplacer $P$ par $a_0^{-1}P$, on peut supposer que $a_0=1$. Soit $v$ le plus petit entier $\geq 1$ tel que $a_v\not= 0$. Ecrivons $a_v=-|a_v|\exp(-iv\theta)$. On a alors $P(r\exp(i\theta))=1-|a_v|r^v+ O_0(r^{v+1}) $ donc, lorsque $x\rightarrow 0$, on a $|P(0)|\leq 1-|a_v|r^v+ O_0(r^{v+1})<1$: contradiction. \\
\item Soit $K/k$ une extension de corps avec $K$ algébriquement clos. Alors $  \overline{k}^K\subset  K$ est un corps algébriquement clos. En effet, on sait déjà que $  \overline{k}^K$ un corps. De plus, comme $K$ est algébriquement clos,  tout $P\in  \overline{k}[T]\setminus k (\subset K[T]\setminus K)$ admet une racine $x\in K$. Mais par construction $x$ est algébrique sur $\overline{k}^K$ et comme $\overline{k}^K$ est par définition algébrique sur $k$, $x$ est algébrique sur $k$ \textit{i.e.} $x\in \overline{k}^K$. 
\end{enumerate}
\subsubsection{}\label{CloAlgDef} Soit $k$ un corps, une \textit{clôture algébrique}\index{Clôture algébrique (Extension de corps)} de $k$ est une extension algébrique $  \overline{k}/k$ avec $ \overline{k}$ algébriquement clos. Il résulte immédiatement de la définition que si $  \overline{k}/k$ est une clôture algébrique alors pour toute sous extension $K/k$ de $\overline{k}/k$, $\overline{k}/K$ est  une côture algébrique de $K$.\\
 
\textbf{Exemple.}
\begin{enumerate}
\item  $\R\subset \C$ est une clôture algébrique de $\R$. \\
\item Soit $K/k$ une extension de corps avec $K$ algébriquement clos. Alors $  \overline{k}^K/k $ est une clôture algébrique de $k$. Par exemple,   $ \overline{\Q}^\C/\Q $ est une clôture algébrique de $\Q$.
\end{enumerate}

 L'objectif des deux paragraphes suivants est de démontrer la\\

\subsubsection{}\label{CloAlgExUni}\textbf{Proposition.} \textit{Tout corps $k$ possède une clôture algébrique $\overline{k}/k$, qui est unique à isomorphisme (non unique!) près.  }



\subsubsection{}\textbf{Unicité à isomorphisme près de la clôture  algébrique.}\\

\paragraph{}\label{MonoPlonge}Si $K/k$ est une extension de corps et $P\in k[X]$, on notera $$Z_K(P):=\lbrace x\in K\; |\;  P(x)=0\rbrace\subset K$$
l'ensemble des racines de $P$ dans $K$. Supposons de plus que $P$ est irréductible sur $k$. Pour tout $x\in Z_K(P)$, $ev_x:k[T]\rightarrow K$ induit un $k$-plongement $\overline{ev}_x :k[T]/P\rightarrow K$. Inversement, pour tout $k$-pongement $\phi:k[T]/P\rightarrow K$, $x:=\phi(\overline{T})\in Z_K(P)$ et $\phi=\overline{ev}_x$. On a donc montré  \\


 \textbf{Lemme.} \textit{Soit $k$ un corps et $P\in k[X]$ un polynôme irréductible. Pour toute extension $K/k$ l'application $\phi\rightarrow \phi(\overline{T})$ induit une bijection 
$$\hbox{\rm Hom}_{Alg/k}(k[T]/P,K)\tilde{\rightarrow} Z_K(P)$$
d'inverse $x\rightarrow \overline{ev}_x$.}\\
 
  Autrement dit, et c'est là l'idée clef de la théorie de Galois, les racines distinctes d'un polynôme irréductible $P\in k[T]$ dans une extension $K/k$  correspondent bijectivement aux $k$-plongement $k[T]/P\rightarrow K$.\\
 

 

\paragraph{}\label{CloAlgPlonge}[Utilise le Lemme de Zorn] \textbf{Lemme.} \textit{Soit $k$ un corps et $  \overline{k}/k$ une clôture algébrique de $k$. Alors pour  toute extension algébrique $  K/k $ il existe un $k$-plongement $K \rightarrow \overline{k} $.}
\begin{proof}  Considérons l'ensemble $\mathcal{E}$ des couples $(K',\phi)$, o\`u $k\subset K'\subset K$ est une sous-extension et $\phi:K'\rightarrow \overline{k}$ un $k$-plongement. On munit $\mathcal{E}$ de la relation d'ordre partiel $(K'_1,\phi_1)\leq (K'_2,\phi_2))$ si  $ K_1'\subset K_2'$ et $\phi_2|_{K_1'}=\phi_1$. L'ensemble $\mathcal{E}$ est non vide   puisqu'il contient $(k,k\subset \overline{k})$ et $(\mathcal{E},\leq)$ est ordonné inductif. Par le Lemme de Zorn, il contient donc un élément maximal $(K',\phi)$. Mais si $K'\subsetneq K$, en appliquant \ref{MonoPlonge} à l'extension $\phi:K'\rightarrow \overline{k}$ (qui est une clôture algébrique de $K'$) et à $x\in K\setminus K'$, on construit un $K'$-plongement $\iota:K'(x)\rightarrow \overline{k}$. Par construction $(K',\phi)\leq (K'(x),\iota\circ \phi)$, ce qui contredit la maximalité de $(K',\phi)$. \end{proof}

\paragraph{}\label{CloAlgUni}\textbf{Lemme.} \textit{Soit $K_1/k$, $K_2/k$ deux  clôtures algébriques de $k$. Alors $Hom_{Alg/k}(K_1,K_2)=Aut_{Alg/k}(K_1,K_2)$}

\begin{proof} Soit $\phi:K_1/k\rightarrow K_2/k$ un $k$-plongement; il faut montrer qu'il est surjectif. Mais d'une part $ \phi(K_1)$ est   algébriquement clos et, d'autre part, $K_2/\phi(K_1)$ est une extension  algébrique puisque c'est une sous-extension de $K_2/k$. Donc  on a forcément $\phi(K_1)=K_2$. \end{proof}



 \textbf{Corollaire.} \textit{Les clôtures algébriques de $k$ sont toutes $k$-isomorphes.}

\begin{proof}Soit $K_1/k$, $K_2/k$ deux  clôtures algébriques de $k$. Par \ref{CloAlgPlonge}, il existe un   $k$-plongement $\phi:K_1 \rightarrow K_2 $. Par le lemme précédent, c'est un isomorphisme.   \end{proof}

\subsubsection{}\textbf{Existence de la clôture  algébrique.}\\ 

\paragraph{}\label{CorpsInductive}\textbf{Lemme.} (Limite inductive de corps) \textit{Soit $k_0\stackrel{\phi_0}{\rightarrow} k_1\stackrel{\phi_1}{\rightarrow} k_2\stackrel{\phi_2}{\rightarrow}\cdots$ une suite de morphismes de corps. Il existe un corps $k_\infty$ et une suite de morphismes de corps $i_n:k_n\rightarrow k_\infty$, $n\geq 0$ tels que  $i_{n+1}\circ \phi_n= i_n$, $n\geq 0$  pour tout corps $K$ et toute   suite de morphismes de corps $j_n:k_n\rightarrow K$, $n\geq 0$ tels que $j_{n+1}\circ \phi_n = j_n$, $n\geq 0$ on a un unique morphisme de corps $j:k_\infty\rightarrow K$ tel que $j\circ i_n=j_n$, $n\geq 0$.}
\begin{proof}On commence par faire la construction en oubliant les structures de produit.  Considérons donc la somme directe $\iota_n:k_n\rightarrow \Sigma:=\oplus_{n\geq 0}k_n$ des $k_0$-modules $k_n$, $n\geq 0$ et le sous-$k_0$-module $R$ engendré par les éléments de la forme 
$$\iota_{n+1}\circ \phi_n(x_n)-\iota_n(x_n),\; x_n\in k_n,\; n\geq 0.$$
Posons $k_\infty:=\Sigma/R$ et $$i_n:k_n\stackrel{\iota_n}{\rightarrow} \Sigma\stackrel{p_R}{\twoheadrightarrow} k_\infty,\; n\geq 0.$$
Observons que
\begin{enumerate}
\item $k_\infty=\cup_{n\geq 0}i_n(k_n)$: cela résulte des relations $\iota_n(x_n)=\iota_{n+1}\circ \phi_n(x_n)+\iota_n(x_n)-\iota_{n+1}\circ \phi_n(x_n)\in \iota_{n+1}\circ \phi_n(x_n)+R$, qui montrent que $i_n(k_n)\subset i_{n+1}(k_{n+1})$, $n\geq 0$.
\item Pour tout  $n\geq 0$, $\ker(i_n)=\lbrace 0\rbrace$: Tout  $0\not= x\in  R$ s'écrit sous la forme 
$$x=\sum_{1\leq i\leq r} \iota_{n_i+1}\circ \phi_{n_i}(x_{n_i})-\iota_{n_i}(x_{n_i})$$  
avec $0\not=x_{n_i}\in k_{n_i}$, $i=1,\dots, r$ et $n_1<n_2<\dots<n_r$. En particulier, la $n_1$-ième composant de $x$ vaut $-x_{n_1}$ et la $n_r+1$-ième vaut $-\phi_{n_r}(x_{n_r})$. Comme $\phi_{n_r}$ est injective, on en déduit que $ x$ a au moins deux composantes non nulles. En particulier, $\iota_n(k_n)\cap R= 0$.
\item Produit sur $k_\infty$: On a donc des carrés commutatifs $$\xymatrix{k_{n+1}\ar[r]^\simeq_{i_{n+1}}&i_{n+1}(k_{n+1})\\
k_n \ar[r]^\simeq_{i_n}\ar[u]^{\phi_n}&i_n(k_n)\ar[u]_{\cup}}$$
ar (1), pour tout $x,y\in k_\infty$, on peut choisir $n\geq 0$ tel que $x,y\in i_n(k_n)$ et par (2), $xy=i_n(i_n^{-1}(x)i_n^{-1}(y))$. Cette définition est indépendante du choix de $n$ car la commutativité du diagramme ci-dessus et le fait que $\phi_n:k_n\rightarrow k_{n+1}$ est un morphisme de corps montrent que 
$$i_{n+1}(i_{n+1}^{-1}(x)i_{n+1}^{-1}(y))=i_{n+1}(\phi_n(i_n^{-1}(x))\phi_n(i_n^{-1}(x)))=i_{n+1}(\phi_n(i_n^{-1}(x) i_n^{-1}(x)))=i_n(i_n^{-1}(x)i_n^{-1}(y)). $$
Enfin, on vérifie facilement qu'avec cette loi, $k_\infty$ est un corps et que   $i_n:k_n\rightarrow k_\infty$ est un morphisme  de corps, $n\geq 0$. 
\end{enumerate}
 Il reste à vérifier que les morphismes de corps $i_n:k_n\rightarrow k_\infty$, $n\geq 0$ ainsi construits vérifient bien la propriété universelle de l'énoncé. Si  $j:k_\infty\rightarrow K$ existe, les conditions $j\circ i_n=j_n$, $n\geq 0$ impose que $j(\overline{(x_n)}_{n\geq 0})=\sum_{n\geq 0}j_n(x_n) $, d'o\`u l'unicité de $j:k_\infty\rightarrow K$ sous réserve de son existence. Par propriété universelle de la somme directe, il existe un unique morphisme de $k_0$-modules $\Sigma \rightarrow K$, $(x_n)_{n\geq 0}\rightarrow \sum_{n\geq 0}j_n(x_n) $ et les conditions $j_{n+1}\circ \phi_n = j_n$, $n\geq 0$ montrent que $R$ est contenu dans le noyau de ce morphisme donc qu'il passe au quotient en un morphisme de $k_0$-module $j:k_\infty\rightarrow K$ tel que $j(\overline{(x_n)}_{n\geq 0})=\sum_{n\geq 0}j_n(x_n) $. Par construction, $j\circ i_n(x_n)=j_n(x_n)$, $n\geq 0$ et on vérifie sur les définitions que $j:k_\infty\rightarrow K$
 est bien un morphisme de corps.\end{proof}
 
  Comme d'habitude les $i_n:k_n\rightarrow k_\infty(=:\displaystyle{\lim_{\longrightarrow}k_n})$, $n\geq 0$ sont uniques à unique isomorphisme près et on dit que c'est la limite inductive (ou la limite directe ou simplement la limite) des $k_0\stackrel{\phi_0}{\rightarrow} k_1\stackrel{\phi_1}{\rightarrow} k_2\stackrel{\phi_2}{\rightarrow}\cdots$.\\ 

 On peut réécrire \ref{CorpsInductive} en disant que l'application canonique 
$$\SHom(k_\infty,K)\rightarrow \lbrace (j_n)_{n\geq 0}\in \prod_{n\geq 0}\SHom(k_n,K)\; |\; j_{n+1}\circ \phi_n=j_n,\; n\geq 0\rbrace,\; j\rightarrow (j\circ i_n)_{n\geq 0}$$
est bijective ou encore, plus visuellement,
$$\xymatrix{k_{n+1} \ar[dr]^{i_{n+1}}\ar@/^1pc/[drr]^{j_{n+1}}&& \\
&k_\infty\ar@{.>}[r]^{\exists ! j}&K\\
k_n\ar[uu]^{\phi_n}\ar[ur]_{i_n}\ar@/_1pc/[urr]_{j_n}&& }$$

\paragraph{}\label{CorpsInductiveAlgClos}\textbf{Lemme.}   \textit{Avec les notations de \ref{CorpsInductive}, supposons de plus que pour tout $n\geq 0$ et $P_n\in k_n[X]\setminus k_n$, $\phi_n(P_n)\in k_{n+1}[X]$ a une racine dans $k_{n+1}$. Alors $k_\infty$ est algébriquement clos.}
\begin{proof}Soit $P\in k_\infty[X]\setminus k_\infty$. Comme $k_\infty=\cup_{n\geq 0}i_n(k_n)$ il existe $n\geq 0$ tel que $P\in  i_n(k_n)[X]\setminus k_n$. Mais par hypothèse $\phi_n\circ i_n^{-1}(P)\in k_{n+1}[X] $ a une racine $x_{n+1}\in k_{n+1}$ donc $i_{n+1}(x_{n+1}) \in i_{n+1}(k_{n+1})\subset k_\infty$ est une racine de $i_{n+1}\circ \phi_n\circ i_n^{-1}(P)=P$.\end{proof}

\paragraph{}\label{Contient}[Utilise le Lemme de Zorn] \textbf{Lemme.}   \textit{Pour tout corps $k$ il existe une extension algébrique $K/k$ telle que tout $P\in k[X]\setminus k$ possède une racine dans $K$.}
\begin{proof} Observons d'abord que si $P\in k[T]\setminus k$, il existe toujours une extension finie $K_P/k$ telle que   $P $ possède une racine dans $K$. En effet, il suffit de considérer l'extension $k\rightarrow K_P:=k[X]/Q$ pour $Q\in k[X]$ un diviseur irréductible unitaire de $P$. En appliquant inductivement ce procédé, pour tout $P_1,\dots, P_r\in k[X]\setminus k$, on peut toujours construire une extension finie $K_{P_1,\dots,P_r}/k$ telle que $P_i$ possède une racine dans $K_{P_1,\dots, P_r}$, $i=1,\dots, r$. En utilisant  \ref{CorpsInductive}, on peut encore étendre cette construction à une suite $(P_n)_{n\geq 0}$ éléments  de $k[X]\setminus k$. Mais  $k[X]\setminus k$ n'est en général pas dénombrable et il faut un peu adapter ces observations.Considérons la $k$-algèbre de  polynômes  $k\rightarrow k[X_P, \; P\in k[X]\setminus k]$ et l'idéal $I\subset  k[X_P, \; P\in k[X]\setminus k]$ engendré par les $P(X_P)$, $P\in \mathcal{P}$. Vérifions d'abord que $I\subsetneq k[X_P, \; P\in k[X]\setminus k]$. Sinon on pourrait écrire $$(*)\;\;1=\sum_{1\leq i\leq r} F_iP_i(X_{P_i})$$ avec $F_i\in k[X_P, \; P\in k[X]\setminus k]$, $i=1,\dots, r$. Chaque $F_i$ ne fait intervenir qu'un sous-ensemble fini   $\mathcal{F}\subset k[X]\setminus k$ et quitte à agrandir $\mathcal{F}$, on peut supposer que $P_1,\dots, P_r\in \mathcal{F}$. On peut donc réécrire la relation précédente sous la forme 
$$(*)\;\; 1=\sum_{P\in\mathcal{F}}F_P P(X_P)$$ avec $F_P\in k[X_Q,\; Q\in \mathcal{F}]$, $P\in \mathcal{F}$.  D'après nos observations préliminaires, il existe une extension finie $K_\mathcal{F}/k$ tel que $P$ possède une racine $x_p\in K_\mathcal{F}$, $P\in \mathcal{F}$. En évaluant $(*)$ en les $x_P$, $P\in \mathcal{F}$, on obtient donc $1=0$. Cela montre que $I\subsetneq  k[X_P, \; P\in k[X]\setminus k]$ donc est contenu dans un idéal maximal $\frak{m}\subset k[X_P, \; P\in k[X]\setminus k]$. Cela nous donne une  extension $k\rightarrow k[X_P, \; P\in k[X]\setminus k]/\frak{m}:=\Omega$ tel que  tout $P\in k[X]\setminus k$ a une racine $ x_P\in \Omega$. De plus, $\Omega$ est engendrée comme extension de $k$ par les $x_P$, $P\in k[X]\setminus k$ donc est algébrique sur $k$.\end{proof}

 

\paragraph{}\label{AlgCloExist} \textbf{Proposition.} \textit{Tout corps $k$ admet une clôture algébrique.}
\begin{proof} Notons $k_0:=k$. D'après \ref{Contient},  il existe une extension $k_1/k$  algébrique sur $k$ et tel que tout $P\in k_0[T]\setminus k_0$ possède une racine dans $k_0$.   En itérant le procédé, on construit une suite de morphisme corps     $k_0\stackrel{\phi_0}{\rightarrow} k_1\stackrel{\phi_1}{\rightarrow} k_2\stackrel{\phi_2}{\rightarrow}\cdots$  telle que pour tout $n\geq 0$ et $P_n\in k_n[X]\setminus k_n$, $\phi_n(P_n)\in k_{n+1}[X]$ a une racine dans $k_{n+1}$. Par \ref{CorpsInductive} et \ref{CorpsInductiveAlgClos},  $ k_\infty/k$ est algébriquement clos.  On peut donc prendre $  \overline{k}:=\overline{k}^{k_\infty}$ (\textit{cf.} \ref{CloAlgDef}).   \end{proof}



 \subsection{Automorphismes, Corps de décomposition, extensions normales}\textit{}
 
 \subsubsection{}\label{NormalePrel}Pour une extension de corps $K/k$ on notera $Aut(K/k):=\SAut_{Alg/k}(K)$ le groupe des automorphismes de la $k$-algèbre $k\rightarrow K$. Si $K_1/k$, $K_2/k$ sont deux extensions de corps tout $k$-isomorphisme $\phi:K_1\tilde{\rightarrow }K_2$    induit un isomorphisme de groupes 
 $$\SAut_{Alg/k}(K_1)\tilde{\rightarrow} \SAut_{Alg/k}(K_2),\; \sigma\rightarrow \phi\circ \sigma\circ \phi^{-1}.$$
 En particulier, si $\overline{k}/k$, $\overline{k}'/k$ sont deux clôtures algébriques de $k$, on sait qu'il existe toujours un $k$-isomorphisme $\phi:\overline{k}\tilde{\rightarrow }\overline{k}'$  donc que $\SAut(\overline{k}/k)$ ne dépend pas - à isomorphisme près - du choix de la cl\^^oture algébrique $\overline{k}/k$.\\
   
   \textbf{Exemple.} $\C=\R[T]/T^2+1$. Notons $\overline{T}:= i$.  On a $\SAut(\C/\R)=\SHom_{Alg/\R}(\R[T]/T^2+1,\C)=\lbrace \overline{ev}_{i}=Id, \overline{ev}_{-i}=\overline{-}\rbrace$. \\
 
  Si $\phi:K_1\tilde{\rightarrow} K_2$ est un isomorphisme de corps, la propriété universelle de $c_1:K_1\rightarrow K_1[T]$ appliquée à la $K_1$-algèbre $K_1\stackrel{\phi}{\rightarrow} K_2\stackrel{c_2}{\rightarrow}  K_2[T]$ donne un unique isomorphisme de $K$-algèbres 
 $\phi:K_1[T]\rightarrow K_2[T]$ tel que $\phi\circ c_1=c_2\circ \phi$ (ici, $c_i:K_i\rightarrow K_i[T]$, $i=1,2$ sont les morphismes canoniques). Explicitement $\phi(\sum_{n\geq 0}a_nT^n)=\sum_{n\geq 0}\phi(a_n)T^n$. Par construction, pour tout $x_1\in K_1$ on a 
 $$\phi\circ ev_{x_1}=ev_{\phi(x_1)}\circ \phi:K_1[T]\rightarrow K_2[T].$$
 En particulier, $\phi:K_1\tilde{\rightarrow }K_2$  se restreint en une bijection 
 $$\phi:Z_{K_1}(P)\tilde{\rightarrow} Z_{K_2}(\phi(P)).$$
 Dans le cas particulier o\`u $K_1=K_2=K/k$, $\phi\in \SAut(K/k)$ et $P\in k[T]$, $\phi P=P$, $\phi\in \SAut(K/k)$ induit une permutation $\phi: Z_K(P)\tilde{\rightarrow} Z_K(P)$. En d'autres termes, $ \SAut(K/k)$ agit sur $ Z_K(P)$ \textit{i.e.} on a un morphisme de groupes 
 $$\SAut(K/k)\rightarrow \mathcal{S}(Z_K(P)).$$
 On peut notamment appliquer cette observation à $K/k=\overline{k}/k$ une clôture algébrique de $k$. 
  
 \subsubsection{}\textbf{Sous-extensions normales de $\overline{k}/k$.}  
 

 \paragraph{}\label{Normale1}\textbf{Lemme.} \textit{Soit $K/k$ une extension algébrique et $\phi_1,\phi_2:K\rightarrow \overline{k}$ deux $k$-plongements. Il existe $\sigma\in  \SAut(\overline{k}/k)$ tel que $\sigma\circ \phi_1=\phi_2$.}
 \begin{proof} C'est un cas particulier de \ref{CloAlgPlonge}. En effet, comme $\phi_1:K\rightarrow\overline{k}$ est une extension algébrique et $\phi_2:K\rightarrow \overline{k}$ est une clôture algébrique de $K$,  il existe un $K$-plongement $\sigma:\overline{k} \rightarrow \overline{k} $, qui est automatiquement un $K$-automorphisme. Comme à gauche l'extension $\overline{k}/K$ est donnée par $\phi_1 :K\rightarrow \overline{k}$ et à droite est donnée par $ \phi_2:K\rightarrow \overline{k}$, dire que  $\sigma:\overline{k} \rightarrow \overline{k} $ est un $k$-plongement signifie bien que $\sigma\circ \phi_1=\phi_2$.  \end{proof}
 
  \paragraph{}\label{Normale2}On rappelle que si $x\in\overline{k}$, on note  $P_x\in k[T]$ le polynôme minimal de $x$ sur $k$.\\
  
  \textbf{Lemme.} \textit{Pour tout $x,y\in\overline{k}$ les propriétés suivantes sont équivalentes.
 \begin{enumerate}
\item  Il existe $\sigma\in  \SAut(\overline{k}/k)$ tel que $\sigma(x)=y$;
\item $P_x=P_y$.
\end{enumerate}}
 \begin{proof}   (1) $\Rightarrow $ (2) D'après \ref{NormalePrel}, $y=\sigma(x) \in Z_{\overline{k}}(P_x)$ donc $P_y|P_x$ dans $k[T]$. Mais comme $P_x$ est irréductible  sur $k$ on a nécessairement $P_x=P_y$.\\
 (2) $\Rightarrow$ (1) Notons $P:=P_x=P_y\in k[T]$. On dispose de deux $k$-plongements $\overline{ev}_x:k[T]/P\rightarrow \overline{k}$, $\overline{T}\rightarrow x$,  $\overline{ev}_y:k[T]/P\rightarrow \overline{k}$, $\overline{T}\rightarrow y$ donc, d'après \ref{Normale1}, il existe $\sigma\in  \SAut(\overline{k}/k)$ tel que $\sigma\circ \overline{ev}_x=\overline{ev}_y$. En particulier $\sigma(x)=\sigma ( \overline{ev}_x(\overline{T}))=\overline{ev}_y(\overline{T})=y$.\end{proof}


 On dit que deux éléments $x,y\in \overline{k}$ qui vérifient les propriétés   équivalentes du lemme \ref{Normale2}  sont conjugués sur $k$. Si $x\in \overline{k}$ les conjugués de $x$ sur $k$ sont donc les éléments de
$$(*)\;\; \mathcal{C}_k(x):=\lbrace \sigma(x)\; |\; \sigma\in \SAut(\overline{k}/k)\rbrace =Z_{\overline{k}}(P_x).$$\\


\textbf{Exemple.}\begin{enumerate}
\item  $  \C/\R$: les conjugués sur $\R$ de $z\in \C$ sont $z$ et $\overline{z}$.    \\
\item $\overline{\Q}/\Q$: les conjugués sur $Q$ de $^3\sqrt{5}$  sont $^3\sqrt{5}$, $j^3\sqrt{5}$, $j^2\;^3\sqrt{5}$ (o\`u $j$ est une racine de $T^2+T+1$ \textit{i.e.} une racine primitive $3$ième de l'unité).
\end{enumerate}

 \paragraph{}\label{Normale3} \textbf{Lemme.} \textit{Soit $k\subset K\subset \overline{k}$ une sous-extension. Les propriétés suivantes sont équivalentes.
 \begin{enumerate}
\item  Pour tout $\sigma\in  \SAut(\overline{k}/k)$, $\sigma(K)\subset K$
\item Pour tout $x\in K$, $Z_K(P_x)=Z_{\overline{k}}(P_x)$ (\textit{i.e.} toutes les racines de $P_x$ dans $\overline{k}$ sont contenues dans $K$).
\end{enumerate}}
 \begin{proof} (1) $\Rightarrow $ (2)  Cela résulte immédiatement de $(*)$.   (2) $\Rightarrow$ (1) Toujours  d'après $(*)$, pour tout $\sigma\in  \SAut(\overline{k}/k)$ et   tout $x\in K$, $\sigma(x)\in Z_{\overline{k}}(P_x)$ or, par (2) $Z_{\overline{k}}(P_x)=Z_K(P_x)\subset K$. \end{proof}


 On dit qu'une sous-extension $k\subset K\subset \overline{k}$ qui  vérifie les propriétés   équivalentes du lemme \ref{Normale3} est une \textit{sous-extension normale}  \index{Normale (Sous-extension)} de $\overline{k}/k$.\\

 \paragraph{}\label{Normale4}\textbf{Corollaire.} \textit{Soit $k\subset K\subset \overline{k}$ une sous-extension normale de   $\overline{k}/k$. Le morphisme de restriction 
 $$\SAut(\overline{k}/k)\rightarrow \SAut(K/k),\; \sigma\rightarrow \sigma|_K$$
 est un morphisme de groupes bien défini, surjectif et de noyaux $\SAut(\overline{k}/K)$.}
 \begin{proof}Le fait que $\sigma\rightarrow \sigma|_K$ est bien défini résulte de \ref{Normale3}; c'est alors automatiquement un morphisme de groupes. Soit $\sigma\in \SAut(K/k)$ et notons $\phi:K\rightarrow \overline{k}$ le $k$-plongement définissant $\overline{k}/K$. On dispose donc de deux $k$-plongements $\phi,\phi\circ \sigma:K\rightarrow \overline{k}$ donc, d'après \ref{Normale1}, il existe $\tilde{\sigma}\in \SAut(\overline{k}/k)$ tel que $\tilde{\sigma}\circ \phi=\phi\circ \sigma$. Cela montre la surjectivité. On a immédiatement   $\SAut(\overline{k}/K)=\ker( \sigma\rightarrow \sigma|_K)$. \end{proof}
 
  La terminologie `normale' vient du fait que le sous-groupe $\SAut(\overline{k}/k)\subset \SAut(\overline{k}/k)$ est normal.  \\
 
 \textbf{Exemple.}
 \begin{itemize}
 \item Toute sous-extension $k\subset K\subset \overline{k}$ de degré $2$ est normale.
\item  La sous-extension $\Q(^3\sqrt{5})/\Q$ de $\overline{\Q}/\Q$ (de degré $3$) n'est pas normale puisqu'elle ne contient pas $j^3\sqrt{5}$, $j^2 \;^3\sqrt{5}$. Par contre la sous-extension $\Q(^3\sqrt{5}, j^3\sqrt{5})/\Q=\Q(^3\sqrt{5}, j)/\Q$ de $\overline{\Q}/\Q$ (de degré $6$)  est normale. 
\end{itemize}
 \subsubsection{}\textbf{Extensions normales, corps de décomposition.}\\
 
  \paragraph{}\label{Normale5}Si $k$ est un corps et $K/k$ une extension de corps, on dit qu'un polynôme $P\in k[T]$ est totalement décomposé sur $K$ si tous ses facteurs irréductibles dans $ K[T]$ sont de degré $1$ \textit{i.e.} $P$ s'écrit sous la forme 
  $$P=a\prod_{1\leq i\leq n}(T-\alpha_i)$$
  avec $a\in k$, $\alpha_1,\dots, \alpha_n \in k$.\\
  
     Observons que si $P\in k[T]$ est totalement décomposé sur $K$, pour toute extension de corps $K'/k$ et $k$-plongement $\phi:K\rightarrow K'$, $P\in k[T]$ est encore totalement décomposé sur $K'$ puisque, en utilisant que  $\phi:K[T]\tilde{\rightarrow} K[T]$  est un automorphisme de $k$-algèbre,  
  $$\phi P=P=\phi(a\prod_{1\leq i\leq n}(T-\alpha_i))= a \prod_{1\leq i\leq n}(T-\phi(\alpha_i)). $$
 En particulie, on a  $\phi (Z_K(P))=Z_{\phi(K)}(P)=Z_{K'}(P)$.\\
  
   \textbf{Lemme.} \textit{Soit $K/k$ une extension algébrique. Les propriétés suivantes sont équivalentes.
 \begin{enumerate}
\item  Pour tout $x\in K$, $P_x\in k[T]$ est totalement décomposé sur $K$;
\item Si $\overline{k}/k$ est une clôture algébrique et $\phi_1,\phi_2:K\rightarrow \overline{k}$ sont deux $k$-plongements, $\phi_1(K)=\phi_2(K)$;
\item  Si $\overline{k}/k$ est une clôture algébrique et $\phi:K\rightarrow \overline{k}$ un $k$-plongement, $k\subset \phi(K)\subset \overline{k}$ est une sous-extension normale de $\overline{k}/k$.
\end{enumerate}}
 \begin{proof} (1) $\Rightarrow$ (2) D'après l'observation ci-dessus appliquée à un $k$-plongement $\phi :K\rightarrow \overline{k}$, pour tout $x\in K$, $\phi (x)\in \phi_i(Z_K(P_x))=Z_{\overline{k}}(P_x)=Z_{\phi (K)}(P_x)\subset \phi (K)$. Autrement dit 
 $$\phi (K)=\cup_{x\in K}Z_{\overline{k}}(P_x)$$
 est indépendant du $k$-plongement $\phi :K\rightarrow \overline{k}$. 
  (2) $\Rightarrow$ (3) Pour tout  $\sigma\in  \SAut(\overline{k}/k)$, $\sigma\circ \phi:K\rightarrow \overline{k}$ est un autre $k$-plongement donc par (2) $\sigma(\phi(K))=\phi(K)$ et on utilise la caractérisation (1) de \ref{Normale3}.\\
    (3) $\Rightarrow$ (1) Par \ref{CloAlgPlonge}, il existe un $k$-plongement $\phi:K\rightarrow \overline{k}$. Par (3) et \ref{Normale3} (2), pour tout $x\in K$, $P_{\phi(x)}=\phi(P_x)=P_x\in k[T]$ est totalement décomposé sur $\phi(K)$ donc $P_x$ est totalement décomposé sur $K$.  \end{proof}


 On dit qu'une  extension $ K/k $ qui  vérifie les propriétés   équivalentes du lemme \ref{Normale5} est une \textit{extension normale}\index{Normale (Extension)}.\\

\textbf{Remarque.} On peut reformuler (1) en (1)' pour tout $P\in k[T]$  irréductible si $P$  a une racine dans $K$ alors $P$ est totalement décomposé sur $K$ et (2) en (2)' Si   $\phi :K\rightarrow K'$ est un $k$-plongement alors $\phi(K)\subset K'$ est la sous-extension de $K'/k$ engendrée par les $Z_{K'}(P_x)$, $x\in K$ (et est, en particulier, indépendante du $k$-plongement $\phi :K\rightarrow K'$).\\ 

  \textbf{Exemple.}
   \begin{itemize}
 \item Toute  extension $K/k$ de degré $2$ est normale.
\item  L'extension $\Q(^3\sqrt{5})\simeq \Q[T]/T^3-5/\Q$  (de degré $3$) n'est pas normale. Par contre    $\Q(^3\sqrt{5}, j^3\sqrt{5})/\Q\simeq\Q(^3\sqrt{5})[T]/\langle T^2+T+1\rangle\simeq \Q[X,T]/\langle X^3-5, T^2+T+1\rangle /\Q  $  (de degré $6$)  est normale.  \\
 
\end{itemize}
  
  \textbf{Contre-exemple.} On prendra garde que la propriété d'être normale se comporte mal par transitivié. Plus précisément, si $K_3/K_2$ et $K_2/K_1$ sont deux extensions de corps, 
  \begin{itemize}
  \item il n'est pas vrai en général que si $K_3/K_2$ et $K_2/K_1$ sont normales alors $K_3/K_1$ est normale (contre-exemple: $\Q(^4\sqrt{2})/\Q(\sqrt{2})$ et $\Q(\sqrt{2})/\Q$ sont normales mais $\Q(^4\sqrt{2})/\Q$ ne l'est pas);
  \item  il n'est pas vrai non plus en général que si $K_3/K_1$ est normale alors $K_2/K_1$ est normale  (contre-exemple: $\Q(^3\sqrt{5},j)/\Q $ est  normale  mais $\Q(3\sqrt{5})/\Q$ ne l'est pas);
  \item par contre si $K_3/K_1$ est normale alors $K_3/K_2$ l'est aussi. En effet, pour tout $x\in K_3$ si on note $P_{x,1}\in K_1[T]$ et $P_{x, 2}\in K_2[T]$ les polynômes minimaux de $x$ sur $K_1 $ et $K_2$ respectivement alors $P_{x,2}|P_{x,1}$ dans $K_2[T]$ donc, comme $P_{x,1}$ est totalement décomposé sur $K_3$, $P_{x,2}$ l'est aussi.
  \end{itemize}
  \paragraph{}\label{Normale6}Soit $P\in k[T]$. On dit qu'une extension $K/k$ est un \textit{corps de décomposition}\index{Corps de décomposition (Polynôme)} de $P\in k[T]$ sur $k$ si $P$ est totalement décomposé sur $k$ et si $K=k( Z_K(P))/k$.\\
  
  \textbf{Lemme.} \textit{Tout polynôme $P\in k[T]$ admet un corps de décomposition  sur $k$, qui est unique à $k$-isomorphisme (non-unique) près et est une extension normale de $k$.}
  \begin{proof}Soit $\overline{k}/k$ une clôture algébrique de $k$ alors $K_0=k(Z_{\overline{k}}(P))/k$ est un corps de décomposition de $P$ sur $k$. Soit $K/k$ un corps de décomposition de $P$ sur $k$.  Par \ref{CloAlgPlonge}, il existe un $k$-plongement $\phi:K\rightarrow  \overline{k} $. Or par $(*)$ et comme $P$ est totalement décomposé sur $K$ donc sur $\phi(K)\subset \overline{k}$, 
   $\phi(K)=\phi(k(Z_K(P)))=k(Z_{\overline{k}}(P))=K_0$.  En particulier, $\phi:K\rightarrow \overline{k}$ induit un $k$-isomorphisme $\phi:K=k(Z_K(P))\tilde{\rightarrow} k(\phi(Z_K(P)))=k(Z_{\overline{k}}(P))=K_0$. Enfin, pour tout   $\sigma\in \SAut(\overline{k}/k)$, $\sigma_{K_0}:K_0\rightarrow \overline{k}$ est un $k$-plongement et l'argument qui précède appliqué avec $K=K_0$ montre que $\sigma(K_0)=K_0$. Donc $K_0/k$ est une sous-extension normale de $\overline{k}/k$ donc une extension normale.  \end{proof}
  
   Si $K/k$ est un corps de décomposition de $P$, on a une action naturelle $\SAut(K/k)\times Z_K(P)\rightarrow Z_K(P)$ qui est fidèle puisque $K=k(Z_K(P))$ d'o\`u un morphisme de groupes injectif $\SAut(K/k)\hookrightarrow \mathcal{S}(Z_K(P))$. C'est essentiellement ce groupe $\SAut(K/k)$  qui va reflèter  les `symétries' des racines de $P$. \\
  
  \textbf{Remarque.} Le Lemme \ref{Normale6} montre qu'une extension $K/k$ finie est normale si et seulement si c'est le corps de décomposition d'un polynôme $P\in k[T]$ sur $k$.\\
  
  \textbf{Exemple.} Le corps de décomposition de $ X^3-5\in \Q[T]$ sur $\Q$ est $\Q(^3\sqrt{5}, j)/\Q$. En considérant l'extension intermédiaire $\Q\subset \Q(j)\subset \Q(j)(^3\sqrt{5})$ et en observant que $\Q(j)/\Q$ est le corps de décomposition de $X^2+X+1\in \Q[T]$ sur $\Q$, on obtient une suite exacte courte de groupes 
  $$1\rightarrow \SAut(\Q(^3\sqrt{5}, j)/\Q(j))\rightarrow \SAut(\Q(^3\sqrt{5}, j)/\Q)\rightarrow \SAut(\Q(j)/\Q)\rightarrow 1.$$ 
 Or la bijection $\SAut(\Q(^3\sqrt{5}, j)/\Q(j))\tilde{\rightarrow} Z_{\Q(^3\sqrt{5},j)}(X^3-5)$ montre que  $\SAut(\Q(^3\sqrt{5}, j)/\Q(j))$ est d'ordre $3$ donc cyclique et engendré par $\Q(^3\sqrt{5}, j)=\Q(j)[T]/T^3-5\tilde{\rightarrow} \Q(^3\sqrt{5}, j)$, $\overline{T}=^3\sqrt{5}\rightarrow j\;^3\sqrt{5}$. De même,  la bijection $\SAut(\Q(  j)/\Q )\tilde{\rightarrow} Z_{\Q( j)}(X^2+X+1)$ montre que  $\SAut(\Q(j)/\Q )$ est d'ordre $2$ donc cyclique et engendré par $\Q(  j)=\Q [T]/T^2+T+1\tilde{\rightarrow} \Q( j)$, $\overline{T}=J\rightarrow j^2$. Donc $\SAut(\Q(^3\sqrt{5}, j)/\Q)$ est un sous-groupe d'ordre $6$ de $\mathcal{S}_3$; c'est $\mathcal{S}_3$... On verra bientôt comment mener de fa\c{c}on plus systématique  ce type de calculs.\\
  
   A ce stade, on aimerait poser, pour un polynôme $P\in k[T]$, $G_P:=\SAut(K/k)$, o\`u $K$ est un corps de décomposition de $P$ sur $k$ et étudier les racines $\alpha_1,\dots , \alpha_n$ de $P$ dans $K$ - ou plutôt les sous-corps $k(\alpha_i)$, $i=1,\dots, n$ de $K$ engendrés par les  racines \textit{via} la structure du groupe $G_P$ en montrant qu'on a une correspondance bijective entre  les sous-groupes $H\subset G_P$ et les sous-corps $k\subset L\subset K$ donnée par $H\rightarrow K^{H}$, $L\rightarrow \SAut(K/L)$. Cependant, ce n'est pas toujours vrai que les applications  $H\rightarrow K^{H}$, $L\rightarrow \SAut(K/L)$ sont bijectives comme le montre l'exemple suivant. Prenons $k:=\F_p(X)$, $P=T^{p^2}-X\in k[T]$ et notons $K/\F_p(X)$ un corps de décomposition. Soit $\alpha\in K$ une racine de $P$. On a alors $P=(T-\alpha)^{p^2}$ sur $K[T]$. En particulier, $K=k(\alpha)=k[T]/P$. Mais comme $P$ n'a qu'une seule racine sur $k$, $\SAut(K/k)=1$ alors que $K/k$ contient est non triviale (et même contient une sous-extension stricte: $k\subsetneq k(\alpha^p)\subsetneq k(\alpha)=K$). Pour faire marcher cette approche, il va falloir imposer une condition supplémentaire sur $P$, celle d'être séparable.\\
   \subsection{Extensions séparables}
 
 \subsubsection{}\label{Derivations}\textbf{Dérivations.} Soit $A$ une $k$-algèbre. Une \textit{$k$-dérivation sur $A$}\index{Dérivation} est un endomorphisme de $k$-module $\partial: A\rightarrow A$ tel que $$\partial (ab)=a\partial b+(\partial a) b,\; a,b\in A.$$ 
Une récurrence facile montre que $\partial (a^n)=na^{n-1}\partial a$, $a\in A$, $n\geq 1$. En particulier, $\partial 1=\partial (1^2)=2\partial 1$ donc $\partial 1=0$ et par $k$-linéarité, $ k\subset \ker(\partial)$.\\

 Sur $A=k[T]$ une dérivation est donc uniquement déterminée par sa valeur en $T$. Considérons la dérivation $\partial : k[T]\rightarrow k[T]$, $T\rightarrow 1$ \textit{i.e.} 
$$\partial (\sum_{n\geq 0}a_nT^n)=\sum_{n\geq 1}na_nT^{n-1}. $$
On note en général $P':=\partial P$, $P\in k[T]$ et on dit que $P'$ est le polynôme dérivé de $P$\index{Polynôme dérivé}.\\

\textbf{Remarques.}
 \begin{enumerate}
\item Pour tout morphisme de corps $\phi:k\rightarrow k'$ on vérifie immédiatement que 
 $\phi(P)'=\phi(P')$, $P\in k[T]$.
\item Notons $p$ la caractéristique de $k$. Soit $P\in k[T]$ tel que $P'=0$.

 \begin{itemize}
\item Si $p=0$, $P\in k$.
\item Si $p>0$, il existe - un unique polynôme $Q\in k[T]$ tel que $P=Q_1(T^p)$. Par récurrence il existe donc un unique $r\in\Z_{\geq 1}$ et un unique polynôme $Q_r\in k[T]$ tels que $Q_r'\not= 0$ et $ P=Q_r(T^{p^r})$
\end{itemize}
Ecrivons $P=\sum_{n\geq 0} a_nT^n$ donc $P'=\sum_{n\geq 1} na_nT^{n-1}=0$ si et seulement si $na_n=0$, $n\geq1$. Si $p=0$, cela impose $a_n=0$, $n\geq 1$ donc $P=a_0$. Si $p>0$, cela impose $a_n=0$, $n\geq 1$, $p\not| n$ donc $P= \sum_{n\geq 0} a_{np}T^{np}=\sum_{n\geq 0} a_{np}(T^p)^n=Q(T^p)$ avec $Q_1=\sum_{n\geq 0} a_{np}T^n$.\\
\end{enumerate}

  \subsubsection{}\label{Separable1}\textbf{Lemme.} \textit{Pour $P\in k[T]$ les propriétés suivantes sont équivalentes.
  \begin{enumerate}
  \item $P$ et $P'$ sont premiers entre eux dans $k[T]$;
  \item Pour toute clôture algébrique $\overline{k}/k$, $P$ et $P'$ sont premiers entre eux dans  $\overline{k}[T]$;
  \item Pour toute clôture algébrique $\overline{k}/k$, $|Z_{\overline{k}}(P)|=deg(P)$;
   \item Pour toute clôture algébrique $\overline{k}/k$, $\overline{k}[T]/P$ est réduite.
  \end{enumerate}}
  \begin{proof} (1) $\Rightarrow$ (2) résulte de Bézout: $P$ et $P'$ sont premiers entre eux dans $k[T]$ si et seulement si $k[T]=k[T]P+k[T]P'$ \textit{i.e.} il existe $U,V\in k[T]$ tels que $UP+VP'=1$. Mais cette relation est  \textit{a fortiori} vraie dans $ \overline{k}[T]$ donc la conclusion résulte de Bézout dans $\overline{k}[T]$. (2) $\Rightarrow$ (1) Par la contraposé, si $P$ et $P'$ ont un diviseur irréductible commun $Q$ dans $k[T]$, celui-ci est \textit{a fortiori} un diviseur commun (non constant...) de $P$ et $P'$ dans $\overline{k}[T]$.  (2) $\Rightarrow$ (3) Par la contraposée, si  $|Z_{\overline{k}}(P)|<deg(P)$ cela signifie que $P$ a au moins une racine double $\alpha$ dans $\overline{k}$ \textit{i.e.} $(T-\alpha)^2| P$ dans $\overline{k}[T]$. En écrivant $P=(T-\alpha)^2Q$ dans $\overline{k}[T]$ et en dérivant on obtient $P'=2(T-\alpha)Q+tT-\alpha)^2Q'$ donc $(T-\alpha)| P'$ dans $\overline{k}[T]$. (3) $\Rightarrow$ (2) Par la contraposée,  si $P,P'$ ont un diviseur irréductible commun $T-\alpha$ dans $\overline{k}[T]$, on peut écrire 
  $P=(T-\alpha)Q$ dans $\overline{k}[T]$ donc $P'=(T-\alpha)Q'+Q$. Comme $(T-\alpha)|P'$ on a forcément $(T-\alpha)|Q$ donc $(T-\alpha)^2| P$ donc $|Z_{\overline{k}}(P)|\leq deg(P)-1$.
    (3) $\Leftrightarrow$ (4) En écrivant $P=a\prod_{1\leq i\leq n}(T-\alpha_i)^{\nu_i}$  dans $\overline{k}[T]$ avec $0\not=a\in k$ et  $\alpha_1,\dots, \alpha_n\in\overline{k}$ deux à deux distincts, le Lemme des restes chinois nous donne un isomorphisme canonique de  $\overline{k}$-algèbres
  $$\overline{k}[T]/P\tilde{\rightarrow} \prod_{1\leq i\leq n}\overline{k}[T]/(T-\alpha_i)^{\nu_i}.$$
  En particulier $\overline{k}[T]/P$ est réduit si et seulement si $\overline{k}[T]/(T-\alpha_i)^{\nu_i}$ est réduit, $i=1,\dots, n$ si et seulement si $\nu_1=\dots=\nu_n=1$.  \end{proof}
  
   On dit qu'un polynôme  $P\in k[T]$  qui vérifie les propriétés équivalentes du Lemme \ref{Separable1} est \textit{séparable}\index{Séparable (Polynôme)}. \\
  
  \textbf{Exemple.} Si $k$ est de caractéristique $0$ tout polynôme $P\in k[T]$ irréductible est séparable sur $k$. Ce n'est plus vrai si $k$ est de caractéristique $p>0$: le polynôme $P=T^p-X\in \F_p(X)[T]$ est irréductible sur $\F_p(X)$ mais pas séparable.
  
    \subsubsection{}\label{Separable2} D'après \ref{Separable1} on a 
  $$\begin{tabular}[t]{c|c|c}
    $P$&  arbitraire&  séparable\\
    \hline
$|Z_{\overline{k}}(P)|$&$\leq deg(P)$&$=deg(P)$
    \end{tabular}$$
 Si $P\in k[T]$ est irréductible, puisque pour toute clôture algébrique $\overline{k}/k$, $|\SHom_{Alg/k}(k[T]/P,\overline{k})|=|Z_{\overline{k}}(P)|$ et $deg(P)=[k[T]/P:k]$ on peut réécrire le tableau précédent sous la forme 
    $$\begin{tabular}[t]{c|c|c}
    $P$&  arbitraire&  séparable\\
    \hline
   $|\SHom_{Alg/k}(k[T]/P,\overline{k})|$&$\leq [k[T]/P:k]$&$=[k[T]/P:k]$
    \end{tabular}$$
  
    
        \textbf{Lemme.} \textit{Soit $k$ un corps algébriquement clos et $A$ une $k$-algèbre de dimension finie sur $k$. Le morphisme de $k$-algèbres 
        $$A\rightarrow k^{\hbox{\rm \small Hom}_{Alg_{/k}}(A,k)},\; a\rightarrow (\sigma(a))_{\sigma\in \hbox{\rm \small Hom}_{Alg_{/k}}(A,k)}$$
        est surjectif de noyau le nilradical de $A$. En particulier, $dim_k(A)\geq |\SHom_{Alg/k}(A,k)|$.}
        
        
        \begin{proof}Il s'agit de la combinaison d'un cas facile du Nullstellensatz et du Lemme des restes chinois. Plus précisément,  comme $A$ est de $k$-dimension finie - donc en particulier de type finie comme $k$-algèbre - et que $k$ est algébriquement clos, le  Nullstellensatz nous donne une bijection canonique 
        $$\SHom_{Alg/k}(A,k)\tilde{\rightarrow} spm(A),\; (\phi:A\rightarrow k)\rightarrow \ker(\phi)$$
        (d'inverse $\frak{m} \rightarrow (A\rightarrow A/\frak{m})=k$). On a vu qu'il résultait aussi du Nullstellensatz
        que le radical de Jacobson et le nilradical de $A$ coincident, d'o\`u l'assertion sur le noyau.  La surjectivité résultera du Lemme des restes chinois si l'on sait que $A$ n'a qu'un nombre fini d'idéaux maximaux. Mais  en notant $n:=dim_kA$, $A$ a au plus $n$ idéaux maximaux deux à deux distincts. En effet, si on avait $n+1$ idéaux maximaux  $\frak{m}_1,\dots, \frak{m}_{n+1}$ deux à deux distincts, le Lemme des restes chinois nous donnerait un morphisme surjectif de $k$-algèbres (donc de $k$-espaces vectoriels)
        $$A\twoheadrightarrow A/\frak{m}_1\times \cdots\times A/\frak{m}_{n+1}=k^{n+1}.$$
        \end{proof}

   Le corollaire suivant généralise la première colonne du tableau à une extension finie arbitraire.\\
    
\textbf{Corollaire.} \textit{Si $K/k$ est une  extension finie  et $\overline{k}/k$ une  clôture algébrique, $|\SHom_{Alg/k}(K,\overline{k})|$ est indépendant de $\overline{k}/k$ et $\leq [K:k]$}.

\begin{proof} L'indépendance de la clôture algébrique vient du fait que si $\overline{k}'/k$ est une autre clôture algébrique tout $k$-isomorphisme $\phi:\overline{k}\tilde{\rightarrow}\overline{k}'$  induit une bijection $\phi\circ-:\SHom_{Alg/k}(K,\overline{k})\tilde{\rightarrow}\SHom_{Alg/k}(K,\overline{k})$. Par la propriété universelle du produit tensoriel de $k$-algèbres on a une bijection canonique 
$$\SHom_{Alg/k}(K,\overline{k})\tilde{\rightarrow} \SHom_{Alg/\overline{k}}(\overline{k} \otimes_kK,\overline{k}) $$
Et comme $[K:k]=dim_{\overline{k}}\overline{k} \otimes_kK$, l'assertion résulte immédiatement du lemme ci-dessous.\end{proof} 
 \subsubsection{}\label{Separable3}Soit $K/k$ une extension de corps. On dit que $x\in K$ est \textit{séparable sur $k$}\index{Séparable (Elément)} si le polynôme minimal $P_x\in k[T]$ de $x$ sur $k$ est séparable. Le corollaire suivant décrit qu'elle est la bonne généralisation - en termes d'extensions de corps - de la notion de polynômes séparables.\\
 
     \textbf{Corollaire.} \textit{Soit $K/k$ une extension algébrique. Les propriétés suivantes sont équivalentes.
    \begin{enumerate}
    \item Tout élément de $K$ est séparable sur $k$;
    \item Pour toute clôture algébrique $\overline{k}/k$, $\overline{k}\otimes_kK$ est une $\overline{k}$-algèbre réduite.
    \end{enumerate}
     Si, de plus, $K/k$ est finie, ces propriétés sont aussi équivalentes à
        \begin{enumerate} 
        \setcounter{enumi}{2}
        \item Pour toute clôture algébrique $\overline{k}/k$, $|\SHom_{Alg/k}(K,\overline{k})|=[K:k]$.
    \end{enumerate}}
 \begin{proof} Montrons d'abord   (1) $\Rightarrow$  (3) $\Leftrightarrow$ (2) lorsque $K/k$ est finie. On a déjà observé que $\SHom_{Alg/k}(K,\overline{k})\tilde{\rightarrow} \SHom_{Alg/\overline{k}}(\overline{k}\otimes_kK,\overline{k})$ et que $[K:k]=dim_{\overline{k}}(\overline{k}\otimes_kK)$ donc (2) $\Leftrightarrow$ (3)  résulte du Lemme \ref{Separable2}.  Pour (1) $\Rightarrow$ (3) rappelons que    si $k\subset K'\subset K$ est une sous-$k$-extension, le morphisme de restriction 
 $$ \SHom_{Alg/k}( K,\overline{k})\rightarrow \SHom_{Alg/k}( K',\overline{k}),\; \phi\rightarrow \phi|_{K'}$$
 est surjectif et la fibre au-dessus de $\phi:K'\rightarrow \overline{k}$ est l'ensemble $\SHom_{Alg/K',\phi}( K,\overline{k})$ des $k$-plongements avec $\overline{k}$ muni de la structure de $K'$-algèbre donnée par $\phi:K'\rightarrow \overline{k}$. Donc 
 $$(*)\;\;| \SHom_{Alg/k}( K,\overline{k})|=\sum_{\phi\in\hbox{\rm \small Hom}_{Alg/k}( K',\overline{k}) }|\SHom_{Alg/K',\phi}( K,\overline{k})|.$$
 Cette observation permet de faire un raisonnement par récurrence sur le nombre minimal $n$ de générateurs de $K/k$ comme $k$-extension. Si $n=1$, $K=k(x)=k[T]/P_x$ et on a vu que dans ce cas
 $$|\SHom_{Alg/k}(K,\overline{k})|=|Z_{\overline{k}}(P_x)|.$$
 Or   comme $P_x\in k[T]$ est séparable,  $|Z_{\overline{k}}(P_x)|=deg(P_x)(=[k(x):k])$. Si le nombre minimal de générateurs de $K/k$ comme $k$-extension est $n$, on peut écrire $K=k(x_1,\dots, x_n)=k(x_1,\dots, x_{n-1})(x_n)$. Puisque $K'=k(x_1,\dots, x_{n-1})/k$ est engendrée par $\leq n-1$ générateurs comme $k$-extension et vérifie (1), l'hypothèse de récurrence montre que $|\SHom_{Alg/k}(K',\overline{k})|=[K':K]$ alors que par le cas $n=1$, pour tout $k$-plongement $\phi:K'\rightarrow\overline{k}$, on a $|\SHom_{Alg/K',\phi}( K,\overline{k})|=[K:K']$. De $(*)$ on déduit
 $$| \SHom_{Alg/k}( K,\overline{k})|=\sum_{\phi\in\hbox{\rm \small Hom}_{Alg/k}( K',\overline{k}) }[K:K']=[K':k][K:K']=[K:k].$$
 Pour  (2) $\Rightarrow$ (1), on n'a pas besoin de supposer $K/k$ finie. En effet, observons que si $k\subset K'\subset K$ est une sous-$k$-extension, le morphisme de $\overline{k}$-algèbres $\overline{k}\otimes_kK'\rightarrow \overline{k}\otimes_kK$ est injectif. En effet, il suffit pour cela de choisir une $k$-base   $e_i$, $i\in I$ de $K$ telle que $e_i$, $i\in I'$ soit une $k$-base de $K'$ pour un certain sous-ensemble $I'\subset I$ et d'observer que $1\otimes e_i$, $i\in I$ est encore une $\overline{k}$-base de $\overline{k}\otimes K$ (le produit tensoriel commute aux sommes directes). Or, d'après \ref{Separable2}, $x\in K$ est séparable sur $k$ si et seulement si $\overline{k}\otimes_kk(x)$ est réduite.  Comme on vient de voir que $\overline{k}\otimes_kk(x)$  est une sous-$\overline{k}$-algèbre de, cela montre  (2) $\Rightarrow$ (1). Enfin, on sait déjà que (1) $\Rightarrow$ (2) lorsque $K/k$ est finie. Pour le cas général, tout élément  de $\overline{k}\otimes_kK$ s'écrit comme combinaison $k$-linéaire finie d'éléments de la forme $\lambda\otimes x$, $\lambda\in \overline{k}$, $x\in K$ donc il est contenu dans une sous-$\overline{k}$-algèbre de la forme $\overline{k}\otimes_kk(x_1,\dots, x_n)$, o\`u $x_1,\dots, x_n\in K$. Comme les $x_1,\dots, x_n$ sont algébriques sur $k$,  $[k(x_1,\dots, x_n):k]$ est fini et $k(x_1,\dots, x_n)/k$ vérifie  (1) donc l'assertion résulte  de (1) $\Rightarrow$ (2) dans le cas o\`u $K/k$ est finie. \end{proof}

 On dit qu'une extension algébrique $K/k$ qui vérifie les propriétés équivalentes du Corollaire ci-dessus est \textit{séparable}\index{Séparable (Extension de corps)}. L'équivalence (1) $\Rightarrow$ (3) lorsque $K/k$ est finie montre en particulier que $x\in K$ est séparable sur $k$ si et seulement si $k(x)=k[T]/P_x/k$ est une extension séparable puisque 
$$| \SHom_{Alg/k}( k[T]/P_x,\overline{k})|=|Z_{\overline{k}}(P_x)|\leq deg(P_x)=[k(x):k]$$
    avec égalité si et seulement si $P_x\in k[T]$ est séparable. \\
    
    \textbf{Exemple.} Si $k$ est de caractéristique $0$, toute extension algébrique de $k$ est séparable (par la caractérisation (1) de \ref{Separable3}). L'extension $\F_p(X)[T]/T^p-X/\F_p(X)$ n'est pas séparable.
    
     \subsubsection{}\label{Separable3}\textbf{Corollaire.} \textit{Soit $K_3/K_2$ et $K_2/K_1$ des extensions algébriques. Alors $K_3/K_1$ est séparable si et seulement si $K_3/K_2$ et $K_2/K_1$ sont séparables. }
     
     \begin{proof} Si $K_3/K_1$ est séparable, la caractérisation (1) de la séparabilité montre immédiatement que $K_2/K_1$ est séparable et, puisque le polynôme minimal $P_{x,K_2}$ d'un élément $x\in K_3$ sur $K_2$ divise dans $K_2[T]$ le polynôme minimal  $P_{x,K_1}$  de $x$ sur $K_1$, que $K_3/K_2$ est aussi séparable. Réciproquement,  si $K_3/K_1$ est finie, $(*)$ dans la preuve du Corollaire de \ref{Separable2} montre que 
      $$| \SHom_{Alg/K_1}( K_3,\overline{k})|=\sum_{\phi\in\hbox{\rm \small Hom}_{Alg/K_1}( K_2,\overline{k}) }|\SHom_{Alg/K_2,\phi}( K_3,\overline{k})|=[K_3:K_2][K_2:K_1]=[K_3:K_1].$$
      Si $K_3/K_1$ n'est pas finie, soit $x\in K_3$. Par hypothèse son polynôme minimal sur $K_2$ s'écrit $P_{x,K_2}=T^n+a_{n-1}T^{n-1}+\cdots+a_0\in K_2[T]$ et est séparable donc $x$ est séparable sur $K_2'=K_1(a_{n-1},\dots, a_0)$ (puisque son  polynôme minimal sur $K_2'$ est encore $ P_{x,K_2}$) \textit{i.e.} $K_2'(x)/K_2'$ est séparable; $K_2'(x)/K_2'$ et $K_2'/K_1$ sont donc finies (car algébriques de type fini) et séparables donc $K_2'(x)/K_1$ est séparable par transitivité.     \end{proof}
     
      \subsubsection{}\label{Separable4}\textbf{Corollaire.} (Elément primitif) \textit{Toute extension $K/k$ séparable finie est monogène \textit{i.e.} de la forme $K=k(x)/k$ pour un certain $x\in K$.}
      
      \begin{proof} Supposons d'abord que $k$ est \textit{infini}. Pour tout $x\in K$, on a vu que  l'application de restriction $$-|_{k(x)}: \SHom_{Alg/k}( K,\overline{k})\rightarrow  \SHom_{Alg/k}( k(x),\overline{k})$$
      était toujours surjective. De plus, puisque $K/k$ est séparable finie, on a $ |\SHom_{Alg/k}( K,\overline{k})|=[K:k]$ et $|\SHom_{Alg/k}( k(x),\overline{k})|=[k(x):k] $. Donc 
      $$K=k(x) \Leftrightarrow [K:k]=[k(x):k] \Leftrightarrow  -|_{k(x)}: \SHom_{Alg/k}( K,\overline{k})\rightarrow  \SHom_{Alg/k}( k(x),\overline{k})\; \hbox{\rm est injective}$$
      Mais comme l'évaluation en $x$,  $ \SHom_{Alg/k}( k(x),\overline{k})\rightarrow  \overline{k}$, $\phi\rightarrow \phi(x)$ est injective,  $-|_{k(x)}: \SHom_{Alg/k}( K,\overline{k})\rightarrow  \SHom_{Alg/k}( k(x),\overline{k})$ est injective si et seulement si l'évaluation en $x$ $ \SHom_{Alg/k}( K,\overline{k})\rightarrow \overline{k}$, $\phi\rightarrow \phi(x)$ est injective autrement dit, si et seulement si $x\notin \ker(\phi_1-\phi_2)$, $\phi_1\not=\phi_2\in \SHom_{Alg/k}( K,\overline{k})$. Comme $ \SHom_{Alg/k}( K,\overline{k})$ est fini et que les $ \ker(\phi_1-\phi_2)\subsetneq K$ sont des sous-$k$-espace vectoriels stricts, la conclusion résulte du Lemme 1 ci-dessous. \\
      
       Supposons maintenant que $k$  est fini. Le corps $K$ est donc également fini (de cardinal $|k|^{[K:k]}$) et, d'après le Lemme 2 ci-dessous, le groupe multiplicatif $K^\times $ est cyclique. Tout générateur $x$ du groupe $K^\times $  vérifie $K=k(x)$.  \end{proof}
    
    \textbf{Lemme 1.} \textit{Soit $k$ un corps infini et $V$ un $k$-espace vectoriel de dimension finie. Si $W_1,\dots, W_r\subsetneq V$ sont des sous-$k$-espaces vectoriels stricts, $V\setminus \cup_{1\leq i\leq r}V_i\not=\emptyset$.}
    
    \begin{proof}On procède par récurrence sur la $k$-dimension $n$ de $V$. Si $n=1$, c'est immédiat. Si $n\geq 1$, il suffit de montrer l'énoncé dans le cas o\`u les $W_i$, $i=1,\dots, r$ sont des hyperplans deux à deux distincts. Comme $W_1\cap W_i\subsetneq W_1$, par hypothèse de récurrence il existe $w\in W_1\setminus \cup_{2\leq i\leq r}W_1\cap W_i$. Fixons également $v\in V\setminus W_1$ et notons $D:=v+kw\subset V$. Puisque $v\in V\setminus W_1$, $D\cap W_1=\emptyset $ et, pour $i=1,\dots, r$, $ D\cap W_i $ contient au plus un élément  car si $v+aw,v+bw\in W_i$, $(a-b)w\in W_i$ donc, puisque $w\in V\setminus W_i$, $a=b$. Cela montre que $D\cap \cup_{1\leq i\leq r}W_i$ est fini et on conclut en utilisant que $D$ est infini puisque $k$ l'est.    \end{proof}
    
        \textbf{Lemme 2.} \textit{Soit $k$ un corps. Tout sous-groupe fini $G$ du groupe multiplicatif $k^\times$ est cyclique.}
    
    \begin{proof} Notons $n=|G|$ et  $m$ le ppcm des ordres des éléments de $G$. Comme $G$ est un groupe abélien fini, il contient un élément d'ordre $m$ (penser au théorème de structure... ou le vérifier à la main). Il suffit donc de montrer que $m=n$. Mais par définition de $m$, 
  $G\subset Z_k(T^m-1)$. Or $|Z_k(T^m-1)|\leq deg(T^m-1)=m$. Donc $n=|G|\leq |Z_k(T^m-1)|=m\leq n$ montre bien que $m=n$. \end{proof}
    
 \textbf{Exemples.}
 \begin{itemize}
 \item Une extension finie non séparable n'est en général  pas monogène. Par exemple $K:=\F_p(X,Y)/k:=\F_p(X^p,Y^p))$ est finie, de $\F_p(X^p,Y^p)$-base $X^iY^j$, $0\leq i,j\leq p-1$ mais elle n'est pas monogène car tout $x\in K$ vérifie $x^p\in k$ donc son polynôme minimal sur $k$ divise $T^p-x^p$ et est donc de degré $ \leq p$.
 \item La  preuve de \ref{Separable4} n'est pas constructive. On verra plus loin comment en trouver. Par exemple, on verra que $[\Q(^3\sqrt{5}+j):\Q]=6$ et donc que $\Q(^2\sqrt{5},j)=\Q(^3\sqrt{5}+j)$.
 \end{itemize}
 
 
   \subsection{Corps finis}
   
  Soit $k$ un corps et $c_k:\Z\rightarrow k$ le morphisme caractéristique de $k$ et $p$ la caractéristique de $k$. Comme $k$ est intègre, 
  \begin{itemize}
 \item soit $p=0$, auquel cas, par propriété universelle de l'anneau des fractions, $c_k:\Z\rightarrow k$ se factorise en un morphisme de corps $c_k:\Q\hookrightarrow k$;
 \item soit $p=0$, auquel cas, par propriété universelle du quotient, $c_k:\Z\rightarrow k$ se factorise en un morphisme de corps $c_k:\F_p\hookrightarrow k$;
 \end{itemize} 
  Si $p=0$ (resp. $p>0$) on dit que $\Q$ (resp. $\F_p$) est le sous-corps premier de $k$ (c'est le plus petit sous-corps de $k$). \\
 
  En particulier, un corps fini $\F$   est nécessairement de caractéristique $p>0$ et c'est un $\F_p$-espace vectoriel de dimension finie. On doit donc avoir 
  $|\F|=p^{[\F:\F_p]}$. \\
 
   L'une des spécificités fondamentales des corps de caractéristique $p>0$ est l'existence du Frobenius.\\
 
 \subsubsection{} \textbf{Lemme.} (Frobenius) \textit{Soit $k$ un corps de  caractéristique $p>0$. L'application $F_k:k\rightarrow k$, $x\rightarrow x^p$ est un endomorphisme de $\F_p$-algèbre.}
 \begin{proof}La seule chose à vérifier est l'additivité. Cela résulte de la formule du binôme de Newton. Pour tout $a,b\in k$
 $$F_k(a+b)=(a+b)^p=\sum_{0\leq i\leq p} \binom{p}{i}a^ib^{p-i}=a^p+b^p$$
 puisque $p|\binom{p}{i} $, $i=1,\dots, p-1$. \end{proof}
 
 \textbf{Remarque.} On prendra garde que si le Frobenius est toujours injectif (puisque c'est un morphisme de corps), il n'est pas toujours surjectif. Par exemple, m'image du Frobenius sur $\F_p(T)$ est le sous-corps strict $\F_p(T^p)\subsetneq \F_p(T)$. Le Frobenius est cependant toujours surjectif sur un corps $\F$ fini (injectif entre deux ensembles de mêmes cardinal) ou algébriquement clos (puisque les polynômes $T^p-x$ ont tous une racine dans $\F$). La surjectivité du Frobenius est lié à la notion de corps parfait.\\
 
 \subsubsection{}\label{CF1} \textbf{Lemme.} \textit{Soit $k$ un corps de  caractéristique $p>0$. Pour tout $r\in \Z_{\geq 1}$, le sous-ensemble
 $\F_p\subset Z_k(T^{p^r}-T)\subset k$ est un sous-corps fini, de cardinal $p^s$ avec $s| r$.}
 \begin{proof} Puisque  $\F:=Z_k(T^{p^r}-T)$ est l'égalisateur de deux endomorphismes de $\F_p$-algèbres: $F_k^{p^r}, Id:k\rightarrow k$, c'est une sous-$\F_p$-algèbre (donc un sous-corps) de $k$. Explicitement, $\F=\ker(F_k^{p^r}- Id)\subset k$ est un sous-$\F_p$-espace vectoriel, $1\in \F$ et pour tout $a, 0\not= b\in \F$, $(ab^{-1})^{p^r}=a^{p^r}(b^{-1})^{p^r}=ab^{-1}$ donc $ab^{-1}\in \F$. Comme $|\F|<+\infty$, $\F$ est un $\F_p$-espace vectoriel de dimension finie - disons $s$ - donc de cardinal $p^s$.  Soit $x\in \F^\times$ un générateur de $\F^\times$. On a donc $\F\subset Z_k(T^{p^s}-T)$. Or $p^s=|\F|\leq |Z_k(T^{p^s}-T)|\leq p^s$ donc $\F=Z_k(T^{p^s}-T)$ et $T^{p^s}-T$ est totalement décomposé sur $\F$. Autrement dit, $\F$ est un corps de décomposition de $T^{p^s}-T$ sur $\F_p$. Comme $  T^{p^s}-T| T^{p^r}-T$ dans $\overline{\F}_p[T]\subset \overline{k}[T]$ donc dans $\F_p[T]$,  on a $$\F_p\subset \F= Z_{k}(T^{p^s}-T)=Z_{\overline{k}}(T^{p^s}-T)\subset Z_{\overline{k}}(T^{p^r}-T)=:\F_{\overline{k}}$$
 Comme $T^{p^r}-T\in \F_p[T]$ est séparable, $|\F_{\overline{k}}|=p^r$. On doit donc avoir 
 $$r=[\F_{\overline{k}}:\F_p]=[\F_{\overline{k}}:\F][\F:\F_p]=[\F_{\overline{k}}:\F]s$$
 donc $s|r$. \end{proof} 
 
  \subsubsection{}\textbf{Corollaire.} (Corps finis) \textit{Pour tout premier $p>0$ et tout $r\in \Z_{\geq 1}$, il existe un corps fini $\F_{p^r}$ à $\F_{p^r}$-éléments, unique à isomorphisme (non-unique près); c'est le corps de décomposition de $T^{p^r}-T\in \F_p[T]$ sur $\F_p$. De plus, 
  \begin{enumerate}
  \item pour tout $r,s\in \Z_{\geq 1}$, $\F_{p^s}\subset \F_{p^r}$ si et seulement si $s|r$;
  \item toute extension algébrique $K/\F_p$ est réunion de ses sous-corps finis. En particulier, $\overline{\F}_p=\cup_{r\geq 1}\overline{\F}_{p^r}$.
  \end{enumerate} } 
  
  \begin{proof}Soit $\overline{\F}_p/\F_p$ une clôture algébrique et $\F_p\subset \F_{p^r}:=\F_p(Z_{\overline{\F}_p}(T^{p^r}-T))\subset \overline{\F}_p$ le  corps de décomposition correspondant de $T?{p^r}-T $ sur $\F_p$. D'après \ref{CF1}, $Z_{\overline{\F}_p}(T^{p^r}-T)$ est une extension algébrique de $\F_p$ vérifiant tautologiquement $Z_{\overline{\F}_p}(T^{p^r}-T)=\F_p(Z_{\overline{\F}_p}(T^{p^r}-T))$;  donc $\F_{p^{r}}=Z_{\overline{\F}_p}(T^{p^r}-T)$. De plus, comme $ T^{p^r}-T\in \F_p[T]$ est séparable et totalement décomposé sur $\overline{\F}_p$, $|\F_{p^{r}}|=|Z_{\overline{\F}_p}(T^{p^r}-T)|=p^r$. Cela montre l'existence de $\F_{p^r}$. Pour l'uncité, soit $\F$ est un corps fini à $p^r$ éléments; c'est une extension finie donc algébrique de $\F_p$ donc on peut le plonger dans $\overline{\F}_p$. Comme $|\F^\times|=p^{r-1}$, tout $0\not= x\in \F$ vérifie $x^{p^r-1}=1$ donc $\F\subset Z_{\overline{\F}_p}(T^{p^r}-T)$. Par cardinalité, $\F=Z_{\overline{\F}_p}(T^{p^r}-T)$. La condition nécessaire de (1) résulte de $$r=[\F_{p^r}:\F_p]=[\F_{p^r}:\F_{p^s}][\F_{p^s}:\F_p]=[\F_{p^r}:\F_{p^s}]s.$$ Pour la condition suffisante, si $s|r$,  on a  $p^s-1|p^r-1$ car 
  $$p^r-1=(p^s)^{r/s}-1=(p^s-1)\sum_{0\leq i\leq r/s-1}p^{si}.$$
Or $0\not= x\in \F_{p^s}$ $\Rightarrow$ $x^{p^s-1}=1$ $\Rightarrow$ $x^{p^r-1}=(x^{p^s-1})^{(p^r-1)/(p^s-1)}=1$ $\Rightarrow$ $x\in \F_{p^r}$. (2) est immédiat.  \end{proof}
 
 
 
    \subsection{Polynôme caractéristique, trace et norme}
    \subsubsection{}Soit $A$ une $k$-algèbre de dimension finie. A tout $a\in K$ on peut associer l'automorphisme de $k$-espace vectoriel $L_a:A\rightarrow A$, $b\rightarrow ab$. On note 
 $$\chi_{A/k}(a):=det(L_a-TId|A)\in k[T]$$
 son \textit{polynôme caractéristique}\index{Polynôme caractéristique (Elément)},    
    $tr_{A/k}(a):=tr_k(L_a:A\rightarrow A)\in k $ sa \textit{trace}\index{Trace} et $N_{A/k}:=det(L_a)\in k$ son déterminant - appelé \textit{norme}\index{Norme (Elément)}. On dispose en particulier 
 d'une forme linéaire $tr_{A/k}:A\rightarrow k$ et d'un morphisme de groupes $N_{A/k}:A^\times\rightarrow k^\times$.\\
 
    \subsubsection{}\textbf{Lemme.} \textit{Soit $K/k$ une extension finie. Pour tout $x\in K$, on a $\chi_{K/k}(x)=P_x^{[K:k(x)]}$ o\`u $P_x\in k[T]$ est le  polynôme minimal de $x$ sur $k$. En particulier, $tr_{K/k}(x)=[K:k(x)]tr_{k(x)/k}(x)$ et $N_{K/k}(x)=N_{k(x)/k}(x)^{[K:k(x)]}$.}
    \begin{proof} Notons $n:=[K:k]$, $q_x:=[K:k(x)]$,  $n_x:=[k(x):k]$. Fixons une $k(x)$-base $y_1,\dots, y_{q_x}$ de $K$.  On a une décomposition $K=\oplus_{1\leq i\leq q_x}k(x)y_i$ en  $k$-espaces vectoriels $k(x)y_i\subset K$  $L_x$-stables et pour chaque $i=1,\dots, q_x$, la matrice de  $L_x:k(x)y_i\rightarrow k(x)y_i$ dans la $k$-base $x^jy_i$, $j=0,\dots, n_x-1$ est la matrice compagnon $C(P_x)$ de $P_x$. Donc  la matrice de  $L_x:K\rightarrow K$ dans la $k$-base $y_ix^j$, $1\leq i\leq q_x$, $0\leq j\leq n_x-1$  de $K$ est la matrice diagonale par blocs de taille $n\times n$ dont tous les blocs diagonaux valent $C(P_x)$.  On conclut en utilisant que le polynôme minimal et le polynôme caractéristique d'une  matrice compagnon $C(P)$ coïncident et sont égaux à $P$.   \end{proof}
    
\subsubsection{}\textbf{Lemme.} \textit{Soit $K_3/K_2$ et $K_2/K_1$  des extensions finies. Alors $tr_{K_2/K_1}\circ tr_{K_3/K_2}=tr_{K_3/K_1}$.}

\begin{proof}Notons $n_3:=[K_3:K_2]$, $ n_2:=[K_2:K_1]$. Soit $e_{3,i}$, $i=1,\dots, n_3$ une $K_2$-base de $K_3$ et $e_{2,i}$, $i=1,\dots, n_2$ une $K_1$-base de $K_2$. Soit $x\in K_3$. Notons $A_3:=(a_{3,i,j})_{1\leq i,j\leq n_3}\in M_{n_3}(K_2)$ la matrice du $K_2$-endomorphisme $L_x:K_3\rightarrow K_3$ dans $e_{3,i}$, $i=1,\dots, n_3$ et pour chaque $a_{3,i,j}$, notons $A_{2,i,j}:=(a_{2,i,j,r,s})_{1\leq r,s\leq n_2}\in M_{n_2}(K_1)$ la matrice du $K_1$-endomorphisme $L_{a_{3,i,j}}:K_2\rightarrow K_2$ dans $e_{2,i}$, $i=1,\dots, n_2$. Dans la $K_1$-base  $e_{3,i}e_{2,j}$, $1\leq i\leq n_3,1\leq j\leq n_2$   de $K_3$, la matrice du $K_1$-endomorphisme $L_x:K_3\rightarrow K_3$ est la matrice par blocs $(A_{2,i,j})_{1\leq i,j\leq n_3}\in M_{n_2n_3}(K_1)$. En particulier
$$tr_{K_2/K_1}(tr_{K_3/K_2}(x))= \sum_{1\leq i \leq n_3}tr_{K_2/K_1}(a_{3,i,i})=\sum_{1\leq i\leq n_3}\sum_{1\leq j\leq n_2}a_{2,i,i,j,j}=tr_{K_3/K_1}(x).$$
\end{proof}
    \textbf{Remarque.} On peut aussi montrer que   $ N_{K_2/K_1}\circ N_{K_3/K_2}=N_{K_3/K_1}$ mais c'est un peu plus délicat.\\
 
    
\subsubsection{}\textbf{Proposition} \textit{Une extension finie $K/k$ est séparable si et seulement si $tr_{K/k}:K\rightarrow k$ est non nulle (\textit{i.e.} surjective).}
    
    \begin{proof} En prenant des bases adaptées, on vérifie facilement que si $k\subset K'\subset K$ est une sous-extension, $tr_{K/k}=tr_{K'/k}\circ tr_{K/K'}$. En particulier, pour tout $x\in K$, $tr_{K/k}=tr_{k(x)/k}\circ tr_{K/k(x)}$. Si $K/k$ est séparable, on sait qu'il existe $x\in K$ tel que $K=k(x)$. Il suffit donc de montrer que $x\in K$ est séparable sur $k$ si et seulement si $tr_{k(x)/k}:k(x)\rightarrow k$ est surjective. De plus, la suite exacte courte de $k$-espaces vectoriels 
    $$0\rightarrow \ker(tr_{K/k})\rightarrow K\stackrel{tr_{K/k}}{\rightarrow}\hbox{\rm im}(tr_{K/k})\rightarrow 0$$
    reste exacte après $\overline{k}\otimes_k-$ 
    $$0\rightarrow \overline{k}\otimes_k\ker(tr_{K/k}) \rightarrow \overline{k}\otimes_kK\stackrel{Id\otimes tr_{K/k}}{\rightarrow}\overline{k}\otimes_k\hbox{\rm im}(tr_{K/k})\rightarrow 0$$
 Autrement dit, $ \overline{k}\otimes_k\ker(tr_{K/k}) =  \ker(Id\otimes tr_{K/k}tr_{K/k}) $, $\overline{k}\otimes_k\hbox{\rm im}(tr_{K/k})=\hbox{\rm im}(Id\otimes tr_{K/k}tr_{K/k})$. Il suffit donc de montrer que
  $x\in K$ est séparable si et seulement si $ Id\otimes tr_{K/k}tr_{K/k}: \overline{k}\otimes_kK\rightarrow \overline{k}$ est non nulle. P   en écrivant  $P_x=\prod_{1\leq i\leq r}(T-x_i)^{n_i}$  dans $\overline{k}[T]$ avec les $x_1,\dots, x_n\in\overline{k}$ deux à deux distincts, le lemme des restes Chinois nous donne un isomorphisme de $\overline{k}$-algèbres explicite
  $$ \overline{k}\otimes_kK= \overline{k}\otimes_kk[T]/P_x=  \overline{k}[T]/P_x\tilde{\rightarrow} \prod_{1\leq i\leq r}\overline{k}[T]/(T-x_i)^{n_i}$$
  et en prenant une $\overline{k}$-base adaptée à cette décomposition, on obtient $ tr_{\overline{k}\otimes_kK/\overline{k} }=\sum_{1\leq i\leq r} tr_{A_i/\overline{k} }$, o\`u $A_i:=\overline{k}[T]/(T-x_i)^{n_i}$, $i=1,\dots ,r$. 
  \begin{itemize}
  \item Si $x$ est séparable sur $k$, $n_1=\dots=n_r=1$ donc $A_i=\overline{k}$ donc t $ tr_{A_i/\overline{k} }= Id$, $i=1,\dots, n$ et $ tr_{\overline{k}\otimes_kK/\overline{k} }:\overline{k}^n\rightarrow \overline{k}$, $(a_1,\dots, a_n)\rightarrow a_1+\cdots+a_n$ est clairement surjective.
  \item Si $x$ n'est pas séparable sur $k$, $k$ est de caractéristique $p>0$ et $P_x=P(T^{p^s})$ avec $P\in k[T]$ séparable et $r\geq 1$ (\textit{cf.} \ref{Derivations}). En écrivant $P=\prod_{1\leq i\leq r}(T-\alpha_i)$ et $\alpha_i=x_i^{p^s}$, $i=1,\dots, r$ dans $\overline{k}$, on en déduit 
  $$P_x(T)=\prod_{1\leq i\leq r}(T^{p^s}-x_i^{p^s})=\prod_{1\leq i\leq r}(T -x_i)^{p^s},$$
  autrement dit $n_1=\dots=n_r=p^s$. Or tout élément $a\in A_i$ s'écrit sous la forme $a=a_0+\nu$ avec $a_0\in\overline{k}$ et $\nu\in A_i$ nilpotent. Donc $tr_{A_i/\overline{k}}(a)=tr_{A_i/\overline{k}}(a_0)+tr_{A_i/\overline{k}}(\nu)= p^sa_0=0$.
  \end{itemize}
  \end{proof}   
 
 

 
  

 \section{Correspondance de Galois} 
 \subsection{Extensions galoisiennes}
 \subsubsection{}\label{Galois1}\textbf{Lemme} \textit{Soit $K/k$ une extension finie. Les propriétés suivantes sont équivalentes.
 \begin{enumerate}
 \item $K/k$ est normale et séparable;
  \item $K/k$ est le corps de décomposition d'un polynôme séparable sur $k$;
  \item $|\SAut(K/k)|=[K:k]$;
 \item $k=K^{\hbox{\rm \footnotesize Aut}(K/k)}$;
 \item Pour tout $x\in K$,  son polynôme minimal   $P_x\in k[T]$  sur $k$ se décompose comme $P_x=\prod_{y\in \hbox{\rm \footnotesize{Aut}}(K/k)\cdot x}(T-y)$ dans $K[T]$; 
 \end{enumerate}}
\begin{proof} On se fixe une clôture algébrique $\overline{k}/k$. Pour $x\in K$ on note   $P_x\in k[T]$  son polynôme minimal    sur $k$. On va montrer que (1) $\Leftrightarrow$ (i), $i=2,3,4$ et (4) $\Leftrightarrow$ (5)\\
\begin{itemize}[leftmargin=* ,parsep=0cm,itemsep=0cm,topsep=0cm] 
\item (1) $\Rightarrow$ (2) résulte de l'élément primitif \ref{Separable4} et de la définition d'une extension normale. (2) $\Rightarrow$ (1) Si $K$ est le corps de décomposition d'un polynôme $P\in k[T]$ séparable, $K/k$ est normale. Mais par \ref{Separable3}, $K/k$ est aussi séparable puisque si on note $Z_K(P)=\lbrace x_1,\dots, x_n\rbrace$, $K=k(x_1,\dots, x_n)/k$ $K=k(x_1,\dots,x_n)/k$ et chacune des extensions $k(x_1,\dots,x_i)/k(x_1,\dots, x_{i-1})$ est séparable (monogène engendrée par un élément séparable \textit{cf.} caractérisation (3) de \ref{Separable3}). \\
\item (1) $\Leftrightarrow$ (3) Pour  toute clôture algébrique $k\subset K\subset \overline{k}$ on a toujours $\SAut(K/k)\hookrightarrow \SHom_{Alg/k}(K,\overline{k})$ et $|\SHom_{Alg/k}(K,\overline{k})|\leq [K:k]$. Donc $|\SAut(K/k)|=[K:k]$ si et seulement si $\SAut(K/k)\tilde{\rightarrow}\SHom_{Alg/k}(K,\overline{k})$ (\textit{i.e.} $K/k$ est normale) et $ |\SHom_{Alg/k}(K,\overline{k})|= |K:k]$ (\textit{i.e.} $K/k$ est séparable).\\
\item  (1) $\Rightarrow$ (5)  Comme $K/k$ est séparable, pour tout $x\in K$,   son polynôme minimal   $P_x\in k[T]$  sur $k$ se décompose comme $P_x= \prod_{y\in Z_{\overline{k}}(P_x)}(T-y)$ dans $\overline{k}[T]$. Mais comme $K/k$ est normale, 
 $$Z_{\overline{k}}(P_x)=\lbrace \sigma(x)\; |\; \sigma\in \SHom_{Alg/k}(K,\overline{k}) \rbrace = \SAut(K/k)\cdot x.$$
 (5) $\Rightarrow$ (1)  Pour tout $x\in K$,  la décomposition   $P_x=\prod_{y\in \hbox{\rm \footnotesize{Aut}}(K/k)\cdot x}(T-y)$ dans $K[T]$ montre que $P_x$ est séparable (donc que $K/k$ est séparable) et est totalement décomposé sur $K$ (donc que $K/k$ est normale).
\item  (5) $\Rightarrow$ (4) Si $x\in K\setminus k$, son polynôme minimal   $P_x\in k[T]$  sur $k$ est de degré $\geq 2$. Or, par hypothèse $P_x=\prod_{y\in \hbox{\rm \footnotesize{Aut}}(K/k)\cdot x}(T-y)$ donc il existe $\sigma\in \SAut(K/k)$ tel que $\sigma(x)\not= x$. Autrement dit $x\in K\setminus K^{\hbox{\rm \footnotesize Aut}(K/k)}$. (4) $\Rightarrow$ (5) Pour tout $x\in K$ notons  $\widetilde{P}_x:=\prod_{y\in \hbox{\rm \footnotesize{Aut}}(K/k)\cdot x}(T-y)\in K[T]$. Dans $\overline{k}[T]$ on a 
 $$\prod_{y\in Z_{\overline{k}}(P_x)}(T-y)=\prod_{y\in \hbox{\rm \footnotesize{Aut}}(\overline{k}/k)\cdot x}(T-y)|P_x.$$
 Et comme pour tout $\sigma\in \SAut(K/k)$ il existe $\widetilde{\sigma}\in \SAut(\overline{k}/k)$ tel que le diagramme suivant commute (\ref{CloAlgPlonge}) $$\xymatrix{\overline{k}\ar[r]^{\widetilde{\sigma}}_\simeq & \overline{k}\\
 K\ar[r]^\sigma_\simeq\ar@{_{(}->}[u]&K,\ar@{_{(}->}[u]}$$
 on a $ \SAut(K/k)\cdot x\subset \SAut(\overline{k}/k)\cdot x$ donc $\widetilde{P}_x|P_x$( dans $\overline{k}[T]$ donc) dans $K[T]$. Par ailleurs,  $\widetilde{P}_x(x)=0$ et pour tout $\sigma\in \SAut(K/k)$ $\sigma \widetilde{P}_x=\widetilde{P}_x$ donc $\widetilde{P}_x\in K^{\hbox{\rm \footnotesize Aut}(K/k)} [T]=k[T]$. Cela impose $P_x|\widetilde{P}_x$ donc $P_x=\widetilde{P}_x$.  
 \end{itemize}
  \end{proof}
  
  
   On dit qu'une extension finie $K/k$ qui vérifie les propriétés équivalentes du Lemme \ref{Galois1} est \textit{galoisienne}\index{Galoisienne (Extension de corps)}. Lorsque $K/k$ est galoisienne, on note $\SGal(K/k):=\SAut(K/k)$ et on dit que c'est le \textit{groupe de Galois}\index{Galois (Groupe de)} de $K/k$. \\
  
   \textbf{Exemple.}  On a vu que $\Q(^3\sqrt{5},j)/\Q$ était galoisienne et que $\SGal(\Q(^3\sqrt{5},j)/\Q)=\mathcal{S}_3$.
  
  \subsection{}\textbf{Exemples classiques.} Si $k$ est corps  on note $\mu_n(k):=Z_k(T^n-1)\subset k^\times$;   c'est un sous-groupe fini donc cyclique (Lemme 2 de la preuve de \ref{Separable4}) de $k^\times$.
 \subsubsection{}\textbf{Corps finis.} Pour tout $r\in \Z_{\geq 1}$, $\F_{p^r}/\F_p$ est galoisienne puisque c'est le corps de décomposition du polynôme séparable $T^{p^r}-T\in \F_p[T]$ sur $\F_p$. En outre, $F:\F_{p^r}\rightarrow \F_{p^r}$, $x\rightarrow x^p\in \SGal(\F_{p^r}/\F_p)$. De plus, si $x\in \F_{p^r}^\times$ est un générateur, les éléments $x,F(x),\dots, F^{r-1}(x)$ sont tous distincts donc $F$ est d'ordre $\geq r$. Mais par la caractérisation (3) d'une extension galoisienne $|\SGal(\F_{p^r}/\F_p)|=[ \F_{p^r}:\F_p]=r$. Donc $F:\F_{p^r}\rightarrow \F_{p^r}$ est d'ordre exactement $r$ et $$\SGal(\F_{p^r}/\F_p)=\langle F\rangle\simeq \Z/r.$$
  
 \subsubsection{}\label{GaloisCyc}\textbf{Extensions cyclotomiques.} Pour tout corps $k$ et $n\in \Z_{\geq 1}$, la $n$-ième \textit{extension cyclotomique}\index{Cyclotomique (Extension de corps)} $k_n/k$ de $k$  est par définition le corps de décomposition  de $T^n-1\in k[T]$ sur $k$. Si on note $p$ la caractéristique de $k$ et si $p>0$ on a, en écrivant $n=p^rm$,  $p\not| m$
 $$T^n-1=(T^m-1)^{p^r}$$
 donc $k_n=k_m$. On peut par conséquent supposer que $p\not| n$ donc que $T^n-1\in k[T]$ est séparable et $k_n/k$ galoisienne. Fixons une clôture algébrique $\overline{k}/k$.  Puisque $T^n-1$ est séparable $\mu_n:=\mu_n(\overline{k})=Z_{\overline{k}}(T^n-1) \subset \overline{k}^\times$ est (cyclique) d'ordre $n$. De plus, pour tout $\sigma\in \SGal(k_n/k)$ et $\zeta, \zeta'\in \mu_n$ on a $\sigma(\zeta\zeta')=\sigma(\zeta)\sigma(\zeta')$ puisque $\sigma$ est un morphisme de corps. Donc la restriction à $\mu_n\subset k_n$ induit un morphisme de groupes $$\chi_k:\SGal(k_n/k)\rightarrow \SAut_{Grp}(\mu_n) $$
 qui est injectif puisque $k_n=k(\mu_n)$. Le choix d'un générateur $\zeta_n$ de $\mu_n$ définit un isomorphisme de groupes  (non canonique) $\Z/n\tilde{\rightarrow} \mu_n$ et donc un isomorphisme de groupes $\SAut_{Grp}(\mu_n)\tilde{\rightarrow} \SAut_{Grp}(\Z/n)=\Z/n^\times$. Modulo ces isomorphismes, on a   $$\sigma(\zeta_n)=\zeta_n^{\chi_k(\sigma)},\; \sigma\in \SGal(k_n/k).$$
 L'image de  $\SGal(k_n/k)$ dans $(\Z/n)^\times$ dépend de l'arithmétique du corps. Voici un exemple.\\
 
 \textbf{Proposition.} \textit{$\chi_\Q:\SGal(\Q_n/Q)\rightarrow \Z/n^\times$ est un isomorphisme de groupes.}
 
 \begin{proof} On sait déjà que $\chi_\Q:\SGal(\Q_n/Q)\rightarrow (\Z/n)^\times$  est injectif. Il suffit donc de montrer que $[\Q_n:\Q](=|\SGal(\Q_n/Q)|)= | \Z/n^\times|=:\varphi(n)$. Mais $\Q_n=\Q(\zeta_n)/\Q$. Il suffit donc de montrer que le polynôme minimal $P_{\zeta_n}\in \Q[T]$ de $\zeta_n$ sur $\Q$ est irréductible. Pour cela, notons $\frak{u}_n\subset \mu_n$ le sous-ensemble des générateurs de $\mu_n$ (les racines primitives $n$-ièmes de $1$). Par construction $\frak{u}_n$ est  $\SGal(\Q_n/\Q)$-stable donc le polynôme 
 $\Phi_n=\prod_{u\in \frak{u}_n}(T-u)$ est dans $\Q[T]$ (par la caractérisation (4) de \ref{Galois1}) et il a $\zeta_n$ pour racine. Donc $P_{\zeta_n}|\Phi_n$.  Comme $deg(\Phi_n)=|\frak{u}_n|=|\Z/n^\times|$, la conclusion résulte du Lemme ci-dessous. \end{proof}
  \textbf{Lemme.} \textit{$\Phi_n\in \Q[T]$ est irréductible sur $\Q[T]$.}
  \begin{proof} On procède en plusieurs étapes.
  \begin{enumerate}[leftmargin=* ,parsep=0cm,itemsep=0cm,topsep=0cm]
  \item $\Phi_n\in \Z[T]$. En effet $T^n-1=\prod_{d|n}\Phi_d$ dans $\Q[T]$. Comme $\Z$ est factoriel, on a en particulier  
 $1=C_\Z(T^n-1)=\prod_{d|n}C_\Z(\Phi_d)$. Mais comme   $\Phi_d$  est unitaire , on a $C_\Z(\Phi_d)=\frac{1}{a_d} $ pour un certain $a_d\in \Z_{\geq 1}$, $d|n$. Cela impose $C_\Z(\Phi_d)=1$ \textit{i.e.} $\Phi_d\in \Z[T]$, $d|n$.
 \item Soit $\zeta\in \frak{u}_n$ et notons $P$ le polynôme minimal de $\zeta$ sur $\Q$. On veut montrer que $P=\Phi_n$. Comme les éléments de $\frak{u}_n$ sont tous de la forme $\zeta^m$ pour un certain $m\in \Z_{\geq 1}$, $gcd(m,n)=1$,  il suffit de prouver que si $p$ est un nombre premier $\not|n$, $\zeta^p$ est aussi une racine de $P$. Sinon, notons $Q$ le polynôme  minimal de $\zeta^p$ sur $\Q$. Comme $P\not= Q$ et $P,Q|\Phi_n$ dans $\Q[T]$, $PQ|\Phi_n$ dans $\Q[T]$. On peut donc écrire $\Phi_n=PQR$ dans $\Q[T]$. Comme $\Phi_n\in \Z[T]$ et $P,Q,R$ sont unitaires, le même argument de contenu qu'en (1) montre que $P,Q,R\in \Z[T]$.
 \item Puisque $Q(\zeta^p)=0$, $P|Q(T^p)$  dans $\Q[T]$ donc - toujours par l'argument de contenu -  dans $\Z[T]$. Notons $F\rightarrow \overline{F}$ le morphisme de réduction modulo $p$, $\Z[T]\rightarrow \Z/p[T]$. On a donc $\overline{P}|\overline{Q}(T^p)=\overline{Q}(T)^p$ dans $\F_p[T]$. Mais puisque $\F_p[T]$ est factoriel, tout diviseur irréductible de $\overline{P}$ est en particulier un diviseur irréductible de $\overline{Q}$. Fixons $\Pi\in \F_p[T]$  un diviseur irréductible  de $\overline{P}$ donc de $\overline{Q}$. On a donc $$\Pi^2|\overline{P}\overline{Q} |\overline{\Phi}_n|T^n-\overline{1}$$ dans $\F_p[T]$. On peut donc écrire $T^n-\overline{1}= \Pi^2\Xi$ dans $\F_p[T]$. En particulier, $nT^{n-1}=2\Pi\Pi'\Xi+\Pi^2\Xi'$ donc $\Pi|T^n$ dans $\F_p[T]$ donc $\Pi| (T^n-\overline{1})-T^n=\overline{1}$ dans $\F_p[T]$: contradiction.
  \end{enumerate}
  \end{proof}
   \subsubsection{}\label{GaloisRad}\textbf{Extensions radicielles.} Soit $k$ un corps de caractéristique $p\geq 0$ et $n\in \Z_{\geq 1}$ tel que $|\mu_n:=\mu_n(k)|=n$ \textit{i.e.} si $p>0$, $p\not| n$ et $k$ contient toutes les racines $n$-ièmes de $1$. Fixons $a\in k$ et notons $k_{n,a}/k$ un corps de décomposition de $T^n-a$; puisque $T^n-a\in k[T]$ est séparable,  $k_{n,a}/k$ est galoisienne. 
   Si $\alpha\in Z_k(T^n-a)$ on a $Z_{\overline{k}}(T^n-a)=Z_k(T^n-a)=\lbrace \zeta \alpha\; |\; \alpha\in \mu_n\rbrace$. Donc $k_{n,a}=k(\alpha)$ et on dispose   d'un morphisme de groupes  injectif 
   $$\begin{tabular}[t]{ccc}
   $Gal(k_{n,a}/k)$&$\hookrightarrow$&$\mu_n$\\
   $\sigma$&$\rightarrow$
&$\frac{\sigma(\alpha)}{\alpha}$   \end{tabular}$$
d'image l'unique sous-groupe  d'ordre $n_a:=[k_{n,a}:k]|n$ \textit{i.e.} $\mu_{n_a}$. En particulier $$\prod_{\zeta\in \mu_{n_a}}(T-\zeta\alpha)=\alpha^{n_a}((\alpha^{-1}T)-\zeta)=T^{n_a}-\alpha^{n_a}\in k[T]$$
donc $\alpha^{n_a}\in k$. Cela montre que $n_a$ est aussi le plus petit $m\in \Z_{\geq 1}$ tel que  $\alpha^m\in k$ ou, encore (puisque $|Z_k(T^n-1)|=n$), tel que $a^m\in k^{\times n}$. Autrement dit, $[k_{n,a}:k]=n_a$ est l'ordre de l'image de $a$ dans $k^\times/k^{\times n}$. On a donc montré la première partie de la proposition suivante.\\
 
 \paragraph{}\label{GaloisRadProp}\textbf{Proposition.} \textit{$k_{n,a}/k$ est Galoisienne de degré l'ordre $n_a$ de $a$ dans $k^\times/k^{\times n}$ et on a un isomorphisme de groupes  $$\begin{tabular}[t]{ccc}
   $Gal(k_{n,a}/k)$&$\tilde{\rightarrow}$&$\mu_{n_a}$\\
   $\sigma$&$\rightarrow$
&$\frac{\sigma(\alpha)}{\alpha}$   \end{tabular}$$
Réciproquement, toute extension $K/k$ galoisienne cyclique de degré $n$ est le corps de décomposition d'un polynôme irréductible  de la forme $T^n-a\in k[T]$.}
   \begin{proof} Il reste a démontrer la réciproque. Soit donc $K/k$ une extension galoisienne de groupe de Galois $Gal(K/k)\simeq \Z/n$ et $\sigma\in Gal(K/k)$ un générateur. On cherche à construire $\alpha\in K$ tel que $\sigma(\alpha)=\zeta \alpha$ pour $\zeta\in \frak{u}_n$. Tout élément non nul de la forme 
   $$\alpha=\sum_{0\leq i\leq n-1}\zeta^{-i}\sigma^i(x)$$
   conviendrait. Il faut donc s'assurer que le $k$-endomorphisme $\sum_{0\leq i\leq n-1}\zeta^{-i}\sigma^i:K\rightarrow K$ est non nul. Cela résulte du classique Lemme \ref{Dedekind}.
   \end{proof}
   
    \paragraph{}\label{Dedekind}\textbf{Lemme.}  \textit{Soit $K,L$ deux   corps. Tout sous-ensemble fini de $\SHom(K,L)$ est $L$-libre.}
   \begin{proof}Soit $\phi_1,\dots, \phi_n:K\rightarrow L$ $n$ morphismes de corps deux à deux distincts.  On raisonne par induction sur $n$. Si $n=1$, l'assertion est claire. Si $n\geq 2$ et si  $\phi_1,\dots, \phi_n:K\rightarrow L$ ne sont pas $L$-libre, il existe $x_1,\dots, x_n\in L$, tous non nuls par hypothèses de récurrence, tels que 
   $$x_1\phi_1+\cdots+x_n\phi_n=0.$$
 On a alors pour tout $x,y\in K$
$$(x_1\phi_1+\cdots+x_n\phi_n)(xy)= x_1\phi_1(x)\phi_1(y)+\cdots+x_n\phi_n(x)\phi_n(y)=0.$$
En particulier, pour tout $x\in K$ on a $$(*)\;\;x_1\phi_1(x)\phi_1 +\cdots+x_n\phi_n(x)\phi_n =0. $$ Mais on a aussi, 
   $$(**)\;\; \phi_1(x)(x_1\phi_1+\cdots+x_n\phi_n)=   x_1\phi_1(x)\phi_1+\cdots+x_n\phi_1(x)\phi_n=0.$$
   Comme $\phi_1\not=\phi_2$, il existe $x\in K$ tel que $\phi_1(x)\not=\phi_2(x)$, ce qui en faisant $(**)-(*)$, implique
   $$x_2(\phi_1(x)-\phi_2(x))\phi_2+\cdots+x_n(\phi_1(x)-\phi_n(x))\phi_n=0$$
   avec $x_2(\phi_1(x)-\phi_2(x))\not=0$, ce qui contredit l'hypothèse de récurrence.   \end{proof}
   
  \paragraph{}\label{RadIt}\textbf{Corollaire.}  \textit{Soit  $K/k$ une extension finie engendrée par des éléments $\alpha_1,\dots ,\alpha_r\in \overline K$ tels que $\alpha_i^n\in k$. Alors $K/k$ est galoisienne de groupe $Gal(K/k)$ abélien.}
   \begin{proof} L'extension $K/k$ est galoisienne puisque c'est le corps de décomposition du polynôme séparable $(T^n-\alpha_1^n)\cdots (T^n-\alpha_r^n)\in k[T]$ sur $k$. De plus, puisque $K=k(\alpha_1,\dots ,\alpha_r)$ le morphisme de groupes 
   
   $$Gal(K/k)\rightarrow \prod_{1\leq i\leq r}Gal(k(\alpha_i/k)),\;\; \sigma\rightarrow (\sigma|_{k(\alpha_i)})_{1\leq i\leq r}$$
   est injectif; la conclusion résulte donc de la première partie de \ref{GaloisRadProp}.      \end{proof}

   
    
     \subsection{}\label{GaloisInv1}\textbf{Lemme.}  \textit{Soit $K$ un corps et $G\subset \SAut(K)$ un sous-groupe fini. Alors $K/K^G$ est galoisienne et $Gal(K/K^G)=G$.}
     \begin{proof}Pour tout  $x\in K$ le polynôme $\Pi_x:=\prod_{y\in G\cdot x}(T-y)$ est par construction séparable, dans $ K^G[T]$ et $P_x(x)=0$. En particulier $x$ est algébrique sur $K^G$ et son polynôme minimal $P_x\in K^G[T]$ sur $K^G$ divise $\Pi_x$ donc est de degré $\leq |G\cdot x|\leq |G|$. Cela montre que $K/K^G$ est algébrique séparable. Elle est de plus de degré fini $\leq |G|$ sinon il existerait une sous-extension séparable finie $K^G\subset L\subset K$ de degré $[L:K^G]>|G|$. Mais par  \ref{Separable4}, $L=K^G(x)$: contradiction. Par ailleurs, puisque $G\subset \SAut(K/K^G)$, on a $  |G|\leq |\SAut(K/K^G)|\leq [K:K^G]\leq |G|$ donc  $|G|=| \SAut(K/K^G)|=[K:K^G]$, $K/K^G $ est galoisienne et  $Gal(K/K^G)=\SAut(K/K^G)=G$.  \end{proof} 
     
\subsection{}\label{GaloisInv2}\textbf{Corollaire.}  \textit{Pour tout groupe fini $G$ il existe une extension galoisienne $K/k$ de groupe  $Gal(K/k)=G $.}
      \begin{proof}D'après \ref{GaloisInv1}, il suffit de montrer qu'il existe un corps $K$ tel que  $G$ est un sous-groupe de $ \SAut(K)$. Or on peut toujours plonger $G$ dans un groupe de permutations $\mathcal{S}_n$ (\textit{e.g} en faisant agir $G$ sur lui-même par translation et en prenant $n=|G|$). Or pour n'importe quel corps $k$, $\mathcal{S}_n$ l'action de $\mathcal{S}_n$ par permutation des coordonnées induit un $k$-plongement canonique   $\mathcal{S}_n\hookrightarrow \SAut(k(X_1,\dots, X_n)/k)$.      \end{proof}
     \textbf{Remarque.} Un problème beaucoup plus difficile est de savoir si, étant donné un corps $k$, tout groupe fini $G$ est le groupe de Galois d'une extension galoisienne $K/k$ ou, de fa\c{c}on équivalente, est un quotient de $\SAut(\overline{k}/k)$; c'est ce qu'on appelle le problème de Galois inverse pour $k$.  Il y a des corps $k$ pour lesquels on sait que la réponse est non (les corps algébriquement clos, les corps finis, $\R$, les extensions finies de $\Q_p$ \textit{etc.}) car le groupe $\SAut(\overline{k}/k)$ est trop simple. En fait, la complexité du  groupe $\SAut(\overline{k}/k)$ est une bonne mesure de la complexité arithmétique de $k$ (plus   $\SAut(\overline{k}/k)$ a de quotients finis, plus il y a de possibilités pour les groupes de Galois des polynômes à coefficients dans $k$ et donc plus c'est difficile de résoudre une équation à coeffiicients dans $k$). Par exemple, on ne sait pas résoudre le problème de GAlois inverse pour  $\Q$ ou  les extensions de type fini de $\Q$ - par exemple  $\Q(T)$ (en fait, si on savait le résoudre pour $\Q(T)$ on saurait le résoudre pour $\Q$; c'est une conséquence du théorème d'irréductibilité de Hilbert). On sait par contre le résoudre - par des méthodes géométriques - pour   des corps comme $\C(T)$, $\overline{\Q}(T)$, $\Q_p(T)$.  
 
 \subsection{}\label{Galois2}\textbf{Proposition} \textit{Soit $K/k$ une extension galoisienne et $k\subset L\subset K$ une sous-extension.
 \begin{enumerate} 
 \item $K/L$ est galoisienne et $L=K^{Gal(K/L)}$;
 \item Pour tout $\sigma\in Gal(K/k)$, $\sigma Gal(K/L)\sigma^{-1}=Gal(K/\sigma(L))$
 \item $L/k$ est galoisienne (de fa\c{c}on équivalente normale) si et seulement si $Gal(K/L)$ est normal dans $Gal(K/k)$, auquel cas, le morphisme de  restriction $ Gal(K/k)\rightarrow Gal(L/k)$, $\sigma\rightarrow \sigma|_K$ est bien défini et induit   une suite exacte courte de groupes
  $$1\rightarrow Gal(K/L)\rightarrow Gal(K/k)\rightarrow Gal(L/k)\rightarrow 1.$$
 \end{enumerate}}
 \begin{proof} (1) $K/L$ est galoisienne puisque normale et séparable (\ref{Normale5} - Contre-Exemple, \ref{Separable3}) et l'égalité $L=K^{Gal(K/L)}$ résulte alors de la caractérisation (5) de \ref{Galois1}. (2) Pour tout $\sigma\in Gal(K/k)$, $\tau\in Gal(K/L)$ et $x\in L$ on a $\sigma \tau\sigma^{-1}(\sigma(x))=\sigma\tau(x)=\sigma(x)$ donc $\sigma Gal(K/L)\sigma^{-1}\subset Gal(K/\sigma(L))$. Par symétrie, $ \sigma^{-1} Gal(K/\sigma(L))\sigma \subset Gal(K/L)$.
  (3) La deuxième partie de l'assertion et la condition nécessaire de la première partie résulte de \ref{Normale4}. Pour la condition suffisante, si $Gal(K/L)$ est normal dans $Gal(K/k)$, d'après (2), pour tout $\sigma\in Gal(K/k)$ on a $Gal(K/L)=Gal(K/\sigma(L))=:H$. Mais par (1) on a alors $L=K^H=\sigma(L)$. Comme par ailleurs $K/k$ est galoisienne, si $K\subset \overline{k}$ est une clôture algébrique, tout $\sigma\in \SAut(\overline{k}/k)$ se restreint en $\sigma|_K\in Gal(K/k)$ donc, en fait, on a bien que pour tout $\sigma\in \SAut(\overline{k}/k)$, $\sigma(L)=L$ \textit{i.e.} $L/k$ est normale. Enfin $L/k$ est séparable par \ref{Separable3}.   \end{proof}
 
  \subsection{}Soit $K/k$ une extension galoisienne. Notons $\mathcal{S}(K/k)$ l'ensemble des sous-extensions de $K/k$ et $\mathcal{S}(Gal(K/k))$ l'ensemble des sous-groupes de $Gal(K/k)$.  \\
  
\subsubsection{}\label{Galois3}\textbf{Corollaire} (Correspondance de Galois) \textit{Les applications 
  $$\begin{tabular}[t]{ccc}
   $\mathcal{S}( Gal(K/k))$&$\rightarrow$&   $\mathcal{S}(K/k)$\\
  $H$&$\rightarrow $&$K^H:=\lbrace x\in K\; |\; \sigma(x)=x,\; \sigma \in H\rbrace$\\
  \end{tabular},\;\; \begin{tabular}[t]{ccc}
   $\mathcal{S}( K/k)$&$\rightarrow$&   $\mathcal{S}(Gal(K/k))$\\
  $L$&$\rightarrow $&$Gal(K/L)$\\
  \end{tabular}$$  induisent des bijections inverses l'une de l'autre, décroissantes pour $\subset$ et telles que les sous-extensions  galoisiennes (de fa\c{c}on équivalente normales) de $K/k$ correspondent aux sous-groupes normaux de $Gal(K/k)$.}
  
  \begin{proof} Résulte de \ref{GaloisInv1} et \ref{Galois2}. \end{proof}
  
\subsection{}\textbf{Exemples.}  
 \subsubsection{Extensions de Kummer de $\Q$} Soit $n\in \Z_{\geq 1}$, $\zeta\in\overline{\Q}$ une racine primitive $n$-ième de l'unité et $a\in \Q$ dont l'image dans $\Q^\times/(\Q^\times)^n$ est d'ordre $n$. On a vu que
   \begin{itemize}
   \item (\ref{GaloisRad}) le polynôme $T^n-a\in \Q(\zeta)[T]$ est  irréductible sur $\Q(\zeta)$ et que l'extension $\Q(\zeta,^n\sqrt{a})/\Q(\zeta)$ est galoisienne de groupe $Gal(\Q(\zeta,^n\sqrt{a})/\Q(\zeta))\tilde{\rightarrow }\mu_n$, $\sigma\rightarrow \sigma(^n\sqrt{a})/^n\sqrt{a}$;
   \item (\ref{GaloisCyc}) le polynôme $\Phi_n\in \Q[T]$ est   irréductible sur $\Q $ et que l'extension $\Q(\zeta)/\Q$ est galoisienne de groupe $\chi_\Q:Gal(\Q(\zeta)/\Q)\tilde{\rightarrow} (\Z/n)^\times$.
   \end{itemize}
   L'extension $\Q(\zeta,^n\sqrt{a})/\Q$ est galoisienne puisque c'est un corps de décomposition du polynôme séparable $T^n -a\in \Q[T]$ sur $\Q$ et on a, d'après \ref{Galois2} une suite exacte courte de groupes
   $$1\rightarrow Gal(\Q(\zeta,^n\sqrt{a})/\Q(\zeta))\rightarrow Gal(\Q(\zeta,^n\sqrt{a})/\Q)\rightarrow Gal(\Q(\zeta)/\Q)\rightarrow 1.$$
   Cette suite exacte se scinde. En effet, considérons le sous-groupe $ Gal(\Q(\zeta,^n\sqrt{a})/\Q(^n\sqrt{a}))\subset  Gal(\Q(\zeta,^n\sqrt{a})/\Q)$.  En effet, on a clairement $$Gal(\Q(\zeta,^n\sqrt{a})/\Q(^n\sqrt{a}))\cap Gal(\Q(\zeta,^n\sqrt{a})/\Q(\zeta))=1$$ 
   donc un morphisme de groupe injectif $ Gal(\Q(\zeta,^n\sqrt{a})/\Q(^n\sqrt{a}))\hookrightarrow Gal(\Q(\zeta)/\Q)$.  Mais puisque $\Q(\zeta,^n\sqrt{a})/\Q(^n\sqrt{a})$ est galoisienne, on a aussi
   $$|Gal(\Q(\zeta,^n\sqrt{a})/\Q(^n\sqrt{a}))|=[\Q(\zeta,^n\sqrt{a}):\Q(\zeta)]=\frac{[\Q(\zeta,^n\sqrt{a}:\Q]}{[\Q(^n\sqrt{a}):\Q]}=\frac{n\phi(n)}{n}=\phi(n)=|Gal(\Q(\zeta)/\Q)|$$
   d'o\`u, en fait, un isomorphisme de groupes $ Gal(\Q(\zeta,^n\sqrt{a})/\Q(^n\sqrt{a}))\tilde{\rightarrow} Gal(\Q(\zeta)/\Q)$. On peut même calculer facilement l'action par conjugaison de $Gal(\Q(\zeta,^n\sqrt{a})/\Q(^n\sqrt{a}))$ sur $Gal(\Q(\zeta,^n\sqrt{a})/\Q(\zeta))$: soit $\sigma\in Gal(\Q(\zeta,^n\sqrt{a})/\Q(^n\sqrt{a}))$, $\tau\in Gal(\Q(\zeta,^n\sqrt{a})/\Q(\zeta))$ alors $$\frac{\sigma\tau\sigma^{-1}(^n\sqrt{a})}{^n\sqrt{a}}=\sigma(\frac{\tau(\sigma^{-1}(^n\sqrt{a}))}{\sigma^{-1}(^n\sqrt{a})})=\sigma(\frac{\tau(^n\sqrt{a})}{^n\sqrt{a}})=(\frac{\tau(^n\sqrt{a})}{^n\sqrt{a}})^{\chi_\Q(\sigma)}=\frac{\tau^{\chi_\Q(\sigma)}(^n\sqrt{a})}{^n\sqrt{a}}$$
   \textit{i.e.} $\sigma\tau\sigma^{-1}=\tau^{\chi_\Q(\sigma)}$. On a donc un isomorphisme de groupes $$Gal(\Q(\zeta,^n\sqrt{a})/\Q)\tilde{\rightarrow} \Z/n\rtimes (\Z/n)^\times,\;\; \sigma\rightarrow (\frac{\sigma(^n\sqrt{a})}{^n\sqrt{a}},\sigma|_{\mu_n})$$
   avec, à gauche, la structure de produit direct `tautologique' (donnée par l'action naturelle de $\Z/n^\times$ sur $\mu_n$ par $u\cdot\zeta=\zeta^u$). 
    \subsection{}\textbf{Clôture normale.} Soit $K/k$ une extension algébrique, $ \overline{k}$ une clôture algébrique (contenant $K$) et $ K_i/k$, $i\in I$ des sous-extensions normales de $\overline{k}/k$ contenant $K$. Alors $\cap_{i\in I}K_i/k$ est encore une extension normale contenant $K$ puisque pour tout $\sigma\in\SAut(\overline{k}/k)$, $\sigma(K_i)=K_i$ et  $\sigma(\cap_{i\in I}K_i)=\cap_{i\in I}\sigma(K_i)$. Il existe donc une plus petit sous-extension normale $\widehat{K}/k$ de $\overline{k}/k$  contenant $K$ appelé \textit{clôture normale}\index{Normale (Clôture)} de $K/k$ dans $\overline{k}/k$.\\
 
 
 \textbf{Lemme.} \textit{Soit $\widetilde{K}/K$ une extension algébrique. Les propriétés suivantes sont équivalentes.
 \begin{enumerate}
 \item $\widetilde{K}$ est $K$-isomorphe à $\widehat{K}$;
 \item $\widetilde{K}/k$ est normale et engendrée, comme $k$-extension de corps par les $\sigma(K)$, $\sigma\in \SHom_k(K,\widetilde{K})$.
 \end{enumerate}}
 
  On dit qu'une extension$\widetilde{K}/k$ vérifiant les propriétés équivalentes du lemme ci-dessus est une clôture normale de $K/k$. La propriété (1) montre qu'elle est unique à isomorphisme (non unique) près. 
 
 \begin{proof} Si $\overline{k}'/k$ est une autre clôture algébrique contenant $K$ et $\sigma:\overline{k}\tilde{\rightarrow}\overline{k}'$ un $k$-isomorphisme,  $\sigma(\widehat{K})/k$ est la clôture normale de $\sigma(K)/k$ dans $\overline{k}'/k$. On peut donc supposer que $\widehat{K},\widetilde{K}\subset \overline{k}$.  Comme  $\widetilde{K}/k$ est normale,  $\widehat{K}\subset \widetilde{K}$. En outre,  $\widetilde{K}/k$   est aussi la sous-$k$-extension de $\overline{k}/k$ engendrée comme $k$-extension par les  $\sigma(K)$, $\sigma\in \SHom_k(K,\overline{k})$. Or  tout $\sigma\in \SHom_k(K,\overline{k})$ s'étend en un $k$-plongement $\widehat{\sigma}\in \SHom_k(\widehat{K},\overline{k})$ qui, comme $\widehat{K}/k$ est normale, vérifie $\widehat{\sigma}(\widehat{K})=\widehat{K}$ Donc $\widetilde{K}\subset \widehat{K}$. \end{proof}
 
 
 \textbf{Exemple.} Si $K/k$ est une extension séparable fini, par \ref{Separable4} il existe $x\in K$ tel que $K=k(x)/k$. Si $P_x\in k[T]$ est le polynôme minimal de $x$ sur $k$, les clôtures normales de $K/k$ sont les corps de décomposition de $P_x$ sur $k$. En particulier, la clôture normale $\widehat{K}/k$ de $K/k$ est alors galoisienne finie (de degré divisant $[K:k]!$) et on dit plutôt que $\widehat{K}/k$ est la \textit{clôture galoisienne}\index{Galoisienne (Clôture)} de $K/k$. Par exemple, $\Q(^3\sqrt{5},j)/\Q$ est la clôture galoisienne de $\Q(^3\sqrt{5})/\Q$.
 

   
      \subsection{}\textbf{Résolubilité par radicaux.} Etant donné un polynôme   séparable $P\in k[T]$, notons $K_P/k$ un corps de décomposition de $P$ sur $k$ et $G_P:=Gal(K_P/k)$. Si $P$ est de degré $n$ et $Z_K(P)=\lbrace x_1,\dots, x_n\rbrace$, la restriction induit un morphisme de groupes injectifs $$Gal(K_P/k)\hookrightarrow \mathcal{S}(\lbrace x_1,\dots, x_n\rbrace) \simeq \mathcal{S}_n.$$
 
 \subsubsection{}\textbf{Lemme.} \textit{$P\in k[T]$ est irréductible sur $k$ si et seulement si l'action de $G_P$ sur $Z_K(P)$ est transitive.}
 
 \begin{proof} Si $x\in Z_K(P)$, $P\in k[T]$ est irréductible sur $k $ si et seulement si c'est le polynôme minimal de $P_x$ de $x$ sur $k$ \textit{i.e.} (caractérisation (5) de \ref{Galois1}) si et seulement si $P=\prod_{y\in Gal(K_P/k)}(T-y)$.  \end{proof}
 
  A l'opposé, si $P$ se factorise en produit de polynômes irréductibles $P=P_1\cdots P_r$ dans $k[T]$, les sous-extensions $K_{P_i}/k$ de $K_P/k$ étant galoisiennes, on a des suites exactes courtes 
 $$1\rightarrow Gal(K/K_i)\rightarrow G_P\stackrel{-|_{K_{P_i}}}{\rightarrow} G_{P_i})\rightarrow 1,\; i=1,\dots, r$$
 et un morphisme $G_P\hookrightarrow \prod_{1\leq i\leq r} G_{P_i}$ induit par le produit des $-|_{K_{P_i}:G_P \twoheadrightarrow} G_{P_i}$ est injectif puisque $K_P=k(Z_K(P_1)\cup\dots\cup Z_K(P_r))$.
 
  \textbf{Exemple.} Soit $\zeta$ une racine primitive $5$ème de $1$. L'extension $\Q(^3\sqrt{5},j,\zeta)/\Q$ est galoisienne puisque c'est le corps de décomposition du polynôme séparable $P=(T^3-5)\Phi_5$. En posant $P_1:=T^3-5$, $P_2=\Phi_5$, on a donc un morphisme de groupes injectif $G_P\hookrightarrow G_{P_1}\times G_{P_2}=\mathcal{S}_3\times (\Z/5)^\times $. De plus $T^3-5$ est irréductible sur $\Q(\zeta)$...\\
 
 
 \subsubsection{}On dit qu'un polynôme $P\in k[T]$ est \textit{résoluble par radicaux sur $k$}\index{Résoluble par radicaux (Polynôme)} si ses racines peuvent s'exprimer à partir des  éléments de $k$ en appliquant successivement les opérations $+,-,/, ^n\sqrt{-}$. Autrement dit, cela signifie qu'on a une suite de sous-extensions $$K_P=K_{n+1}\supset K_n\supset\cdots \supset K_1\supset K_0=k$$
 telles que $K_{i+1}=K_i(x_i)$ et $x_i^{n_i}\in K_i$, $i=0,\dots, n$.\\
 
 
  On dit qu'un  groupe  $G$ est \textit{résoluble}\index{Résoluble (Groupe)} s'il existe une suite de sous-groupes $$G=:G_0\supset G_1\supset\cdots\supset G_n\supset G_{n+1}=1$$
 tels que $G_{i+1 }$ est normal dans $G_{i }$ et $G_i/G_{i+1}$ est abélien, $i=0,\dots, n$. (Il n'est pas difficile de vérifier que cette définition est équivalente à celle donnée précédemment à savoir que $D^nG=1$, $n\gg 0$). \\
 
 \textbf{Exercice.} Montrer que tout sous-groupe et tout quotient d'un groupe résoluble est résoluble et que toute extension de groupes résolubles est résoluble.\\
 
  \subsubsection{}\label{GaloisReso}\textbf{Théorème.} \textit{Un polynôme séparable $P\in k[T]$ est résoluble par radicaux sur $k$ si et seulement si $G_P$ est résoluble.}
 
 \begin{proof} Notons $p\geq 0$ la caractéristique de $k$. Supposons d'abord $P\in k[T]$  résoluble par radicaux sur $k$ et soit  $$K_P={n+1}\supset K_n\supset\cdots \supset K_1\supset K_0=k$$
 une suite de sous-extensions telles que $K_{i+1}=K_i(x_i)$ et $x_i^{n_i}\in K_i$, $i=0,\dots, n$. On peut supposer que $p\not| n_i$ et que $T^{n_i}-x_i^{n_i}\in K_i[T]$ est irréductible sur $K_i$, $i=0,\dots, n$. Comme $K_P/K_i$ est galoisienne et contient $x_i$, $K_P$ contient les racines $n_i$ième de $1$. Donc, en posant $n=pgcm(n_0,\dots, n_n)$, $K_P$ contient les racines $n$ième de $1$. On peut donc supposer que $K_1=k(\zeta)$ pour $\zeta\in K_P$ une racine primitive $n$ième de $1$. Chacune des extensions $K_{i+1}/K_i$ est alors galoisienne de groupe $Gal(K_{i+1}/K_i)$ abélien. Mais $K_i/k$ n'a aucune raison d'être galoisienne pour $n\geq 2$. Rempla\c{c}ons donc les $K_i/k$ par leur clôture galoisienne. Comme $K_{n+1}/k$ est galoisienne et par minimalité de la clôture galoisienne on a des inclusions
 
 $$\xymatrix{K_{n+1}\ar@{=}[d]&\supset&K_n\ar@{_{(}->}[d]&\supset&\cdots&\supset&K_2\ar@{_{(}->}[d]&\supset &K_1\ar@{=}[d]\supset& K_0=k\ar@{=}[d]\\
K_{n+1} &\supset&\widehat{K}_n&\supset&\cdots&\supset&\widehat{K}_2&\supset & K_1\supset& K_0=k\\}$$
 Il reste à voir que $Gal(\widehat{K}_{i+1}/\widehat{K}_i)$ est abélien, $i=0,\dots, n$. Mais $\widehat{K}_{i+1}$ est engendrée comme $k$-extension par les $x_{i+1},\sigma}:=\sigma(x_{i+1})$, $\sigma\in Gal(K_{n+1}/k)$. Or $\sigma(x_{i+1})^{n_i}=\sigma(x_{i+1}^{n_i})\in \sigma(K_i)\subset \widehat{K}_i$; la conclusion résulte donc de   \ref{RadIt}. Inversement, supposons $G_P$ résoluble. Notons $n:=[K_P:k]$ et $\widetilde{K}_P/K_P$ un corps de décomposition de $T^n-1\in k[T]\subset K_P[T]$ sur $K_P$. L'extension $\widetilde{K}_P/k$ est encore galoisienne car c'est le corps de décomposition du polynôme séparable $P(T^n-1)/pgcd(P,T^n-1)\in k[T]$ sur $k$. En particulier, on a une suite exacte courte de groupes finis 
 $$1\rightarrow Gal(\widetilde{K}_P/K_P)\rightarrow Gal(\widetilde{K}_P/k)\rightarrow G_P=Gal(K_P/k)\rightarrow 1,$$
 ce qui montre que $Gal(\widetilde{K}_P/k)$ est extension d'un groupe résoluble  par un groupe abélien donc est résoluble.
  \end{proof}
 
 \subsection{}\textbf{Spécialisation.}  Si le degré de $P$ est $\geq 5$, on ne dispose  pas de formule universelle pour calculer les racines de $P$ et déterminer $G_P$  est en général une question difficile. Le critère de spécialisation que nous allons voir permet souvent de réduire la question au problème de la factorisation des polynômes sur les corps finis, problème qu'on sait résoudre algorithmiquement. 



 
 
 
 
 $$***$$ 
 
 Syllabus prochaines séances:\\
$$ \begin{tabular}[t]{l}
 \\
\end{tabular}$$
  \begin{tabular}[t]{l}
\textit{anna.cadoret@imj-prg.fr}\\
 IMJ-PRG, Sorbonne Université\\
 Paris, FRANCE
\end{tabular}
 \printindex

\end{document}
%%%%%%%%%%%%%%%%%%%
%%%%%%%%%%%%%%%%%%%
%%%%%%%%%%%%%%%%%%%
%%%%%%%%%%%%%%%%%%%
%%%%%%%%%%%%%%%%%%%
 
 
  
